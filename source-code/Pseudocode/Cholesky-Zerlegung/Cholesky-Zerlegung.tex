\documentclass{article}
\usepackage[pdftex,active,tightpage]{preview}
\setlength\PreviewBorder{2mm}

\usepackage[utf8]{inputenc} % this is needed for umlauts
\usepackage[ngerman]{babel} % this is needed for umlauts
\usepackage[T1]{fontenc}    % this is needed for correct output of umlauts in pdf
\usepackage{amssymb,amsmath,amsfonts} % nice math rendering
\usepackage{braket} % needed for \Set
\usepackage{algorithm,algpseudocode}

\begin{document}
\begin{preview}
    Sei $n \in \mathbb{N}_{\geq 1}$, $A \in \mathbb{R}^{n \times n}$ und 
    positiv definit sowie symmetrisch.

    Dann existiert eine Zerlegung $A = L \cdot L^T$, wobei $L$ eine
    untere Dreiecksmatrix ist. Diese wird von folgendem Algorithmus 
    berechnet:

    \begin{algorithm}[H]
        \begin{algorithmic}
            \Function{Cholesky}{$A \in \mathbb{R}^{n \times n}$}
                \State $L = \Set{0} \in \mathbb{R}^{n \times n}$ \Comment{Initialisiere $L$}
                \For{($k=1$; $\;k \leq n$; $\;k$++)}
                    \State $L_{k,k} = \sqrt{A_{k,k} - \sum_{i=1}^{k-1} L_{k,i}^2}$
                    \For{($i=k+1$; $\;i \leq n$; $\;i$++)}
                        \State $L_{i,k} = \frac{A_{i,k} - \sum_{j=1}^{k-1} L_{i,j} \cdot L_{k,j}}{L_{k,k}}$
                    \EndFor
                \EndFor
                \State \Return $L$
            \EndFunction
        \end{algorithmic}
    \caption{Cholesky-Zerlegung}
    \label{alg:seq1}
    \end{algorithm}
\end{preview}
\end{document}
