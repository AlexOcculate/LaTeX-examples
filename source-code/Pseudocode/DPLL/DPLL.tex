\documentclass{article}
\usepackage[pdftex,active,tightpage]{preview}
\setlength\PreviewBorder{2mm}

\usepackage[utf8]{inputenc} % this is needed for umlauts
\usepackage[ngerman]{babel} % this is needed for umlauts
\usepackage[T1]{fontenc}    % this is needed for correct output of umlauts in pdf
\usepackage{amssymb,amsmath,amsfonts} % nice math rendering
\usepackage{braket} % needed for \Set
\usepackage{algorithm,algpseudocode}

\algnewcommand\True{\textbf{true}\space} 
\algnewcommand\False{\textbf{false}\space} 
\algnewcommand{\LineComment}[1]{\State \(\triangleright\) #1}
\begin{document}
\begin{preview}
    \begin{algorithm}[H]
        \begin{algorithmic}
            \Require Klauselmenge $K$, Symbolmenge $S$, (partielles) Modell $M$

			\Procedure{DPLL}{$K$, $S$, $M$}
            \If{Alle Klauseln sind wahr in $M$}
                \State \Return \texttt{true}
            \EndIf

            \If{Eine Klausel ist falsch in $M$}
                \State \Return \texttt{false}
            \EndIf
			\\

            \State $P, Wert \gets \Call{FindeEinheitsklausel}{K, S, M}$
            \If{$P$ existiert}
                \State $M \gets \Call{SetzePInModell}{P, Wert, M}$
                \State \Return $\Call{DPLL}{k, S \setminus \Set{P}, M}$
            \EndIf
			\\

			\LineComment{Ein reines Literal ist eines, das in jeder Klausel}
			\LineComment{immer wahr bzw. immer falsch ist}
            \State $P, Wert \gets \Call{FindeReinesLiteral}{K, M}$
            \If{$P$ existiert}
                \State $m \gets \Call{SetzePInModell}{P, Wert, M}$
                \State \Return $\Call{DPLL}{K, S \setminus \Set{P}, M}$
            \EndIf
			\\

            \State $P \gets S.pop()$ \Comment{Symbolmenge schrumpft}
            \State $M_1 \gets \Call{SetzePInModell}{P, \texttt{true}, M}$
            \State $M_2 \gets \Call{SetzePInModell}{P, \texttt{false}, M}$
            \State \Return $\Call{DPLL}{K, S, M_1} \lor \Call{DPLL}{K, S, M_2}$
			\EndProcedure
        \end{algorithmic}
    \caption{DPLL-Verfahren}
    \label{alg:dpll}
    \end{algorithm}
\end{preview}
\end{document}
