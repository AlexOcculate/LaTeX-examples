\documentclass[a4paper,9pt]{scrartcl}
\usepackage{amssymb, amsmath} % needed for math
\usepackage[utf8]{inputenc} % this is needed for umlauts
\usepackage[ngerman]{babel} % this is needed for umlauts
\usepackage[T1]{fontenc}    % this is needed for correct output of umlauts in pdf
\usepackage[margin=2.5cm]{geometry} %layout
\usepackage{hyperref}   % links im text
\usepackage{color}
\usepackage{framed}
\usepackage{enumerate}  % for advanced numbering of lists
\usepackage{braket} % needed for nice printing of sets
\usepackage{xcolor}
\usepackage{lastpage}
\clubpenalty  = 10000   % Schusterjungen verhindern
\widowpenalty = 10000   % Hurenkinder verhindern

\hypersetup{ 
  pdfauthor   = {Martin Thoma}, 
  pdfkeywords = {Diskrete Mathematik}, 
  pdftitle    = {Graphentheorie I: Handout} 
} 

\usepackage{fancyhdr}
\pagestyle{fancy}
\lhead{Diskrete Mathematik}
\chead{Graphentheorie I (Martin Thoma)}
\rhead{Seite \thepage\ von \pageref{LastPage}}

%%%%%%%%%%%%%%%%%%%%%%%%%%%%%%%%%%%%%%%%%%%%%%%%%%%%%%%%%%%%%%%%%%%%%
% Custom definition style, by                                       %
% http://mathoverflow.net/questions/46583/what-is-a-satisfactory-way-to-format-definitions-in-latex/58164#58164
%%%%%%%%%%%%%%%%%%%%%%%%%%%%%%%%%%%%%%%%%%%%%%%%%%%%%%%%%%%%%%%%%%%%%
\makeatletter
\newdimen\errorsize \errorsize=0.2pt
% Frame with a label at top
\newcommand\LabFrame[2]{%
    \fboxrule=\FrameRule
    \fboxsep=-\errorsize
    \textcolor{FrameColor}{%
    \fbox{%
      \vbox{\nobreak
      \advance\FrameSep\errorsize
      \begingroup
        \advance\baselineskip\FrameSep
        \hrule height \baselineskip
        \nobreak
        \vskip-\baselineskip
      \endgroup
      \vskip 0.5\FrameSep
      \hbox{\hskip\FrameSep \strut
        \textcolor{TitleColor}{\textbf{#1}}}%
      \nobreak \nointerlineskip
      \vskip 1.3\FrameSep
      \hbox{\hskip\FrameSep
        {\normalcolor#2}%
        \hskip\FrameSep}%
      \vskip\FrameSep
    }}%
}}
\definecolor{FrameColor}{rgb}{0.25,0.25,1.0}
\definecolor{TitleColor}{rgb}{1.0,1.0,1.0}

\newenvironment{contlabelframe}[2][\Frame@Lab\ (cont.)]{% 
  % Optional continuation label defaults to the first label plus
  \def\Frame@Lab{#2}%
  \def\FrameCommand{\LabFrame{#2}}%
  \def\FirstFrameCommand{\LabFrame{#2}}%
  \def\MidFrameCommand{\LabFrame{#1}}%
  \def\LastFrameCommand{\LabFrame{#1}}%
  \MakeFramed{\advance\hsize-\width \FrameRestore} 
}{\endMakeFramed} 
\newcounter{definition}
\newenvironment{definition}[1]{%
  \par
  \refstepcounter{definition}%
  \begin{contlabelframe}{Definition \thedefinition:\quad #1}
 \noindent\ignorespaces}
{\end{contlabelframe}} 
\makeatother
%%%%%%%%%%%%%%%%%%%%%%%%%%%%%%%%%%%%%%%%%%%%%%%%%%%%%%%%%%%%%%%%%%%%%
% Begin document                                                    %
%%%%%%%%%%%%%%%%%%%%%%%%%%%%%%%%%%%%%%%%%%%%%%%%%%%%%%%%%%%%%%%%%%%%%
\begin{document}
\begin{definition}{Graph}
Ein Graph ist ein Tupel $(E, K)$, wobei $E \neq \emptyset$ die Eckenmenge und 
$K \subseteq E \times E$ die 
Kantenmenge bezeichnet.
\end{definition}

\begin{definition}{Inzidenz}
Sei $e \in E$ und $k = \Set{e_1, e_2} \in K$.

$e$ heißt \textbf{inzident} zu $k :\Leftrightarrow e = e_1$ oder $e = e_2$
\end{definition}

\begin{definition}{Vollständiger Graph}
Sei $G = (E, K)$ ein Graph.

$G$ heißt \textbf{vollständig} $:\Leftrightarrow K = E \times E \setminus \Set{e \in E: \Set{e, e}}$
\end{definition}

Ein vollständiger Graph mit $n$ Ecken wird als $K_n$ bezeichnet.

\begin{definition}{Bipartiter Graph}
Sei $G = (E, K)$ ein Graph und $A, B \subset E$ zwei disjunkte Eckenmengen mit
$E \setminus A = B$.

$G$ heißt \textbf{bipartit} $:\Leftrightarrow \forall_{k = \Set{e_1, e_2} \in K}: (e_1 \in A \text{ und } e_2 \in B) \text{ oder } (e_1 \in B \text{ und } e_2 \in A) $
\end{definition}

\begin{definition}{Vollständig bipartiter Graph}
Sei $G = (E, K)$ ein bipartiter Graph und $\Set{A, B}$ bezeichne die Bipartition.

$G$ heißt \textbf{vollständig bipartit} $:\Leftrightarrow A \times B = K$
\end{definition}

\begin{definition}{Kantenzug, Länge eines Kantenzuges und Verbindung von Ecken}
Sei $G = (E, K)$ ein Graph.

Dann heißt eine Folge $k_1, k_2, \dots, k_s$ von Kanten, zu denen es Ecken
$e_0, e_1, e_2, \dots, e_s$ gibt, so dass
\begin{itemize}
    \item $k_1 = \Set{e_0, e_1}$
    \item $k_2 = \Set{e_1, e_2}$
    \item \dots
    \item $k_s = \Set{e_{s-1}, e_s}$
\end{itemize}
gilt ein \textbf{Kantenzug}, der $e_0$ und $e_s$ \textbf{verbindet} und $s$ 
seine \textbf{Länge}.
\end{definition}

\begin{definition}{Geschlossener Kantenzug}
Sei $G = (E, K)$ ein Graph und $A = (e_0, e_1, \dots, e_s)$ ein Kantenzug.

A heißt \textbf{geschlossen} $:\Leftrightarrow e_s = e_0$ .
\end{definition}

\begin{definition}{Weg}
Sei $G = (E, K)$ ein Graph und $A = (k_1, k_2 \dots, k_s)$ ein Kantenzug.

A heißt \textbf{Weg} $:\Leftrightarrow \forall_{i, j \in 1, \dots, s}: i \neq j \Rightarrow k_i \neq k_j$ .
\end{definition}

\begin{definition}{Einfacher Kreis}
Sei $G = (E, K)$ ein Graph und $A = (k_1, k_2 \dots, k_s)$ ein Kantenzug.

A heißt \textbf{Kreis} $:\Leftrightarrow A$ ist geschlossen und ein Weg.
\end{definition}

\begin{definition}{Zusammenhängender Graph}
Sei $G = (E, K)$ ein Graph.

$G$ heißt \textbf{zusammenhängend} $:\Leftrightarrow \forall e_1, e_2 \in E: $ 
Es ex. ein Kantenzug, der $e_1$ und $e_2$ verbindet
\end{definition}

\begin{definition}{Grad einer Ecke}
Der \textbf{Grad} einer Ecke ist die Anzahl der Kanten, die von dieser Ecke
ausgehen.
\end{definition}

\begin{definition}{Isolierte Ecke}
Hat eine Ecke den Grad 0, so nennt man ihn \textbf{isoliert}.
\end{definition}

\begin{definition}{Eulerscher Kreis}
Sei $G$ ein Graph und $A$ ein Kreis in $G$.

$A$ heißt \textbf{eulerscher Kreis} $:\Leftrightarrow \forall_{e \in E}: e \in A$.
\end{definition}

\begin{definition}{Eulerscher Graph}
Ein Graph heißt \textbf{eulersch}, wenn er einen eulerschen Kreis enthält.
\end{definition}

\begin{definition}{Offene eulersche Linie}
Sei $G$ ein Graph und $A$ ein Weg, der kein Kreis ist.

$A$ heißt \textbf{offene eulersche Linie} von $G :\Leftrightarrow$ Jede Kante 
in $G$ kommt genau ein mal in $A$ vor.
\end{definition}

\end{document}
