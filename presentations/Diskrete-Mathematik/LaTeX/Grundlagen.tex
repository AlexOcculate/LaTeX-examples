\subsection{Grundlagen}
\begin{frame}{Graph}
\begin{block}{Graph}
Ein Graph ist ein Tupel $(E, K)$, wobei $E \neq \emptyset$ die Eckenmenge und 
$K \subseteq E \times E$ die 
Kantenmenge bezeichnet.
\end{block}
\pause
\tikzstyle{vertex}=[draw,fill=black,circle,minimum size=10pt,inner sep=0pt]

\begin{gallery}
    \galleryimage[Green]{graphs/graph-1}
    \galleryimage[Green]{graphs/graph-2}
    \galleryimage[Green]{graphs/k-3-3}
    \galleryimage[Green]{graphs/k-5}\\
    \galleryimage[Green]{graphs/k-16}
    \galleryimage[Green]{graphs/graph-6}
    \galleryimage[Green]{graphs/star-graph}
    \galleryimage[Green]{graphs/tree}
\end{gallery}
\end{frame}

\begin{frame}{Synonyme}

\begin{center}
\Huge{Knoten $\Leftrightarrow$ Ecken}
\end{center}

\end{frame}

\framedgraphic{Modellierung, Flüsse, Netzwerke}{../images/Unit_disk_graph.png}
\framedgraphic{Karten}{../images/map.png}
\framedgraphic{Good Will Hunting}{../images/good-will-hunting.jpg}
\framedgraphic{Graham's Number}{../images/hypercube.png}

\begin{frame}{Isomorphe Graphen}
\begin{center}
\href{http://www.martin-thoma.de/uni/graph.html}{martin-thoma.de/uni/graph.html}
\end{center}
\end{frame}

\begin{frame}{Grad einer Ecke}
\begin{block}{Grad einer Ecke}
Der \textbf{Grad} einer Ecke ist die Anzahl der Kanten, die von dieser Ecke
ausgehen.
\end{block}

\begin{block}{Isolierte Ecke}
Hat eine Ecke den Grad 0, so nennt man ihn \textbf{isoliert}.
\end{block}

\begin{gallery}
    \galleryimage{graphs/graph-1}
    \galleryimage{graphs/graph-2}
    \galleryimage{graphs/k-3-3}
    \galleryimage{graphs/k-5}\\
    \galleryimage{graphs/k-16}
    \galleryimage{graphs/graph-6}
    \galleryimage{graphs/star-graph}
    \galleryimage{graphs/tree}
\end{gallery}
\end{frame}

\begin{frame}{Schlinge}
\begin{block}{Schlinge}
Sei $G=(E, K)$ ein Graph und $k=\Set{e_1, e_2} \in K$ eine Kante.

$k$ heißt \textbf{Schlinge} $:\Leftrightarrow e_1 = e_2$ 
\end{block}

Ein Graph ohne Schlingen heißt \enquote{schlingenfrei}

\begin{gallery}
    \galleryimage{graphs/graph-1}
    \galleryimage{graphs/graph-2-schlinge}
    \galleryimage{graphs/k-3-3}
    \galleryimage{graphs/k-5-schlinge}
\end{gallery}
\end{frame}

\begin{frame}{Aufgabe 1}
Zeichnen Sie alle schlingenfreien Graphen mit genau vier Ecken.

\only<2>{
    \begin{gallery}
        \galleryimage{aufgabe-1/graph-8}    % vier einzelne Punkte
        \galleryimage{aufgabe-1/graph-7}    % nur eine Kante
        \galleryimage{aufgabe-1/graph-6}    % zwei Kanten
        \galleryimage{aufgabe-1/graph-11}   % zwei Kanten -------------
        \galleryimage{aufgabe-1/graph-12}   % drei Kanten: umgedrehtes u
        \galleryimage{aufgabe-1/graph-5}    % drei Kanten
        \galleryimage[red]{aufgabe-1/graph-4}    % drei Kanten: S3 - fehlt im Buch
        \galleryimage{aufgabe-1/graph-10}   % vier Kanten: Viereck
        \galleryimage{aufgabe-1/graph-3}    % vier Kanten: Dreieck mit Spitze
        \galleryimage[red]{aufgabe-1/graph-2} % fünf kanten - fehlt im Buch
        \galleryimage{aufgabe-1/graph-9}    %  fünf Kanten: nur Diagonale fehlt
        \galleryimage{aufgabe-1/graph-1}    % sechs Kanten: K_4
    \end{gallery}
}
\end{frame}

\begin{frame}{Inzidenz}
\begin{block}{Inzidenz}
Sei $e \in E$ und $k = \Set{e_1, e_2} \in K$.

$e$ heißt \textbf{inzident} zu $k :\Leftrightarrow e = e_1$ oder $e = e_2$
\end{block}

\pause
\tikzstyle{vertex}=[draw,fill=black,circle,minimum size=10pt,inner sep=0pt]

\begin{gallery}
    \galleryimage[Green]{inzidenz/graph-1}
    \galleryimage[Green]{inzidenz/graph-2}
    \galleryimage[Green]{inzidenz/k-3-3}
    \galleryimage[Green]{inzidenz/k-5}\\
    \galleryimage[Green]{inzidenz/k-16}
    \galleryimage[red]{inzidenz/graph-6}
    \galleryimage[Green]{inzidenz/star-graph}
    \galleryimage[Green]{inzidenz/tree}
\end{gallery}
\end{frame}
