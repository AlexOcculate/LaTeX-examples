\subsection{Königsberger Brückenproblem}

\framedgraphic{Königsberg heute}{../images/koenigsberg-bruecken-luftbild}

\framedgraphic{Königsberger Brückenproblem}{../images/Konigsberg_bridges.png}

\framedgraphic{Übersetzung in einen Graphen}{../images/Konigsberg_bridges-graph.png}

\begin{frame}{Übersetzung in einen Graphen}
\begin{center}
\adjustbox{max size={\textwidth}{0.8\textheight}}{
\documentclass[varwidth=true, border=2pt]{standalone}
\usepackage{tikz}
\usetikzlibrary{arrows,positioning} 
\tikzset{
    %Define standard arrow tip
    >=stealth',
    % Define arrow style
    pil/.style={->,thick}
}

\begin{document}
  \begin{tikzpicture}
      \node (a)[vertex] at (0,8) {$a$};
      \node (b)[vertex] at (0,4) {$b$};
      \node (c)[vertex] at (0,0) {$c$};
      \node (d)[vertex] at (4,4) {$d$};

      \foreach \from/\to/\pos in {a/b/20,a/b/-20,a/d/0,b/c/20,b/c/-20,b/d/0,c/d/0}
        \draw[line width=2pt] (\from) to [bend left=\pos] (\to);
  \end{tikzpicture}
\end{document}

}
\end{center}
\end{frame}

\begin{frame}{Eulerscher Kreis}
\begin{block}{Eulerscher Kreis}
Sei $G$ ein Graph und $A$ ein Kreis in $G$.

$A$ heißt \textbf{eulerscher Kreis} $:\Leftrightarrow \forall_{e \in E}: e \in A$.
\end{block}

\begin{block}{Eulerscher Graph}
Ein Graph heißt \textbf{eulersch}, wenn er einen eulerschen Kreis enthält.
\end{block}
\end{frame}

\pgfdeclarelayer{background}
\pgfsetlayers{background,main}
\begin{frame}{Eulerscher Kreis}
    \newcommand\n{5}
    \begin{center}
    \adjustbox{max size={\textwidth}{0.8\textheight}}{
    \begin{tikzpicture}
        \foreach \number in {1,...,\n}{
            \node[vertex] (N-\number) at ({\number*(360/\n)}:5.4cm) {};
        }

        \foreach \number in {1,...,\n}{
            \foreach \y in {1,...,\n}{
                \draw (N-\number) -- (N-\y);
            }
        }

        \node<2->[vertex,red] (N-1) at ({1*(360/\n)}:5.4cm) {};

        \begin{pgfonlayer}{background}
            \path<2->[selected edge] (N-1.center) edge node {} (N-2.center);
            \path<3->[selected edge] (N-2.center) edge node {} (N-3.center);
            \path<4->[selected edge] (N-3.center) edge node {} (N-4.center);
            \path<5->[selected edge] (N-4.center) edge node {} (N-5.center);
            \path<6->[selected edge] (N-5.center) edge node {} (N-1.center);
            \path<7->[selected edge] (N-1.center) edge node {} (N-3.center);
            \path<8->[selected edge] (N-3.center) edge node {} (N-5.center);
            \path<9->[selected edge] (N-5.center) edge node {} (N-2.center);
            \path<10->[selected edge] (N-2.center) edge node {} (N-4.center);
            \path<11->[selected edge](N-4.center) edge node {} (N-1.center);
        \end{pgfonlayer}
    \end{tikzpicture}
    }
    \end{center}
\end{frame}

\subsection{Satz von Euler}
\begin{frame}{Satz von Euler}
\begin{block}{Satz von Euler}
Wenn ein Graph $G$ eulersch ist, dann hat jede Ecke von $G$ geraden Grad.
\end{block}

\pause

$\Rightarrow$ Wenn $G$ eine Ecke mit ungeraden Grad hat, ist $G$ nicht eulersch.

\pause

\begin{gallery}
    \galleryimage{vollstaendig/k-5}
    \galleryimage{koenigsberg/koenigsberg-1}
\end{gallery}
\end{frame}

\begin{frame}{Umkehrung des Satzes von Euler}
\begin{block}{Umkehrung des Satzes von Euler}
Wenn in einem zusammenhängenden Graphen $G$ jede Ecke geraden Grad hat, dann 
ist $G$ eulersch.
\end{block}

Beweis per Induktion

TODO
\end{frame}

\begin{frame}{Offene eulersche Linie}
\begin{block}{Offene eulersche Linie}
Sei $G$ ein Graph und $A$ ein Weg, der kein Kreis ist.

$A$ heißt \textbf{offene eulersche Linie} von $G :\Leftrightarrow$ Jede Kante in $G$ kommt genau ein mal in $A$ vor.
\end{block}

Ein Graph kann genau dann "`in einem Zug"' gezeichnet werden, wenn er eine 
offene eulersche Linie besitzt.
\end{frame}

\begin{frame}{Offene eulersche Linie}
\begin{block}{Satz 8.2.3}
Sei $G$ ein zusammenhängender Graph.

$G$ hat eine offene eulersche Linie $:\Leftrightarrow G$ hat genau zwei Ecken 
ungeraden Grades.
\end{block}

\pause

\begin{block}{Beweis "`$\Rightarrow"'$}
Sei $G=(E, K)$ ein zusammenhängender Graph und $L = (e_0, \dots, e_s)$ eine offene
eulersche Linie. \pause
Sei $G^* = (E, K \cup \Set{e_s, e_0})$. \pause
Es gibt einen Eulerkreis in $G^*$ \pause \\
$\xRightarrow{\text{Satz von Euler}}$ In $G^*$ hat jede Ecke geraden Grad \pause \\
Der Grad von nur zwei Kanten wurde um jeweils 1 erhöht \pause \\
$\Rightarrow$ in $G$ haben genau 2 Ecken ungeraden Grad $\blacksquare$
\end{block}
\end{frame}

\pgfdeclarelayer{background}
\pgfsetlayers{background,main}
\begin{frame}{Haus des Nikolaus}
    \tikzstyle{selected edge} = [draw,line width=5pt,-,red!50]
    \begin{center}
    \adjustbox{max size={\textwidth}{0.8\textheight}}{
    \begin{tikzpicture}
        \node[vertex] (a) at (0,0) {};
        \node[vertex] (b) at (2,0) {};
        \node[vertex] (c) at (2,2) {};
        \node[vertex] (d) at (0,2) {};
        \node[vertex] (e) at (1,4) {};

        \draw (a) -- (d);
        \draw (d) -- (b);
        \draw (b) -- (c);
        \draw (c) -- (d);
        \draw (d) -- (e);
        \draw (e) -- (c);
        \draw (c) -- (a);
        \draw (a) -- (b);

        \node<2->[vertex, red] (a) at (0,0) {};

        \begin{pgfonlayer}{background}
            \path<2->[selected edge] (a.center) edge node {} (d.center);
            \path<3->[selected edge] (d.center) edge node {} (b.center);
            \path<4->[selected edge] (b.center) edge node {} (c.center);
            \path<5->[selected edge] (c.center) edge node {} (d.center);
            \path<6->[selected edge] (d.center) edge node {} (e.center);
            \path<7->[selected edge] (e.center) edge node {} (c.center);
            \path<8->[selected edge] (c.center) edge node {} (a.center);
            \path<9->[selected edge] (a.center) edge node {} (b.center);
        \end{pgfonlayer}
    \end{tikzpicture}
    }
    \end{center}
\end{frame}
