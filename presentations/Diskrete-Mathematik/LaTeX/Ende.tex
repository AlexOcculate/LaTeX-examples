\subsection{Aufgabe 3}
\begin{frame}{Aufgabe 3}
Zeigen Sie: Ein Kreis ist genau dann bipartit, wenn er gerade Länge hat.
\end{frame}

\pgfdeclarelayer{background}
\pgfsetlayers{background,main}
\begin{frame}{Aufgabe 3 - Lösung}
Idee: Knoten abwechselnd färben

    \tikzstyle{selected edge} = [draw,line width=5pt,-,black!50]
    \begin{center}
    \adjustbox{max size={\textwidth}{0.8\textheight}}{
    \begin{tikzpicture}
        \node[vertex] (a) at (0,0) {};
        \node[vertex] (b) at (2,0) {};
        \node[vertex] (c) at (2,2) {};
        \node[vertex] (d) at (0,2) {};
        \node[vertex] (e) at (1,4) {};

        \draw (a) -- (b) -- (c) -- (e) -- (d) -- (a);

        \node<2->[vertex, red] (a) at (0,0) {};
        \node<3->[vertex, blue] (b) at (2,0) {};
        \node<4->[vertex, red] (c) at (2,2) {};
        \node<5->[vertex, blue] (e) at (1,4) {};
        \node<6->[vertex, red] (d) at (0,2) {};

        \begin{pgfonlayer}{background}
            \path<3->[selected edge] (a.center) edge node {} (b.center);
            \path<4->[selected edge] (b.center) edge node {} (c.center);
            \path<5->[selected edge] (c.center) edge node {} (e.center);
            \path<6->[selected edge] (e.center) edge node {} (d.center);
            \path<7->[selected edge,lime] (d.center) edge node {} (a.center);
        \end{pgfonlayer}
    \end{tikzpicture}
    }
    \end{center}
\end{frame}

\subsection{Aufgabe 4}
\begin{frame}{Aufgabe 4}
Zeigen Sie: Ein Graph $G$ ist genau dann bipartit, wenn er nur Kreise
gerade Länge hat.
\end{frame}

\begin{frame}{Aufgabe 4: Lösung, Teil 1}
\underline{Vor.:} Sei $G = (E, K)$ ein zus. Graph. \pause

\underline{Beh.:} $G$ ist bipartit $\Rightarrow G$ hat keine Kreis ungerader Länge \pause

\underline{Bew.:} durch Widerspruch \pause

\underline{Annahme:} $G$ hat Kreis ungerader Länge \pause

$\xRightarrow[]{A.4}$ Ein Subgraph von $G$ ist nicht bipartit \pause

$\Rightarrow$ Widerspruch zu \enquote{$G$ ist bipartit} \pause

$\Rightarrow$ $G$ hat keinen Kreis ungerader Länge $\blacksquare$
\end{frame}

\begin{frame}{Aufgabe 4: Lösung, Teil 2}
\underline{Vor.:} Sei $G = (E, K)$ ein zus. Graph. \pause

\underline{Beh.:} $G$ hat keinen Kreis ungerader Länge $\Rightarrow G$ ist bipartit \pause

\underline{Bew.:} Konstruktiv \pause

Färbe Graphen mit Breitensuche $\blacksquare$
\end{frame}

\pgfdeclarelayer{background}
\pgfsetlayers{background,main}
\begin{frame}{Aufgabe 4 - Beispiel}
    \tikzstyle{selected edge} = [draw,line width=5pt,-,black!50]
    \begin{center}
    \adjustbox{max size={\textwidth}{0.8\textheight}}{
    \begin{tikzpicture}
        \node[vertex] (a) at (1,1) {};
        \node[vertex] (b) at (2,0) {};
        \node[vertex] (c) at (4,0) {};
        \node[vertex] (d) at (1,2) {};
        \node[vertex] (e) at (2,2) {};
        \node[vertex] (f) at (3,2) {};
        \node[vertex] (g) at (2,4) {};
        \node[vertex] (h) at (3,3) {};
        \node[vertex] (i) at (4,2) {};
        \node[vertex] (j) at (1,3) {};

        \draw (a) -- (b);
        \draw (a) -- (d);
        \draw (b) -- (e);
        \draw (b) -- (c);
        \draw (c) -- (f);
        \draw (d) -- (e);
        \draw (d) -- (j);
        \draw (e) -- (f);
        \draw (f) -- (i);
        \draw (g) -- (j);
        \draw (g) -- (h);

        \node<2->[vertex, red]  (a) at (1,1) {};
        \node<3->[vertex, blue] (b) at (2,0) {};
        \node<3->[vertex, blue] (d) at (1,2) {};
        \node<4->[vertex, red]  (c) at (4,0) {};
        \node<4->[vertex, red]  (e) at (2,2) {};
        \node<4->[vertex, red]  (j) at (1,3) {};
        \node<5->[vertex, blue] (f) at (3,2) {};
        \node<5->[vertex, blue] (g) at (2,4) {};
        \node<6->[vertex, red]  (h) at (3,3) {};
        \node<6->[vertex, red]  (i) at (4,2) {};

        \begin{pgfonlayer}{background}
            \path<3->[selected edge] (a.center) edge node {} (b.center);
            \path<3->[selected edge] (a.center) edge node {} (d.center);
            \path<4->[selected edge] (b.center) edge node {} (c.center);
            \path<4->[selected edge] (b.center) edge node {} (e.center);
            \path<4->[selected edge] (d.center) edge node {} (j.center);
            \path<4->[selected edge] (d.center) edge node {} (e.center);
            \path<5->[selected edge] (j.center) edge node {} (g.center);
            \path<5->[selected edge] (e.center) edge node {} (f.center);
            \path<5->[selected edge] (c.center) edge node {} (f.center);
            \path<6->[selected edge] (g.center) edge node {} (h.center);
            \path<6->[selected edge] (f.center) edge node {} (i.center);
        \end{pgfonlayer}
    \end{tikzpicture}
    }
    \end{center}
\end{frame}

\subsection{Aufgabe 9}
\begin{frame}{Aufgabe 9, Teil 1}
Im folgenden sind die ersten drei Graphen $G_1, G_2, G_3$ einer
Folge $(G_n)$ aus Graphen abgebildet. Wie sieht $G_4$ aus?

\begin{gallery}
    \galleryimage{graphs/triangular-1}
    \galleryimage{graphs/triangular-2}
    \galleryimage{graphs/triangular-3}
\end{gallery}
\end{frame}

\begin{frame}{Aufgabe 9, Teil 1 (Lösung)}
    \begin{center}
        \documentclass[varwidth=true, border=2pt]{standalone}
\usepackage{ifthen}
\usepackage{tikz}
\usetikzlibrary{calc} 

\begin{document}
\tikzstyle{vertex}=[draw,red,fill=red,circle,
minimum size=10pt,inner sep=0pt]
\tikzstyle{edge}=[red, very thick]
\begin{tikzpicture}
    \newcommand{\n}{4}
    \foreach \y in {0, ..., \n}{
        \pgfmathsetmacro{\loopend}{{2*\n-\y}}
        \pgfmathsetmacro{\second}{{\y+2}}
        \foreach \x in {\y, \second,..., \loopend}{
            \ifthenelse{\n=\y}{\breakforeach}{}
            \node (n-\x\y)[vertex] at (\x,\y) {};

            \ifthenelse{\y=0}{}{\draw[edge] (\x,\y) -- (\x+1,\y-1);}
            \pgfmathtruncatemacro\X{\x}
            \ifthenelse{\X<\loopend}{\draw[edge] (\x,\y) -- (\x+2,\y);}{}
            \ifthenelse{\X=\loopend}{}{\draw[edge] (\x,\y) -- (\x+1,\y+1);}
            
        }
    }
\end{tikzpicture}
\end{document}

    \end{center}
\end{frame}

\begin{frame}{Aufgabe 9, Teil 1 (Lösung)}
    \begin{center}
        \documentclass[varwidth=true, border=2pt]{standalone}
\usepackage{ifthen}
\usepackage{tikz}
\usetikzlibrary{calc} 

\begin{document}
\tikzstyle{vertex}=[draw,red,fill=red,circle,
minimum size=10pt,inner sep=0pt]
\tikzstyle{edge}=[red, very thick]
\begin{tikzpicture}
    \newcommand{\n}{5}
    \foreach \y in {0, ..., \n}{
        \pgfmathsetmacro{\loopend}{{2*\n-\y}}
        \pgfmathsetmacro{\second}{{\y+2}}
        \foreach \x in {\y, \second,..., \loopend}{
            \ifthenelse{\n=\y}{\breakforeach}{}
            \node (n-\x\y)[vertex] at (\x,\y) {};

            \ifthenelse{\y=0}{}{\draw[edge] (\x,\y) -- (\x+1,\y-1);}
            \pgfmathtruncatemacro\X{\x}
            \ifthenelse{\X<\loopend}{\draw[edge] (\x,\y) -- (\x+2,\y);}{}
            \ifthenelse{\X=\loopend}{}{\draw[edge] (\x,\y) -- (\x+1,\y+1);}
            
        }
    }
\end{tikzpicture}
\end{document}

    \end{center}
\end{frame}

\begin{frame}{Aufgabe 9, Teil 1 (Lösung)}
    \begin{center}
        \documentclass[varwidth=true, border=2pt]{standalone}
\usepackage{ifthen}
\usepackage{tikz}
\usetikzlibrary{calc} 

\begin{document}
\tikzstyle{vertex}=[draw,red,fill=red,circle,
minimum size=10pt,inner sep=0pt]
\tikzstyle{edge}=[red, very thick]
\begin{tikzpicture}
    \newcommand{\n}{6}
    \foreach \y in {0, ..., \n}{
        \pgfmathsetmacro{\loopend}{{2*\n-\y}}
        \pgfmathsetmacro{\second}{{\y+2}}
        \foreach \x in {\y, \second,..., \loopend}{
            \ifthenelse{\n=\y}{\breakforeach}{}
            \node (n-\x\y)[vertex] at (\x,\y) {};

            \ifthenelse{\y=0}{}{\draw[edge] (\x,\y) -- (\x+1,\y-1);}
            \pgfmathtruncatemacro\X{\x}
            \ifthenelse{\X<\loopend}{\draw[edge] (\x,\y) -- (\x+2,\y);}{}
            \ifthenelse{\X=\loopend}{}{\draw[edge] (\x,\y) -- (\x+1,\y+1);}
            
        }
    }
\end{tikzpicture}
\end{document}

    \end{center}
\end{frame}

\begin{frame}{Aufgabe 9, Teil 2}
Wie viele Ecken und wie viele Kanten hat $G_i$?

\begin{gallery}
    \galleryimage{graphs/triangular-1}
    \galleryimage{graphs/triangular-2}
    \galleryimage{graphs/triangular-3}
\end{gallery}
\end{frame}

\begin{frame}{Aufgabe 9, Teil 2: Antwort}
Ecken:

\[|E_n| = |E_{n-1}| + (n+1) = \sum_{i=1}^{n+1} = \frac{n^2 + 2n+2}{2}\]

Kanten:

\begin{align}
|K_n| &= |K_{n-1}| + \underbrace{((n+1)-1)+2}_{\text{außen}} + (n-1) \cdot 2\\
    &= |K_{n-1}| + n+2+2n-2\\
    &= |K_{n-1}| + 3n\\
    &= \sum_{i=1}^{n} 3i = 3 \sum_{i=1}^{n} i \\
    &= 3 \frac{n^2 + n}{2}
\end{align}
\end{frame}

\begin{frame}{Aufgabe 9, Teil 3}
Gebe $G_i$ formal an.

\begin{gallery}
    \galleryimage{graphs/triangular-1}
    \galleryimage{graphs/triangular-2}
    \galleryimage{graphs/triangular-3}
\end{gallery}
\end{frame}

\begin{frame}{Aufgabe 9, Teil 3 (Lösung)}
Gebe $G_n$ formal an.

\begin{gallery}
    \galleryimage{graphs/triangular-1}
    \galleryimage{graphs/triangular-2}
    \galleryimage{graphs/triangular-3}
\end{gallery}

\begin{align*}
    E_n &= \Set{e_{x,y} | y \in 1, \dots, n;\; x \in y, \dots, 2 \cdot n - y \text{ mit } x-y \equiv 0 \mod 2}\\
    K_n &= \Set{\Set{e_{x,y}, e_{i,j}} \in E_n^2 | (x+2=i \land y=j) \lor (x+1=i \land y\pm1=j)}\\
    G_n &= (E_n, K_n)
\end{align*}

\end{frame}

\subsection{Bildquelle}
\begin{frame}{Bildquelle}
\begin{itemize}
    \item \href{http://commons.wikimedia.org/wiki/File:Konigsberg\_bridges.png}{http://commons.wikimedia.org/wiki/File:Konigsberg\_bridges.png}
    \item \href{http://commons.wikimedia.org/wiki/File:Unit\_disk\_graph.svg}{http://commons.wikimedia.org/wiki/File:Unit\_disk\_graph.svg}
    \item \href{http://goo.gl/maps/WnXRh}{Google Maps} (Grafiken \TCop 2013 Cnes/Spot Image, DigitalGlobe)
\end{itemize}
\end{frame}

\subsection{Literatur}
\begin{frame}{Literatur}
\begin{itemize}
    \item A. Beutelspacher: \textit{Diskrete Mathematik für Einsteiger}, 4. Auflage, ISBN 978-3-8348-1248-3
\end{itemize}
\end{frame}
