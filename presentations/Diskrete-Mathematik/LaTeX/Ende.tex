\subsection{Weitere Aufgaben}
\begin{frame}{Aufgabe 3}
Zeigen Sie: Ein Kreis ist genau dann bipartit, wenn er gerade Länge hat.
\end{frame}

% TODO

\subsection{Bildquelle}
\begin{frame}{Bildquelle}
\begin{itemize}
    \item \href{http://commons.wikimedia.org/wiki/File:Konigsberg\_bridges.png}{http://commons.wikimedia.org/wiki/File:Konigsberg\_bridges.png}
    \item \href{http://goo.gl/maps/WnXRh}{Google Maps} (Grafiken \TCop 2013 Cnes/Spot Image, DigitalGlobe)
\end{itemize}
\end{frame}

\subsection{Literatur}
\begin{frame}{Literatur}
\begin{itemize}
    \item A. Beutelspacher: \textit{Diskrete Mathematik für Einsteiger}, 4. Auflage, ISBN 978-3-8348-1248-3
\end{itemize}
\end{frame}
