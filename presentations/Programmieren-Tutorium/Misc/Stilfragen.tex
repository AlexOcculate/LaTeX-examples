\section{Stilfragen}
\subsection{So nicht}
\begin{frame}{So nicht}
    \inputminted[linenos, numbersep=5pt, tabsize=4, frame=lines, label=pi.c, fontsize=\tiny]{c}{pi.c}
    Quelle: \href{http://www.cise.ufl.edu/~manuel/obfuscate/obfuscate.html}{Obfuscated C Code}
\end{frame}

\subsection{Aber so}
\begin{frame}{Aber so}
    \inputminted[linenos, numbersep=5pt, tabsize=4, frame=lines, label=pi-good.c]{c}{pi-good.c}

    Nur halt in Java (Ich will keine C Abgaben sehen! In dem
    Modul "`Programmieren"' wird ausschließlich Java behandelt.)
\end{frame}

\subsection{Hinweise zu gutem Programmierstil}
\begin{frame}{Hinweise zu gutem Programmierstil}
    \begin{itemize}
        \item Sinnvolle Modellierung ("`ist ein"' bzw. "`hat"')
        \item Aussagekräftige Namen (Klassen, Methoden, Variablen)
        \item JavaDoc
        \item Hilfreiche Kommentare (für die Abschlussaufgabe: lieber zu viele)
        \item Kurze Funktionseinheiten
    \end{itemize}
\end{frame}
