\subsection{Aufbau}
\begin{frame}
\frametitle{Der Klassiker}
\begin{block}{Dreiteiler}
\begin{itemize}
	\item{Einleitung einer Präsentation \quad -- \textcolor{green}{ca. 15\%}}
	\item{Hauptteil einer Präsentation \quad -- \textcolor{green}{ca. 75\%}}
	\item{Schluss einer Präsentation \qquad -- \textcolor{green}{ca. 10\%}}
\end{itemize}
Die einzelnen Hauptteile jeweils am Ende zusammenfassen und zum nächsten Teil überleiten.
\end{block}
\vspace{1cm}
\begin{alertblock}{Wer hat an der Uhr gedreht?}
Unwesentliche Unterpunkte im Hauptteil kürzen. Dazu ist Planung im Voraus nötig!
\end{alertblock}
\end{frame}


\begin{frame}
\frametitle{Der Neue}
\begin{block}{5-Satz}
\begin{itemize}
	\item{Einleitung in eine Präsentation \quad -- \textcolor{green}{ca. 15\%}}
	\item{Hauptteil einer Präsentation \qquad -- \textcolor{green}{ca. 75\%}}
	\begin{itemize}
		\item<2->{Ist -- Ziel -- Weg}
		\item<2->{Anlass -- Ziel -- Appell}
		\item<2->{These -- Antithese -- Synthese}
		\item<2->{Position A -- Position B -- Meinung}
		\item<2->{\dots}
	\end{itemize}
	\item{Schluss einer Präsentation \qquad -- \textcolor{green}{ca. 10\%}}
\end{itemize}
\end{block}
\end{frame}
