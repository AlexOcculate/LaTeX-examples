\subsection{Aufgabe 3}
\begin{frame}{Aufgabe 3}
Zeigen Sie: Wenn der Graph eines Kreises bipartit ist, dann hat der Kreis gerade Länge.
\end{frame}

\subsection{Bildquellen}
\begin{frame}{Bildquellen}
\begin{itemize}
	\item \href{http://commons.wikimedia.org/wiki/File:Hypercube.svg}{http://commons.wikimedia.org/wiki/File:Hypercube.svg}
    \item \href{http://commons.wikimedia.org/wiki/File:Konigsberg\_bridges.png}{http://commons.wikimedia.org/wiki/File:Konigsberg\_bridges.png}
    \item \href{http://commons.wikimedia.org/wiki/File:Unit\_disk\_graph.svg}{http://commons.wikimedia.org/wiki/File:Unit\_disk\_graph.svg}
    \item \href{http://goo.gl/maps/WnXRh}{Google Maps} (Grafiken \TCop 2013 Cnes/Spot Image, DigitalGlobe)
	\item \href{http://cf.drafthouse.com/\_uploads/galleries/29140/good_will\_hunting\_3.jpg}{cf.drafthouse.com/\_uploads/galleries/29140/good\_will\_hunting\_3.jpg}
\end{itemize}
\end{frame}

\subsection{Literatur}
\begin{frame}{Literatur}
\begin{itemize}
    \item A. Beutelspacher: \textit{Diskrete Mathematik für Einsteiger}, 4. Auflage, ISBN 978-3-8348-1248-3
\end{itemize}
\end{frame}

\subsection{Folien, \LaTeX und Material}
\begin{frame}{Folien, \LaTeX und Material}
Der Foliensatz und die \LaTeX und Ti\textit{k}Z-Quellen sind unter

\href{https://github.com/MartinThoma/LaTeX-examples/tree/master/presentations/Diskrete-Mathematik}{github.com/MartinThoma/LaTeX-examples/tree/master/presentations/Diskrete-Mathematik}
\\

Kurz-URL:
\href{http://goo.gl/uTgam}{goo.gl/uTgam}
\end{frame}
