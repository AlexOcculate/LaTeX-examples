\subsection{Artikulationspunkt}
\begin{frame}{Artikulationspunkt}{Articulation vertex or cut vertices}
	\begin{block}{Definition: Artikulationspunkt (auch "Gelenkpunkt" genannt)}
		Ein Knoten $v \in V$ eines Graphen $G(V,E)$ heißt Artikulationspunkt
		$: \Leftrightarrow$ Durch das Entfernen von v zerfällt G in mehr zusammenhängende Teilgraphen,
		als G bereits hat.
	\end{block}
	\begin{figure}
		\begin{tikzpicture}[scale=1.8, auto,swap]
			% Draw a 7,11 network
			% First we draw the vertices
			\foreach \pos/\name in {{(0,0)/a}, {(1,0)/b}, {(2,0)/c},
									{(3,0)/d}, {(2.5,0.7)/e}}
				\node[vertex] (\name) at \pos {$\name$};
			% Connect vertices with edges and draw weights
			\foreach \source/ \dest /\pos in {a/b/,b/c/,c/d/,c/e/,d/e/}
				\path (\source) edge [\pos] node {} (\dest);
		\end{tikzpicture}
	\end{figure}
\end{frame}


\begin{frame}
	\frametitle{Algorithmus}
	Fast wie für Brücken. Unterschiede: 
	\begin{itemize}
		\item $L(v)$ hier anders definiert: bei Rückwärtskanten erst nach mindestens einer Baumkante
		\item Hier vergleichen wir für alle $v \in V$ $L(v)$ und $N(v)$: \\
	$v$ ist Art.-punkt $\Leftrightarrow L(v) = N(v)$
	\end{itemize}
	\begin{figure}
		\begin{tikzpicture}[scale=1.8, auto,swap]
			% Draw a 7,11 network
			% First we draw the vertices
			\foreach \pos/\name in {{(0,0)/a}, {(1,0)/b}, {(2,0)/c},
									{(3,0)/d}, {(2.5,0.7)/e}}
				\node[vertex] (\name) at \pos {$\name$};
			% Connect vertices with edges and draw weights
			\foreach \source/ \dest /\pos in {a/b/,b/c/,c/d/,c/e/,d/e/}
				\path (\source) edge [\pos] node {} (\dest);
		\end{tikzpicture}
	\end{figure}
%	\begin{figure}
%		\begin{tikzpicture}[scale=1.8, auto, swap]
%			\foreach \pos/\name in {{(0.5,1.5)/e}, {(1,1)/c}, {(1.5,0.5)/d}, {(0.5,0.5)/b}, {(0,0)/a}}
%				\node[vertex] (\name) at \pos {$\name$};
%			% Connect vertices with edges and draw weights
%			\foreach \source/ \dest /\pos in {a/b/,b/c/,d/c/,e/c/,d/e/}
%				\path (\source) edge [\pos] node {} (\dest);
%		\end{tikzpicture}
%	\end{figure}
%	<Beispiel eines Algorithmus-Durchlaufs>
\end{frame}
