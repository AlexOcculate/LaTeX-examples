\subsection{What will I do next?}
\begin{frame}{What will I do next?}
    \begin{itemize}
        \item Get classification performance with cross-validation
        \item Implement neural net for classification
        \begin{itemize}
            \item preprocessing: compute cubic spline for each line
            \begin{itemize}
                \item equi-spaced points or
                \item get equi-timed points
            \end{itemize}
            \item 5 - 20 input neurons for each line
            \item 1076 output neurons (one for each symbol)
        \end{itemize}
        \item Get a language model (e.g. by parsing Wikipedia)
        \item Use ANN with HMM (?)
    \end{itemize}
\end{frame}

\subsection{Far future}
\begin{frame}{What could be done?}
    \begin{itemize}
        \item Make use of audio data in a multimodal approach\\
              e.g. $R$ and $\mathcal{R}$
        \item Currently, the Lecture Translation system doesn't recognize math.\\
              You get \enquote{integral of e raised to the power of x d x} instead
              of $\int e^x \mathrm{d} x$.
        \item Spoken math is ambigous: $\sqrt{a+b}$ vs. $\sqrt{a} + b$
        \item The language model I create could help to find probable formulas
        \item The platform could be used to get more input data of users
    \end{itemize}
\end{frame}