%%%%%%%%%%%%%%%%%%%%%%%%%%%%%%%%%%%%%%%%%
% Two Column Curriculum Vitae XeLaTeX Template
%
% This template has been downloaded from:
% http://www.latextemplates.com/template/two-column-one-page-cv
%
% Original author:
% Alessandro (The CV Inn)
%
% IMPORTANT: THIS TEMPLATE NEEDS TO BE COMPILED WITH XeLaTeX
%
% This template uses several fonts not included with Windows/Linux by
% default. If you get compilation errors saying a font is missing, find the line
% on which the font is used and either change it to a font included with your
% operating system or comment the line out to use the default font.
% 
%%%%%%%%%%%%%%%%%%%%%%%%%%%%%%%%%%%%%%%%%

%----------------------------------------------------------------------------------------
%	PACKAGES AND OTHER DOCUMENT CONFIGURATIONS
%----------------------------------------------------------------------------------------

\documentclass[a4paper,10pt]{article} % Font size (10pt, 11pt or 12pt)

\usepackage[ngerman]{babel} % this is needed for umlauts

\usepackage[hmargin=1.25cm, vmargin=1.0cm]{geometry} % Document margins

\usepackage{marvosym} % Required for symbols in the colored box
\usepackage{ifsym} % Required for symbols in the colored box
\usepackage{pdfpages}  % Signatureinbingung und includepdf

\usepackage{xcolor} % Allows the definition of hex colors

% Fonts and tweaks for XeLaTeX
\usepackage{fontspec,xltxtra,xunicode}
\defaultfontfeatures{Mapping=tex-text}
\setromanfont[Mapping=tex-text]{Times New Roman} % Main document font
\setsansfont[Scale=MatchLowercase,Mapping=tex-text]{Arial} % Font for your name at the top
%\setmonofont[Scale=MatchLowercase]{Andale Mono}

% Colors for links, text and headings
\usepackage{hyperref}
\definecolor{linkcolor}{HTML}{506266} % Blue-gray color for links
\definecolor{shade}{HTML}{F5DD9D} % Peach color for the contact information box
\definecolor{text1}{HTML}{2b2b2b} % Main document font color, off-black
\definecolor{headings}{HTML}{701112} % Dark red color for headings
% Other color palettes: shade=B9D7D9 and linkcolor=A40000; shade=D4D7FE and linkcolor=FF0080

\hypersetup{colorlinks,breaklinks, urlcolor=linkcolor, linkcolor=linkcolor} % Set up links and colors

\usepackage{fancyhdr}
\pagestyle{fancy}
\fancyhf{}
% Headers and footers can be added with the \lhead{} \rhead{} \lfoot{} \rfoot{} commands
% Example footer:
%\rfoot{\color{headings} {\sffamily Last update: \today}. Typeset with Xe\LaTeX}

\renewcommand{\headrulewidth}{0pt} % Get rid of the default rule in the header

\usepackage{titlesec} % Allows creating custom \section's

% Format of the section titles
\titleformat{\section}{\color{headings}
\scshape\Large\raggedright}{}{0em}{}[\color{black}\titlerule]

\titlespacing{\section}{0pt}{0pt}{5pt} % Spacing around titles

\newcommand{\ts}{\textsuperscript}

\hypersetup{ 
  pdfauthor   = {Martin Thoma}, 
  pdfkeywords = {Martin Thoma,KIT,CV}, 
  pdftitle    = {Curriculum Vitae of Martin Thoma} 
} 

\begin{document}

\color{text1} % Sets the default text color for the whole document

%----------------------------------------------------------------------------------------
%	TITLE
%----------------------------------------------------------------------------------------

\par{\centering{\sffamily\Huge Martin Thoma}\\ % Your name
{\Huge \color{headings}\fontspec{Zapfino Linotype One} Curriculum {Vit\fontspec{Zapfino Linotype One}\ae}\\[15pt]\par}
	
%----------------------------------------------------------------------------------------

% Start the left-hand side of the page
\begin{minipage}[t]{0.5\textwidth}
\vspace{0pt} % Trick for alignment
	
%----------------------------------------------------------------------------------------
%	WORK EXPERIENCE
%----------------------------------------------------------------------------------------

\section{Work Experience} 

%----------------------------------------------------------------------------------------
% WORK EXPERIENCE -0-

{\raggedleft\textsc{2013}\par}

{\raggedright\large Programmer\\
\textit{improving KIT lecture translator}\\[5pt]}

\normalsize{I have to implement and integrate an unsupervised acoustic model training framework into KIT lecture translator system for automatic model adaption.}\\

%----------------------------------------------------------------------------------------
% WORK EXPERIENCE -0-

{\raggedleft\textsc{2013}\par}

{\raggedright\large Scientific lector\\
\textit{\LaTeX{}, German and computer science}\\[5pt]}

\normalsize{I've corrected a script for computer engineering.}\\

%----------------------------------------------------------------------------------------
% WORK EXPERIENCE -0-

{\raggedleft\textsc{2012}\par}

{\raggedright\large Tutor for programming\\
\textit{teaching students programming at university}\\[5pt]}

\normalsize{I taught people about 30 students how to program in Java.
Coding conventions and basic OOP was part of the course. All of my German presentations are online.}\hfill \href{http://martin-thoma.com/programmieren-tutorium/#Folien}{$\rightarrow$ presentations}\\

%----------------------------------------------------------------------------------------
% WORK EXPERIENCE -1-

{\raggedleft\textsc{2011}\par}

{\raggedright\large Freelancer at KTC\\
\textit{programming for a consulting company}\\[5pt]}

\normalsize{At KTC, I gained first experiences with buisness-logic
and a big, but algorithmically not challenging project. To be honest,
I only fixed some Java bugs.}\\

%----------------------------------------------------------------------------------------
% WORK EXPERIENCE -2-

{\raggedleft\textsc{2011}\par}

{\raggedright\large Student research assistant at \textsc{ Institute of Toxicology and Genetics}, KIT\\
\textit{participating in a university research project}\\[5pt]}

\normalsize{In summer 2011 I worked for over a month for a 
research project at KIT. I have written bash scripts for file
conversions, fixed some bugs and re-written a slow Mathematica script
in a much faster Python version. But it quickly turned out that
this project had a lot of C++ source which was rarely commented or
documented. I realized, that I wouldn't have time for this project
after beginning my studies at university.}\\

%----------------------------------------------------------------------------------------
% WORK EXPERIENCE -3-

{\raggedleft\textsc{since 2011}\par}

{\raggedright\large Freelance Work\\
\textit{building an online service}\\[5pt]}

\normalsize{I have started to work as a freelancer at the beginning 
of 2011. I have developed an online-service which helped
schools to coordinate their dates. I have sold this online service to
two schools in bavaria and three other schools were interested. 
Unfortunately, the ministry of education of Bavaria 
released an application with similar functionality in
2012. This was the reason why I decided to shut down my service.}\\

%----------------------------------------------------------------------------------------
% WORK EXPERIENCE -4-

%{\raggedleft\textsc{2010}\par}

%{\raggedright\large Compulsory community service\\
%\textit{District Office Augsburg}\\[5pt]}

%\normalsize{I have worked in the district office of Augsburg in my
%as compulsory community service. I had the task to controll nature 
%conservation conditions. To do so, I had to use a geographic 
%information system (which could definitely be improved).}\\

%----------------------------------------------------------------------------------------	

%----------------------------------------------------------------------------------------	

\end{minipage} % End left-hand side of the page
\hfill
% Start the right-hand side of the page
\begin{minipage}[t]{0.44\textwidth} 
\vspace{0pt} %trick for alignment

%----------------------------------------------------------------------------------------
%	COLORED BOX
%----------------------------------------------------------------------------------------

\colorbox{shade}{\textcolor{text1}{
\begin{tabular}{c|p{7cm}}
\raisebox{-4pt}{\textifsymbol{18}} & Parkstraße 17, 76131 Karlsruhe \\ % Address
\raisebox{-3pt}{\Mobilefone} & +49 $($1636$)$ 28 04 91 \\ % Phone number
\raisebox{-1pt}{\Letter} & \href{mailto:info@martin-thoma.de}{info@martin-thoma.de} \\ % Email address
\Keyboard & \href{http://martin-thoma.com}{martin-thoma.com} \\ % Website
\end{tabular}
}
}\\[10pt]

%----------------------------------------------------------------------------------------
%	EDUCATION
%----------------------------------------------------------------------------------------

\section{Education} 

\begin{tabular}{rl} % Start a table with two columns, one for dates and one for qualifications


%----------------------------------------------------------------------------------------
% EDUCATION -1-

from 2011 & \textbf{Bachelor of Science} \\ 
& \textsc{Computer Science} \\ 
& \textit{Karlsruhe Institute of Technology}\\
&\\
	 
%----------------------------------------------------------------------------------------
% EDUCATION -2-

2004 -- 2010 & \textbf{Abitur}\\ 
& \textsc{Intensive course physics and mathematics} \\ 
& \textit{Paul-Klee-Gymnasium Gersthofen}\\
&\\

%----------------------------------------------------------------------------------------

\end{tabular}\\[10pt]

%----------------------------------------------------------------------------------------
%	AWARDS
%----------------------------------------------------------------------------------------

\section{Awards} 

\begin{tabular}{rl}
2010	 & \textbf{Winner}\\
& \textit{Federal Competition for Computer Science}\\ \\

%----------------------------------------------------------------------------------------

2009	 & \textbf{2nd price - regional competition}\\
& \textit{Youth Research Competition}\\[10pt]

%----------------------------------------------------------------------------------------

2008	 & \textbf{1st price}\\
& \textit{data analysis competition at University of Augsburg}\\[10pt]

%----------------------------------------------------------------------------------------

2008	 & \textbf{Award for social commitment}\\
& \textit{Paul-Klee-Gymnasium}
\\[10pt]

%----------------------------------------------------------------------------------------

2007	 & \textbf{Price for science and research}\\
& \textit{FOCUS pupils competition}
\end{tabular}\\[10pt]

%----------------------------------------------------------------------------------------
%	COMPUTER SKILLS
%----------------------------------------------------------------------------------------

\section{Computer Skills} 

\begin{tabular}{rl}
Basic Knowledge         & \textsc{JavaScript}\\
                        & \textsc{Linux}, \textsc{SQL}, \textsc{PHP}\\ \\
Intermediate Knowledge  & \LaTeX, \textsc{Java}, \textsc{HTML}\\ \\
Good Knowledge          & \textsc{Python}\\ \\
\end{tabular}

%----------------------------------------------------------------------------------------
%	COMMUNICATION SKILLS
%----------------------------------------------------------------------------------------

\section{Language Skills} 

\begin{tabular}{rl}
\textsc{German}
& mother tongue\\
& \\
\textsc{English}
& Cambridge Certificate – C1\\
& \\
\textsc{French}
& DELF A2 \\
\end{tabular}\\[10pt]

%----------------------------------------------------------------------------------------
	
\end{minipage} % End right-hand side of the page
%-----------------------------------------------------------------------------------------------------------------------------------------------------

% Start the left-hand side of the page
\begin{minipage}[t]{0.5\textwidth}
\vspace{0pt} % Trick for alignment
	
%----------------------------------------------------------------------------------------
%	WORK EXPERIENCE
%----------------------------------------------------------------------------------------

\section{Work Experience} 

%----------------------------------------------------------------------------------------

{\raggedleft\textsc{since 2006}\par}

{\raggedright\large HackIts, Puzzles and Challenges\\
\textit{ProjectEuler, bright-shadows.net and many more}\\[5pt]}

\normalsize{I really love solving logical, algorithmical or math 
puzzles and participated in competitions. I started to solve puzzles 
in 2006 and I still like them. This was the reason why I participated 
in a practical curse at KIT for preparation for ICPC. It was fun, 
but I found out that many people are much faster in producing C++ 
code that passed the tests than I am.
However, as I've been very successfull at the Federal Competition for 
Computer Science (``Bundeswettbewerb Informatik'') it seems as if I'm 
better in problem solving if I get more time to think about it.}\\

%----------------------------------------------------------------------------------------	

\section{Future plans and motivation}

I watched a video about Google Driverless Car in June 2013 and was 
totally amazed. I've started two online courses on Udacity to learn
more about this topic. Currently, I try to find a good way to
compute the minimum distance of a point to a cubic spline which
is necessary to apply the PID algorithm for steering control.\\

I'm currently employed as a research assistant at Institute for Anthropomatics (KIT).
My task is to implement some changes that were proposed in a thesis 
to the KIT lecture translation system. The work helps me to understand
how speech recognition works.

I will definitely also make a masters degree in computer science. 

A topic about which I would like to learn more
are self driving cars. But I think there are no courses in Karlsruhe
about this.\\

I also would like to spend about half a year abroad, preferably
in an English speaking country which is not in Europe.



\end{minipage} % End left-hand side of the page
\hfill
% Start the right-hand side of the page
\begin{minipage}[t]{0.44\textwidth} 
\vspace{0pt} %trick for alignment

%----------------------------------------------------------------------------------------
%	AWARDS
%----------------------------------------------------------------------------------------

\section{Projects} 

\begin{tabular}{rl}
11/2013	 & \textbf{Minimum distance paper}\\
& \textit{research to compute the minimum distance}\\
& \textit{ to a cubic spline}\hfill \href{https://github.com/MartinThoma/LaTeX-examples/tree/master/documents/math-minimal-distance-to-cubic-function}{$\rightarrow$ read more}\\

%----------------------------------------------------------------------------------------
06/2013	 & \textbf{Interpolation}\\
& \textit{creating an interactive HTML5/JS-example}\\
& \textit{for interpolation} \hfill \href{http://martin-thoma.com/polynomial-interpolation/}{$\rightarrow$ read more} \\\\

%----------------------------------------------------------------------------------------
02/2013	 & \textbf{Line segment intersection}\\
& \textit{creating a simple check for line segment}\\
& \textit{intersection} \hfill \href{http://martin-thoma.com/how-to-check-if-two-line-segments-intersect/}{$\rightarrow$ read more} \\\\

%----------------------------------------------------------------------------------------
06/2012	 & \textbf{Matrix multiplication}\\
& \textit{examining algorithms and libraries for}\\
& \textit{matrix multiplication} \hfill \href{http://martin-thoma.com/matrix-multiplication-python-java-cpp/}{$\rightarrow$ read more}\\\\

%----------------------------------------------------------------------------------------
09/2011	 & \textbf{Blogging on martin-thoma.com}\\
& \textit{about Algorithms, the Web, University, \dots}\\ \\

%----------------------------------------------------------------------------------------

06/2011	 & \textbf{Community Chess}\\
& \textit{This is a platform for programmers. They}\\
& \textit{can use the API to create A.I.s that play}\\
& \textit{chess agains each other. } \hfill \href{https://github.com/MartinThoma/community-chess}{$\rightarrow$ read more}\\\\
\end{tabular}\\[10pt]

%----------------------------------------------------------------------------------------
%	COMPUTER SKILLS
%----------------------------------------------------------------------------------------

\section{Online Courses} 

\begin{tabular}{rll}
09/2013	 & \textbf{Artificial Intelligence} & Udacity\\
    	 & \textbf{for Robotics}            &\\
    	 & \textit{finished 10/2013}        &\\\\
06/2013	 & \textbf{Introduction to }        & Udacity\\
    	 & \textbf{Artificial Intelligence} &\\
    	 & \textit{finished 08/2013}        &\\\\
05/2012	 & \textbf{Algorithms I}            & Stanford\\
    	 & \textit{finished 07/2012}        &\\\\
06/2010	 & \textbf{Introduction to Computer}& MIT\\
    	 & \textbf{Science and Programming} &\\
    	 & \textit{finished 09/2010}        &\\\\
\end{tabular}\\[10pt]

%----------------------------------------------------------------------------------------
	
\end{minipage} % End right-hand side of the page

%\includepdf[pages=1-2]{zeugnis}

\end{document}  
