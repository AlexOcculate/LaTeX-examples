In diesem Kapitel seien $X,Y,Z$ Mengen ($\ne\emptyset$) und 
$f: X\to Y,\; g:Y\to Z$ Abbildungen.

\begin{enumerate}
    \index{Potenzmenge}
    \index{Disjunktheit}
    \item 
    \begin{enumerate}
        \item $\mathcal{P}(X):=\{A:A\subseteq X\}$ heißt 
              \textbf{Potenzmenge} von $X$.
        \item Sei $\fm\subseteq\mathcal{P}(X)$, so heißt $\fm$ 
              \textbf{disjunkt}, genau dann wenn $A\cap B=\emptyset$ 
              für $A,B\in\fm$ mit $A\ne B$.
        \item Sei $(A_j)$ eine Folge in $\mathcal{P}(X)$ (also 
              $A_j\subseteq X$), so heißt $(A_j)$ \textbf{disjunkt},
              genau dann wenn $\{A_1,A_2,\dots\}$ disjunkt ist.\\
              \textbf{Schreibweise}:\\
              \begin{align*}
                \dot{\bigcup}_{j=1}^\infty &:=\bigcup_{j=1}^\infty A_j\\
                \bigcup_{j=1}^\infty A_j   &:=\bigcup A_j\\
                \bigcap_{j=1}^\infty A_j   &:=\bigcap A_j\\
                \sum_{j=1}^\infty a_j      &=: \sum a_j
              \end{align*}
    \end{enumerate}
    \item Sei $A\subseteq X$, $\mathds{1}_A : X \rightarrow R$ 
          definiert durch:
          \[\mathds{1}_A(x):= \begin{cases}
                1 &\text{falls } x\in A\\
                0 &\text{falls } x\in A^c
            \end{cases}\]
          wobei $A^c:=X\setminus A$. $\mathds{1}_A$ heißt die
          \textbf{charakteristische Funktion} oder 
          \textbf{Indikatorfunktion von A}.
    \item Sei $B\subseteq Y$ dann ist $f^{-1}(B):=\{x\in X: f(x)\in B\}$ 
          und es gelten folgende Eigenschaften:
          \begin{enumerate}
            \item $f^{-1}(B^c)=f^{-1}(B)^c$
            \item Ist $B_j$ eine Folge in $\mathcal{P}(Y)$, so gilt:
                  \begin{align*}
                  f^{-1}(\bigcup B_j)=\bigcup f^{-1}(B_j)\\
                  f^{-1}(\bigcap B_j)=\bigcap f^{-1}(B_j)\\
                  \end{align*}
            \item Ist $C\subseteq Z$, so gilt:
                  \[(g\circ f)^{-1}(C)=f^{-1}(g^{-1}(C))\]
          \end{enumerate}
\end{enumerate}

\begin{definition}
    \index{offen}
    Sei $n \in \mdn$ und $\emptyset \neq X \subseteq \mdr^n$ und 
    $A \subseteq X$.

    $A$ heißt $\stackrel{\text{offen}}{\text{abgeschlossen}}$ in 
    $X :\Leftrightarrow \exists B \subseteq \mdr^n$. 
    $B$ ist $\stackrel{\text{offen}}{\text{abgeschlossen}}$ und 
    $A = B \cap X$
\end{definition}

\begin{satz}
    Sei $\emptyset \neq X \subseteq \mdr^n,\; A \subseteq X$ und
    $f: X \rightarrow \mdr^n$.

    \begin{enumerate}
        \item $A$ ist offen in $X \Leftrightarrow \forall x \in A$ 
              ex. eine Umgebung $U$ von $x$ mit $U \cap X \subseteq A$
        \item $A$ ist abgeschlossen in $X$\\
              $\Leftrightarrow X \setminus A$ ist offen in $X$\\
              $\Leftrightarrow$ für jede konvergente Folge $(a_k)$ 
              in $A$ mit $\lim a_k \in X$ ist $\lim a_k \in A$
        \item Die folgenden Aussagen sind äquivalent:
              \begin{enumerate}
                \item $f \in C(X, \mdr^m)$
                \item für jede offene Menge $B \subseteq \mdr^m$ ist 
                      $f^{-1}(B)$ offen in $X$
                \item für jede abgeschlossene Menge $B \subseteq \mdr^m$ ist 
                      $f^{-1}(B)$ abgeschlossen in $X$
              \end{enumerate}
    \end{enumerate}
\end{satz}
