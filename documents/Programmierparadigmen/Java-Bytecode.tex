%!TEX root = Programmierparadigmen.tex
\chapter{Java Bytecode}
\index{Java Bytecode|(}
\begin{definition}[Bytecode]\xindex{Bytecode}%
    Der Bytecode ist eine Sammlung von Befehlen für eine virtuelle Maschine.
\end{definition}

Bytecode ist unabhängig von realer Hardware.

\begin{definition}[Heap]\xindex{Heap}\xindex{Speicher!dynamischer}%
    Der dynamische Speicher, auch Heap genannt, ist ein Speicherbereich, aus dem
    zur Laufzeit eines Programms zusammenhängende Speicherabschnitte angefordert
    und in beliebiger Reihenfolge wieder freigegeben werden können.
\end{definition}

\index{Java Bytecode|)}