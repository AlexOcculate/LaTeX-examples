%!TEX root = Programmierparadigmen.tex
\markboth{Symbolverzeichnis}{Symbolverzeichnis}
\twocolumn
\chapter*{Symbolverzeichnis}
\addcontentsline{toc}{chapter}{Symbolverzeichnis}
%%%%%%%%%%%%%%%%%%%%%%%%%%%%%%%%%%%%%%%%%%%%%%%%%%%%%%%%%%%%%%%%%%%%%
% Reguläre Ausdrücke                                                 %
%%%%%%%%%%%%%%%%%%%%%%%%%%%%%%%%%%%%%%%%%%%%%%%%%%%%%%%%%%%%%%%%%%%%%
\section*{Reguläre Ausdrücke}
$\emptyset\;\;\;$ Leere Menge\\
$\epsilon\;\;\;$ Das leere Wort\\
$\alpha, \beta\;\;\;$ Reguläre Ausdrücke\\
$L(\alpha)\;\;\;$ Die durch $\alpha$ beschriebene Sprache\\
$L(\alpha | \beta) = L(\alpha) \cup L(\beta)$\\
$L(\alpha \cdot \beta) = L(\alpha) \cdot L(\beta)$\\
$\alpha^+ = L(\alpha)^+$ TODO: Was ist $L(\alpha)^+$\\
$\alpha^* = L(\alpha)^*$ TODO: Was ist $L(\alpha)^*$\\
\onecolumn