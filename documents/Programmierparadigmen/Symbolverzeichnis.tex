\twocolumn
\chapter*{Symbolverzeichnis\markboth{Symbolverzeichnis}{Symbolverzeichnis}}
\addcontentsline{toc}{chapter}{Symbolverzeichnis}
%%%%%%%%%%%%%%%%%%%%%%%%%%%%%%%%%%%%%%%%%%%%%%%%%%%%%%%%%%%%%%%%%%%%%
% Mengenoperationen                                                 %
%%%%%%%%%%%%%%%%%%%%%%%%%%%%%%%%%%%%%%%%%%%%%%%%%%%%%%%%%%%%%%%%%%%%%
\section*{Mengenoperationen}
$A^C\;\;\;$ Komplement der Menge $A$\\
$\mathcal{P}(M)\;\;\;$ Potenzmenge von $M$\\
$\overline{M}\;\;\;$ Abschluss der Menge $M$\\
$\partial M\;\;\;$ Rand der Menge $M$\\
$M^\circ\;\;\;$ Inneres der Menge $M$\\
$A \times B\;\;\;$ Kreuzprodukt zweier Mengen\\
$A \subseteq B\;\;\;$ Teilmengenbeziehung\\
$A \subsetneq B\;\;\;$ echte Teilmengenbeziehung\\
$A \setminus B\;\;\;$ $A$ ohne $B$\\
$A \cup B\;\;\;$ Vereinigung\\
$A \dcup B\;\;\;$ Disjunkte Vereinigung\\
$A \cap B\;\;\;$ Schnitt\\

\section*{Geometrie}
$AB\;\;\;$ Gerade durch die Punkte $A$ und $B$\\
$\overline{AB}\;\;\;$ Strecke mit Endpunkten $A$ und $B$\\
$\triangle ABC\;\;\;$ Dreieck mit Eckpunkten $A, B, C$\\
