%!TEX root = Programmierparadigmen.tex
\chapter{Programmiertechniken}
\section{Rekursion}
\index{Rekursion|(}

\begin{definition}[rekursive Funktion]\xindex{Funktion!rekursive}%
    Eine Funktion $f: X \rightarrow X$ heißt rekursiv definiert,
    wenn in der Definition der Funktion die Funktion selbst wieder
    steht.
\end{definition}

\begin{beispiel}[rekursive Funktionen]
    \begin{bspenum}
        \item Fibonacci-Funktion:\xindex{Fibonacci-Funktion}
                \begin{align*}
                    fib: \mdn_0 &\rightarrow \mdn_0\\
                    fib(n) &= \begin{cases}
                            n                   &\text{falls } n \leq 1\\
                            fib(n-1) + fib(n-2) &\text{sonst}
                        \end{cases}
                \end{align*}
              Erzeugt die Zahlen $0, 1, 1, 2, 3, 5, 8, 13, \dots$
        \item Fakultät:\xindex{Fakultät}
                \begin{align*}
                    !  &: \mdn_0 \rightarrow \mdn_0\\
                    n! &= \begin{cases}
                            1              &\text{falls } n \leq 1\\
                            n\cdot (n-1)!  &\text{sonst}
                        \end{cases}
                \end{align*}
        \item \label{bsp:binomialkoeffizient} Binomialkoeffizient:\xindex{Binomialkoeffizient}
                \begin{align*}
                    \binom{\cdot}{\cdot}    &: \mdn_0 \times \mdn_0 \rightarrow \mdn_0\\
                            \binom{n}{k}    &= \begin{cases}
                        1                               &\text{falls } k=0 \lor k = n\\
                        \binom{n-1}{k-1}+\binom{n-1}{k} &\text{sonst}
                        \end{cases}
                \end{align*}
    \end{bspenum}
\end{beispiel}

Ein Problem von rekursiven Funktionen in Computerprogrammen ist der 
Speicherbedarf. Für jeden rekursiven Aufruf müssen alle lokalen Variablen
der aufrufenden Funktion (\enquote{stack frame}) gespeichert bleiben,
bis der rekursive Aufruf beendet ist. Im Fall der Fibonacci-Funktion
sieht ist der Call-Stack in \cref{fig:fib-callstack} abgebildet.

\begin{figure}[htp]
    \centering
    \includegraphics*[width=0.5\linewidth, keepaspectratio]{figures/fib-callstack.png} 
    \caption{Call-Stack der Fibonacci-Funktion}
    \label{fig:fib-callstack}
\end{figure}

\begin{bemerkung}
    Die Anzahl der rekursiven Aufrufe der Fibonacci-Funktion $f_C$ ist:
        \[f_C(n) = \begin{cases}
                        1 &\text{falls } n=0\\
                    2 \cdot fib(n) - 1 &\text{falls } n \geq 1
                    \end{cases}\]
\end{bemerkung}
\begin{beweis}\leavevmode
    \begin{itemize}
        \item Offensichtlich gilt $f_C(0) = 1$
        \item Offensichtlich gilt $f_C(1) = 1 = 2 \cdot fib(1) - 1$
        \item Offensichtlich gilt $f_C(2) = 3 = 2 \cdot fib(2) - 1$
        \item Für $n \geq 3$:
           \begin{align*}
                f_C(n) &= 1 + f_C(n-1) + f_C(n-2)\\
                       &= 1 + (2\cdot fib(n-1)-1) + (2 \cdot fib(n-2)-1)\\
                       &=2\cdot (fib(n-1) + fib(n-2)) - 1\\
                       &=2 \cdot fib(n) - 1
            \end{align*}
    \end{itemize}
\end{beweis}

Mit Hilfe der Formel von Moivre-Binet folgt:

\[f_C \in \mathcal{O} \left (\frac{\varphi^n- \psi^n}{\varphi - \psi} \right) \text{ mit } \varphi := \frac{1+ \sqrt{5}}{2} \text{ und }\psi := 1 - \varphi\]

Dabei ist der Speicherbedarf $\mathcal{O}(n)$. Dieser kann durch
das Benutzen eines Akkumulators signifikant reduziert werden.\todo{TODO}

\begin{definition}[linear rekursive Funktion]\xindex{Funktion!linear rekursive}%
    Eine Funktion heißt linear rekursiv, wenn in jedem Definitionszweig
    der Funktion höchstens ein rekursiver Aufruf vorkommt.
\end{definition}

\begin{definition}[endrekursive Funktion]\xindex{Funktion!endrekursive}\xindex{tail recursive}%
    Eine Funktion heißt endrekursiv, wenn in jedem Definitionszweig
    der Rekursive aufruf am Ende des Ausdrucks steht. Der rekursive
    Aufruf darf also insbesondere nicht in einen anderen Ausdruck
    eingebettet sein.
\end{definition}

Auf Englisch heißen endrekursive Funktionen \textit{tail recursive}.

\begin{beispiel}[Linear- und endrekursive Funktionen]
    \begin{bspenum}
        \item \texttt{fak n = if (n==0) then 1 else (n * fak (n-1))}\\
              ist eine linear rekursive Funkion, aber nicht endrekursiv,
              da nach der Rückgabe von \texttt{fak (n-1)} noch die Multiplikation
              ausgewertet werden muss.
        \item \texttt{fakAcc n acc = if (n==0) then acc else fakAcc (n-1) (n*acc)}\\
              ist eine endrekursive Funktion.
        \item \texttt{fib n = n <= 1 ? n : fib(n-1) + fib (n-2)}\\
              ist weder linear- noch endrekursiv.
    \end{bspenum}
\end{beispiel}

Wenn eine rekursive Funktion nicht terminiert oder wenn 

\index{Rekursion|)}
\section{Backtracking}
\index{Backtracking|(}
Unter \textit{Backtracking} versteht man eine Programmiertechnik, die
(eventuell implizit) auf einem Suchbaum arbeitet und mittels Tiefensuche versucht
eine Lösung zu finden.

\begin{beispiel}[Backtracking]
    Probleme, bei deren (vollständigen) Lösung Backtracking verwendet wird, sind:
    \begin{bspenum}
        \item Damenproblem
        \item Springerproblem
        \item Rucksackproblem
    \end{bspenum}
\end{beispiel}
\index{Backtracking|)}

\section{Funktionen höherer Ordnung}
Funktionen höherer Ordnung sind Funktionen, die auf Funktionen arbeiten.
Bekannte Beispiele sind:
\begin{itemize}
    \item \texttt{map(function, list)}\xindex{map}\\
          \texttt{map} wendet \texttt{function} auf jedes einzelne
          Element aus \texttt{list} an.
    \item \texttt{filter(function, list)}\xindex{filter}\\
          \texttt{filter} gibt eine Liste aus Elementen zurück, für 
          die \texttt{function} mit \texttt{true} evaluiert.
    \item \texttt{reduce(function, list)}\xindex{reduce}\\
          \texttt{function} ist für zwei Elemente aus \texttt{list}
          definiert und gibt ein Element des gleichen Typs zurück.
          Nun steckt \texttt{reduce} zuerst zwei Elemente aus \texttt{list}
          in \texttt{function}, merkt sich dann das Ergebnis und nimmt
          so lange weitere Elemente aus \texttt{list}, bis jedes 
          Element genommen wurde.\\
          Bei \texttt{reduce} ist die Assoziativität wichtig (vgl. \cpageref{bsp:foldl-und-foldr})
\end{itemize}
