%!TEX root = Programmierparadigmen.tex
\chapter{X10}\index{X10|(}%
X10 ist eine objektorientierte Programmiersprache, die 2004 bei IBM entwickelt
wurde.

X10 nutzt das PGAS-Modell:

\begin{definition}[PGAS\footnotemark]\xindex{PGAS}%
    PGAS (partitioned global address space) ist ein Programmiermodell für 
    Mehrprozessorsysteme und massiv parallele Rechner. Dabei wird der globale 
    Adressbereich des Arbeitsspeichers logisch unterteilt. Jeder Prozessor 
    bekommt jeweils einen dieser Adressbereiche als lokalen Speicher zugeteilt. 
    Trotzdem können alle Prozessoren auf jede Speicherzelle zugreifen, wobei auf 
    den lokalen Speicher mit wesentlich höherer Geschwindigkeit zugegriffen 
    werden kann als auf den von anderen Prozessoren.
\end{definition}
\footnotetext{\url{https://de.wikipedia.org/wiki/PGAS}}

\section{Erste Schritte}
Als erstes sollte man x10 von \url{http://x10-lang.org/x10-development/building-x10-from-source.html?id=248} herunterladen.

Dann kann man die \texttt{bin/x10c} zum erstellen von ausführbaren Dateien nutzen.
Der Befehl \texttt{x10c HelloWorld.x10} erstellt eine ausführbare Datei namens
\texttt{a.out}.

\inputminted[numbersep=5pt, tabsize=4, frame=lines, label=HelloWorld.x10]{cpp}{scripts/x10/HelloWorld.x10}

\section{Syntax}
\section{Datentypen}
Byte, UByte, Short, UShort, Char, Int, UInt, Long, ULong, Float, Double, Boolean, Complex, String, Point, Region, Dist, Array

\section{Beispiele}

\section{Weitere Informationen}
\begin{itemize}
    \item \url{http://x10-lang.org/}
\end{itemize}
\index{X10|)}