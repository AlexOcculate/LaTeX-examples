%!TEX root = Programmierparadigmen.tex
\chapter{X10}\index{X10|(}%
X10 ist eine objektorientierte Programmiersprache, die 2004 bei IBM entwickelt
wurde.

\section{Erste Schritte}
Als erstes sollte man x10 von \url{http://x10-lang.org/x10-development/building-x10-from-source.html?id=248} herunterladen.

Dann kann man die bin/x10c++ zum erstellen von ausführbaren Dateien nutzen.
Der Befehl \texttt{x10c++ hello-world.x10} erstellt eine ausführbare Datei namens
\texttt{a.out}.

\inputminted[numbersep=5pt, tabsize=4, frame=lines, label=hello-world.x10]{cpp}{scripts/x10/hello-world.x10}

\section{Syntax}
\section{Beispiele}

\section{Weitere Informationen}
\begin{itemize}
    \item \url{http://x10-lang.org/}
\end{itemize}
\index{X10|)}