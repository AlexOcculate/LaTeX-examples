%!TEX root = Programmierparadigmen.tex
\chapter{$\lambda$-Kalkül}
Der $\lambda$-Kalkül (gesprochen: Lambda-Kalkül) ist eine formale Sprache.
In diesem Kalkül gibt es drei Arten von Termen $T$:

\begin{itemize}
    \item Variablen: $x$
    \item Applikationen: $(T S)$
    \item Lambda-Abstraktion: $\lambda x. T$
\end{itemize}

In der Lambda-Abstraktion nennt man den Teil vor dem Punkt die \textit{Parameter}
der $\lambda$-Funktion. Wenn etwas dannach kommt, auf die die Funktion angewendet
wird so heißt dieser Teil das \textit{Argument}:

\[(\lambda \underbrace{x}_{\mathclap{\text{Parameter}}}. x^2) \overbrace{5}^{\mathclap{\text{Argument}}} = 5^2\]

\begin{beispiel}[$\lambda$-Funktionen]
    \begin{bspenum}
        \item $\lambda x. x$ heißt Identität.
        \item $(\lambda x. x^2)(\lambda y. y + 3) = \lambda y. (y+3)^2$
        \item \label{bsp:lambda-3} $\begin{aligned}[t]
                           &(\lambda x.\Big (\lambda y.yx \Big ))~ab\\
                \Rightarrow&(\lambda y.ya)b\\
                \Rightarrow&ba
               \end{aligned}$
    \end{bspenum}

    In \cref{bsp:lambda-3} sieht man, dass $\lambda$-Funktionen die Argumente
    von Links nach rechts einziehen.
\end{beispiel}

Die Funktionsapplikation sei linksassoziativ. Es gilt also:

\[a~b~c~d = ((a~b)~c)~d\]

\begin{definition}[Gebundene Variable]\xindex{Variable!gebundene}%
    Eine Variable heißt gebunden, wenn sie der Parameter einer $\lambda$-Funktion ist.
\end{definition}

\begin{definition}[Freie Variable]\xindex{Variable!freie}%
    Eine Variable heißt \textit{frei}, wenn sie nicht gebunden ist.
\end{definition}

\begin{satz}
    Der untypisierte $\lambda$-Kalkül ist Turing-Äquivalent.
\end{satz}

\section{Reduktionen}\index{Reduktion|(}
\begin{definition}[Redex]\xindex{Redex}%
    Eine $\lambda$-Term der Form $(\lambda x. t_1) t_2$ heißt Redex.
\end{definition}

\begin{definition}[$\alpha$-Äquivalenz]\xindex{Reduktion!Alpha ($\alpha$)}\xindex{Äquivalenz!Alpha ($\alpha$)}%
    Zwei Terme $T_1, T_2$ heißen $\alpha$-Äquivalent, wenn $T_1$ durch 
    konsistente Umbenennung in $T_2$ überführt werden kann.

    Man schreibt dann: $T_1 \overset{\alpha}{=} T_2$.
\end{definition}

\begin{beispiel}[$\alpha$-Äquivalenz]
    \begin{align*}
        \lambda x.x    &\overset{\alpha}{=} \lambda y. y\\
        \lambda x. x x &\overset{\alpha}{=} \lambda y. y y\\
        \lambda x. (\lambda y. z (\lambda x. z y) y) &\overset{\alpha}{=}
        \lambda a. (\lambda x. z (\lambda c. z x) x)
    \end{align*}
\end{beispiel}

\begin{definition}[$\beta$-Äquivalenz]\xindex{Reduktion!Beta ($\beta$)}\xindex{Äquivalenz!Beta ($\beta$)}%
    Eine $\beta$-Reduktion ist die Funktionsanwendung auf einen Redex:
    \[(\lambda x. t_1) t_2 \Rightarrow t_1 [x \mapsto t_2]\]
\end{definition}

\begin{beispiel}[$\beta$-Äquivalenz]
    \begin{defenum}
        \item $(\lambda x.x) y \overset{\beta}{\Rightarrow} x[x \mapsto y] = y$
        \item $(\lambda x. x (\lambda x. x)) (y z) \overset{\beta}{\Rightarrow} (x(\lambda x. x))[x \mapsto y z] (y z) (\lambda x. x)$
    \end{defenum}
\end{beispiel}

\begin{definition}[$\eta$-Äquivalenz]\xindex{Reduktion!Eta ($\eta$)}\xindex{Äquivalenz!Eta ($\eta$)}%
    Zwei Terme $\lambda x. f~x$ und $f$ heißen $\eta$-Äquivalent, wenn
    $x$ nicht freie Variable von $f$ ist.
\end{definition}

\begin{beispiel}[$\eta$-Äquivalenz]
    TODO
\end{beispiel}
\index{Reduktion|)}

\section{Auswertungsstrategien}
\begin{definition}[Normalenreihenfolge]\xindex{Normalenreihenfolge}%
    In der Normalenreihenfolge-Auswertungsstrategie wird der linkeste äußerste
    Redex ausgewertet.
\end{definition}

\begin{definition}[Call-By-Name]\xindex{Call-By-Name}%
    In der Call-By-Name Auswertungsreihenfolge wird der linkeste äußerste Redex
    reduziert, der nicht von einem $\lambda$ umgeben ist.
\end{definition}

Die Call-By-Name Auswertung wird in Funktionen verwendet.

Haskell verwendet die Call-By-Name Auswertungsreihenfolge zusammen mit \enquote{sharing}. Dies nennt man \textit{Lazy Evaluation}. Ein spezialfall der Lazy-Evaluation ist die sog. \textit{Kurzschlussauswertung}.\xindex{Kurzschlussauswertung}\xindex{Short-circuit evaluation}
Das bezeichnet die Lazy-Evaluation von booleschen Ausdrücken.

\todo[inline]{Was ist sharing?}

\begin{definition}[Call-By-Value]\xindex{Call-By-Value}%
    In der Call-By-Value Auswertung wird der linkeste Redex reduziert, der
    nicht von einem $\lambda$ umgeben ist und dessen Argument ein Wert ist.
\end{definition}

Die Call-By-Value Auswertungsreihenfolge wird in C und Java verwendet.
Auch in Haskell werden arithmetische Ausdrücke in der Call-By-Name Auswertungsreihenfolge reduziert.

\section{Church-Zahlen}
Im $\lambda$-Kalkül lässt sich jeder mathematische Ausdruck darstellen, also 
insbesondere beispielsweise auch $\lambda x. x+3$. Aber \enquote{$3$} und
\enquote{$+$} ist hier noch nicht das $\lambda$-Kalkül.

Zuerst müssen wir uns also Gedanken machen, wie man natürliche Zahlen $n \in \mdn$
darstellt. Dafür dürfen wir nur Variablen und $\lambda$ verwenden. Eine Möglichkeit
das zu machen sind die sog. \textit{Church-Zahlen}.

Dabei ist die Idee, dass die Zahl angibt wie häufig eine Funktion $f$ auf eine
Variable $z$ angewendet wird. Also:
\begin{itemize}
    \item $0 := \lambda f~z. z$
    \item $1 := \lambda f~z. f z$
    \item $2 := \lambda f~z. f (f z)$
    \item $3 := \lambda f~z. f (f (f z))$
\end{itemize}

Auch die gewohnten Operationen lassen sich so darstellen.

\begin{beispiel}[Nachfolger-Operation]
    \begin{align*}
        \succ :&= \lambda n f z. f (n f z)\\
               &= \lambda n. (\lambda f  (\lambda z f (n f z)))
    \end{align*}
    Dabei ist $n$ die Zahl. 

    Will man diese Funktion anwenden, sieht das wie folgt aus:
    \begin{align*}
    \succ 1&= (\lambda n f z. f(n f z)) 1\\
           &= (\lambda n f z. f(n f z)) \underbrace{(\lambda f~z. f z)}_{n}\\
           &= \lambda f z. f (\lambda f~z. f z) f z\\
           &= \lambda f z. f (f z)\\
           &= 2
    \end{align*}
\end{beispiel}

\begin{beispiel}[Vorgänger-Operation]
    \begin{align*}
        \pair&:= \lambda a. \lambda b. \lambda f. f a b\\
        \fst &:= \lambda p. p (\lambda a. \lambda b. a)\\
        \snd &:= \lambda p. p (\lambda a. \lambda b. b)\\
        \nxt &:= \lambda p. \pair (\snd p)~(\succ (\snd p))\\
        \pred&:= \lambda n. \fst (n \nxt (\pair c_0 c_0))
    \end{align*}
\end{beispiel}
\begin{beispiel}[Addition]
    \begin{align*}
        \text{plus} &:= \lambda m n f z. m f (n f z)
    \end{align*}
    Dabei ist $m$ der erste Summand und $n$ der zweite Summand.
\end{beispiel}

\begin{beispiel}[Multiplikation]
    \begin{align*}
     \text{times} :&= \lambda m n f. m~s~(n~f~z)\\
                   &\overset{\eta}{=} \lambda m n f z. n (m s) z
    \end{align*}
    Dabei ist $m$ der erste Faktor und $n$ der zweite Faktor.
\end{beispiel}

\begin{beispiel}[Potenz]
    \begin{align*}
     \text{exp} :&= \lambda b e. eb\\
                   &\overset{\eta}{=} \lambda b e f z. e b f z
    \end{align*}
    Dabei ist $b$ die Basis und $e$ der Exponent.
\end{beispiel}

\section{Church-Booleans}
\begin{definition}[Church-Booleans]\xindex{Church-Booleans}%
    \texttt{True} wird zu $c_{\text{true}} := \lambda t. \lambda f. t$.\\
    \texttt{False} wird zu $c_{\text{false}} := \lambda t. \lambda f. f$.
\end{definition}

\section{Weiteres}
\begin{satz}[Satz von Curch-Rosser]
    Wenn zwei unterschiedliche Terme $a$ und $b$ äquivalent sind, d.h. mit Reduktionsschritten beliebiger Richtung ineinander transformiert werden können, dann gibt es einen weiteren Term $c$, zu dem sowohl $a$ als auch $b$ reduziert werden können.
\end{satz}

\section{Fixpunktkombinator}
\begin{definition}[Fixpunkt]\xindex{Fixpunkt}%
    Sei $f: X \rightarrow Y$ eine Funktion mit $\emptyset \neq A = X \cap Y$ und
    $a \in A$.

    $a$ heißt \textbf{Fixpunkt} der Funktion $f$, wenn $f(a) = a$ gilt.
\end{definition}

\begin{beispiel}[Fixpunkt]
    \begin{bspenum}
        \item $f_1: \mdr \rightarrow \mdr; f(x) = x^2 \Rightarrow x_1 = 0$ ist 
              Fixpunkt von $f$, da $f(0) = 0$. $x_2 = 1$ ist der einzige weitere
              Fixpunkt dieser Funktion.
        \item $f_2: \mdn \rightarrow \mdn$ hat ganz $\mdn$ als Fixpunkte, also
              insbesondere unendlich viele Fixpunkte.
        \item $f_3: \mdr \rightarrow \mdr; f(x) = x+1$ hat keinen einzigen Fixpunkt.
        \item $f_4: \mdr[X] \rightarrow \mdr[X]; f(p) = p^2$ hat $p_1(x) = 0$ und
              $p_2(x)=1$ als Fixpunkte.
    \end{bspenum}
\end{beispiel}

\begin{definition}[Kombinator]\xindex{Kombinator}%
    Ein Kombinator ist eine Abbildung ohne freie Variablen.
\end{definition}

\begin{beispiel}[Kombinatoren\footnotemark]%
    \begin{bspenum}
        \item $\lambda a.\ a$
        \item $\lambda a.\ \lambda b.\ a$
        \item $\lambda f.\ \lambda a.\ \lambda b. f\ b\ a$
    \end{bspenum}
\end{beispiel}
\footnotetext{Quelle: \url{http://www.haskell.org/haskellwiki/Combinator}}

\begin{definition}[Fixpunkt-Kombinator]\xindex{Fixpunkt-Kombinator}%
    Sei $f$ ein Kombinator, der $f\ g = g\ (f\ g)$ erfüllt. Dann heißt $f$
    \textbf{Fixpunktkombinator}.
\end{definition}

Insbesondere ist also $f \ g$ ein Fixpunkt von $g$.

\begin{definition}[Y-Kombinator]\xindex{Y-Kombinator}%
    Der Fixpunktkombinator
    \[Y := \lambda f.\ (\lambda x.\ f\ (x\ x))\ (\lambda x.\ f\ (x\ x))\]
    heißt $Y$-Kombinator.
\end{definition}

\begin{behauptung}
    Der $Y$-Kombinator ist ein Fixpunktkombinator.
\end{behauptung}

\begin{beweis}\footnote{Quelle: Vorlesung WS 2013/2014, Folie 175}\leavevmode

    \textbf{Teil 1:} Offensichtlich ist $Y$ ein Kombinator.

    \textbf{Teil 2:} z.~Z.: $Y f \Rightarrow^* f \ (Y \ f)$

    \begin{align*}
        Y\ f     &=\hphantom{^\beta f\ } (\lambda f.\ (\lambda x.\ f\ (x\ x))\ (\lambda x.\ f\ (x\ x)))\ f\\
                 &\Rightarrow^\beta\hphantom{f \ (\lambda f.\ }  (\lambda x. f\ (x\ x))\ (\lambda x.\ f\ (x\ x))\\
                 &\Rightarrow^\beta  f \ (\hphantom{\lambda f.\ }(\lambda x.\ f\ (x\ x))\ (\lambda x.\ f\ (x\ x)))\\
                 &\Rightarrow^\beta  f \ (\lambda f.\ (\lambda x.\ f\ (x\ x))\ (\lambda x.\ f\ (x\ x))\ f)\\
                 &=\hphantom{^\beta} f \ (Y \ f)
    \end{align*}
    $\qed$
\end{beweis}

\begin{definition}[Turingkombinator]\xindex{Turingkombinator}%
    Der Fixpunktkombinator
    \[\Theta := (\lambda x. \lambda y. y\ (x\ x\ y)) (\lambda x.\ \lambda y.\ y\ (x\ x\ y))\]
    heißt \textbf{Turingkombinator}.
\end{definition}

\begin{behauptung}
    Der Turing-Kombinator $\Theta$ ist ein Fixpunktkombinator.
\end{behauptung}

\begin{beweis}\footnote{Quelle: Übungsblatt 6, WS 2013/2014}

    \textbf{Teil 1:} Offensichtlich ist $\Theta$ ein Kombinator.

    \textbf{Teil 2:} z.~Z.: $\Theta f \Rightarrow^* f \ (\Theta \ f)$

    Sei $\Theta_0 := (\lambda x.\ \lambda y.\ y\ (x\ x\ y))$. Dann gilt:
    \begin{align*}
        \Theta\ f &= ((\lambda x.\ \lambda y.\ y\ (x\ x\ y))\ \Theta_0)\ f\\
                 &\Rightarrow^\beta (\lambda y. y\ (\Theta_0 \ \Theta_0 \ y))\ f\\
                 &\Rightarrow^\beta f \ (\Theta_0 \Theta_0 f)\\
                 &= f \ (\Theta \ f)
    \end{align*}
    $\qed$
\end{beweis}



Für den Turing-Operator gilt 