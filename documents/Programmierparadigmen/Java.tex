%!TEX root = Programmierparadigmen.tex
\chapter{Java}
\index{Java|(}

Im Folgenden wird in aller Kürze erklärt, wie man in Java Programme schreibt,
die auf mehreren Prozessoren laufen.

\section{Thread, ThreadPool und Runnable}
%%%%%%%%%%%%%%%%%%%%%%%%%%%%%%%%%%%%%%%%%%%%%%%%%%%%%%%%%%%%%%%%%%%%%%%%%%%%%%%%
\texttt{Interface Runnable}\\
\-\hspace{0.8cm}$\leftharpoonup$ \texttt{java.lang.Thread}\xindex{Runnable}%
\begin{itemize}
    \item Methods: 
    \begin{itemize}
        \item \texttt{void run()}: When an object implementing interface 
          Runnable is used to create a thread, starting the thread causes the 
          object's run method to be called in that separately executing thread.
    \end{itemize}
\end{itemize}
%%%%%%%%%%%%%%%%%%%%%%%%%%%%%%%%%%%%%%%%%%%%%%%%%%%%%%%%%%%%%%%%%%%%%%%%%%%%%%%%
\texttt{Class Thread}\\
\-\hspace{0.8cm}$\leftharpoonup$ \texttt{java.lang.Thread}\xindex{Thread}%
\begin{itemize}
    \item implements Runnable
\end{itemize}
%%%%%%%%%%%%%%%%%%%%%%%%%%%%%%%%%%%%%%%%%%%%%%%%%%%%%%%%%%%%%%%%%%%%%%%%%%%%%%%%
\texttt{Class ThreadPoolExecutor}\xindex{ThreadPoolExecutor}\\
\-\hspace{0.8cm}$\leftharpoonup$ \texttt{java.util.concurrent.ThreadPoolExecutor}
%%%%%%%%%%%%%%%%%%%%%%%%%%%%%%%%%%%%%%%%%%%%%%%%%%%%%%%%%%%%%%%%%%%%%%%%%%%%%%%%
\texttt{Interface Callable<V>}\xindex{Callable}\\
\-\hspace{0.8cm}$\leftharpoonup$ \texttt{java.util.concurrent}
\begin{itemize}
    \item Parameter: \texttt{V} - the result type of method call
 \end{itemize} 
%%%%%%%%%%%%%%%%%%%%%%%%%%%%%%%%%%%%%%%%%%%%%%%%%%%%%%%%%%%%%%%%%%%%%%%%%%%%%%%%
\section{Futures}\xindex{Future}\index{Promise|see{Future}}
\enquote{Ein Future (engl. \enquote{Zukunft}) oder ein Promise (engl. \enquote{Versprechen}) bezeichnet in der Programmierung einen Platzhalter (Proxy) für ein Ergebnis, das noch nicht bekannt ist, meist weil seine Berechnung noch nicht abgeschlossen ist.}

\texttt{Interface Future<V>}\xindex{Future}\\
\-\hspace{0.8cm}$\leftharpoonup$ \texttt{java.util.concurrent}

\textbf{Parameter}:
\begin{itemize}
    \item \texttt{V}: The result type returned by this Future's get method
\end{itemize}
%%%%%%%%%%%%%%%%%%%%%%%%%%%%%%%%%%%%%%%%%%%%%%%%%%%%%%%%%%%%%%%%%%%%%%%%%%%%%%%%

\textbf{Beispiel}:
% Zu lang, geht das kürzer?
%\inputminted[numbersep=5pt, tabsize=4]{java}{scripts/java/matrix-multiplication.java}
\begin{beispiel}[Runnable, ExecutorService, ThreadPool\footnotemark]
    \inputminted[numbersep=5pt, tabsize=4]{java}{scripts/java/vorlesung-futures-basics.java}
\end{beispiel}
\footnotetext{WS 2013/2014, Kapitel 41, Folie 26}


\section{Literatur}
\begin{itemize}
    \item \href{http://openbook.galileocomputing.de/javainsel9/javainsel_14_004.htm}{Java ist auch eine Insel}: Kapitel 14 -
          Threads und nebenläufige Programmierung
    \item \href{http://www.vogella.com/tutorials/JavaConcurrency/article.html}{vogella.com}: Java concurrency (multi-threading) - Tutorial
    \item Links zur offiziellen Java 8 Dokumentation:
        \begin{itemize}
            \item \href{http://docs.oracle.com/javase/8/docs/api/java/util/concurrent/ThreadPoolExecutor.html}{ThreadPoolExecutor}
            \item \href{http://docs.oracle.com/javase/8/docs/api/java/lang/Runnable.html}{Runnable}
            \item \href{http://docs.oracle.com/javase/8/docs/api/java/lang/Thread.html}{Thread}
            \item \href{http://docs.oracle.com/javase/8/docs/api/java/util/concurrent/Callable.html}{Callable}
            \item \href{http://docs.oracle.com/javase/8/docs/api/java/util/concurrent/Future.html}{Future}
        \end{itemize}
\end{itemize}

\index{Java|)}