%!TEX root = Programmierparadigmen.tex
\chapter{Typinferenz}
\begin{definition}[Datentyp]\index{Typ|see{Datentyp}}\xindex{Datentyp}%
	Ein \textit{Datentyp} oder kurz \textit{Typ} ist eine Menge von Werten, mit
	denen eine Bedeutung verbunden ist.
\end{definition}

\begin{beispiel}[Datentypen]
	\begin{itemize}
		\item $\text{\texttt{bool}} = \Set{\text{True}, \text{False}}$
		\item $\text{\texttt{char}} = \text{vgl. \cpageref{sec:ascii-tabelle}}$
		\item $\text{\texttt{int}}_{\text{Haskell}} = [-2^{29}, 2^{29}-1] \cap \mathbb{N}$
		\item $\text{\texttt{int}}_{\text{C90}} = [-2^{15}-1, 2^{15}-1] \cap \mathbb{N}$\footnote{siehe ISO/IEC 9899:TC2, Kapitel 7.10: Sizes of integer types <limits.h>}
		\item \texttt{float} = siehe IEEE 754
		\item Funktionstypen, z.~B. $\text{\texttt{int}} \rightarrow \text{\texttt{int}}$ oder
			  $\text{\texttt{char}} \rightarrow \text{\texttt{int}}$
	\end{itemize}
\end{beispiel}

\underline{Hinweis:} Typen sind unabhängig von ihrer Repräsentation. So kann ein
\texttt{bool} durch ein einzelnes Bit repräsentiert werden oder eine Bitfolge
zugrunde liegen.

Auf Typen sind Operationen definiert. So kann man auf numerischen Typen eine 
Addition (+), eine Subtraktion (-), eine Multiplikation (*) und eine Division (/)
definieren.\\
Ich schreibe hier bewusst \enquote{eine} Multiplikation und nicht \enquote{die}
Multiplikation, da es verschiedene Möglichkeiten gibt auf Gleitpunktzahlen
Multiplikationen zu definieren. So kann man beispielsweise die Assoziativität
unterschiedlich wählen.

\begin{beispiel}[Multiplikation ist nicht assoziativ]
	In Python 3 ist die Multiplikation linksassoziativ. Also:
	\inputminted[numbersep=5pt, tabsize=4]{python}{scripts/python/multiplikation.py}
\end{beispiel}

\begin{definition}[Typvariable]\xindex{Typvariable}%
	Eine Typvariable repräsentiert einen Typen.
\end{definition}

\underline{Hinweis:} Üblicherweise werden kleine griechische Buchstaben ($\alpha, \beta, \tau_1, \tau_2, \dots$) als Typvariablen gewählt.

Genau wie Typen bestimmte Operationen haben, die auf ihnen definiert sind, 
kann man sagen, dass Operationen bestimmte Typen, auf die diese Anwendbar sind. So ist
\[\alpha+\beta\]
für numerische $\alpha$ und $\beta$ wohldefiniert, auch wenn $\alpha$ und $\beta$ boolesch sind
oder beides Strings sind könnte das Sinn machen. Es macht jedoch z.~B. keinen Sinn,
wenn $\alpha$ ein String ist und $\beta$ boolesch.

Die Menge aller Operationen, die auf die Variablen angewendet werden, nennt man
\textbf{Typkontext}\xindex{Typkontext}. Dieser wird üblicherweise mit $\Gamma$
bezeichnet.

Das Ableiten einer Typisierung für einen Ausdruck nennt man \textbf{Typinferenz}\xindex{Typinferenz}.
Man schreibt: $\vdash(\lambda x.2): \alpha \rightarrow \text{int}$.

Bei solchen Ableitungen sind häufig viele Typen möglich. So kann der Ausdruck
\[\lambda x.2\]
Mit folgenderweise typisiert werden:
\begin{itemize}
	\item $\vdash(\lambda x.2): \text{bool} \rightarrow int$
	\item $\vdash(\lambda x.2): \text{int} \rightarrow int$
	\item $\vdash(\lambda x.2): \text{Char} \rightarrow int$
	\item $\vdash(\lambda x.2): \alpha \rightarrow int$
\end{itemize}

In der letzten Typisierung stellt $\alpha$ einen beliebigen Typen dar.

\begin{definition}[Typsystem $\Gamma \vdash t: T$]\label{def:typsystem-t1}
	Ein Typkontext $\Gamma$ ordnet jeder freien Variable $x$ einen Typ $\Gamma(x)$
	durch folgende Regeln zu:

	\begin{align*}
		\CONST:&\frac{c \in \text{Const}}{\Gamma \vdash c: \tau_c}\\
			   &\\
		\VAR:  &\frac{\Gamma(x) = \tau}{\Gamma \vdash c: \tau}\\
			   &\\
		\ABS:  &\frac{\Gamma, x: \tau_1 \vdash t: \tau_2}{\Gamma \vdash \lambda x. t: \tau_1 \rightarrow \tau_2}\\
			   &\\
		\APP:  &\frac{\Gamma \vdash t_1, \tau_2 \tau\;\;\; \Gamma \vdash t_2: \tau_2}{\Gamma \vdash t_1 t_2: \tau}
	\end{align*}
\end{definition}



Dabei ist der lange Strich kein Bruchstrich, sondern ein Symbol der Logik das als
\textbf{Schlussstrich}\xindex{Schlussstrich} bezeichnet wird. Dabei ist der 
Zähler als Voraussetzung und der Nenner als Schlussfolgerung zu verstehen.

\begin{definition}[Typsubstituition]\xindex{Typsubstituition}%
	Eine \textit{Typsubstituition} ist eine endliche Abbildung von Typvariablen auf
	Typen.
\end{definition}

Für eine Menge von Typsubsitutionen wird überlicherweise $\sigma$ als Symbol
verwendet. Man schreibt also beispielsweise:
\[\sigma = [\alpha_1 \text{\pointer} \text{\texttt{bool}}, \alpha_2 \text{\pointer} \alpha_1 \rightarrow \alpha_1]\]

\begin{definition}[Lösung eines Typkontextes]
	Sei $t$ eine beliebige freie Variable, $\tau = \tau(t)$ ein beliebiger Typ
	$\sigma$ eine Menge von Typsubstitutionen und $\Gamma$ ein Typkontext.

	$(\sigma, \tau)$ heißt eine Lösung für $(\Gamma, t)$, falls gilt:
	\[\sigma \Gamma \vdash t : \tau\]
\end{definition}

\section{Beispiele}
Im Folgenden wird die Typinferenz für einige $\lambda$-Funktionen durchgeführt.

\subsection[$\lambda x.\ \lambda y.\ x\ y$]{$\lambda x.\ \lambda y.\ x\ y$\footnote{Lösung von Übungsblatt 6, WS 2013 / 2014}}
Gesucht ist ein Typ $\tau$, sodass sich $\vdash \lambda x.\ \lambda y.\ x\ y: \tau$
mit einem Ableitungsbaum nachweisen lässt. Es gibt mehrere solche $\tau$, aber
wir suchen das allgemeinste. Die Regeln unseres Typsystems (siehe \cpageref{def:typsystem-t1}) 
sind \textit{syntaxgerichtet}, d.~h. zu jedem $\lambda$-(Teil)-Term gibt es genau
eine passende Regel.

Für $\lambda x.\ \lambda y.\ x\ y$ wissen wir also schon, dass jeder Ableitungsbaum\xindex{Ableitungsbaum}
von folgender Gestalt ist. Dabei sind $\alpha_i$ Platzhalter:

\[\ABS \frac{\ABS\frac{\textstyle\ABS \frac{\textstyle\VAR \frac{(x: \alpha_2, y: \alpha_4)\ (x) = \alpha_6}{x: \alpha_2, y: \alpha_4 \vdash x: \alpha_6}\ \ \VAR \frac{(x:\alpha_2, y: \alpha_4)\ (y) = \alpha_7}{x: \alpha_2, y: \alpha_4 \vdash y : \alpha_7}}{\textstyle x: \alpha_2, y: \alpha_4 \vdash x\ y: \alpha_5}}{x:\alpha_2 \vdash \lambda y.\ x\ y\ :\ \alpha_3}}{\vdash \lambda x.\ \lambda \ y.\ x\ y: \alpha_1}\]

Das was wir haben wollen steht am Ende, also unter dem unterstem Schlussstrich.
Dann bedeutet die letzte Zeile 
\[\vdash \lambda x.\ \lambda \ y.\ x\ y: \alpha_1\]

Ohne (weitere) Voraussetzungen lässt sich sagen, dass der Term
\[\lambda x.\ \lambda \ y.\ x\ y\]
vom Typ $\alpha_1$ ist.

Links der Schlussstriche steht jeweils die Regel, die wir anwenden. Also entweder
$\ABS$, $\VAR$, $\CONST$ oder $\APP$.

Nun gehen wir eine Zeile höher:

\[x:\alpha_2 \vdash \lambda y.\ x\ y\ :\ \alpha_3\]

Diese Zeile ist so zu lesen: Mit der Voraussetzung, dass $x$ vom Typ $\alpha_2$
ist, lässt sich syntaktisch Folgern, dass der Term $\lambda y.\ x\ y$ vom
Typ $\alpha_3$ ist.

\underline{Hinweis:} Alles was in Zeile $i$ dem $\vdash$ steht, steht auch in 
jedem \enquote{Nenner} in Zeile $j < i$ vor jedem einzelnen $\vdash$.

Folgende Typgleichungen $C$ lassen sich aus dem Ableitungsbaum ablesen:

\begin{align*}
	C &= \Set{\alpha_1 = \alpha_2 \rightarrow \alpha_3}\\
	  &\cup \Set{\alpha_3 = \alpha_4 \rightarrow \alpha_5}\\
	  &\cup \Set{\alpha_6 = \alpha_7 \rightarrow \alpha_5}\\
	  &\cup \Set{\alpha_6 = \alpha_2}\\
	  &\cup \Set{\alpha_7 = \alpha_4}
\end{align*}

Diese Bedingungen (engl. \textit{Constraints})\xindex{Constraints} haben eine
allgemeinste Lösung mit einem allgemeinsten Unifikator $\sigma_C$:

\begin{align*}
	\sigma_C = [&\alpha_1 \Parr (\alpha_4 \rightarrow \alpha_5) \rightarrow \alpha_4 \rightarrow \alpha_5,\\
	&\alpha_2 \Parr \alpha_4 \rightarrow \alpha_5,\\
	&\alpha_3 \Parr \alpha_4 \rightarrow \alpha_5,\\
	&\alpha_6 \Parr \alpha_4 \rightarrow \alpha_5,\\
	&\alpha_7 \Parr \alpha_4]
\end{align*}

\underline{Hinweis:} Es gilt $(\alpha_4 \rightarrow \alpha_5) \rightarrow \alpha_4 \rightarrow \alpha_5 = (\alpha_4 \rightarrow \alpha_5) \rightarrow (\alpha_4 \rightarrow \alpha_5)$

Also gilt: Der allgemeinste Typ von $\lambda x.\ \lambda y.\ x\ y$ ist $\sigma_C (\alpha_1) = (\alpha_4 \rightarrow \alpha_5) \rightarrow \alpha_4 \rightarrow \alpha_5$.

\subsection[Selbstapplikation]{Selbstapplikation\footnote{Lösung von Übungsblatt 6, WS 2013 / 2014}}\xindex{Selbstapplikation}
Im Folgenden wird eine Typinferenz für die Selbstapplikation, also
\[\lambda x.\ x\ x\]
durchgeführt.

Zuerst erstellt man den Ableitungsbaum:

\[\ABS\frac{\APP \frac{\VAR \frac{(x:\alpha_2)\ (x) = \alpha_5}{x:\alpha_2 \vdash x: \alpha_5} \;\;\; \VAR \frac{(x:\alpha_2)\ (x) = \alpha_4}{x:\alpha_2 \vdash x:\alpha_4}}{x: \alpha_2 \vdash x\ x\ :\ \alpha_3}}{\vdash \lambda x.\ x\ x: \alpha_1}\]

Dies ergibt die Constraint-Menge
\begin{align}
	C&= \Set{\alpha_1 = \alpha_2 \rightarrow \alpha_3}    &\text{$\ABS$-Regel}\label{eq:bsp2.c1}\\ 
     &\cup \Set{\alpha_5 = \alpha_4 \rightarrow \alpha_3} &\text{$\APP$-Regel}\label{eq:bsp2.c2}\\
     &\cup \Set{\alpha_5 = \alpha_2}                      &\text{Linke $\VAR$-Regel}\label{eq:bsp2.c3}\\
     &\cup \Set{\alpha_4 = \alpha_2}                      &\text{Rechte $\VAR$-Regel}\label{eq:bsp2.c4}
\end{align}

Aus \cref{eq:bsp2.c3} und \cref{eq:bsp2.c4} folgt:
\[\alpha_2 = \alpha_4 = \alpha_5\]

Also lässt sich \cref{eq:bsp2.c2} umformulieren:
\[\alpha_2 = \alpha_2 \rightarrow \alpha_3\]

Offensichtlich ist diese Bedingung nicht erfüllbar. Daher ist ist die Selbstapplikation
nicht typisierbar. Dies würde im Unifikationsalgorithmus 
(vgl. \cref{alg:klassischer-unifikationsalgorithmus})
durch den \textit{occur check} festgestellt werden.