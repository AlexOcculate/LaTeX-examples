%!TEX root = Programmierparadigmen.tex
\chapter{C}
\index{C|(}
C ist eine imperative Programmiersprache. Sie wurde in vielen Standards
definiert. Die wichtigsten davon sind:\todo{Wo sind unterschiede?}

\begin{itemize}
    \item C89
    \item C99
    \item ANSI C
    \item C11
\end{itemize}

\section{Datentypen}\xindex{Datentypen}
Die grundlegenden C-Datentypen sind
\begin{table}[htp]
    \centering
    \begin{tabular}{|l|l|}
    \hline
    \textbf{Typ}    & \textbf{Größe}   \\ \hline\hline
    char   & 1 Byte  \\ \hline
    int    & 4 Bytes \\ \hline
    float  & 4 Bytes \\ \hline
    double & 8 Bytes \\ \hline
    void   & 0 Bytes \\ \hline
    \end{tabular}
\end{table}

zusätzlich kann man \texttt{char}\xindex{char} und \texttt{int}\xindex{int}
noch in \texttt{signed}\xindex{signed} und \texttt{unsigned}\xindex{unsigned}
unterscheiden.

\section{ASCII-Tabelle}\label{sec:ascii-tabelle}
\begin{table}[htp]
    \centering
    \begin{tabular}{|l|l||l|l||l|l||l|l|}
    \hline
    \textbf{Dez.} & \textbf{Z.} & \textbf{Dez.} & \textbf{Z.} & \textbf{Dez.} & \textbf{Z.} & \textbf{Dez.} & \textbf{Z.} \\ \hline\hline
    0    & ~       & 31   & ~       & 64   & @       & 96   & '       \\ \hline
    1    & ~       & ~    & ~       & 65   & A       & 97   & a       \\ \hline
    2    & ~       & ~    & ~       & 66   & B       & 98   & b       \\ \hline
    3    & ~       & ~    & ~       & ~    & C       & 99   & c       \\ \hline
    4    & ~       & ~    & ~       & ~    & D       & 100  & d       \\ \hline
    5    & ~       & ~    & ~       & ~    & E       & ~    & ~       \\ \hline
    6    & ~       & ~    & ~       & ~    & F       & ~    & ~       \\ \hline
    7    & ~       & ~    & ~       & ~    & G       & ~    & ~       \\ \hline
    8    & ~       & ~    & ~       & ~    & H       & ~    & ~       \\ \hline
    9    & ~       & ~    & ~       & ~    & I       & ~    & ~       \\ \hline
    10   & ~       & ~    & ~       & ~    & ~       & ~    & ~       \\ \hline
    11   & ~       & ~    & ~       & ~    & ~       & ~    & ~       \\ \hline
    12   & ~       & ~    & ~       & ~    & ~       & ~    & ~       \\ \hline
    13   & ~       & ~    & ~       & ~    & ~       & ~    & ~       \\ \hline
    14   & ~       & ~    & ~       & ~    & ~       & ~    & ~       \\ \hline
    15   & ~       & ~    & ~       & ~    & ~       & ~    & ~       \\ \hline
    16   & ~       & ~    & ~       & ~    & ~       & ~    & ~       \\ \hline
    17   & ~       & ~    & ~       & ~    & ~       & ~    & ~       \\ \hline
    18   & ~       & ~    & ~       & ~    & ~       & ~    & ~       \\ \hline
    19   & ~       & ~    & ~       & ~    & ~       & ~    & ~       \\ \hline
    20   & ~       & ~    & ~       & ~    & ~       & ~    & ~       \\ \hline
    21   & ~       & ~    & ~       & ~    & ~       & ~    & ~       \\ \hline
    22   & ~       & ~    & ~       & ~    & ~       & ~    & ~       \\ \hline
    23   & ~       & ~    & ~       & ~    & ~       & ~    & ~       \\ \hline
    24   & ~       & ~    & ~       & ~    & ~       & ~    & ~       \\ \hline
    25   & ~       & ~    & ~       & ~    & ~       & ~    & ~       \\ \hline
    26   & ~       & ~    & ~       & ~    & ~       & ~    & ~       \\ \hline
    27   & ~       & ~    & ~       & ~    & ~       & ~    & ~       \\ \hline
    28   & ~       & ~    & ~       & ~    & ~       & ~    & ~       \\ \hline
    29   & ~       & ~    & ~       & ~    & ~       & ~    & ~       \\ \hline
    31   & ~       & ~    & ~       & ~    & ~       & 127  & ~       \\ \hline\hline
    \end{tabular}
\end{table}

\section{Syntax}
\section{Beispiele}
\subsection{Hello World}
Speichere den folgenden Text als \texttt{hello-world.c}:

\inputminted[linenos, numbersep=5pt, tabsize=4, frame=lines, label=hello-world.c]{c}{scripts/c/hello-world.c}

Compiliere ihn mit \texttt{gcc hello-world.c}. Es wird eine ausführbare
Datei namens \texttt{a.out} erzeugt.

\index{C|)}

