\chapter{Haskell}
\index{Haskell|(}
Haskell ist eine funktionale Programmiersprache, die von Haskell 
Brooks Curry entwickelt wurde und 1990 in Version~1.0 veröffentlicht 
wurde.

Wichtige Konzepte sind:
\begin{enumerate}
    \item Funktionen höherer Ordnung
    \item anonyme Funktionen (sog. Lambda-Funktionen)
    \item Pattern Matching
    \item Unterversorgung
    \item Typinferenz
\end{enumerate}

Haskell kann mit \enquote{Glasgow Haskell Compiler} mittels 
\texttt{ghci} interpretiert und mittels 

\section{Erste Schritte}
Haskell kann unter \href{http://www.haskell.org/platform/}{\path{www.haskell.org/platform/}}
für alle Plattformen heruntergeladen werden. Unter Debian-Systemen
ist das Paket \texttt{ghc} bzw. \texttt{haskell-platform} relevant.

\section{Typen}
Siehe \cref{fig:haskell-type-hierarchy}:

\begin{figure}[htp]
    \centering
    \resizebox{0.9\linewidth}{!}{\documentclass[varwidth=false, border=2pt]{standalone}
\usepackage{tikz}
\usetikzlibrary{shapes}

\begin{document}
\begin{tikzpicture}
    \tikzstyle{node}=[ellipse,thick,fill=white,draw=black,inner sep=0pt,text width=3cm,align=center]
    \tikzstyle{edge}=[->, ultra thick]
    \matrix[row sep=0.5cm,column sep=0.5cm] {
    \node[node] (Eq)         {\textbf{Eq}\\All except IO, (->)};   &
    \node[node] (Show)       {\textbf{Show}\\All except IO, (->)}; &
    \node[node] (Read)       {\textbf{Read}\\All except IO, (->)}; \\
    \node[node] (Ord)        {\textbf{Ord}\\All except IO, (->), IOError};   &
    \node[node] (Num)        {\textbf{Num}\\Int, Integer, Float, Double}; &
    \node[node] (Bounded)    {\textbf{Bounded}\\Int, Char, Bool, (), Ordering, tuples}; \\
    \node[node] (Enum)       {\textbf{Enum}\\{\small (), Bool, Char, Ordering, Int, Integer, Float, Double}};   &
    \node[node] (Real)       {\textbf{Real}\\Int, Integer, Float, Double}; &
    \node[node] (Fractional) {\textbf{Fractional}\\Float, Double}; \\
    \node[node] (Integral)   {\textbf{Integral}\\Int, Integer};   &
    \node[node] (RealFrac)   {\textbf{RealFrac}\\Float, Double}; &
    \node[node] (Floating)   {\textbf{Floating}\\Float, Double}; \\
    \node[node] (Monad)      {\textbf{Monad}\\IO, (), Maybe};   &
    \node[node] (RealFloat)  {\textbf{RealFloat}\\Float, Double}; &
     \\
    \node[node] (MonadPlus)  {\textbf{MonadPlus}\\IO, (), Maybe};   &
    \node[node] (Functor)    {\textbf{Functor}\\IO, (), Maybe}; &
     \\
    }; 
    \draw[edge] (Eq) -- (Ord);
    \draw[edge] (Eq) -- (Num);
    \draw[edge] (Show) -- (Num);
    \draw[edge] (Ord) -- (Real);
    \draw[edge] (Num) -- (Real);
    \draw[edge] (Num) -- (Fractional);
    \draw[edge] (Enum) -- (Integral);
    \draw[edge] (Real) -- (Integral);
    \draw[edge] (Real) -- (RealFrac);
    \draw[edge] (Floating) -- (RealFloat);
    \draw[edge] (RealFrac) -- (RealFloat);
    \draw[edge] (Monad) -- (MonadPlus);
\end{tikzpicture}
\end{document}
}
    \caption{Hierarchie der Haskell Standardklassen}
    \label{fig:haskell-type-hierarchy}
\end{figure}

\section{Syntax}
\subsection{Klammern}
Haskell verzichtet an vielen Stellen auf Klammern. So werden im
Folgenden die Funktionen $f(x) := \frac{\sin x}{x}$ und $g(x) := x \cdot f(x^2)$
definiert:

\inputminted[numbersep=5pt, tabsize=4]{haskell}{scripts/haskell/einfaches-beispiel-klammern.hs}

\subsection{if / else}
Das folgende Beispiel definiert den Binomialkoeffizienten (vgl. \cref{bsp:binomialkoeffizient})

\inputminted[numbersep=5pt, tabsize=4]{haskell}{scripts/haskell/binomialkoeffizient.hs}
\inputminted[numbersep=5pt, tabsize=4]{bash}{scripts/haskell/compile-binom.sh}

\todo[inline]{Guards}

\subsection{Rekursion}
Die Fakultätsfunktion wurde wie folgt implementiert:
\[fak(n) := \begin{cases}
        1              &\text{falls } n=0\\
        n \cdot fak(n) &\text{sonst}
    \end{cases}\]
\inputminted[numbersep=5pt, tabsize=4]{haskell}{scripts/haskell/fakultaet.hs}

Diese Implementierung benötigt $\mathcal{O}(n)$ rekursive Aufrufe und
hat einen Speicherverbrauch von $\mathcal{O}(n)$. Durch einen
\textbf{Akkumulator}\xindex{Akkumulator} kann dies verhindert werden:
\inputminted[numbersep=5pt, tabsize=4]{haskell}{scripts/haskell/fakultaet-akkumulator.hs}

\subsection{Listen}
\todo[inline]{Cons-Operator, Unendliche Listen}

\subsubsection{Beispiel in der interaktiven Konsole}
\inputminted[numbersep=5pt, tabsize=4]{haskell}{scripts/haskell/listenoperationen.sh}

\section{Beispiele}
\subsection{Hello World}
Speichere folgenden Quelltext als \texttt{hello-world.hs}:
\inputminted[linenos, numbersep=5pt, tabsize=4, frame=lines, label=hello-world.hs]{haskell}{scripts/haskell/hello-world.hs}

Kompiliere ihn mit \texttt{ghc -o hello hello-world.hs}. Es wird eine
ausführbare Datei erzeugt.

\subsection{Fibonacci}
\inputminted[linenos, numbersep=5pt, tabsize=4, frame=lines, label=fibonacci.hs]{haskell}{scripts/haskell/fibonacci.hs}

\subsection{Quicksort}

\section{Weitere Informationen}
\begin{itemize}
    \item \href{http://www.haskell.org/hoogle/}{\path{haskell.org/hoogle}}: Suchmaschine für das Haskell-Manual
    \item \href{http://wiki.ubuntuusers.de/Haskell}{\path{wiki.ubuntuusers.de/Haskell}}: Hinweise zur Installation von Haskell unter Ubuntu
\end{itemize}

\index{Haskell|)}

