\chapter{Haskell}
\index{Haskell|(}
Haskell ist eine funktionale Programmiersprache, die von Haskell 
Brooks Curry entwickelt und 1990 in Version~1.0 veröffentlicht 
wurde.

Wichtige Konzepte sind:
\begin{enumerate}
    \item Funktionen höherer Ordnung
    \item anonyme Funktionen (sog. Lambda-Funktionen)
    \item Pattern Matching
    \item Unterversorgung
    \item Typinferenz
\end{enumerate}

Haskell kann mit \enquote{Glasgow Haskell Compiler} mittels 
\texttt{ghci} interpretiert und mittels 

\section{Erste Schritte}
Haskell kann unter \href{http://www.haskell.org/platform/}{\path{www.haskell.org/platform/}}
für alle Plattformen heruntergeladen werden. Unter Debian-Systemen
ist das Paket \texttt{ghc} bzw. \texttt{haskell-platform} relevant.

\subsection{Hello World}
Speichere folgenden Quelltext als \texttt{hello-world.hs}:
\inputminted[linenos, numbersep=5pt, tabsize=4, frame=lines, label=hello-world.hs]{haskell}{scripts/haskell/hello-world.hs}

Kompiliere ihn mit \texttt{ghc -o hello hello-world.hs}. Es wird eine
ausführbare Datei erzeugt.

\section{Syntax}
\subsection{Klammern und Funktionsdeklaration}
Haskell verzichtet an vielen Stellen auf Klammern. So werden im
Folgenden die Funktionen $f(x) := \frac{\sin x}{x}$ und $g(x) := x \cdot f(x^2)$
definiert:

\inputminted[numbersep=5pt, tabsize=4]{haskell}{scripts/haskell/einfaches-beispiel-klammern.hs}

Die Funktionsdeklarationen mit den Typen sind nicht notwendig, da 
die Typen aus den benutzten Funktionen abgeleitet werden.

Zu lesen ist die Deklaration wie folgt:

\begin{center}
\texttt{[Funktionsname] :: \texttt{[Typendefinitionen]} => \texttt{Signatur}}
\end{center}

\begin{itemize}
    \item[T. Def.] Die Funktion \texttt{f} benutzt als Parameter bzw. Rückgabewert
          einen Typen. Diesen Typen nennen wir \texttt{a} und er ist
          vom Typ \texttt{Floating}. Auch \texttt{b}, \texttt{wasweisich}
          oder etwas ähnliches wäre ok.
    \item[Signatur] Die Signatur liest man am einfachsten von hinten:
        \begin{itemize}
            \item \texttt{f} bildet auf einen Wert vom Typ \texttt{a} ab und
            \item \texttt{f} hat genau einen Parameter \texttt{a}
        \end{itemize}
\end{itemize}

\todo[inline]{Gibt es Funktionsdeklarationen, die äquivalent? (bis auf wechsel des namens und der Reihenfolge)}

\subsection{if / else}
Das folgende Beispiel definiert den Binomialkoeffizienten (vgl. \cref{bsp:binomialkoeffizient})

\inputminted[numbersep=5pt, tabsize=4]{haskell}{scripts/haskell/binomialkoeffizient.hs}
\inputminted[numbersep=5pt, tabsize=4]{bash}{scripts/haskell/compile-binom.sh}

\todo[inline]{Guards}

\subsection{Rekursion}
Die Fakultätsfunktion wurde wie folgt implementiert:
\[fak(n) := \begin{cases}
        1              &\text{falls } n=0\\
        n \cdot fak(n) &\text{sonst}
    \end{cases}\]
\inputminted[numbersep=5pt, tabsize=4]{haskell}{scripts/haskell/fakultaet.hs}

Diese Implementierung benötigt $\mathcal{O}(n)$ rekursive Aufrufe und
hat einen Speicherverbrauch von $\mathcal{O}(n)$. Durch einen
\textbf{Akkumulator}\xindex{Akkumulator} kann dies verhindert werden:
\inputminted[numbersep=5pt, tabsize=4]{haskell}{scripts/haskell/fakultaet-akkumulator.hs}

\subsection{Listen}
\begin{itemize}
    \item \texttt{[]} erzeugt die leere Liste,
    \item \texttt{[1,2,3]} erzeugt eine Liste mit den Elementen $1, 2, 3$
    \item \texttt{:} wird \textbf{cons}\xindex{cons} genannt und ist
          der Listenkonstruktor.
    \item \texttt{head list} gibt den Kopf von \texttt{list} zurück,
          \texttt{tail list} den Rest:
          \inputminted[numbersep=5pt, tabsize=4]{haskell}{scripts/haskell/list-basic.sh}
    \item \texttt{null list} prüft, ob \texttt{list} leer ist.
    \item \texttt{length list} gibt die Anzahl der Elemente in \texttt{list} zurück.
    \item \texttt{maximum [1,9,1,3]} gibt 9 zurück (analog: \texttt{minimum}).
    \item \texttt{last [1,9,1,3]} gibt 3 zurück.
    \item \texttt{reverse [1,9,1,3]} gibt \texttt{[3,1,9,1]} zurück.
    \item \texttt{elem item list} gibt zurück, ob sich \texttt{item} in \texttt{list} befindet.
\end{itemize}

\subsubsection{Beispiel in der interaktiven Konsole}
\inputminted[numbersep=5pt, tabsize=4]{haskell}{scripts/haskell/listenoperationen.sh}

\subsubsection{List-Comprehensions}\xindex{List-Comprehension}
List-Comprehensions sind kurzschreibweisen für Listen, die sich an 
der Mengenschreibweise in der Mathematik orientieren. So entspricht
die Menge
\begin{align*}
    myList &= \Set{1,2,3,4,5,6}\\
    test   &= \Set{x \in myList | x > 2}
\end{align*}
in etwa folgendem Haskell-Code:
\inputminted[numbersep=5pt, tabsize=4]{haskell}{scripts/haskell/list-comprehensions.sh}

\subsection{Strings}
\begin{itemize}
    \item Strings sind Listen von Zeichen:\\
          \texttt{tail "ABCDEF"} gibt \texttt{"BCDEF"} zurück.
\end{itemize}

\section{Typen}
In Haskell werden Typen aus den Operationen geschlossfolgert. Dieses
Schlussfolgern der Typen, die nicht explizit angegeben werden müssen,
nennt man \textbf{Typinferent}\xindex{Typinferenz}.


Haskell kennt die Typen aus \cref{fig:haskell-type-hierarchy}.

Ein paar Beispiele zur Typinferenz:
\inputminted[numbersep=5pt, tabsize=4]{haskell}{scripts/haskell/typinferenz.sh}

\begin{figure}[htp]
    \centering
    \resizebox{0.9\linewidth}{!}{\documentclass[varwidth=false, border=2pt]{standalone}
\usepackage{tikz}
\usetikzlibrary{shapes}

\begin{document}
\begin{tikzpicture}
    \tikzstyle{node}=[ellipse,thick,fill=white,draw=black,inner sep=0pt,text width=3cm,align=center]
    \tikzstyle{edge}=[->, ultra thick]
    \matrix[row sep=0.5cm,column sep=0.5cm] {
    \node[node] (Eq)         {\textbf{Eq}\\All except IO, (->)};   &
    \node[node] (Show)       {\textbf{Show}\\All except IO, (->)}; &
    \node[node] (Read)       {\textbf{Read}\\All except IO, (->)}; \\
    \node[node] (Ord)        {\textbf{Ord}\\All except IO, (->), IOError};   &
    \node[node] (Num)        {\textbf{Num}\\Int, Integer, Float, Double}; &
    \node[node] (Bounded)    {\textbf{Bounded}\\Int, Char, Bool, (), Ordering, tuples}; \\
    \node[node] (Enum)       {\textbf{Enum}\\{\small (), Bool, Char, Ordering, Int, Integer, Float, Double}};   &
    \node[node] (Real)       {\textbf{Real}\\Int, Integer, Float, Double}; &
    \node[node] (Fractional) {\textbf{Fractional}\\Float, Double}; \\
    \node[node] (Integral)   {\textbf{Integral}\\Int, Integer};   &
    \node[node] (RealFrac)   {\textbf{RealFrac}\\Float, Double}; &
    \node[node] (Floating)   {\textbf{Floating}\\Float, Double}; \\
    \node[node] (Monad)      {\textbf{Monad}\\IO, (), Maybe};   &
    \node[node] (RealFloat)  {\textbf{RealFloat}\\Float, Double}; &
     \\
    \node[node] (MonadPlus)  {\textbf{MonadPlus}\\IO, (), Maybe};   &
    \node[node] (Functor)    {\textbf{Functor}\\IO, (), Maybe}; &
     \\
    }; 
    \draw[edge] (Eq) -- (Ord);
    \draw[edge] (Eq) -- (Num);
    \draw[edge] (Show) -- (Num);
    \draw[edge] (Ord) -- (Real);
    \draw[edge] (Num) -- (Real);
    \draw[edge] (Num) -- (Fractional);
    \draw[edge] (Enum) -- (Integral);
    \draw[edge] (Real) -- (Integral);
    \draw[edge] (Real) -- (RealFrac);
    \draw[edge] (Floating) -- (RealFloat);
    \draw[edge] (RealFrac) -- (RealFloat);
    \draw[edge] (Monad) -- (MonadPlus);
\end{tikzpicture}
\end{document}
}
    \caption{Hierarchie der Haskell Standardklassen}
    \label{fig:haskell-type-hierarchy}
\end{figure}

\section{Beispiele}
\subsection{Quicksort}
\inputminted[linenos, numbersep=5pt, tabsize=4, frame=lines, label=qsort.hs]{haskell}{scripts/haskell/qsort.hs}

\begin{itemize}
    \item Die leere Liste ergibt sortiert die leere Liste.
    \item Wähle das erste Element \texttt{p} als Pivotelement und
          teile die restliche Liste \texttt{ps} in kleinere und 
          gleiche sowie in größere Elemente mit \texttt{filter} auf.
          Konkateniere diese beiden Listen mit \texttt{++}.
\end{itemize}

Durch das Ausnutzen von Unterversorgung\xindex{Unterversorgung} lässt
sich das ganze sogar noch kürzer schreiben:

\inputminted[linenos, numbersep=5pt, tabsize=4, frame=lines, label=qsort.hs]{haskell}{scripts/haskell/qsort-unterversorg.hs}

\subsection{Fibonacci}\xindex{Fibonacci}
\inputminted[linenos, numbersep=5pt, tabsize=4, frame=lines, label=fibonacci.hs]{haskell}{scripts/haskell/fibonacci.hs}
\inputminted[linenos, numbersep=5pt, tabsize=4, frame=lines, label=fibonacci-akk.hs]{haskell}{scripts/haskell/fibonacci-akk.hs}
\inputminted[linenos, numbersep=5pt, tabsize=4, frame=lines, label=fibonacci-zip.hs]{haskell}{scripts/haskell/fibonacci-zip.hs}
\inputminted[linenos, numbersep=5pt, tabsize=4, frame=lines, label=fibonacci-pattern-matching.hs]{haskell}{scripts/haskell/fibonacci-pattern-matching.hs}

\subsection{Quicksort}
\subsection{Funktionen höherer Ordnung}


\section{Weitere Informationen}
\begin{itemize}
    \item \href{http://www.haskell.org/hoogle/}{\path{haskell.org/hoogle}}: Suchmaschine für das Haskell-Manual
    \item \href{http://wiki.ubuntuusers.de/Haskell}{\path{wiki.ubuntuusers.de/Haskell}}: Hinweise zur Installation von Haskell unter Ubuntu
\end{itemize}

\index{Haskell|)}

