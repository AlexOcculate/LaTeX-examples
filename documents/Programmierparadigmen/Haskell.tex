\chapter{Haskell}
\index{Haskell|(}
Haskell ist eine funktionale Programmiersprache, die von Haskell 
Brooks Curry entwickelt wurde und 1990 in Version~1.0 veröffentlicht 
wurde.

Wichtige Konzepte sind:
\begin{enumerate}
    \item Funktionen höherer Ordnung
    \item anonyme Funktionen (sog. Lambda-Funktionen)
    \item Pattern Matching
    \item Unterversorgung
    \item Typinferenz
\end{enumerate}

Haskell kann mit \enquote{Glasgow Haskell Compiler} mittels 
\texttt{ghci} interpretiert und mittels 

\section{Erste Schritte}
Haskell kann unter \href{http://www.haskell.org/platform/}{\path{www.haskell.org/platform/}}
für alle Plattformen heruntergeladen werden. Unter Debian-Systemen
ist das Paket \texttt{ghc} bzw. \texttt{haskell-platform} relevant.

\section{Syntax}
\subsection{Klammern}
Haskell verzichtet an vielen Stellen auf Klammern. So werden im
Folgenden die Funktionen $f(x) := \frac{\sin x}{x}$ und $g(x) := x \cdot f(x^2)$
definiert:

\inputminted[numbersep=5pt, tabsize=4]{haskell}{scripts/haskell/einfaches-beispiel-klammern.hs}

\subsection{if / else}
Das folgende Beispiel definiert den Binomialkoeffizienten 
\[\binom{n}{k} := \begin{cases}
        1                               &\text{falls } k=0 \lor k = n\\
        \binom{n-1}{k-1}+\binom{n-1}{k} &\text{sonst}
        \end{cases}\]
für $n,k \geq 0$:

\inputminted[numbersep=5pt, tabsize=4]{haskell}{scripts/haskell/binomialkoeffizient.hs}
\inputminted[numbersep=5pt, tabsize=4]{bash}{scripts/haskell/compile-binom.sh}

\todo[inline]{Guards}

\subsection{Rekursion}
Die Fakultätsfunktion wurde wie folgt implementiert:
\[fak(n) := \begin{cases}
        1              &\text{falls } n=0\\
        n \cdot fak(n) &\text{sonst}
    \end{cases}\]
\inputminted[numbersep=5pt, tabsize=4]{haskell}{scripts/haskell/fakultaet.hs}

Diese Implementierung benötigt $\mathcal{O}(n)$ rekursive Aufrufe und
hat einen Speicherverbrauch von $\mathcal{O}(n)$. Durch einen
\textbf{Akkumulator}\xindex{Akkumulator} kann dies verhindert werden:
\inputminted[numbersep=5pt, tabsize=4]{haskell}{scripts/haskell/fakultaet-akkumulator.hs}

\todo[inline]{Endrekursion ... macht für mich unter "Haskell" wenig sinn. Vielleicht einen neuen Abschnitt mit Techniken? Was würde da noch landen?}

\section{Beispiele}
\subsection{Hello World}
Speichere folgenden Quelltext als \texttt{hello-world.hs}:
\inputminted[linenos, numbersep=5pt, tabsize=4, frame=lines, label=hello-world.hs]{haskell}{scripts/haskell/hello-world.hs}

Kompiliere ihn mit \texttt{ghc -o hello hello-world.hs}. Es wird eine
ausführbare Datei erzeugt.

\subsection{Fibonacci}
\inputminted[linenos, numbersep=5pt, tabsize=4, frame=lines, label=fibonacci.hs]{haskell}{scripts/haskell/fibonacci.hs}

\subsection{Quicksort}

\section{Weitere Informationen}
\begin{itemize}
    \item \href{http://www.haskell.org/hoogle/}{\path{haskell.org/hoogle}}: Suchmaschine für das Haskell-Manual
    \item \href{http://wiki.ubuntuusers.de/Haskell}{\path{wiki.ubuntuusers.de/Haskell}}: Hinweise zur Installation von Haskell unter Ubuntu
\end{itemize}

\index{Haskell|)}

