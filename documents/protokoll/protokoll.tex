\documentclass[11pt,a4paper,oneside]{scrartcl}
\usepackage{amssymb, amsmath} % needed for math
\usepackage[utf8]{inputenc} % this is needed for umlauts
\usepackage[ngerman]{babel} % this is needed for umlauts
\usepackage[T1]{fontenc}    % this is needed for correct output of umlauts in pdf
\usepackage[margin=2.5cm]{geometry} %layout
\usepackage{hyperref}   % links im text
\usepackage{enumerate}  % for advanced numbering of lists
\clubpenalty  = 10000   % Schusterjungen verhindern
\widowpenalty = 10000   % Hurenkinder verhindern
\usepackage{fancyheadings}  % Kopfzeile
\usepackage{array}      % needed for m{1cm} in tabular
\usepackage{enumitem}
\usepackage{pdfpages}   % Signatureinbingung
\usepackage{myStyle}    % some shortcuts

%%%%%%%%%%%%%%%%%%%%%%%%%%%%%%%%%%%%%%%%%%%%%%%%%%%%%%%%%%%%%%%%%%%%%
% Change the following lines:                                       %
%%%%%%%%%%%%%%%%%%%%%%%%%%%%%%%%%%%%%%%%%%%%%%%%%%%%%%%%%%%%%%%%%%%%%
\newcommand\yourTitle{UpToDatE: Protokoll}
\newcommand\yourKeywords{Informatik;PSE}
\newcommand\protokollNr{1}
\newcommand\Schriftfuehrer{Martin Thoma}
\newcommand\Ort{Karlsruhe}
\newcommand\Datum{\today}   %change \today to DD. Month YYYY
%%%%%%%%%%%%%%%%%%%%%%%%%%%%%%%%%%%%%%%%%%%%%%%%%%%%%%%%%%%%%%%%%%%%%

\hypersetup { 
  pdfauthor   = {\Schriftfuehrer}, 
  pdfkeywords = {\yourKeywords}, 
  pdftitle    = {\yourTitle} 
}

\pagestyle{fancy}% eigenen Seitestil aktivieren}

\lhead[\thepage]{Protokoll-Nr. \protokollNr{} vom Teamtreffen}
\rhead[\thesection]{UpToDatE}

\cfoot{\thepage}

\newcommand*{\tabbox}[2][t]{%
    \vspace{-15pt}\parbox[#1][3.7\baselineskip]{1cm}{\strut#2\strut}}

\begin{document}
 \author{\Schriftfuehrer}
 \title{\yourTitle}

% \maketitle

%%%%%%%%%%%%%%%%%%%%%%%%%%%%%%%%%%%%%%%%%%%%%%%%%%%%%%%%%%%%%%%%%%%%%
% Start editing your content here                                   %
%%%%%%%%%%%%%%%%%%%%%%%%%%%%%%%%%%%%%%%%%%%%%%%%%%%%%%%%%%%%%%%%%%%%%
\begin{center}
    % See http://en.wikibooks.org/wiki/LaTeX/Tables
    \begin{tabular}{p{4.0cm} p{11.2cm}}
        \textbf{Protokoll-Nr.}      & \protokollNr\\
        \textbf{Datum}              & Dienstag, der 23. Oktober 2012\\
        \textbf{Beginn}             & 15:45 Uhr\\
        \textbf{Ende}               & 17:15 Uhr\\
        \textbf{Anwesende}          & Martin Thoma, Max Mustermann\\
        \textbf{Abwesende}          & -\\
        \textbf{Tagesordnung}       & \vspace{-7mm}%
                \begin{enumerate}[leftmargin=1.3em]%
                    \setlength{\itemsep}{-2pt}% 
                    \item TOP 1
                    \item TOP 2
                    \item TOP 3
                    \item TOP 4
                \end{enumerate}
    \end{tabular}
\end{center}

\section*{TOP 1}
Ist das \href{http://www.digitale-schule-bayern.de/dsdaten/8/825.pdf}{hier}
ein gutes Beispiel?

Fusce libero nulla, euismod vel suscipit nec, elementum vel massa. 
Mauris ut sapien sed neque dignissim sodales. Proin accumsan, lectus 
non gravida dapibus, lorem leo tincidunt odio, in semper ligula 
libero bibendum lorem. Pellentesque venenatis massa a neque porttitor
congue. Maecenas ornare lacus ac orci mattis a placerat sapien 
euismod. In sed eros enim, non interdum nisi. Curabitur quis magna 
et tortor interdum pharetra. Donec sit amet turpis neque, quis congue
leo. Proin sit amet placerat dolor.

\section*{TOP 2}
You can go quite deep into details. You should use chapter, section, 
subsection and subsubsection in this order. No need for fancy bold or
underlined text.

One equation:
\begin{equation}
    \sin^2(x) + \cos^2(x) = 1
\end{equation}

Vivamus ante est, dictum at placerat id, semper auctor tellus. Donec 
\arrow sed ipsum enim, eget lacinia mi. Duis vulputate auctor ligula, sit 
amet suscipit lectus malesuada ut. Donec adipiscing rutrum dolor sit 
amet euismod. Aenean condimentum nibh vitae neque rhoncus ultrices. 
Vestibulum ultrices commodo mattis. Morbi aliquam elementum est, a 
pulvinar arcu viverra quis. Vivamus sed fermentum nisl. Cras 
bibendum, justo tincidunt dictum venenatis, sem turpis vestibulum 
nibh, ut dapibus nunc enim ut justo. 

\section*{TOP 3}
Pellentesque commodo, nisi nec feugiat vehicula, augue erat convallis ipsum, adipiscing cursus purus dolor venenatis velit. Vivamus enim augue, lacinia in ultrices a, varius sit amet sapien. Aliquam enim velit, molestie vitae bibendum eget, laoreet at ante. Suspendisse pulvinar leo at nisi accumsan nec malesuada neque ullamcorper. Aenean quis mi lectus, quis porttitor nisi. Curabitur interdum luctus lectus et egestas. Aenean sapien ligula, aliquam sed fermentum id, blandit at orci. Integer a turpis ac tellus commodo suscipit. Vivamus massa orci, pharetra eu consequat eu, vulputate eget lacus. Suspendisse non justo arcu. Nullam lacus augue, dapibus vitae convallis a, consectetur at elit.

Pellentesque commodo, nisi nec feugiat vehicula, augue erat convallis ipsum, adipiscing cursus purus dolor venenatis velit. Vivamus enim augue, lacinia in ultrices a, varius sit amet sapien. Aliquam enim velit, molestie vitae bibendum eget, laoreet at ante. Suspendisse pulvinar leo at nisi accumsan nec malesuada neque ullamcorper. Aenean quis mi lectus, quis porttitor nisi. Curabitur interdum luctus lectus et egestas. Aenean sapien ligula, aliquam sed fermentum id, blandit at orci. Integer a turpis ac tellus commodo suscipit. Vivamus massa orci, pharetra eu consequat eu, vulputate eget lacus. Suspendisse non justo arcu. Nullam lacus augue, dapibus vitae convallis a, consectetur at elit.
\\\nopagebreak
\\\nopagebreak
\noindent \Ort, den \Datum\\\nopagebreak
\\\nopagebreak
\IfFileExists{signature.pdf}{ % if you have your signature as a digital version, save it as "signature.pdf" in this folder
    \includegraphics[height=10mm]{signature.pdf}\\\nopagebreak
}{
    \\\nopagebreak
}

\noindent \Schriftfuehrer,\\\nopagebreak
Schriftführer

\end{document}
