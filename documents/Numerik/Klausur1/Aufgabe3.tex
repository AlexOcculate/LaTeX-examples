\section*{Aufgabe 3}
Die Jacobi-Matrix von $f$ lautet:
\[f' (x,y) = \begin{pmatrix}
	3     & \cos y\\
	3 x^2 & e^y
\end{pmatrix}\]
Hierfür wurde in in der ersten Spalte nach $x$ abgeleitet und in der
zweiten Spalte nach $y$.

\subsection*{Lösungsvorschlag 1 (Numerische Lösung)}
Eine Iteration des Newton-Verfahren ist durch
\begin{align}
x_{k+1}&=x_{k}-f'(x_k)^{-1}\cdot f(x_k)
\end{align}
gegeben (vgl. Skript, S. 35).

Zur praktischen Durchführung lösen wir
\begin{align}
    f'(x_0, y_0)\Delta x &= -f(x_0,y_0)\\
    L \cdot \underbrace{R \cdot \Delta x}_{=: c} &= -f(x_0, y_0)
\end{align}
mit Hilfe der LR Zerlegung nach $\Delta x$ auf:

\begin{align}
%
	f'(x_0,y_0)	&= L \cdot R \\
	\Leftrightarrow f'(-\nicefrac{1}{3}, 0)	&= L \cdot R \\
	\Leftrightarrow \begin{pmatrix}
		3     & 1\\
		\frac{1}{3} & 1
	\end{pmatrix}
	&=
	\overbrace{\begin{pmatrix}
		1      & 0\\
		\frac{1}{9} & 1
	\end{pmatrix}}^{=: L} \cdot 
	\overbrace{\begin{pmatrix}
		3 & 1\\
		0      & \frac{8}{9}
	\end{pmatrix}}^{=: R}\\
%
	L \cdot c	&= -f(x_0,y_0) \\
	\Leftrightarrow
	\begin{pmatrix}
		1      & 0\\
		\frac{1}{9} & 1
	\end{pmatrix}
	\cdot c
	&= -
		\begin{pmatrix}
		2\\
		\frac{26}{27}
	\end{pmatrix}\\
	\Rightarrow
	c &=		\begin{pmatrix}
		-2\\
		-\frac{20}{27}
	\end{pmatrix}\footnotemark\\
%
	R\cdot \Delta x &= c\\
	\Leftrightarrow
	\begin{pmatrix}
		3 & 1\\
		0      & \frac{8}{9}
	\end{pmatrix}
	\cdot \Delta x &=
	\begin{pmatrix}
		-2\\
		-\frac{20}{27}
	\end{pmatrix}\\
	\Rightarrow \Delta x &= \frac{1}{18}
	\begin{pmatrix}
		-7\\
		-15
	\end{pmatrix}
\end{align}
\footnotetext{Dieser Schritt wird durch Vorwärtssubsitution berechnet.}

Anschließend berechnen wir
\begin{align}
	\begin{pmatrix}
		x_1\\
		y_1
	\end{pmatrix} &= 
	\begin{pmatrix}
		x_0\\
		y_0
	\end{pmatrix}+\Delta x \\
	\Leftrightarrow\begin{pmatrix}
		x_1\\
		y_1
	\end{pmatrix} &= 
	\begin{pmatrix}
		-\frac{1}{3}\\
		0
	\end{pmatrix} +
    \frac{1}{18}
	\begin{pmatrix}
		-7\\
		-15
	\end{pmatrix} \\
	\Leftrightarrow\begin{pmatrix}
		x_1\\
		y_1
	\end{pmatrix} &= 
	\begin{pmatrix}
		-\nicefrac{13}{18}\\
		-\nicefrac{15}{18}
	\end{pmatrix}
\end{align}


\subsection*{Lösungsvorschlag 2 (Analytische Lösung)}
Und jetzt die Berechnung %TODO: Was ist hiermit gemeint?

\[f'(x, y) \cdot (x_0, y_0) = f(x,y)\] %TODO: Was ist hiermit gemeint?

LR-Zerlegung für $f'(x, y)$ kann durch scharfes hinsehen durchgeführt
werden, da es in $L$ nur eine unbekannte (links unten) gibt. Es gilt
also ausführlich:

\begin{align}
	\begin{pmatrix}
		3     & \cos y\\
		3 x^2 & e^y
	\end{pmatrix}
	&=
	\overbrace{\begin{pmatrix}
		1      & 0\\
		l_{12} & 1
	\end{pmatrix}}^L \cdot 
	\overbrace{\begin{pmatrix}
		r_{11} & r_{12}\\
		0      & r_{22}
	\end{pmatrix}}^R\\
	\Rightarrow r_{11} &= 3\\
	\Rightarrow r_{12} &= \cos y\\
	\Rightarrow \begin{pmatrix}
		3     & \cos y\\
		3 x^2 & e^y
	\end{pmatrix}
	&=
	\begin{pmatrix}
		1      & 0\\
		l_{12} & 1
	\end{pmatrix} \cdot 
	\begin{pmatrix}
		3 & \cos y\\
		0 & r_{22}
	\end{pmatrix}\\
	\Rightarrow 3x^2 &\stackrel{!}{=} l_{12} \cdot 3 + 1 \cdot 0\\
	\Leftrightarrow l_{12} &= x^2\\
	\Rightarrow e^y &\stackrel{!}{=} x^2 \cdot \cos y + 1 \cdot r_{22}\\
	\Leftrightarrow r_{22} &= -x^2 \cdot \cos y + e^y\\
	\Rightarrow \begin{pmatrix}
		3     & \cos y\\
		3 x^2 & e^y
	\end{pmatrix}
	&=
	\begin{pmatrix}
		1   & 0\\
		x^2 & 1
	\end{pmatrix} \cdot 
	\begin{pmatrix}
		3 & \cos y\\
		0 & -x^2 \cdot \cos y + e^y
	\end{pmatrix}\\
	P &= I_2
\end{align}

TODO: Eigentlich sollten sich ab hier die Lösungsvorschläge gleichen\dots

Es folgt:
\begin{align}
-f ( \nicefrac{-1}{3}, 0) &= \begin{pmatrix} -2\\ -\frac{26}{27}\end{pmatrix}\\
c &= \begin{pmatrix} 2\\ \nicefrac{82}{27} \end{pmatrix}\\ %TODO: Was ist c?
(x_1, y_1) &= \begin{pmatrix} \nicefrac{5}{3}\\ \nicefrac{82}{27}\end{pmatrix}
\end{align}
