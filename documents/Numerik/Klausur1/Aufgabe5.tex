\section*{Aufgabe 5}
\subsection*{Teilaufgabe a}
Eine Quadraturformel $(b_i, c_i)_{i=1, \dots, s}$ hat die Ordnung
$p$, falls sie exakte Lösungen für alle Polynome vom Grad $\leq p -1$
liefert.

\subsection*{Teilaufgabe b}
\[\sum_{i=1}^s b_i c_i^{q-1} = \frac{1}{q} \text{ für } q = 1, \dots, p\]

\subsection*{Teilaufgabe c}
\paragraph{Aufgabe} Bestimmen Sie zu den Knoten $c_1 = 0$ und $c_2 = \frac{2}{3}$ Gewichte, um eine Quadraturformel
maximaler Ordnung zu erhalten. Wie hoch ist die Ordnung?

\paragraph{Lösung}

$b_1 = \frac{1}{4}$ und $b_2 = \frac{3}{4}$ erreichen laut Felix Ordnung 3.

Damit is

Die möglichen Quadraturformeln lauten:
\begin{align}
	Q(f) &= (b-a)\sum_{i=1}^2 b_i f (a+ c_i (b-a))\\
		 &= (b-a) \cdot \left ( b_1 f(a) + b_2 f \left (a + \frac{2}{3}(b-a) \right ) \right )
\end{align}

$\stackrel{\text{Satz 28}}{\Rightarrow}$ Wenn wir Ordnung $s = 2$ fordern, sind die Gewichte eindeutig bestimmt.
