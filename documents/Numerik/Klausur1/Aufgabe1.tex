\section*{Aufgabe 1}
\subsection*{Teilaufgabe a}
\textbf{Gegeben:}

\[A = 
\begin{pmatrix}
    3 & 15 & 13 \\
    6 & 6  & 6  \\
    2 & 8  & 19
\end{pmatrix}\]

\textbf{Aufgabe:} LR-Zerlegung von $A$ mit Spaltenpivotwahl

\textbf{Lösung:} 

\[P = 
\begin{pmatrix}
    0 & 1 & 0 \\
    1 & 0 & 0  \\
    0 & 0 & 1
\end{pmatrix}\]

durch scharfes hinsehen.

Nun $L, R$ berechnen:

\begin{align}
	&\begin{gmatrix}[p]
		6 & 6  & 6  \\
		3 & 15 & 13 \\
		2 & 8  & 19
	 \rowops
	 \add[\cdot (-\frac{1}{2})]{0}{1}
	 \add[\cdot (-\frac{1}{3})]{0}{2}
	\end{gmatrix}
	\\
  = \begin{pmatrix}
		          1 & 0 & 0 \\
	   -\frac{1}{2} & 1 & 0  \\
	   -\frac{1}{3} & 0 & 1
	\end{pmatrix} \cdot
	&\begin{gmatrix}[p]
		6 & 6  & 6  \\
		0 & 12 & 10 \\
		0 & 6  & 17
	 \rowops
	 \add[\cdot (-\frac{1}{2})]{1}{2}
	\end{gmatrix}
	\\
  = \begin{pmatrix}
		          1 & 0 & 0 \\
	              0 & 1 & 0  \\
	              0 & -\frac{1}{2} & 1
	\end{pmatrix} \cdot
    \begin{pmatrix}
		          1 & 0 & 0 \\
	   -\frac{1}{2} & 1 & 0  \\
	   -\frac{1}{3} & 0 & 1
	\end{pmatrix} \cdot
	&\begin{gmatrix}[p]
		6 & 6  & 6  \\
		0 & 12 & 10 \\
		0 & 0  & 12
	 \colops
	 \add[\cdot (-1)]{0}{1}
	 \add[\cdot (-1)]{0}{2}
	\end{gmatrix}
	\\
  = \begin{pmatrix}
		          1 & 0 & 0 \\
	   -\frac{1}{2} & 1 & 0  \\
	   -\frac{1}{12} & - \frac{1}{2} & 1
	\end{pmatrix} \cdot
	&\begin{gmatrix}[p]
		6 & 0  & 0  \\
		0 & 12 & 10 \\
		0 & 0  & 12
	 \colops
	 \add[\cdot (-\frac{10}{12})]{1}{2}
	\end{gmatrix}
	\cdot
	\begin{pmatrix}
		          1 & -1 & -1 \\
	              0 &  1 &  0  \\
	              0 &  0 &  1
	\end{pmatrix}
	\\
  = \begin{pmatrix}
		          1 & 0 & 0 \\
	   -\frac{1}{2} & 1 & 0  \\
	   -\frac{1}{12} & - \frac{1}{2} & 1
	\end{pmatrix} \cdot
	&\begin{gmatrix}[p]
		6 & 0  & 0 \\
		0 & 12 & 0 \\
		0 & 0  & 12
	 \colops
	  	\mult{0}{\cdot \frac{1}{6}}
	  	\mult{1}{\cdot \frac{1}{12}}
	  	\mult{2}{\cdot \frac{1}{12}}
	\end{gmatrix}
	\cdot
	\begin{pmatrix}
		          1 & -1 & -1 \\
	              0 &  1 &  0 \\
	              0 &  0 &  1
	\end{pmatrix}
	\cdot
	\begin{pmatrix}
		          1 &  0 &  0 \\
	              0 &  1 &  -\frac{10}{12} \\
	              0 &  0 &  1
	\end{pmatrix}
	\\
  = \begin{pmatrix}
		          1 & 0 & 0 \\
	   -\frac{1}{2} & 1 & 0  \\
	   -\frac{1}{12} & - \frac{1}{2} & 1
	\end{pmatrix} \cdot
	&\begin{gmatrix}[p]
		1 & 0 & 0 \\
		0 & 1 & 0 \\
		0 & 0 & 1
	\end{gmatrix}
	\cdot
	\begin{pmatrix}
		          1 & -1 & -\frac{1}{6} \\
	              0 &  1 & -\frac{5}{6} \\
	              0 &  0 &  1
	\end{pmatrix}
	\cdot
	\begin{pmatrix}
	    \frac{1}{6} &  0 & 0 \\
	              0 &  \frac{1}{12} & 0 \\
	              0 &  0 & \frac{1}{12}
	\end{pmatrix}
	\\
  = \underbrace{\begin{pmatrix}
		          1 & 0 & 0 \\
	   -\frac{1}{2} & 1 & 0  \\
	   -\frac{1}{12} & - \frac{1}{2} & 1
	\end{pmatrix}}_L \cdot
	&\begin{gmatrix}[p]
		1 & 0 & 0 \\
		0 & 1 & 0 \\
		0 & 0 & 1
	\end{gmatrix}
	\cdot \underbrace{\frac{1}{72}
	\begin{pmatrix}
		          12 & -6 & -1 \\
	               0 &  6 & -5 \\
	               0 &  0 &  6
	\end{pmatrix}}_R
\end{align}

ACHTUNG: Ich habe mich irgendwo verrechnet!
Siehe \href{http://www.wolframalpha.com/input/?i=%7B%7B1%2C0%2C0%7D%2C%7B-1%2F2%2C1%2C0%7D%2C%7B-1%2F12%2C-1%2F2%2C1%7D%7D*%7B%7B12%2C-6%2C-1%7D%2C%7B0%2C6%2C-5%7D%2C%7B0%2C0%2C6%7D%7D}{WolframAlpha}

\subsection*{Teilaufgabe b}

\textbf{Gegeben:}

\[A = 
\begin{pmatrix}
    9 & 4 & 12 \\
    4 & 1  & 4 \\
   12 & 4  & 17
\end{pmatrix}\]

\textbf{Aufgabe:} $A$ auf positive Definitheit untersuchen, ohne Eigenwerte zu berechnen.

\textbf{Lösung:}
Eine Matrix $A \in \mathbb{R}^{n \times n}$ heißt positiv Definit $\dots$
\begin{align*}
  \dots & \Leftrightarrow \forall x \in \mathbb{R}^n: x^T A x > 0\\
	& \Leftrightarrow \text{Alle Eigenwerte sind größer als 0}
\end{align*}

Falls $A$ symmetrisch ist, gilt:
\begin{align*}
 \text{$A$ ist pos. Definit} & \Leftrightarrow \text{alle führenden Hauptminore von $A$ sind positiv}\\
	& \Leftrightarrow \text{es gibt eine Cholesky-Zerlegung $A=GG^T$ mit $G$ ist reguläre untere Dreiecksmatrix}\\
\end{align*}

Mit dem Hauptminor-Kriterium gilt:

\begin{align}
	\det(A_1) &= 9 > 0\\
	\det(A_2) &= 
		\begin{vmatrix}
			9 & 4 \\
			4 & 1 \\
		\end{vmatrix} = 9 - 16 < 0\\
	&\Rightarrow \text{$A$ ist nicht positiv definit}
\end{align}
