\section*{Aufgabe 31}
\subsection*{Gesucht:}
Eine Quadraturformel maximaler Ordnung mit:
\begin{align}
    s   &= 3\\
    c_1 &= 0\\
    c_3 &= 1\\
\end{align}

\subsection*{Lösung:}

Nach Satz 28 können Ordnungen $\geq s = 3$ erreicht werden.

Die Ordnung kann nach Satz 31 höchstens $2s = 6$ sein. Da $c_1 = 0$
ist, kann es jedoch keine Gauß-Quadraturformel sein. Also kann
die Ordnung höchstens $5$ sein.

\subsubsection*{Ordnung 5}

Es gibt mindestens zwei Möglichkeiten, zu zeigen, dass es keine
QF der Ordnung 5 mit den Knoten $c_1 = 0$ und $c_3 = 1$ gibt:
Mit hilfe von Satz 29 oder über die Ordnungsbedingungen.

\paragraph*{Mit Satz 29}

\begin{align}
    M(x) &= (x-c_1) (x-c_2) (x-c_3)\\
      &= x (x-c_2) (x-1)\\
      &= (x^2- x) (x-c_2)\\
      &= x^3 - (1+c_2)x^2 + c_2 x\\
    \int_0^1 M(x) \cdot g(x) \mathrm{d} x &\stackrel{!}{=} 0
\end{align}

Da wir Ordnung $5 = s + 2$ erreichen wollen, muss $g$ ein beliebiges
Polynom vom Grad $\leq 2-1 = 1$ sein. Also:
\begin{align}
    g(x) &= ax + b\\
    M(x) \cdot g(x) &= ax^4 + (b-a-ac_2)x^3 + (ac_2-bc_2-b)x^2 + b c_2 x\\
    \int_0^1 M(x) g(x) \mathrm{d} x &= \frac{a}{5} + \frac{b-a-ac_2}{4} + \frac{ac_2 - bc_2-b}{3} + \frac{b c_2}{2}\\
    &= \frac{a c_2}{12}-\frac{a}{20}+\frac{b c_2}{6}-\frac{b}{12}\\
    0 &\stackrel{!}{=}\frac{a c_2}{12}-\frac{a}{20}+\frac{b c_2}{6}-\frac{b}{12}\\
    \Leftrightarrow 0 &\stackrel{!}{=} 5 a c_2 - 3a + 10 b c_2 - 5 b\\
    \Leftrightarrow -5 a c_2 - 10 b c_2&\stackrel{!}{=}  - 3a - 5 b\\
    \Leftrightarrow 5 a c_2 + 10 b c_2&\stackrel{!}{=}  3a + 5 b\\
    \Leftrightarrow c_2(5 a + 10 b)&\stackrel{!}{=}  3a + 5 b\\
    \Leftrightarrow c_2 &\stackrel{!}{=}  \frac{3a + 5 b}{5 a + 10 b}
\end{align}

Da diese Bedingung für alle $a, b \in \mathbb{R}$ gelten soll, muss
sie auf jeden Fall für $a=1, b=0$ sowie für $a=1, b=1$ gelten. Aber:

\begin{align}
    \frac{2\cdot1+5\cdot0}{5\cdot1+10\cdot0} = \frac{3}{5} &\neq \frac{8}{15} = \frac{3\cdot1+5\cdot1}{5\cdot1+10\cdot1}
\end{align}

Offensichtlich gibt also es kein $c_2$, dass diese Bedingung für jedes $a,b \in \mathbb{R}$ 
erfüllt. Daher kann es keine Quadraturformel der Ordnung $5$ mit den Knoten
$0$ und $1$ geben.

\paragraph*{Mit Ordnungsbedingungen}
Wir kennen $c_1 = 0$ und $c_3=1$, was die Ordnungsbedingungen
sehr vereinfacht:
\begin{align}
    1 &\stackrel{!}{=} b_1 + b_2 + b_3\\
    \nicefrac{1}{2} &\stackrel{!}{=} b_2 \cdot c_2 + b_3 \label{eq:bed2}\\
    \nicefrac{1}{3} &\stackrel{!}{=} b_2 \cdot c_2^2 + b_3 \label{eq:bed3}\\
    \nicefrac{1}{4} &\stackrel{!}{=} b_2 \cdot c_2^3 + b_3\\
    \nicefrac{1}{5} &\stackrel{!}{=} b_2 \cdot c_2^4 + b_3
\end{align}

Aus \ref{eq:bed2} folgt:
\begin{align}
    c_2 &= \frac{\nicefrac{1}{2} - b_3}{b_2}
\end{align}

Und damit:
\begin{align}
    \nicefrac{1}{3} &\stackrel{!}{=} b_2 \cdot \left (\frac{\nicefrac{1}{2} - b_3}{b_2} \right )^2 + b_3\\
                &= \frac{(\nicefrac{1}{2} - b_3)^2}{b_2} + b_3\\
\Leftrightarrow \frac{1}{3} b_2 - b_2 b_3&= (\nicefrac{1}{2} - b_3)^2\\
\Leftrightarrow b_2 (\frac{1}{3} - b_3) &= (\nicefrac{1}{2} - b_3)^2\\
\Leftrightarrow b_2  &= \frac{(\nicefrac{1}{2} - b_3)^2}{\frac{1}{3} - b_3}
\end{align}

Nun könnte man das ganze in die 4. Ordnungsbedinung einsetzen \dots aber ich
glaube nicht, dass das schön wird. Mache das, wer will. 

\subsubsection*{Ordnung 4}
Die Simpson-Regel erfüllt offensichtlich alle Bedinungen und hat
Ordnung 4:

\begin{align}
    c_2 &= \nicefrac{1}{2}\\
    b_1 &= \nicefrac{1}{6}\\
    b_2 &= \nicefrac{4}{6}\\
    b_3 &= \nicefrac{1}{6}
\end{align}

Dass die Simpson-Regel Ordnung 4 hat, lässt sich schnell über
die Ordnungsbedingungen zeigen. 
