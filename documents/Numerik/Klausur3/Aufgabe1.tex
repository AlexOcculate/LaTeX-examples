\section*{Aufgabe 1}
\subsection*{Teilaufgabe a)}
\paragraph{Gegeben:} Sei $A \in \mathbb{R}^{3 \times 3}$.
\paragraph{Gesucht:} Cholesky-Zerlegung $A = L \cdot L^T$
\paragraph{Rechnung:}

Erste Spalte:
\begin{align}
	l_{11} &= \sqrt{a_{11}} \\
	l_{21} &= \frac{a_{21}}{l_{11}}\\
	l_{31} &= \frac{a_{31}}{l_{11}}\\
\end{align}
Zweite Spalte:
\begin{align}
	l_{22} &= \sqrt{a_{22} - {l_{21}}^2}\\
	l_{32} &= \frac{a_{32} -l_{21} \cdot l_{31}}{l_{22}} \\
\end{align}
Dritte Spalte:
\begin{align}
	l_{33} &= \sqrt{a_{33}-{l_{32}^2}-{l_{31}}^2}
\end{align}

\subsection*{Teilaufgabe b)}
\begin{align}
	l_{11} &= 2 \\
	l_{21} &= 1 \\
	l_{31} &= -2 \\
	l_{22} &= 3 \\
	l_{32} &= 1 \\
	l_{33} &= 1 \\
\end{align}
Die restlichen Einträge sind $0$. ($L$ ist immer eine untere Dreiecksmatrix)

\subsection*{Teilaufgabe c)}

\begin{align}
	A \cdot x = b \Leftrightarrow L \cdot L^T \cdot x = b \\
	L \cdot c = b \label{a1}
\end{align}
Löse \ref{a1} mit Vorwärtssubstitution.
\begin{align}
	L^T \cdot x = c \label{a2}
\end{align}
Löse \ref{a2} mit Rückwärtssubstitution.
\begin{align}
	x_3 &= 3 \\
	x_2 &= 1 \\
	x_1 &= 2
\end{align}
