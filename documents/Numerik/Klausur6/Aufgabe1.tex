\section*{Aufgabe 1}
\textbf{Gegeben:}

\[
A = \begin{pmatrix}
    1 & 2 & 3\\
    2 & 8 & 14\\
    3 & 14 & 34
\end{pmatrix}\]

\textbf{Aufgabe:} Durch Gauß-Elimination die Cholesky-Zerlegung $A = \overline{L} \overline{L}^T$
berechnen

\textbf{Lösung mit Gauß-Elimination:}
\begin{align*}
    A &=
	\begin{gmatrix}[p]
        1 & 2 & 3\\
        2 & 8 & 14\\
        3 & 14 & 34
        \rowops
        \add[\cdot (-2)]{0}{1}
        \add[\cdot (-3)]{0}{2}
    \end{gmatrix}\\
    \leadsto
    L^{(1)} &=
    \begin{pmatrix}
		1 & 0 & 0\\
	   -2 & 1 & 0\\
       -3 & 0 & 1
	\end{pmatrix},&
    A^{(1)} &=
	\begin{gmatrix}[p]
        1 & 2 & 3\\
        0 & 4 & 8\\
        0 & 8 & 25
        \rowops
        \add[\cdot (-2)]{1}{2}
    \end{gmatrix}\\
    \leadsto
    L^{(2)} &=
    \begin{pmatrix}
		1 & 0 & 0\\
	    0 & 1 & 0\\
        0 & -2 & 1
	\end{pmatrix},&
    A^{(2)} &=
	\begin{gmatrix}[p]
        1 & 2 & 3\\
        0 & 4 & 8\\
        0 & 0 & 9
        \rowops
        \add[\cdot (-2)]{1}{2}
    \end{gmatrix}\\
\end{align*}

TODO: Und wie gehts weiter?


\textbf{Lösung ohne Gauß-Elimination:}
\[
    A = 
    \underbrace{
	\begin{pmatrix}
        1 & 0 & 0\\
        2 & 2 & 0\\
        3 & 4 & 3
    \end{pmatrix}}_{=: L} \cdot \underbrace{\begin{pmatrix}
        1 & 2 & 3\\
        0 & 2 & 4\\
        0 & 0 & 3
    \end{pmatrix}}_{=: L^T}
\]
