\section*{Aufgabe 1}
\textbf{Gegeben:}

\[
A = \begin{pmatrix}
    1 & 2 & 3\\
    2 & 8 & 14\\
    3 & 14 & 34
\end{pmatrix}\]

\textbf{Aufgabe:} Durch Gauß-Elimination die Cholesky-Zerlegung $A = \overline{L} \overline{L}^T$
berechnen

\begin{align*}
    A &=
	\begin{gmatrix}[p]
        1 & 2 & 3\\
        2 & 8 & 14\\
        3 & 14 & 34
        \rowops
        \add[\cdot (-2)]{0}{1}
        \add[\cdot (-3)]{0}{2}
    \end{gmatrix}\\
    \leadsto
    L^{(1)} &=
    \begin{pmatrix}
		1 & 0 & 0\\
	   -2 & 1 & 0\\
       -3 & 0 & 1
	\end{pmatrix},&
    A^{(1)} &=
	\begin{gmatrix}[p]
        1 & 2 & 3\\
        0 & 4 & 8\\
        0 & 8 & 25
        \rowops
        \add[\cdot (-2)]{1}{2}
    \end{gmatrix}\\
    \leadsto
    L^{(2)} &=
    \begin{pmatrix}
		1 & 0 & 0\\
	    0 & 1 & 0\\
        0 & -2 & 1
	\end{pmatrix},&
    A^{(2)} &=
	\begin{gmatrix}[p]
        1 & 2 & 3\\
        0 & 4 & 8\\
        0 & 0 & 9
    \end{gmatrix} =: R\\
    L &= (L^{(2)} \cdot L^{(1)})^{-1}\footnotemark
    &L &= \begin{pmatrix}
        1 & 0 & 0\\
        2 & 1 & 0\\
        3 & 2 & 1
    \end{pmatrix}
\end{align*}
\footnotetext{Da dies beides Frobeniusmatrizen sind, kann einfach die negierten Elemente unter der Diagonalmatrix auf die Einheitsmatrix addieren um das Ergebnis zu erhalten}

Nun gilt:
\begin{align}
    A &= LR = L (DL^T)\\
\Rightarrow A &= \underbrace{(L D^\frac{1}{2})}_{=: \overline{L}} (D^\frac{1}{2} L^T)\\
    \begin{pmatrix}d_1 &0&0\\0&d_2&0\\0&0&d_3\end{pmatrix} \cdot
\begin{pmatrix}
        1 & 2 & 3\\
        0 & 1 & 2\\
        0 & 0 & 1
    \end{pmatrix}
 &= \begin{pmatrix}
        1 & 2 & 3\\
        0 & 4 & 8\\
        0 & 0 & 9
    \end{pmatrix}\\
\Rightarrow D &= \begin{pmatrix}1 &0&0\\0&4&0\\0&0&9\end{pmatrix}\\
\Rightarrow D^\frac{1}{2} &= \begin{pmatrix}1 &0&0\\0&2&0\\0&0&3\end{pmatrix}\\
\overline{L} &= \begin{pmatrix}
        1 & 0 & 0\\
        2 & 1 & 0\\
        3 & 2 & 1
    \end{pmatrix} \cdot \begin{pmatrix}1 &0&0\\0&2&0\\0&0&3\end{pmatrix}\\
    &= \begin{pmatrix}
        1 & 0 & 0\\
        2 & 2 & 0\\
        3 & 4 & 3
    \end{pmatrix}
\end{align}
