\section*{Aufgabe 4}
Nach der Substitutionsregel gilt:
\[\int_{x_2}^{x_3} f(x) \mathrm{d}x = (x_3 - x_2) \cdot \int_0^1 f(x_2 + \tau (b-a)) \mathrm{d} \tau\]

Wenn $f$ ein Polynom vom Grad $q$ war, so ist auch das neue Integral ein Polynom
vom Grad $q$.

Ein Polynom, das vier Punkte interpoliert, hat maximal Grad 3.
Da wir das Integral über dieses Polynom im Bereich $[x_2, x_3]$
exakt berechnen sollen, muss die Quadraturformel vom Grad $p=4$ sein.



TODO
