\section*{Aufgabe 2}
\subsection*{Teilaufgabe i}
Es gilt:

\begin{align}
    2x - e^{-x} &= 0\\
    \Leftrightarrow 2x &= e^{-x}\\
\end{align}

Offensichtlich ist $g(x) := 2x$ streng monoton steigend und $h(x) := e^{-x}$ streng
monoton fallend.

Nun gilt: $g(0) = 0 < 1 = e^0 = h(0)$. Das heißt, es gibt keinen
Schnittpunkt für $x \leq 0$.

Außerdem: $g(1) = 2$ und $h(1) = e^{-1} = \frac{1}{e} < 2$.
Das heißt, für $x \geq 1$ haben $g$ und $h$ keinen Schnittpunkt.

Da $g$ und $h$ auf $[0,1]$ stetig sind und $g(0) < h(0)$ sowie $g(1) > h(1)$
gilt, müssen sich $g$ und $h$ im Intervall mindestens ein mal schneiden.
Da beide Funktionen streng monoton sind, schneiden sie sich genau
ein mal.

Ein Schnittpunkt der Funktion $g,h$ ist äquivalent zu einer
Nullstelle der Funktion $f$. Also hat $f$ genau eine Nullstelle
und diese liegt in $[0,1]$.

\subsection*{Teilaufgabe ii}
