\section*{Aufgabe 2}
\subsection*{Teilaufgabe i}
Es gilt:

\begin{align}
    2x - e^{-x} &= 0\\
    \Leftrightarrow 2x &= e^{-x}\\
\end{align}

Offensichtlich ist $g(x) := 2x$ streng monoton steigend und $h(x) := e^{-x}$ streng
monoton fallend.

Nun gilt: $g(0) = 0 < 1 = e^0 = h(0)$. Das heißt, es gibt keinen
Schnittpunkt für $x \leq 0$.

Außerdem: $g(1) = 2$ und $h(1) = e^{-1} = \frac{1}{e} < 2$.
Das heißt, für $x \geq 1$ haben $g$ und $h$ keinen Schnittpunkt.

Da $g$ und $h$ auf $[0,1]$ stetig sind und $g(0) < h(0)$ sowie $g(1) > h(1)$
gilt, müssen sich $g$ und $h$ im Intervall mindestens ein mal schneiden.
Da beide Funktionen streng monoton sind, schneiden sie sich genau
ein mal.

Ein Schnittpunkt der Funktion $g,h$ ist äquivalent zu einer
Nullstelle der Funktion $f$. Also hat $f$ genau eine Nullstelle
und diese liegt in $[0,1]$.

\subsection*{Teilaufgabe ii}
    \begin{align}
        2x - e^{-x} &= 0\\
    \Leftrightarrow 2x &= e^{-x}\\
    \Leftrightarrow x &= \frac{1}{2} \cdot e^{-x} = F_1(x) \label{a2iif1}\\
    \stackrel{x \in \mathbb{R}^+}{\Rightarrow} \ln(2x) &= -x\\
    \Leftrightarrow x &= - \ln(2x) = F_2(x)\label{a2iif2}
    \end{align}

Gleichung \ref{a2iif1} zeigt, dass der Fixpunkt von $F_1$ mit der 
Nullstelle von $f$ übereinstimmt.

Gleichung \ref{a2iif2} zeigt, dass der Fixpunkt von $F_1$ mit der 
Nullstelle von $f$ übereinstimmt. Da es nur in $[0,1]$ eine Nullstelle
gibt (vgl. Teilaufgabe i), ist die Einschränkung von $x$ auf $\mathbb{R}^+$
irrelevant.

TODO: Ich vermute, man soll die Kontraktionszahlen ermitteln.
Die Funktion mit der niedrigern Kontraktionszahl ist besser, da man
bessere Abschätzungen machen kann.

$F_1$ ist auf $[0,1]$ eine Kontraktion mit Kontraktionszahl $\theta = \frac{1}{2}$:
\begin{align}
    \|\frac{1}{2} e^{-x} - \frac{1}{2} e^{-y}\| &\leq \frac{1}{2} \cdot \|x-y\|\\
    \Leftrightarrow \frac{1}{2} \cdot \| e^{-x} - e^{-y}\| &\leq \frac{1}{2} \cdot \|x-y\|\\
    \Leftrightarrow \| e^{-x} - e^{-y}\| &\leq \|x-y\|\\
    \Leftrightarrow \| -e^{-x-y}(e^{x} - e^{y})\| &\leq \|x-y\|\\
    \Leftrightarrow \|-e^{-x-y} \| \cdot \|e^{x} - e^{y}\| &\leq \|x-y\|\\
    \Leftrightarrow \underbrace{e^{-(x+y)}}_{\leq 1} \cdot \|e^{x} - e^{y}\| &\leq \|x-y\|
\end{align}

TODO: Beweis ist noch nicht fertig

$F_2$ ist auf $(0,1]$ eine Kontraktion mit Kontraktionszahl $\theta$:
\begin{align}
    \|- \ln (2x) + \ln(2y) \| &\leq \theta \cdot \|x-y\|\\
    \Leftrightarrow \| \ln(\frac{2y}{2x}) \| &\leq \theta \cdot \|x-y\|\\
    \Leftrightarrow \| \ln(\frac{y}{x}) \| &\leq \theta \cdot \|x-y\|
\end{align}

TODO: Beweis ist nicht mal wirklich angefangen

Gegen $F_2$ spricht auch, dass $\log$ nur auf $\mathbb{R}^+$ definiert
ist. Das kann bei Rundungsfehlern eventuell zu einem Fehler führen.
(vgl. Python-Skript)

\subsection*{Teilaufgabe iii}
\[x_{k+1} = x_k - \frac{2x_k - e^{-x_k}}{2 + e^{-x_k}}\]

Laut \href{http://www.wolframalpha.com/input/?i=2x-e%5E(-x)%3D0}{Wolfram|Alpha} ist die Lösung etwa 0.35173371124919582602
