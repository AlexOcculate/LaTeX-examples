\section*{Aufgabe 4}
\subsection*{Teilaufgabe a)}
\begin{enumerate}
    \item Ordnung 3 kann durch geschickte Gewichtswahl erzwungen werden.
    \item Ordnung 4 ist automatisch gegeben, da die QF symmetrisch sein soll.
    \item Aufgrund der Symmetrie gilt Äquivalenz zwischen Ordnung 5 und 6.
          Denn eine hätte die QF Ordnung 5, so wäre wegen der
          Symmetrie Ordnung 6 direkt gegeben. Ordnung 6 wäre aber
          bei der Quadraturformel mit 3 Knoten das Maximum, was nur
          mit der Gauß-QF erreicht werden kann. Da aber $c_1 = 0$ gilt,
          kann es sich hier nicht um die Gauß-QF handeln. Wegen
          erwähnter Äquivalenz kann die QF auch nicht Ordnung 5 haben.
\end{enumerate}

Da $c_1 = 0$ gilt, muss $c_3 = 1$ sein (Symmetrie). Und dann muss $c_2 = \frac{1}{2}$
sein. Es müssen nun die Gewichte bestimmt werden um Ordnung 3 zu
garantieren mit:

\begin{align}
    b_i &= \int_0^1 L_i(x) \mathrm{d}x\\
    b_1 &= \frac{1}{6},\\
    b_2 &= \frac{4}{6},\\
    b_3 &= \frac{1}{6}
\end{align}

\subsection*{Teilaufgabe b)}
Als erstes ist festzustellen, dass es sich hier um die Simpsonregel handelt und die QF
\begin{align}
    \int_a^b f(x) \mathrm{d}x &= (b-a) \cdot \frac{1}{6} \cdot \left ( f(a) + 4 \cdot f(\frac{a+b}{2}) + f(b) \right )
\end{align}

ist. Wenn diese nun auf $N$ Intervalle aufgepflittet wird gilt folgendes:

\begin{align}
    \int_a^b f(x) \mathrm{d}x &= (b-a) \cdot \frac{1}{6} \cdot \left [ f(a) + f(b) + 2 \cdot \sum_{i=1}^{N-1} f(i \cdot \frac{1}{N}) + 4 \cdot \sum_{i=1}^N f(i \cdot \frac{1}{2N})\right ]
\end{align}

$\sum_{i=1}^{N-1} f(i \cdot \frac{1}{N})$  sind die Grenzknoten der Intervalle
 (deshalb werden sie doppelt gezählt). Von den Grenzknoten gibt es
insgesamt $s-2$ Stück, da die tatsächlichen Integralgrenzen $a$ und $b$
nur einmal in die Berechnung mit einfließen.

$\sum_{i=1}^N f(i \cdot \frac{1}{2N})$ sind die jeweiligen
mittleren Knoten der Intervalle. Davon gibt es $s-1$ Stück.

\begin{figure}[h]
    \centering
    \includegraphics*[width=\linewidth, keepaspectratio]{aufgabe4-b.png}
\end{figure}

\subsection*{Teilaufgabe c)}
TODO
