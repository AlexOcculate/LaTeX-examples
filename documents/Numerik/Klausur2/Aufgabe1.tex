\section*{Aufgabe 1}
\subsection*{Teilaufgabe a)}

$
L =
\begin{pmatrix}
2 & 0 & 0 \\
1 & 2 & 0 \\
4 & 2 & 3 \\
\end{pmatrix}
$


\subsection*{Teilaufgabe b)}
\textbf{Gesucht}: $\det(A)$

Sei $P \cdot L = L \cdot R$, die gewohnte LR-Zerlegung.

Dann gilt:

\[\det(A) = \det(L) \cdot \det(R) / \det(P)\]

$\det(L) = 1$, da alle Diagonalelemente 1 sind und es sich um eine untere Dreiecksmatrix handelt.

$\det(R) = r_{11} \cdot \ldots \cdot r_{nn} $ da es sich um eine obere Dreiecksmatrix handelt.


$\det(P) = 1$ oder $-1$

Das Verfahren ist also:
\begin{enumerate}
\item Berechne Restmatrix R mit dem Gaußverfahren.
\item \label{manker} Multipliziere die Diagonalelemente von R
\item falls die Anzahl an Zeilenvertauschungen ungerade ist negiere das Produkt aus \ref{manker} (eine Zeilenvertauschung verändert lediglich das Vorzeichen und P ist durch Zeilenvertauschungen aus der Einheitsmatrix hervorgegangen)
\end{enumerate}
