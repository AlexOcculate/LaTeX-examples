\section*{Aufgabe 2}
\subsection*{Teilaufgabe a)}

\textbf{Behauptung:} Für $x \in \mathbb{R}$ gilt, dass $cos(x_k) = x_{k+1}$ gegen den einzigen Fixpunkt $x^{*} = cos(x^{*})$ konvergiert.

\textbf{Beweis:} 
Sei $ D := [-1, 1]$.\\
Trivial: $D$ ist abgeschlossen.

Sei $ x \in D$, so gilt:
\begin{align*}
	0 < cos(x) \leq 1
\end{align*}
Also: $cos(x) \in D$.\\ Wenn $x \not\in D$, so gilt $y := cos(x)$ und $cos(y) \in D$. D.h. bereits nach einem Iterationschritt wäre $cos(x) \in D$ für $x \in \mathbb{R}$! Dies ist wichtig, da damit gezeigt ist, dass $cos(x_k) = x_{k+1}$ für jedes $x \in \mathbb{R}$ konvergiert! Es kommt nur dieser einzige Iteratationsschritt für $x \not\in \mathbb{R}$ hinzu.

Nun gilt mit $ x, y \in D, x < y, \xi \in (x,y) $ und dem Mittelwert der Differentialrechnung:
\begin{align*}
	\frac{cos(x) - cos(y)}{x - y} = cos'(\xi) \\
	\Leftrightarrow cos(x) - cos(y) =  cos'(\xi) * (x - y)  \\
	\Leftrightarrow | cos(x) - cos(y) | = | cos'(\xi) * (x - y) | \leq | cos'(\xi) | * | (x - y) | 
\end{align*}
Da $ \xi \in (0, 1) $ gilt:
\begin{align*}
	0 \leq | cos'(\xi) | = | sin(\xi) | < 1 
\end{align*}
Damit ist gezeigt, dass $cos(x) : D \rightarrow D$ Kontraktion auf $D$.

Damit sind alle Voraussetzung des Banachschen Fixpunktsatzes erfüllt.

Nach dem Banachschen Fixpunktsatz folgt die Aussage.