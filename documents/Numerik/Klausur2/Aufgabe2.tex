\section*{Aufgabe 2}

\subsection*{Lösungsalternative 1:}

\textbf{Voraussetzung:} 
Gegeben sei eine Funktion $F$:
\begin{align*}
    F: \mathbb{R} &\rightarrow [-1, 1]\\
    F(x) &:= \cos(x)
\end{align*}

sowie eine Folge $(x)_k$ mit $x_{k+1} := F(x_k)$.

\textbf{Behauptung:} $\displaystyle \exists! x^*: \forall x \in \mathbb{R}: \lim_{k \rightarrow \infty} x_k = x^*$

\paragraph{Beweis:} über den Banachschen Fixpunktsatz.

Teil 1: Es gibt genau einen Fixpunkt und dieser ist in $(0,1)$
\begin{proof}
Sei $ x \in \mathbb{R}$, so gilt:
\begin{align*}
	-1 \leq \cos(x) \leq 1
\end{align*}
Also genügt es $x \in [-1, 1]$ zu betrachten.

Sei nun $x \in [-1, 0)$. Dann gilt: $\cos(x) > 0$. Da $x <0$ aber $F(x) > 0$,
kann kein Fixpunkt in $[-1, 0)$ sein. Es genügt also sogar,
nur $[0, 1]$ zu betrachten.

Offensichtlich ist $F(0) \neq 0$ und $F(1) \neq 1$, also ist der 
Fixpunkt - falls vorhanden - in $(0,1)$. $F$ ist in $(0,1)$ stetig
und streng monoton fallend. Da auch $-x$ in $(0,1)$ streng monoton
fallend ist, folgt, dass $\cos(x) - x$ in $(0,1)$ streng monoton 
fallend ist.

$x=0 \Rightarrow \cos(x) - x = \cos(0) - 0 = 1$

$x=45^\circ = \frac{1}{4} \pi < 1 \Rightarrow \cos(45^\circ) - \frac{\pi}{4} = \frac{\sqrt{2}}{2} - \frac{\pi}{4} <0$, da
\begin{align}
    8 &< 9 < \pi^2\\
    \Rightarrow \sqrt{8} &< \pi\\
    \Leftrightarrow 2 \sqrt{2} &< \pi\\
    \Leftrightarrow \frac{\sqrt{2}}{2} &< \frac{\pi}{4}
\end{align}

$\stackrel{\text{Zwischenwertsatz}}{\Rightarrow} \exists x^*: \cos(x^*) - x^* = 0 \Leftrightarrow \exists x^*: \cos(x^*) = x^*$.

Dieses $x^*$ ist eindeutig, da $\cos(x)-x$ \emph{streng} monoton fallend ist.
\end{proof}

Teil 2: Jeder Startwert $x \in \mathbb{R}$ konvergiert gegen $x^*$.

\begin{proof}
Er genügt zu zeigen, dass $F$ auf $[0,1]$ eine Kontraktion ist, da
bereits in Teil 1 gezeigt wurde, dass man bereits $x_2 = \cos(\cos(x)) \in (0,1)$ ist.

Sei $0 \leq x < y \leq 1$. Dann folgt:
\begin{align}
    \stackrel{\text{Mittelwertsatz}}{\Rightarrow} \exists L \in (x,y): \frac{\cos(y) - \cos(x)}{y-x} &= f'(L)\\
    \Rightarrow \exists L \in [0,1]: \| \cos y - \cos x \| &= \| - \sin(L) \cdot (y-x)\| \\
    &= \underbrace{\sin(L)}_{[0,1)} (y-x)\\
   \Rightarrow F \text{ ist Kontraktion auf [0,1]}
\end{align}

Da $F|_{[0,1]}$ eine Selbstabbildung und eine Kontraktion ist und
offensichtlich $[0,1]$ abgeschlossen ist, greift der 
Banachsche Fixpunktsatz. Es folgt direkt, dass auch für alle $x \in [0,1]$
die Folge $(x)_k$ gegen den einzigen Fixpunkt $x^*$ konvergiert.

\end{proof}

\subsection*{Lösungsalternative 2:}

\textbf{Behauptung:} Für $x \in \mathbb{R}$ gilt, dass $cos(x_k) = x_{k+1}$ gegen den einzigen Fixpunkt $x^{*} = cos(x^{*})$ konvergiert.

\textbf{Beweis:} 
Sei $ D := [-1, 1]$.\\
Trivial: $D$ ist abgeschlossen.

Sei $ x \in D$, so gilt:
\begin{align*}
	0 < cos(x) \leq 1
\end{align*}
Also: $cos(x) \in D$.\\ Wenn $x \not\in D$, so gilt $y := cos(x)$ und $cos(y) \in D$. D.h. bereits nach einem Iterationschritt wäre $cos(x) \in D$ für $x \in \mathbb{R}$! Dies ist wichtig, da damit gezeigt ist, dass $cos(x_k) = x_{k+1}$ für jedes $x \in \mathbb{R}$ konvergiert! Es kommt nur dieser einzige Iteratationsschritt für $x \not\in \mathbb{R}$ hinzu.

Nun gilt mit $ x, y \in D, x < y, \xi \in (x,y) $ und dem Mittelwert der Differentialrechnung:
\begin{align*}
	\frac{cos(x) - cos(y)}{x - y} = cos'(\xi) \\
	\Leftrightarrow cos(x) - cos(y) =  cos'(\xi) * (x - y)  \\
	\Leftrightarrow | cos(x) - cos(y) | = | cos'(\xi) * (x - y) | \leq | cos'(\xi) | * | (x - y) | 
\end{align*}
Da $ \xi \in (0, 1) $ gilt:
\begin{align*}
	0 \leq | cos'(\xi) | = | sin(\xi) | < 1 
\end{align*}
Damit ist gezeigt, dass $cos(x) : D \rightarrow D$ Kontraktion auf $D$.

Damit sind alle Voraussetzung des Banachschen Fixpunktsatzes erfüllt.

Nach dem Banachschen Fixpunktsatz folgt die Aussage.
