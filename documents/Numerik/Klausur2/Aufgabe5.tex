\section*{Aufgabe 5}

Zunächst ist nach der Familie von Quadraturformeln gefragt, für die gilt: ($p := $ Ordnung der QF)
\begin{align}
	s = 3 \\
	0 = c_1 < c_2 < c_3 \\
	p \ge 4
\end{align}

Nach Satz 29 sind in der Familie genau die QFs, für die gilt: \\
Für alle Polynome $g(x)$ mit Grad $\le m-1$ gilt:
\begin{align}
	 \int_0^1 M(x) \cdot g(x) \mathrm{d}x = 0 \label{a3}
\end{align}

Da eine Quadraturformel höchstens Grad $2s=6$ (Satz 30) haben kann und es wegen
$c_1 = 0$ nicht die Gauss-Quadratur sein kann (Satz 31), kommt nur Ordnung $p=4$
und $p=5$ in Frage.

\subsection*{Ordnung 4}
Es gilt $g(x) = c$ für eine Konstante $c$, da $\text{Grad}(g(x))=0$ ist. 
Also ist \ref{a3} gleichbedeutend mit:
\begin{align}
	 \int_0^1 M(x) \cdot c \mathrm{d}x &= 0 \\
	 \Leftrightarrow c \cdot \int_0^1 M(x) \mathrm{d}x &= 0 \\
 	 \Leftrightarrow \int_0^1 M(x) \mathrm{d}x &= 0 \\
 	 \Leftrightarrow \int_0^1 (x-c_1)(x-c_2)(x-c_3) \mathrm{d}x &= 0 \\
 	 \Leftrightarrow \frac{1}{4} - \frac{1}{3} \cdot (c_2 + c_3) + \frac{1}{2} \cdot c_2 \cdot c_3 &= 0 \\
 	 \Leftrightarrow \frac{\frac{1}{4} - \frac{1}{3} \cdot c_3}
 	                      {\frac{1}{3} - \frac{1}{2} \cdot c_3} &= c_2
\end{align}

Natürlich müssen auch die Gewichte optimal gewählt werden. Dafür wird Satz 28 genutzt:
Sei $b^T = (b_1, b_2, b_3)$ der Gewichtsvektor. Sei zudem $C :=
\begin{pmatrix}
    {c_1}^0 & {c_2}^0 & {c_3}^0 \\
    {c_1}^1 & {c_2}^1 & {c_3}^1 \\
    {c_1}^2 & {c_2}^2 & {c_3}^2
\end{pmatrix}
$. \\
Dann gilt: $C$ ist invertierbar und $b = C^{-1} \cdot
\begin{pmatrix}
    1 \\
    \frac{1}{2} \\
    \frac{1}{3}
\end{pmatrix}
$.

Es gibt genau eine symmetrische QF in der Familie. Begründung: \\
Aus $c_1 = 0 $ folgt, dass $c_3 = 1$ ist. Außerdem muss $c_2 = \frac{1}{2} $ sein. Also sind die Knoten festgelegt. Da wir die Ordnung $\ge s = 3$ fordern, sind auch die Gewichte eindeutig. \\
Es handelt sich um die aus der Vorlesung bekannte Simpsonregel.

\subsection*{Ordnung 5}
Es gilt $g(x) = ax+c$ für Konstanten $a \neq 0, c$, da $\text{Grad}(g(x))=1$ ist. 
Also ist \ref{a3} gleichbedeutend mit:
\begin{align}
	 \int_0^1 M(x) \cdot (ax+c) \mathrm{d}x &= 0 \\
    \Leftrightarrow a \int_0^1 x M(x) \mathrm{d}x + c \int_0^1 M(x) \mathrm{d}x &= 0 \\
    \Leftrightarrow a \int_0^1 x (x-c_1)(x-c_2)(x-c_3) \mathrm{d}x + c \int_0^1 (x-c_1)(x-c_2)(x-c_3) \mathrm{d}x &= 0 \\
    \stackrel{c_1=0}{\Leftrightarrow} a \int_0^1 x^2(x-c_2)(x-c_3) \mathrm{d}x + c \int_0^1 x(x-c_2)(x-c_3) \mathrm{d}x &= 0 \\
    \Leftrightarrow a \left (\frac{c_2 c_3}{3}-\frac{c_2}{4}-\frac{c_3}{4}+\frac{1}{5} \right ) + c \left ( \frac{c_2 c_3}{2}-\frac{c_2}{3}-\frac{c_3}{3}+\frac{1}{4} \right ) &= 0 \\
    \Leftrightarrow \left (\frac{c_2 c_3}{3}-\frac{c_2}{4}-\frac{c_3}{4}+\frac{1}{5} \right ) + \underbrace{\frac{c}{a}}_{=: d} \left ( \frac{c_2 c_3}{2}-\frac{c_2}{3}-\frac{c_3}{3}+\frac{1}{4} \right ) &= 0
\end{align}

Nun habe ich \href{http://www.wolframalpha.com/input/?i=(1%2F5+-+c%2F4+%2B+(b+(-3+%2B+4+c))%2F12)%2B+d*(3+-+4+c+%2B+b+(-4+%2B+6+c))%2F12%3D0}{Wolfram|Alpha} lösen lassen:
\begin{align}
    c_2 &= \frac{6-\sqrt{6}}{10} \approx 0.355\\
    c_3 &= \frac{6+\sqrt{6}}{10} \approx 0.845
\end{align}

Wegen der Ordnungsbedingungen gilt nun:
\begin{align}
    1 &= b_1 + b_2 + b_3\\
    \frac{1}{2} &= b_2 \cdot \frac{6-\sqrt{6}}{10} + b_3 \cdot \frac{6+\sqrt{6}}{10}\\
    \frac{1}{3} &= b_2 \cdot \left (\frac{6-\sqrt{6}}{10} \right )^2 + b_3 \cdot \left (\frac{6+\sqrt{6}}{10} \right )^2\\
    \Leftrightarrow \frac{1}{3} - b_3 \cdot \left (\frac{6+\sqrt{6}}{10} \right )^2 &= b_2 \cdot \left (\frac{6-\sqrt{6}}{10} \right )^2\\
    \Leftrightarrow \frac{\frac{1}{3} - b_3 \cdot \left (\frac{6+\sqrt{6}}{10} \right )^2}{\left (\frac{6-\sqrt{6}}{10} \right )^2} &= b_2\\
    \Leftrightarrow b_2 &= \frac{100}{3 \cdot (6-\sqrt{6})^2} - b_3 \cdot \frac{(6+\sqrt{6})^2}{(6-\sqrt{6})^2}\\
    \Rightarrow \frac{1}{2} &= \left ( \frac{100}{3 \cdot (6-\sqrt{6})^2} - b_3 \cdot \frac{(6+\sqrt{6})^2}{(6-\sqrt{6})^2} \right ) \cdot \frac{6-\sqrt{6}}{10} + b_3 \cdot \frac{6+\sqrt{6}}{10}\\
    &= \left (\frac{10}{3 \cdot (6 - \sqrt{6})} - b_3 \cdot \frac{(6+\sqrt{6})^2}{10 \cdot (6 - \sqrt{6})} \right ) + b_3 \cdot \frac{6+\sqrt{6}}{10}\\
    &= b_3 \cdot \left (\frac{6+\sqrt{6}}{10} - \frac{(6+\sqrt{6})^2}{10 \cdot (6 - \sqrt{6})} \right ) + \frac{10}{3 \cdot (6 - \sqrt{6})}\\
    &= b_3 \cdot \left (\frac{30-(6+\sqrt{6})^2}{10 \cdot (6 - \sqrt{6})} \right ) + \frac{10}{3 \cdot (6 - \sqrt{6})}\\
\Leftrightarrow \frac{1}{2} - \frac{10}{3 \cdot (6 - \sqrt{6})} &= b_3 \cdot \left (\frac{30-(6+\sqrt{6})^2}{10 \cdot (6 - \sqrt{6})} \right )\\
\Leftrightarrow \frac{3 \cdot (6 - \sqrt{6}) - 20}{6\cdot (6 - \sqrt{6})} &= b_3 \cdot \left (\frac{30-(6+\sqrt{6})^2}{10 \cdot (6 - \sqrt{6})} \right )\\
\Leftrightarrow b_3 &= \frac{(3 \cdot (6 - \sqrt{6}) - 20) \cdot 10 \cdot (6 - \sqrt{6})}{6\cdot (6 - \sqrt{6}) \cdot (30-(6+\sqrt{6})^2)}\\
&= \frac{(3 \cdot (6 - \sqrt{6}) - 20) \cdot 5}{3 \cdot (30-(6+\sqrt{6})^2)}\\
&= \frac{15 \cdot (6 - \sqrt{6}) - 100}{90-3 \cdot (6+\sqrt{6})^2}\\
&= \frac{16-\sqrt{6}}{36} \approx 0.3764\\
   b_2 &= \frac{16+\sqrt{6}}{36} \approx 0.5125\\
   \stackrel{\text{Ordnungsbedinung 1}}{\Rightarrow} b_1 &= \frac{1}{9}\\
    \frac{1}{4} &\stackrel{?}{=} \frac{16+\sqrt{6}}{36} \cdot \left (\frac{6-\sqrt{6}}{10} \right )^3 + \frac{16-\sqrt{6}}{36} \cdot \left (\frac{6+\sqrt{6}}{10} \right)^3 \text{ \cmark}\\
    \frac{1}{5} &\stackrel{?}{=} \frac{16+\sqrt{6}}{36} \cdot \left (\frac{6-\sqrt{6}}{10} \right )^4 + \frac{16-\sqrt{6}}{36} \cdot \left (\frac{6+\sqrt{6}}{10} \right)^4 \text{ \cmark}\\
    \frac{1}{6} &\stackrel{?}{=} \frac{16+\sqrt{6}}{36} \cdot \left (\frac{6-\sqrt{6}}{10} \right )^5 + \frac{16-\sqrt{6}}{36} \cdot \left (\frac{6+\sqrt{6}}{10} \right)^5 = \frac{33}{200} \text{ \xmark}
\end{align}
