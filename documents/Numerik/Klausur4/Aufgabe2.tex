\section*{Aufgabe 2}
Die \emph{Kondition} eines Problems ist die Frage, wie sich Störungen
der Eingabegrößen unabhängig vom gewählten Algorithmus auf die
Lösung des Problems auswirken.

Bei dem lösen von linearen Gleichungssystemen sind die Eingabegrößen
die Koeffizientenmatrix $A$ und der Vektor $b$.

Der Begriff \emph{Stabilität} ist auf einen konkreten Algorithmus
zu beziehen und beschäftigt sich mit der Frage, wie sich Rundungsfehler,
welche während der Durchführung des Algorithmus entstehen, auf
die Lösung auswirken.

Die Stabilität eines Algorithmus bezeichnet, wie stark der Algorithmus das Ergebnis verfälschen kann. Man kann also die Stabilität der Gauß-Elimination angeben. Man kann allerdings nicht von einer Stabilität des Problems $A \cdot x = b$ sprechen.