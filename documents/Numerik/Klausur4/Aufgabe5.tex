\section*{Aufgabe 5}
\subsection*{Teilaufgabe a}
Eine Quadraturformel $(b_i, c_i)_{i=1,\dots,s}$ hat die Ordnung
$p$, falls sie exakte Lösungenfür alle Polynome vom Grad $\leq p-1$ liefern.\footnote{Kapitel 4, S. 4 des Skripts}

Die Ordnungsbedinungen liefern ein hinreichendes Kriterium zum Überprüfen
der Ordnung einer Quadraturformel.

Für die Mittelpunktsregel $c_1 = \frac{1}{2}, b_1 = 1$ gilt:
\begin{align}
    \frac{1}{1} &\stackrel{?}{=} b_1 = 1 \text{ \cmark}\\
    \frac{1}{2} &\stackrel{?}{=} b_1 c_1 = \frac{1}{2} \text{ \cmark}\\
    \frac{1}{3} &\stackrel{?}{=} b_1 c_1^2 = \frac{1}{4} \text{ \xmark}
\end{align}

Die Ordnung der Mittelpunktsregel ist also $p=2$.

\subsection*{Teilaufgabe b}
\paragraph*{Aufgabe:}
Das Integral
\[I = \int_0^1 \frac{1}{1+4x} \mathrm{d}x\]
soll näherungsweise mit der Mittelpunktsregel, angwendet auf eine
äquistante Unterteilung des Intervalls $[0,1]$ in zwei Teilintervalle
angewendet werden.

\paragraph*{Lösung:}

\begin{align}
    I &= \int_0^\frac{1}{2} \frac{1}{1+4x} \mathrm{d}x + \int_\frac{1}{2}^1 \frac{1}{1+4x} \mathrm{d}x\\
    &\approx \frac{1}{2} \cdot \frac{1}{1+1} + \frac{1}{2} \frac{1}{1+ 4 \cdot \frac{3}{4}} \\
    &= \frac{3}{8}
\end{align}
