\section*{Aufgabe 3}
\subsection*{Teilaufgabe i}
relativer Fehler:

\begin{align}
	\frac{ | \frac{x}{y} - \frac{x \cdot (1 + \epsilon_x)}{y \cdot (1 + \epsilon_y}|}{|\frac{x}{y}|}
	= \ldots = |\frac{\epsilon_y - \epsilon_x }{1 + \epsilon_y} |
	\le \frac{|\epsilon_y | + | \epsilon_x |}{|1 + \epsilon_y|} \le \frac{2 \cdot \text{eps}}{|1 + \epsilon_y|} 
\end{align}

Der letzte Ausdruck ist ungefähr gleich $2 \cdot \text{eps}$, da $1 + \epsilon_y$ ungefähr gleich $1$ ist.
Also: Der relative Fehler kann sich maximal verdoppeln.

\subsection*{Teilaufgabe ii}
Die zweite Formel ist vorzuziehen, also $f(x) = -\ln (x + \sqrt{x^2-1})$, da es bei Subtraktion zweier annähernd gleich-großer Zahlen zur Stellenauslöschung kommt. Bei der ersten Formel, also $f(x) = \ln (x - \sqrt{x^2-1})$, tritt genau dieses Problem auf: $x$ und $\sqrt{x^2-1}$ sind für große $x$ ungefähr gleich groß. \\
Bei der zweiten Formel tritt das Problem nicht auf: $x$ ist positiv und $\sqrt{x^2 - 1}$ auch, also gibt es in dem Ausdruck keine Subtraktion zweier annähernd gleich-großer Zahlen.
