Bei der Anwendung des in \cite{aggarwal2011} vorgestellten Algorithmus
auf reale Datensätze könnten zwei Probleme auftreten,
die im Folgenden erläutert werden. Außerdem werden Verbesserungen
vorgeschlagen, die es allerdings noch zu untersuchen gilt.

\subsection{Anzahl der Knotenbeschriftungen}
So, wie der DYCOS-Algorithmus vorgestellt wurde, können nur Graphen bearbeitet werden, 
deren Knoten jeweils höchstens eine Beschriftung haben. In vielen Fällen, wie z.~B. 
Wikipedia mit Kategorien als Knotenbeschriftungen haben Knoten jedoch viele Beschriftungen.

Auf einen ersten Blick ist diese Schwäche einfach zu beheben, indem 
man beim zählen der Knotenbeschriftungen für jeden Knoten jedes Beschriftung zählt. Dann
wäre noch die Frage zu klären, mit wie vielen Beschriftungen der betrachtete
Knoten beschriftet werden soll.

Jedoch ist z.~B. bei Wikipedia-Artikeln auf den Knoten eine 
Hierarchie definiert. So ist die Kategorie \enquote{Klassifikationsverfahren}
eine Unterkategorie von \enquote{Klassifikation}. Bei dem Kategorisieren
von Artikeln sind möglichst spezifische Kategorien vorzuziehen, also
kann man nicht einfach bei dem Auftreten der Kategorie \enquote{Klassifikationsverfahren}
sowohl für diese Kategorie als auch für die Kategorie \enquote{Klassifikation}
zählen.


\subsection{Überanpassung und Reklassifizierung}
Aggarwal und Li beschreiben in \cite{aggarwal2011} nicht, auf welche
Knoten der Klassifizierungsalgorithmus angewendet werden soll. Jedoch
ist die Reihenfolge der Klassifizierung relevant. Dazu folgendes 
Minimalbeispiel:

Gegeben sei ein dynamischer Graph ohne textuelle Inhalte. Zum Zeitpunkt
$t=1$ habe dieser Graph genau einen Knoten $v_1$ und $v_1$  sei
mit dem $A$ beschriftet. Zum Zeitpunkt $t=2$ komme ein nicht beschrifteter
Knoten $v_2$ sowie die Kante $(v_2, v_1)$ hinzu.\\
Nun wird der DYCOS-Algorithmus auf diesen Knoten angewendet und
$v_2$ mit $A$ beschriftet.\\
Zum Zeitpunkt $t=3$ komme ein Knoten $v_3$, der mit $B$ beschriftet ist,
und die Kante $(v_2, v_3)$ hinzu.

\begin{figure}[ht]
    \centering
    \subfloat[$t=1$]{
        \tikzstyle{vertex}=[draw,black,circle,minimum size=10pt,inner sep=0pt]
\tikzstyle{edge}=[very thick]
\begin{tikzpicture}[scale=1,framed]
    \node (a)[vertex,label=$A$] at (0,0) {$v_1$};
    \node (b)[vertex, white] at (1,0) {$v_2$};
    \node (struktur)[label={[label distance=-0.2cm]0:$t=1$}] at (-2,1) {};
\end{tikzpicture}

        \label{fig:graph-t1}
    }%
    \subfloat[$t=2$]{
        \tikzstyle{vertex}=[draw,black,circle,minimum size=10pt,inner sep=0pt]
\tikzstyle{edge}=[very thick]
\begin{tikzpicture}[scale=1,framed]
    \node (a)[vertex,label=$A$] at (0,0) {$v_1$};
    \node (b)[vertex,label={\color{blue}$A$}] at (1,0) {$v_2$};
    \draw[->] (b) -- (a);
    \node (struktur)[label={[label distance=-0.2cm]0:$t=2$}] at (-2,1) {};
\end{tikzpicture}

        \label{fig:graph-t2}
    }

    \subfloat[$t=3$]{
        \tikzstyle{vertex}=[draw,black,circle,minimum size=10pt,inner sep=0pt]
\tikzstyle{edge}=[very thick]
\begin{tikzpicture}[scale=1,framed]
    \node (a)[vertex,label=$A$] at (0,0) {$v_1$};
    \node (b)[vertex,label={\color{blue}$A$}] at (1,0) {$v_2$};
    \node (c)[vertex,label=$B$] at (2,0) {$v_3$};
    \draw[->] (b) -- (a);
    \draw[->] (b) -- (c);
    \node (struktur)[label={[label distance=-0.2cm]0:$t=3$}] at (-1,1) {};
\end{tikzpicture}

        \label{fig:graph-t3}
    }%
    \subfloat[$t=4$]{
        \tikzstyle{vertex}=[draw,black,circle,minimum size=10pt,inner sep=0pt]
\tikzstyle{edge}=[very thick]
\begin{tikzpicture}[scale=1,framed]
    \node (a)[vertex,label=$A$] at (0,0) {$v_1$};
    \node (b)[vertex,label=45:{\color{blue}$A$}] at (1,0) {$v_2$};
    \node (c)[vertex,label=$B$] at (2,0) {$v_3$};
    \node (d)[vertex] at (1,1) {$v_4$};
    \draw[->] (b) -- (a);
    \draw[->] (b) -- (c);

    \draw[->] (d) -- (a);
    \draw[->] (d) -- (b);
    \draw[->] (d) -- (c);
    \node (struktur)[label={[label distance=-0.2cm]0:$t=3$}] at (-1,1) {};
\end{tikzpicture}

        \label{fig:graph-t4}
    }%
    \label{Formen}
    \caption{Minimalbeispiel für den Einfluss früherer DYCOS-Anwendungen}
\end{figure}

Würde man nun den DYCOS-Algorithmus erst jetzt, also anstelle von
Zeitpunkt $t=2$ zum Zeitpunkt $t=3$ auf den Knoten $v_2$ anwenden, so
würde eine $50\%$-Wahrscheinlichkeit bestehen, dass dieser mit $B$ 
beschriftet wird. Aber in diesem Beispiel wurde der Knoten schon
zum Zeitpunkt $t=2$ beschriftet. Obwohl es in diesem kleinem Beispiel
noch keine Rolle spielt, wird das Problem klar, wenn man weitere
Knoten einfügt:

Wird zum Zeitpunkt $t=4$ ein unbeschrifteter Knoten $v_4$ und die Kanten
$(v_1, v_4)$, $(v_2, v_4)$, $(v_3, v_4)$ hinzugefügt, so ist die 
Wahrscheinlichkeit, dass $v_4$ mit $A$ beschriftet wird bei $\frac{2}{3}$.
Werden die unbeschrifteten Knoten jedoch erst jetzt und alle gemeinsam
beschriftet, so ist die Wahrscheinlichkeit für $A$ als Beschriftung bei nur $50\%$.
Bei dem DYCOS-Algorithmus findet also eine Überanpassung an vergangene
Beschriftungen statt.

Das Reklassifizieren von Knoten könnte eine mögliche Lösung für dieses
Problem sein. Knoten, die durch den DYCOS-Algorithmus beschriftet wurden
könnten eine Lebenszeit bekommen (TTL, Time to Live). Ist diese 
abgelaufen, wird der DYCOS-Algorithmus erneut auf den Knoten angewendet.


