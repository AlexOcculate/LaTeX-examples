Für den DYCOS-Algorithmus wurde in \cite{aggarwal2011} bewiesen,
dass sich nach Ausführung von DYCOS für einen unbeschrifteten
Knoten mit einer Wahrscheinlichkeit von höchstens
$(|\L_t|-1)\cdot e^{-l \cdot b^2 / 2}$ eine Knotenbeschriftung ergibt, deren
relative Häufigkeit weniger als $b$ der häufigsten Beschriftung ist.
Dabei ist $|\L_t|$ die Anzahl der Beschriftungen und $l$ die Länge der 
Random-Walks.

Außerdem wurde experimentell anhand des DBLP-Datensatzes\footnote{http://dblp.uni-trier.de/}
und des CORA-Datensatzes\footnote{http://people.cs.umass.edu/~mccallum/data/cora-classify.tar.gz}
gezeigt (vgl. \cref{tab:datasets}), dass die Klassifikationsgüte nicht wesentlich von der Anzahl der Wörter mit
höchstem Gini-Koeffizient $m$ abhängt. Des Weiteren betrug die Ausführungszeit
auf einem Kern eines Intel Xeon $\SI{2.5}{\GHz}$ Servers mit 
$\SI{32}{\giga\byte}$ RAM für den DBLP-Datensatz unter $\SI{25}{\second}$,
für den CORA-Datensatz sogar unter $\SI{5}{\second}$. Dabei wurde eine
für CORA eine Klassifikationsgüte von 82\% - 84\% und auf den DBLP-Daten
von 61\% - 66\% erreicht.

\begin{table}[htp]
    \centering
    \begin{tabular}{|l||r|r|r|r|}\hline
    \textbf{Name} & \textbf{Knoten} & \textbf{davon beschriftet} & \textbf{Kanten}  & \textbf{Beschriftungen} \\ \hline\hline
    \textbf{CORA} & \num{19396}  & \num{14814}             & \num{75021}   & 5              \\
    \textbf{DBLP} & \num{806635} & \num{18999 }            & \num{4414135} & 5              \\\hline
    \end{tabular}
    \caption{Datensätze, die für die experimentelle analyse benutzt wurden}
    \label{tab:datasets}
\end{table}

Obwohl es sich nicht sagen lässt,
wie genau die Ergebnisse aus \cite{aggarwal2011} zustande gekommen sind,
eignet sich das Kreuzvalidierungsverfahren zur Bestimmung der Klassifikationsgüte
wie es in \cite{Lavesson,Stone1974} vorgestellt wird:
\begin{enumerate}
    \item Betrachte nur $V_{L,T}$.
    \item Unterteile $V_{L,T}$ zufällig in $k$ disjunkte Mengen $M_1, \dots, M_k$.
    \item \label{schritt3} Teste die Klassifikationsgüte, wenn die Knotenbeschriftungen
          aller Knoten in $M_i$ für DYCOS verborgen werden für $i=1,\dots, k$.
    \item Bilde den Durchschnitt der Klassifikationsgüten aus \cref{schritt3}.
\end{enumerate}

Es wird $k=10$ vorgeschlagen.


