\documentclass[9pt,technote]{IEEEtran}
\usepackage{amssymb, amsmath} % needed for math
\usepackage{hyperref}   % links im text
\usepackage{parskip}
\usepackage[pdftex,final]{graphicx}
\usepackage{csquotes}
\usepackage{braket}
\usepackage{booktabs}
\usepackage{multirow}
\usepackage{pgfplots}
\usepackage[noadjust]{cite}
\usepackage[nameinlink,noabbrev]{cleveref} % has to be after hyperref, ntheorem, amsthm
\usepackage[binary-units]{siunitx}
\sisetup{per-mode=fraction,binary-units=true}
\DeclareSIUnit\pixel{px}
\usepackage{glossaries}
\loadglsentries[main]{glossary}
\makeglossaries

\title{On-line Recognition of Handwritten Mathematical Symbols}
\author{Martin Thoma, Kevin Kilgour, Sebastian St{\"u}ker and Alexander Waibel}

\hypersetup{ 
  pdfauthor   = {Martin Thoma, Kevin Kilgour, Sebastian St{\"u}ker and Alexander Waibel},
  pdfkeywords = {Mathematics,Symbols,recognition},
  pdftitle    = {On-line Recognition of Handwritten Mathematical Symbols}
}
\newcommand{\totalCollectedRecordings}{166898}  % ACTUALITY
\newcommand{\detexifyCollectedRecordings}{153423}
\newcommand{\trainingsetsize}{134804}
\newcommand{\validtionsetsize}{15161}
\newcommand{\testsetsize}{17012}
\newcommand{\totalClasses}{1111}
\newcommand{\totalClassesAnalyzed}{369}
\newcommand{\totalClassesAboveFifty}{680}
\newcommand{\totalClassesNotAnalyzedBelowFifty}{431}
\newcommand{\detexifyPercentage}{$\SI{91.93}{\percent}$}
\newcommand{\recordingsWithDots}{$\SI{2.77}{\percent}$}  % excluding i,j, ...
\newcommand{\recordingsWithDotsSizechange}{$\SI{0.85}{\percent}$}  % excluding i,j, ...

%%%%%%%%%%%%%%%%%%%%%%%%%%%%%%%%%%%%%%%%%%%%%%%%%%%%%%%%%%%%%%%%%%%%%
% Begin document                                                    %
%%%%%%%%%%%%%%%%%%%%%%%%%%%%%%%%%%%%%%%%%%%%%%%%%%%%%%%%%%%%%%%%%%%%%
\begin{document}
\maketitle
\begin{abstract}
Writing mathematical formulas with \LaTeX{} is easy as soon as one is used to
commands like \verb+\alpha+ and \verb+\propto+. However, for people who have
never used \LaTeX{} or who don't know the English name of the command, it can
be difficult to find the right command. Hence the automatic recognition of
handwritten mathematical symbols is desirable. This paper presents a system
which uses the pen trajectory to classify handwritten symbols. Five
preprocessing steps, one data multiplication algorithm, five features and five
variants for multilayer Perceptron training were evaluated using $\num{166898}$
recordings which were collected with two crowdsourcing projects. The evaluation
results of these 21~experiments were used to create an optimized recognizer
which has a TOP-1 error of less than $\SI{17.5}{\percent}$ and a TOP-3 error of
$\SI{4.0}{\percent}$. This is a relative improvement of $\SI{18.5}{\percent}$ for the
TOP-1 error and $\SI{29.7}{\percent}$ for the TOP-3 error compared to the
baseline system.
\end{abstract}

\section{Introduction}
On-line recognition makes use of the pen trajectory. This means the data is
given as groups of sequences of tuples $(x, y, t) \in \mathbb{R}^3$, where each
group represents a stroke, $(x, y)$ is the position of the pen on a canvas and
$t$ is the time. One handwritten symbol in the described format is called a
\textit{recording}. One approach to classify recordings into symbol classes
assigns a probability to each class given the data. The classifier can be
evaluated by using recordings which were classified by humans and were not used
to train the classifier. The set of those recordings is called \textit{test
set}. The TOP-$n$ error is defined as the fraction of the symbols where
the correct class was not within the top $n$ classes of the highest
probability.

Various systems for mathematical symbol recognition with on-line data have been
described so far~\cite{Kosmala98,Mouchere2013}, but most of them have neither
published their source code nor their data which makes it impossible to re-run
experiments to compare different systems. This is unfortunate as the choice of
symbols is crucial for the TOP-$n$ error and all systems used different symbol
sets. For example, the symbols $o$, $O$, $\circ$ and $0$ are very similar and
systems which know all those classes will certainly have a higher TOP-$n$ error
than systems which only accept one of them.

Daniel Kirsch describes in~\cite{Kirsch} a system called Detexify which uses
time warping to classify on-line handwritten symbols and claims to achieve a
TOP-3 error of less than $\SI{10}{\percent}$ for a set of $\num{100}$~symbols.
He also published his data on \url{https://github.com/kirel/detexify-data},
which was collected by a crowdsourcing approach via
\url{http://detexify.kirelabs.org}. Those recordings as well as some recordings
which were collected by a similar approach via \url{http://write-math.com} were
used to train and evaluated different classifiers. A complete description of
all involved software, data and experiments is given in~\cite{Thoma:2014}.

\section{Steps in Handwriting Recognition}
The following steps are used in many classifiers:

\begin{enumerate}
    \item \textbf{Preprocessing}: Recorded data is never perfect. Devices have
          errors and people make mistakes while using the devices. To tackle
          these problems there are preprocessing algorithms to clean the data.
          The preprocessing algorithms can also remove unnecessary variations
          of the data that do not help in the classification process, but hide
          what is important. Having slightly different sizes of the same symbol
          is an example of such a variation. Four preprocessing algorithms that
          clean or normalize recordings are explained in
          \cref{sec:preprocessing}.
    \item \textbf{Data multiplication}: Learning algorithms need lots of data
          to learn internal parameters. If there is not enough data available,
          domain knowledge can be considered to create new artificial data from
          the original data. In the domain of on-line handwriting recognition,
          data can be multiplied by adding rotated variants.
    \item \textbf{Segmentation}: The task of formula recognition can eventually
          be reduced to the task of symbol recognition combined with symbol
          placement. Before symbol recognition can be done, the formula has
          to be segmented. As this paper is only about single-symbol
          recognition, this step will not be further discussed.
    \item \textbf{Feature computation}: A feature is high-level information
          derived from the raw data after preprocessing. Some systems like
          Detexify simply take the result of the preprocessing step, but many
          compute new features. This might have the advantage that less
          training data is needed since the developer can use knowledge about
          handwriting to compute highly discriminative features. Various
          features are explained in \cref{sec:features}.
    \item \textbf{Feature enhancement}: Applying PCA, LDA, or
          feature standardization might change the features in ways that could
          improve the performance of learning algorithms.
\end{enumerate}

After these steps, we are faced with a classification learning task which
consists of two parts:
\begin{enumerate}
    \item \textbf{Learning} parameters for a given classifier. This process is
          also called \textit{training}.
    \item \textbf{Classifying} new recordings, sometimes called
          \textit{evaluation}. This should not be confused with the evaluation
          of the classification performance which is done for multiple
          topologies, preprocessing queues, and features in
          \Cref{ch:Evaluation}.
\end{enumerate}

The classification learning task can be solved with \glspl{MLP} if the number
of input features is the same for every recording. There are many ways how to
adjust \glspl{MLP} and how to adjust their training. Some of them are
described in~\cref{sec:mlp-training}.

\section{Algorithms}
\subsection{Preprocessing}\label{sec:preprocessing}
Preprocessing in symbol recognition is done to improve the quality and
expressive power of the data. It should make follow-up tasks like segmentation
and feature extraction easier, more effective or faster. It does so by resolving
errors in the input data, reducing duplicate information and removing irrelevant
information.

Preprocessing algorithms fall into two groups: Normalization and noise
reduction algorithms.

A very important normalization algorithm in single-symbol recognition is
\textit{scale-and-shift}~\cite{Thoma:2014}. It scales the recording so that
its bounding box fits into a unit square. As the aspect ratio of a recording
is almost never 1:1, only one dimension will fit exactly in the unit square.
There are multiple ways how to shift the recording. For this paper, it was
chosen to shift the bigger dimension to fit into the $[0,1] \times [0,1]$ unit
square whereas the smaller dimension is centered in the $[-1,1] \times [-1,1]$
square.

Another normalization preprocessing algorithm is resampling. As the data points
on the pen trajectory are generated asynchronously and with different
time-resolutions depending on the used hardware and software, it is desirable
to resample the recordings to have points spread equally in time for every
recording. This was done by linear interpolation of the $(x,t)$ and $(y,t)$
sequences and getting a fixed number of equally spaced points per stroke.

\textit{Connect strokes} is a noise reduction algorithm. It happens sometimes
that the hardware detects that the user lifted the pen where the user certainly
didn't do so. This can be detected by measuring the Euclidean distance between
the end of one stroke and the beginning of the next stroke. If this distance is
below a threshold, then the strokes are connected.

Due to a limited resolution of the recording device and due to erratic
handwriting, the pen trajectory might not be smooth. One way to smooth is
calculating a weighted average and replacing points by the weighted average of
their coordinate and their neighbors coordinates. Another way to do smoothing
would be to reduce the number of points with the Douglas-Peucker algorithm to
the most relevant ones and then interpolate the stroke between those points.
The Douglas-Peucker stroke simplification algorithm is usually used in
cartography to simplify the shape of roads. It works recursively to find a
subset of points of a stroke that is simpler and still similar to the original
shape. The algorithm adds the first and the last point $p_1$ and $p_n$ of a
stroke to the simplified set of points $S$. Then it searches the point $p_i$ in
between that has maximum distance from the line $p_1 p_n$. If this
distance is above a threshold $\varepsilon$, the point $p_i$ is added to $S$.
Then the algorithm gets applied to $p_1 p_i$ and $p_i p_n$ recursively. It is
described as \enquote{Algorithm 1} in~\cite{Visvalingam1990}.

\subsection{Features}\label{sec:features}
Features can be \textit{global}, that means calculated for the complete
recording or complete strokes. Other features are calculated for single points
on the pen trajectory and are called \textit{local}.

Global features are the \textit{number of strokes} in a recording, the
\textit{aspect ratio} of a recordings bounding box or the
\textit{ink} being used for a recording. The ink feature gets calculated by
measuring the length of all strokes combined. The re-curvature, which was
introduced in~\cite{Huang06}, is defined as
\[\text{re-curvature}(stroke) := \frac{\text{height}(stroke)}{\text{length}(stroke)}\]
and a stroke-global feature.

The simplest local feature is the coordinate of the point itself. Speed,
curvature and a local small-resolution bitmap around the point, which was
introduced by Manke, Finke and Waibel in~\cite{Manke1995}, are other local
features.

\subsection{Multilayer Perceptrons}\label{sec:mlp-training}
\Glspl{MLP} are explained in detail in~\cite{Mitchell97}. They can have
different numbers of hidden layers, the number of neurons per layer and the
activation functions can be varied. The learning algorithm is parameterized by
the learning rate $\eta \in (0, \infty)$, the momentum $\alpha \in [0, \infty)$
and the number of epochs.

The topology of \glspl{MLP} will be denoted in the following by separating the
number of neurons per layer with colons. For example, the notation
$160{:}500{:}500{:}500{:}369$ means that the input layer gets 160~features,
there are three hidden layers with 500~neurons per layer and one output layer
with 369~neurons.

\glspl{MLP} training can be executed in various different ways, for example
with \gls{SLP}. In case of a \gls{MLP} with the topology
$160{:}500{:}500{:}500{:}369$, \gls{SLP} works as follows: At first a \gls{MLP}
with one hidden layer ($160{:}500{:}369$) is trained. Then the output layer is
discarded, a new hidden layer and a new output layer is added and it is trained
again, resulting in a $160{:}500{:}500{:}369$ \gls{MLP}. The output layer is
discarded again, a new hidden layer is added and a new output layer is added
and the training is executed again.

Denoising auto-encoders are another way of pretraining. An
\textit{auto-encoder} is a neural network that is trained to restore its input.
This means the number of input neurons is equal to the number of output
neurons. The weights define an \textit{encoding} of the input that allows
restoring the input. As the neural network finds the encoding by itself, it is
called auto-encoder. If the hidden layer is smaller than the input layer, it
can be used for dimensionality reduction~\cite{Hinton1989}. If only one hidden
layer with linear activation functions is used, then the hidden layer contains
the principal components after training~\cite{Duda2001}.

Denoising auto-encoders are a variant introduced in~\cite{Vincent2008} that
is more robust to partial corruption of the input features. It is trained to
get robust by adding noise to the input features.

There are multiple ways how noise can be added. Gaussian noise and
randomly masking elements with zero are two possibilities. \cite{Deeplearning-Denoising-AE}
describes how such a denoising auto-encoder with masking noise can be
implemented. The \texttt{corruption} is the probability of a feature being
masked.

\section{Evaluation}\label{ch:Evaluation}
In order to evaluate the effect of different preprocessing algorithms, features
and adjustments in the \gls{MLP} training and topology, the following baseline
system was used:

Scale the recording to fit into a unit square while keeping the aspect ratio,
shift it into $[-1,1] \times [-1,1]$ as described in \cref{sec:preprocessing},
resample it with linear interpolation to get 20~points per stroke, spaced
evenly in time. Take the first 4~strokes with 20~points per stroke and
2~coordinates per point as features, resulting in 160~features which is equal
to the number of input neurons. If a recording has less than 4~strokes, the
remaining features were filled with zeroes.

All experiments were evaluated with four baseline systems $B_i$, $i \in \Set{1,
2, 3, 4}$, where $i$ is the number of hidden layers as different topologies
could have a severe influence on the effect of new features or preprocessing
steps. Each hidden layer in all evaluated systems has $500$ neurons.

Each \gls{MLP} was trained with a learning rate of $\eta = 0.1$ and a momentum
of $\alpha = 0.1$. The activation function of every neuron in a hidden layer is
the sigmoid function $\text{sig}(x) := \frac{1}{1+e^{-x}}$. The neurons in the
output layer use the softmax function. For every experiment, exactly one part
of the baseline systems was changed.

\subsection{Random Weight Initialization}
The neural networks in all experiments got initialized with a small random
weight

\[w_{i,j} \sim U(-4 \cdot \sqrt{\frac{6}{n_l + n_{l+1}}}, 4 \cdot \sqrt{\frac{6}{n_l + n_{l+1}}})\]

where $w_{i,j}$ is the weight between the neurons $i$ and $j$, $l$ is the layer
of neuron $i$, and $n_i$ is the number of neurons in layer $i$. This random
initialization was suggested on
\cite{deeplearningweights} and is done to break symmetry.

This might lead to different error rates for the same systems just because the
initialization was different.

In order to get an impression of the magnitude of the influence on the different
topologies and error rates the baseline models were trained 5 times with
random initializations.
\Cref{table:baseline-systems-random-initializations-summary}
shows a summary of the results. The more hidden layers are used, the more do
the results vary between different random weight initializations.

\begin{table}[h]
    \centering
    \begin{tabular}{crrr|rrr} %chktex 44
    \toprule
    \multirow{3}{*}{System}  & \multicolumn{6}{c}{Classification error}\\
    \cmidrule(l){2-7}
          & \multicolumn{3}{c}{TOP-1}   & \multicolumn{3}{c}{TOP-3}\\
          & min                    & max                    & range                 & min                   & max                   & range\\\midrule
    $B_1$ & $\SI{23.08}{\percent}$ & $\SI{23.44}{\percent}$ & $\SI{0.36}{\percent}$ & $\SI{6.67}{\percent}$ & $\SI{6.80}{\percent}$ & $\SI{0.13}{\percent}$ \\
    $B_2$ & \underline{$\SI{21.45}{\percent}$} & \underline{$\SI{21.83}{\percent}$}& $\SI{0.38}{\percent}$ & $\SI{5.68}{\percent}$ & \underline{$\SI{5.75}{\percent}$} & $\SI{0.07}{\percent}$\\
    $B_3$ & $\SI{21.54}{\percent}$ & $\SI{22.28}{\percent}$ & $\SI{0.74}{\percent}$ & \underline{$\SI{5.50}{\percent}$} & $\SI{5.82}{\percent}$ & $\SI{0.32}{\percent}$\\
    $B_4$ & $\SI{23.19}{\percent}$ & $\SI{24.84}{\percent}$ & $\SI{1.65}{\percent}$ & $\SI{5.98}{\percent}$ & $\SI{6.44}{\percent}$ & $\SI{0.46}{\percent}$\\
    \bottomrule
    \end{tabular}
    \caption{The systems $B_1$ -- $B_4$ were randomly initialized, trained
             and evaluated 5~times to estimate the influence of random weight
             initialization.}
\label{table:baseline-systems-random-initializations-summary}
\end{table}

\subsection{Connect strokes}
In order to solve the problem of interrupted strokes, pairs of strokes
can be connected with stroke connect algorithm. The idea is that for
a pair of consecutively drawn strokes $s_{i}, s_{i+1}$ the last point $s_i$ is
close to the first point of $s_{i+1}$ if a stroke was accidentally split
into two strokes.

$\SI{59}{\percent}$ of all stroke pair distances in the collected data are
between $\SI{30}{\pixel}$ and $\SI{150}{\pixel}$. Hence the stroke connect
algorithm was tried with $\SI{5}{\pixel}$, $\SI{10}{\pixel}$ and
$\SI{20}{\pixel}$.
All models TOP-3 error improved with a threshold of $\theta = \SI{10}{\pixel}$
by at least $\SI{0.17}{\percent}$, except $B_4$ which improved only by
$\SI{0.01}{\percent}$ which could be a result of random weight initialization.

\subsection{Douglas-Peucker Smoothing}
The Douglas-Peucker algorithm can be used to find
points that are more relevant for the overall shape of a recording. After that,
an interpolation can be done. If the interpolation is a cubic spline
interpolation, this makes the recording smooth.

The Douglas-Peucker algorithm was applied with a threshold of $\varepsilon =
0.05$, $\varepsilon = 0.1$ and $\varepsilon = 0.2$ after scaling and shifting,
but before resampling. The interpolation in the resampling step was done
linearly and with cubic splines in two experiments. The recording was scaled
and shifted again after the interpolation because the bounding box might have
changed.

The result of the application of the Douglas-Peucker smoothing with $\varepsilon
> 0.05$ was a high rise of the TOP-1 and TOP-3 error for all models $B_i$.
This means that the simplification process removes some relevant information and
does not --- as it was expected --- remove only noise. For $\varepsilon = 0.05$
with linear interpolation some models TOP-1 error improved, but the
changes were small. It could be an effect of random weight initialization.
However, cubic spline interpolation made all systems perform more than
$\SI{1.7}{\percent}$ worse for TOP-1 and TOP-3 error.

The lower the value of $\varepsilon$, the less does the recording change after
this preprocessing step. As it was applied after scaling the recording such that
the biggest dimension of the recording (width or height) is $1$, a value of
$\varepsilon = 0.05$ means that a point has to move at least $\SI{5}{\percent}$
of the biggest dimension.

\subsection{Global Features}
Single global features were added one at a time to the baseline systems. Those
features were re-curvature $\text{re-curvature}(stroke) = \frac{\text{height}(stroke)}{\text{length}(stroke)}$
as described in \cite{Huang06}, the ink feature which is the summed length
of all strokes, the stroke count, the aspect ratio and the stroke center points
for the first four strokes. The stroke center point feature improved the system
$B_1$ by $\SI{0.27}{\percent}$ for the TOP-3 error and system $B_3$ for the
TOP-1 error by $\SI{0.74}{\percent}$, but all other systems and error measures
either got worse or did not improve much.

The other global features did improve the systems $B_1 -- B_3$, but not $B_4$.
The highest improvement was achieved with the re-curvature feature. It
improved the systems $B_1 -- B_4$ by more than $\SI{0.6}{\percent}$ TOP-1 error.


\subsection{Data Multiplication}
Data multiplication can be used to make the model invariant to transformations.
However, this idea seems not to work well in the domain of on-line handwritten
mathematical symbols. It was tried to triple the data by adding a rotated
version that is rotated 3 degrees to the left and another one that is rotated
3 degrees to the right around the center of mass. This data multiplication
made all classifiers for most error measures perform worse by more than
$\SI{2}{\percent}$ for the TOP-1 error.

\subsection{Pretraining}\label{subsec:pretraining-evaluation}
Pretraining is a technique used to improve the training of \glspl{MLP} with
multiple hidden layers.

\Cref{fig:training-and-test-error-for-different-topologies-pretraining} shows
the evolution of the TOP-1 error over 1000~epochs with supervised
layer-wise pretraining and without pretraining. It clearly shows that this
kind of pretraining improves the classification performance by $\SI{1.6}{\percent}$
for the TOP-1 error and $\SI{1.0}{\percent}$ for the TOP-3 error.

\begin{figure}[htb]
    \centering
    \begin{tikzpicture}
    \begin{axis}[
            axis x line=middle,
            axis y line=middle,
            enlarge y limits=true,
            xmin=0,
            % xmax=1000,
            ymin=0.18, ymax=0.4,
            minor ytick={0, 0.01, ..., 1},
            % width=15cm, height=8cm,     % size of the image
            grid = both,
            minor grid style={dashed, gray!30},
            major grid style={gray!40},,
            %grid style={dashed, gray!30},
            ylabel=error,
            xlabel=epoch,
            legend cell align=left,
            legend style={
                at={(0.5,-0.1)},
                anchor=north,
                legend columns=2
            }
         ]
          \addplot[mark=x,green] table [each nth point=20,x=epoch, y=testerror, col sep=comma] {baseline-1.csv};
          \addplot[mark=x,orange] table [each nth point=20,x=epoch, y=testerror, col sep=comma] {baseline-2.csv};
          \addplot[mark=x,red] table [each nth point=20,x=epoch, y=testerror, col sep=comma] {baseline-2-pretraining.csv};
          \legend{{1 hidden layer},
                  {2 hidden layers},
                  {2 hidden layers with pretraining}}
    \end{axis}
\end{tikzpicture}
    \caption{Training- and test error by number of trained epochs for different
             topologies with \gls{SLP}. The plot shows
             that all pretrained systems performed much better than the systems
             without pretraining. All plotted systems did not improve
             with more epochs of training.}
\label{fig:training-and-test-error-for-different-topologies-pretraining}
\end{figure}

Pretraining with denoising auto-encoder lead to the much worse results listed in
\cref{table:pretraining-denoising-auto-encoder}. The first layer used a $\tanh$
activation function. Every layer was trained for $1000$ epochs and the 
\gls{MSE} loss function. A learning-rate of $\eta = 0.001$, a corruption of
$0.3$ and a $L_2$ regularization of $\lambda = 10^{-4}$ were chosen. This
pretraining setup made all systems with all error measures perform much worse.

\begin{table}[tb]
    \centering
    \begin{tabular}{lrrrr}
    \toprule
    \multirow{2}{*}{System}  & \multicolumn{4}{c}{Classification error}\\
    \cmidrule(l){2-5}
              & TOP-1                  & change                 & TOP-3                 & change                 \\\midrule
    $B_{1,p}$ & $\SI{23.75}{\percent}$ & $\SI{+0.41}{\percent}$ & $\SI{7.19}{\percent}$ & $\SI{+0.39}{\percent}$\\
    $B_{2,p}$ & \underline{$\SI{22.76}{\percent}$} & $\SI{+1.25}{\percent}$ & $\SI{6.38}{\percent}$ & $\SI{+0.63}{\percent}$\\
    $B_{3,p}$ & $\SI{23.10}{\percent}$ & $\SI{+1.17}{\percent}$ & \underline{$\SI{6.14}{\percent}$} & $\SI{+0.40}{\percent}$\\
    $B_{4,p}$ & $\SI{25.59}{\percent}$ & $\SI{+1.71}{\percent}$ & $\SI{6.99}{\percent}$ & $\SI{+0.87}{\percent}$\\
    \bottomrule
    \end{tabular}
    \caption{Systems with denoising auto-encoder pretraining compared to pure
             gradient descent. The pretrained systems clearly performed worse.}
\label{table:pretraining-denoising-auto-encoder}
\end{table}

\subsection{Optimized Recognizer}
All preprocessing steps and features that were useful were combined to
create a recognizer that should perform best.

All models were much better than everything that was tried before. The results
of this experiment show that single-symbol recognition with
\totalClassesAnalyzed{} classes and usual touch devices and the mouse can be
done with a TOP1 error rate of $\SI{18.56}{\percent}$ and a TOP3 error of
$\SI{4.11}{\percent}$. This was
achieved by a \gls{MLP} with a $167{:}500{:}500{:}\totalClassesAnalyzed{}$ topology.

It used an algorithm to connect strokes of which the ends were less than
$\SI{10}{\pixel}$ away, scaled each recording to a unit square and shifted this
unit square to $(0,0)$. After that, a linear resampling step was applied to the
first 4 strokes to resample them to 20 points each. All other strokes were
discarded.

The 167 features were

\begin{itemize}
     \item the first 4 strokes with 20 points per stroke resulting in 160
           features,
     \item the re-curvature for the first 4 strokes,
     \item the ink,
     \item the number of strokes and
     \item the aspect ratio
\end{itemize}

\Gls{SLP} was applied with $\num{1000}$ epochs per layer, a
learning rate of $\eta=0.1$ and a momentum of $\alpha=0.1$. After that, the
complete model was trained again for $1000$ epochs with standard mini-batch
gradient descent.

After the models $B_{1,c}$ -- $B_{4,c}$ were trained the first $1000$ epochs,
they were trained again for $1000$ epochs with a learning rate of $\eta = 0.05$.
\Cref{table:complex-recognizer-systems-evaluation} shows that
this improved the classifiers again.

\begin{table}[htb]
    \centering
    \begin{tabular}{lrrrr}
    \toprule
    \multirow{2}{*}{System}  & \multicolumn{4}{c}{Classification error}\\
    \cmidrule(l){2-5}
              & TOP1                   & change                 & TOP3                  & change\\\midrule
    $B_{1,c}$ & $\SI{20.96}{\percent}$ & $\SI{-2.38}{\percent}$ & $\SI{5.24}{\percent}$ & $\SI{-1.56}{\percent}$\\
    $B_{2,c}$ & $\SI{18.26}{\percent}$ & $\SI{-3.25}{\percent}$ & $\SI{4.07}{\percent}$ & $\SI{-1.68}{\percent}$\\
    $B_{3,c}$ & \underline{$\SI{18.19}{\percent}$} & $\SI{-3.74}{\percent}$ & \underline{$\SI{4.06}{\percent}$} & $\SI{-1.68}{\percent}$\\
    $B_{4,c}$ & $\SI{18.57}{\percent}$ & $\SI{-5.31}{\percent}$ & $\SI{4.25}{\percent}$ & $\SI{-1.87}{\percent}$\\\midrule
    $B_{1,c}'$ & $\SI{19.33}{\percent}$ & $\SI{-1.63}{\percent}$ & $\SI{4.78}{\percent}$ & $\SI{-0.46}{\percent}$ \\
    $B_{2,c}'$ & \underline{$\SI{17.52}{\percent}$} & $\SI{-0.74}{\percent}$ & \underline{$\SI{4.04}{\percent}$} & $\SI{-0.03}{\percent}$\\
    $B_{3,c}'$ & $\SI{17.65}{\percent}$ & $\SI{-0.54}{\percent}$ & $\SI{4.07}{\percent}$ & $\SI{+0.01}{\percent}$\\
    $B_{4,c}'$ & $\SI{17.82}{\percent}$ & $\SI{-0.75}{\percent}$ & $\SI{4.26}{\percent}$ & $\SI{+0.01}{\percent}$\\
    \bottomrule
    \end{tabular}
    \caption{Error rates of the optimized recognizer systems. The systems
             $B_{i,c}'$ were trained another $1000$ epochs with a learning rate
             of $\eta=0.05$. The value of the column \enquote{change} of the
             systems $B_{i,c}'$ is relative to $B_{i,c}$.}
\label{table:complex-recognizer-systems-evaluation}
\end{table}


\section{Conclusion}
Four baseline recognition systems were adjusted in many experiments and their
recognition capabilities were compared in order to build a recognition system
that can recognize 396 mathematical symbols with low error rates as well as to
evaluate which preprocessing steps and features help to improve the recognition
rate.

All recognition systems were trained and evaluated with
$\num{\totalCollectedRecordings{}}$ recordings for \totalClassesAnalyzed{}
symbols. These recordings were collected by two crowdsourcing projects 
(\href{http://detexify.kirelabs.org/classify.html}{Detexify} and
\href{write-math.com}{write-math.com}) and created with various devices. While
some recordings were created with standard touch devices such as tablets and
smartphones, others were created with the mouse.

\Glspl{MLP} were used for the classification task. Four baseline systems with
different numbers of hidden layers were used, as the number of hidden layer
influences the capabilities and problems of \glspl{MLP}.

All baseline systems used the same preprocessing queue. The recordings were
scaled to fit into a unit square, shifted to $(0,0)$, resampled with linear
interpolation so that every stroke had exactly 20~points which are spread
equidistant in time. The 80~($x,y$) coordinates of the first 4~strokes were used
to get exactly $160$ input features for every recording. The baseline system
$B_2$ has a TOP-3 error of $\SI{5.75}{\percent}$.

Adding two slightly rotated variants for each recording and hence tripling the
training set made the systems $B_3$ and $B_4$ perform much worse, but improved
the performance of the smaller systems.

The global features re-curvature, ink, stoke count and aspect ratio improved the
systems $B_1$--$B_3$, whereas the stroke center point feature made $B_2$ perform
worse.

Denoising auto-encoders were evaluated as one way
to use pretraining, but by this the error rate increased notably. However,
supervised layer-wise pretraining improved the performance decidedly.

The stroke connect algorithm was added to the preprocessing steps of the
baseline system as well as the re-curvature feature, the ink feature, the number
of strokes and the aspect ratio. The training setup of the baseline system was
changed to supervised layer-wise pretraining and the resulting model was trained
with a lower learning rate again. This optimized recognizer $B_{2,c}'$ had a TOP-3
error of $\SI{4.04}{\percent}$. This means that the TOP-3 error dropped by over
$\SI{1.7}{\percent}$ in comparison to the baseline system $B_2$.

A TOP-3 error of $\SI{4.04}{\percent}$ makes the system usable for symbol lookup.
It could also be used as a starting point for the development of a
multiple-symbol classifier.

The aim of this work was to develop a symbol recognition system which is easy
to use, fast and has high recognition rates as well as evaluating ideas for
single symbol classifiers. Some of those goals were reached. The recognition
system $B_{2,c}'$ evaluates new recordings in a fraction of a second and has
acceptable recognition rates. Many algorithms were evaluated.
However, there are still many other algorithms which could be evaluated and, at
the time of this work, the best classifier $B_{2,c}'$ is only available
through the Python package \texttt{hwrt}. It is planned to add an web version
of that classifier online.

\bibliographystyle{IEEEtranSA}
\bibliography{write-math-ba-paper}
\end{document}
