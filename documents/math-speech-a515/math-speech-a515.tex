\documentclass[a4paper]{scrartcl}
\usepackage{amssymb, amsmath} % needed for math
\usepackage[utf8]{inputenc} % this is needed for umlauts
\usepackage[ngerman]{babel} % this is needed for umlauts
\usepackage[T1]{fontenc}    % this is needed for correct output of umlauts in pdf
\usepackage[margin=2.5cm]{geometry} %layout
\usepackage{hyperref}   % links im text
\usepackage{parskip}

\title{Praktikum Spracherkennung}
\author{Martin Thoma}

\hypersetup{ 
  pdfauthor   = {Martin Thoma}, 
  pdfkeywords = {}, 
  pdftitle    = {Praktikum Spracherkennung} 
} 

%%%%%%%%%%%%%%%%%%%%%%%%%%%%%%%%%%%%%%%%%%%%%%%%%%%%%%%%%%%%%%%%%%%%%
% Begin document                                                    %
%%%%%%%%%%%%%%%%%%%%%%%%%%%%%%%%%%%%%%%%%%%%%%%%%%%%%%%%%%%%%%%%%%%%%
\begin{document}
\section{Aufgabe}
Berechnen Sie den Logarithmus der Wahrscheinlichkeit des Musters $m$, wenn die
eben definierte Gauß-Mixtur gegeben ist (d.h. die HMM
Emissionswahrscheinlichkeit).

\section{Gegeben}

\begin{align}
    m &= \begin{pmatrix}2\\2\end{pmatrix} \in \mathbb{R}^d & \Sigma_{s,i} &= \begin{pmatrix}1 & 0 \\0 & 1\end{pmatrix}  \in \mathbb{R}^{d \times d}\\
    c_{s,1} &= 0.3 \in [0,1]  & c_{s,2} &= 0.7 \in [0,1]\\
    \mu_{s,1} &= \begin{pmatrix}1\\3\end{pmatrix}  \in \mathbb{R}^d & \mu_{s,2} &= \begin{pmatrix}2\\4\end{pmatrix}  \in \mathbb{R}^d
\end{align}

\[p(x|s) = \sum_{i=1}^{n_s} c_{s, i} \frac{1}{\sqrt{(2 \pi)^d |\Sigma_{s,i}|}} e^{-\frac{1}{2} {(x-\mu_{s,i})}^T \Sigma_{s,i}^{-1}(x-\mu_{s,i})}\]

mit

\begin{itemize}
    \item $s$ (für \textit{senone}) ist die kleinste Einheit die der
          automatische Spracherkenner zu erkennen in der lage sein soll.
    \item $m$ ist das zu klassifizierende Muster.
    \item $c_{s,i}$ sind die Gewichte der Gauss-Verteilungen. Ihre Summe muss
          1 ergeben, damit das Ergebnis wieder eine
          Wahrscheinlichkeitsdichtefunktion ist.
    \item $\Sigma_{s,i}$ ist die Kovarianzmatrix. Sie ist invertierbar und
          ihre Determinate $|\Sigma_{s,i}| \neq 0$.
    \item $d$ ist die Dimension der Feature-Vektoren.
    \item $n_s$ ist die Anzahl der Gauss-Verteilungen
    \item $p(x|s)$ ist die Wahrscheinlichkeitsdichtefunktion, gegeben das
          Senone $s$ für das Muster $x$.
\end{itemize}

\section{Erklärung}

Eine Gauss-Verteilung hat die Wahrscheinlichkeitsdichte $f:\mathbb{R} \rightarrow [0,1]$
\[f(x) := \frac{1}{\sqrt{2 \pi \sigma^2}} e^{-\frac{1}{2}(\frac{x - \mu}{\sigma})^2}\]
wobei $\mu$ der Erwartungswert und $\sigma^2$ die Varianz ist.
Man schreibt
\[X \sim \mathcal{N}(\mu, \sigma^2)\]

Eine multivariate Gauss-Verteilung ist eine Verallgemeinerung auf mehrdimensionale
Zufallsvariablen. Sie hat die Wahrscheinlichkeitsdichte $f: \mathbb{R}^n \rightarrow [0,1]$
\[f(x) := \frac{1}{\sqrt{(2\pi)^n|\Sigma|}} e^{-\tfrac{1}{2}(x-\mu)^{T}\Sigma^{-1}(x-\mu)}\]
wobei $\mu$ der Erwartungswert und $\Sigma$ die Kovarianz-Matrix ist.
Man schreibt
\[X \sim \mathcal{N}_n(\mu, \Sigma)\]

Eine Multinomialverteilung ist eine diskrete Wahrscheinlichkeitsverteilung.


\section{Einsetzen}

\begin{align}
    p(m|s) &= \sum_{i=1}^{2} c_{s, i} \frac{1}{\sqrt{(2 \pi)^2}} e^{-\frac{1}{2} {(m_1 - \mu_{1,s,i})^2+(m_2 - \mu_{2,s,i})^2}}\\
    &= 0.3 \frac{1}{\sqrt{(2 \pi)^2}} e^{-\frac{1}{2} ({1^2+ (-1)^2})} + 0.7 \frac{1}{\sqrt{(2 \pi)^2}} e^{-\frac{1}{2} ({0^2+ (-2)^2}})\\
    &= \frac{1}{2 \pi} (0.3 e^{-1} + 0.7 e^{-2})\\
    &\approx 0.0326\\
    \log(p(m|s)) &\approx -3.4234
\end{align}

\section{Fragen}

$p(m|s)$ ist nicht die Wahrscheinlichkeit von $m$. Die Wahrscheinlichkeit
eines einzelnen Wertes einer kontinuierlichen Zufallsvariable ist immer $0$,
da sonst die Summe der Wahrscheinlichkeiten unendlich wäre. Ist das
dennoch der gewünschte Wert?

Was hat das mit den HMMs zu tun? Sollte da eventuell GMM stehen?

Was ist eine multinominale multivariate Gauss-Verteilung? Wo ist der
Unterschied zur multivariaten Gauss-Verteilung?

\end{document}
