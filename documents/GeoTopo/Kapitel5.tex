%%%%%%%%%%%%%%%%%%%%%%%%%%%%%%%%%%%%%%%%%%%%%%%%%%%%%%%%%%%%%%%%%%%%%
% Mitschrieb vom 30.01.2014                                         %
%%%%%%%%%%%%%%%%%%%%%%%%%%%%%%%%%%%%%%%%%%%%%%%%%%%%%%%%%%%%%%%%%%%%%
\chapter{Krümmung}
\section{Krümmung von Kurven}
\begin{definition}%In Vorlesung: Def+Bem. 16.1
    Sei $\gamma: I = [a, b] \rightarrow \mdr^n$ eine $C^\infty$-Funktion.
    
    \begin{defenum}
        \item $\gamma$ heißt \textbf{durch Bogenlänge parametrisiert}\xindex{parametrisiert!durch Bogenlänge},
              wenn $\|\gamma'(t)\|_2 = 1$ für alle $t \in I$. Dabei
              ist $\gamma'(t) = \left (\gamma_1'(t), \gamma_2'(t), \dots, \gamma_n'(t) \right)$
        \item $l(\gamma) = \int_a^b \|\gamma'(t)\| \mathrm{d} t$ heißt
              \textbf{Länge von $\gamma$}\xindex{Kurve!Länge einer}
    \end{defenum}    
\end{definition}

\begin{bemerkung}%In Vorlesung: Def+Bem. 16.1
    Sei $\gamma: I = [a, b] \rightarrow \mdr^n$ eine $C^\infty$-Funktion.

    \begin{bemenum}
        \item Ist $\gamma$ durch Bogenlänge parametrisiert, so ist $l(\gamma) = b-a$.
        \item \label{bem:16.1d} Ist $\gamma$ durch Bogenlänge parametrisiert, so ist 
              $\gamma'(t)$ orthogonal zu $\gamma''(t)$ für alle $t \in I$.
    \end{bemenum}
\end{bemerkung}

\begin{beweis}
    von \cref{bem:16.1d}:

    $1 = \|\gamma'(t)\| = \|\gamma'(t)\|^2 = \langle \gamma'(t), \gamma'(t) \rangle$\\
    \begin{align*}
        \Rightarrow 0 &= \frac{\mathrm{d}}{\mathrm{d}t} \langle \gamma'(t), \gamma'(t) \rangle\\
                      &= \frac{\mathrm{d}}{\mathrm{d}t} (\gamma_1'(t)\gamma_1'(t) + \gamma_2'(t)\gamma_2'(t))\\
                      &= 2 (\gamma_1''(t) \cdot \gamma_1'(t) + \gamma_2''(t) \cdot \gamma_2'(t))\\
                      &= 2 \langle \gamma''(t), \gamma'(t) \rangle
     \end{align*}
\end{beweis}

\begin{definition}%In Vorlesung: Definition 16.2
    Sei $\gamma: I \rightarrow \mdr^2$ eine durch Bogenlänge
    parametrisierte Kurve.

    \begin{defenum}
        \item Für $t \in I$ sei $n(t)$ \textbf{Normalenvektor}\xindex{Normalenvektor}
              an $\gamma$ in $t$, d.~h.
              \[\langle n(t), \gamma'(t) \rangle = 0, \;\;\; \|n(t)\|=1 \]
              und $\det((\gamma_1(t), n(t))) = +1$
        \item Nach \cref{bem:16.1d} sind $n(t)$ und $\gamma''(t)$ linear
              abhängig, d.~h. es gibt $\kappa(t) \in \mdr$ mit
              \[\gamma''(t) = \kappa(t) \cdot n(t)\]
              $\kappa(t)$ heißt \textbf{Krümmung}\xindex{Krümmung}
              von $\gamma$ in $t$.
    \end{defenum}
\end{definition}

\begin{beispiel}%In Vorlesung: Beispiel 16.3
    Gegeben sei ein Kreis mit Radius $r$, d.~h. mit Umfang $2\pi r$.
    Es gilt:

    \[\gamma(t) = (r \cdot \cos \frac{t}{r}, r \cdot \sin \frac{t}{r}) \text{ für } t \in [0, 2\pi r]\]
    ist parametrisiert durch Bogenlänge.

    \begin{align*}
        \gamma'(t)  &= ((r \cdot \frac{1}{r}) (- \sin \frac{t}{r}), r \frac{1}{r} \cos \frac{t}{r})\\
                    &= (- \sin \frac{t}{r}, \cos \frac{t}{r})\\
        \Rightarrow n(t) &= (- \cos \frac{t}{r}, - \sin \frac{t}{r})\\
        \gamma''(t) &= (- \frac{1}{r} \cos \frac{t}{r}, - \frac{1}{r} \sin \frac{t}{r})\\
                    &= \frac{1}{r} \cdot (- \cos \frac{t}{r}, - \sin \frac{t}{r})\\
        \Rightarrow \kappa(t) &= \frac{1}{r}
    \end{align*}
\end{beispiel}

\begin{definition}%In Vorlesung: Def+Bem 16.4
    Sei $\gamma: I \rightarrow \mdr^3$ durch Bogenlänge parametrisierte
    Kurve.

    \begin{defenum}
        \item Für $t \in I$ heißt $\kappa(t) := \|\gamma''(t)\|$ die
              \textbf{Krümmung}\xindex{Krümmung} von $\gamma$ in $t$.
        \item Ist für $t \in I$ die Ableitung $\gamma''(t) \neq 0$,
              so heißt $\gamma''(t)$ \textbf{Normalenvektor}\xindex{Normalenvektor}
              an $\gamma$ in $t$.
        \item \label{def:16.4c} $b(t)$ sei ein Vektor, der $\gamma'(t), n(t)$
              zu einer orientierten Orthonormalbasis von $\mdr^3$ ergänzt.
              Also $\det(\gamma'(t), n(t), b(t)) = 1$;
              $b(t)$ heißt \textbf{Binormalenvektor}\xindex{Binormalenvektor},
              die Orthonormalbasis $\Set{\gamma'(t), n(t), b(t)}$
              heißt \textbf{begleitendes Dreibein}\xindex{Dreibein!begreitendes}.
    \end{defenum}
\end{definition}

\begin{bemerkung}%In Vorlesung: Def+Bem 16.4
    Sei $\gamma: I \rightarrow \mdr^3$ durch Bogenlänge parametrisierte
    Kurve.

    \begin{bemenum}
        \item $n(t)$ ist orthogonal zu $\gamma'(t)$.
        \item $b(t)$ aus \cref{def:16.4c} ist eindeutig.
    \end{bemenum}
\end{bemerkung}

\section{Tangentialebene}
Erinnerung Sie sich an \cref{def:8.5} \enquote{reguläre Fläche}.

Äquivalent dazu ist: $S$ ist lokal von der Form
\[V(f) = \Set{x \in \mdr^3 | f(x) = 0 }\]
für eine $C^\infty$-Funktion $f: \mdr^\infty \rightarrow \mdr$.\todo{Wirklich $\mdr^\infty$?}

\begin{definition}%In Vorlesung: 17.1
    Sei $S \subseteq \mdr^3$ eine reguläre Fläche, $s \in S$,
    $F: U \rightarrow V \cap S$ eine lokale Parametrisierung um $s$
    (d.~h. $s \in V$)
    \[(u,v) \mapsto (x(u,v), y(u,v), z(u,v))\]
    Für $p=F^{-1}(s) \in U$ sei
    \[        J_F(u,v) = \begin{pmatrix}
            \frac{\partial x}{\partial u} (p) & \frac{\partial x}{\partial v} (p)\\
            \frac{\partial y}{\partial u} (p) & \frac{\partial y}{\partial v} (p)\\
            \frac{\partial z}{\partial u} (p) & \frac{\partial z}{\partial v} (p)
        \end{pmatrix}\]
    und $D_P F: \mdr^2 \rightarrow \mdr^3$ die durch $J_F (p)$
    definierte lineare Abbildung.

    Dann heißt $T_s S := \Bild(D_p F)$ die \textbf{Tangentialebene}\xindex{Tangentialebene}
    an $S \in s$.
\end{definition}

\begin{bemerkung}%In Vorlesung: 17.2
    $T_s S$ ist $2$-dimensionaler Untervektorraum von $\mdr^3$.
\end{bemerkung}

\begin{bemerkung}%In Vorlesung: 17.3
    $T_s S$ hängt nicht von der gewählten Parametrisierung ab.
\end{bemerkung}

\begin{beweis}\leavevmode
    \begin{behauptung}
        $T_s S = \Set{x \in \mdr^3 | \exists \text{parametrisierte Kurve } \gamma:[- \varepsilon, + \varepsilon] \rightarrow S \text{ für ein } \varepsilon > 0 \text{ mit } \gamma(0) = S \text{ und } \gamma'(0) = x}$
    \end{behauptung}
\end{beweis}
