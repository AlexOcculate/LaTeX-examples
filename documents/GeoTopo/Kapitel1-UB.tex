\clearpage
\section*{Übungsaufgaben}
\addcontentsline{toc}{section}{Übungsaufgaben}

\begin{aufgabe}[Sierpińskiraum]\label{ub1:aufg1}\xindex{Sierpińskiraum}
    Es sei $X := \Set{0,1}$ und $\fT_X := \Set{\emptyset, \Set{0}, X}$.
    Dies ist der sogenannte Sierpińskiraum.
    \begin{enumerate}[label=(\alph*)]
        \item Beweisen Sie, dass $(X, \fT_X)$ ein topologischer Raum ist.
        \item Ist $(X, \fT_X)$ hausdorffsch?
        \item Ist $\fT_X$ von einer Metrik erzeugt?
    \end{enumerate}
\end{aufgabe}

\begin{aufgabe}\label{ub1:aufg4}
    Es sei $\mdz$ mit der von den Mengen $U_{a,b} := a + b \mdz (a \in \mdz, b \in \mdz \setminus \Set{0})$
    erzeugten Topologie versehen.

    Zeigen Sie:
    \begin{enumerate}[label=(\alph*)]
        \item Jedes $U_{a,b}$ und jede einelementige Teilmenge von $\mdz$ ist abgeschlossen.
        \item $\Set{-1, 1}$ ist nicht offen.
        \item Es gibt unendlich viele Primzahlen.
    \end{enumerate}
\end{aufgabe}

\begin{aufgabe}[Cantorsches Diskontinuum]\label{ub2:aufg4}\xindex{Cantorsches Diskontinuum}
    Für jedes $i \in \mdn$ sei $P_i := \Set{0,1}$ mit der diskreten
    Topologie. Weiter Sei $P := \prod_{i \in \mdn} P_i$.

    \begin{enumerate}[label=(\alph*)]
        \item Wie sehen die offenen Mengen von $P$ aus?
        \item Was können Sie über den Zusammenhang von $P$ sagen?
    \end{enumerate}
\end{aufgabe}

\begin{aufgabe}[Kompaktheit]\label{ub3:aufg1}
    \begin{enumerate}[label=(\alph*)]
        \item Ist $\text{GL}_n(\mdr) = \Set{A \in \mdr^{n \times n} | \det(A) \neq 0}$ kompakt?
        \item Ist $\text{SL}_n(\mdr) = \Set{A \in \mdr^{n \times n} | \det(A) = 1}$ kompakt?
        \item Ist $\praum(\mdr)$ kompakt?
    \end{enumerate}
\end{aufgabe}
