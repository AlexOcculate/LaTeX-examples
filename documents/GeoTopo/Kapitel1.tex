\chapter{Topologische Grundbegriffe}
\section{Topologische Räume}
\begin{definition} \index{Topologischer Raum} \index{offen} \index{abgeschlossen}
    Ein \textbf{topologischer Raum} ist ein Paar $(X, \fT)$ bestehend
    aus einer Menge $X$ und $\fT \subseteq \powerset{X}$ mit
    folgenden Eigenschaften
    \begin{enumerate}[(i)]
        \item $\emptyset, X \in \fT$
        \item Sind $U_1, U_2 \in \fT$, so ist $U_1 \cap U_2 \in \fT$
        \item Ist $I$ eine Menge und $U_i \in \fT$ für jedes $i \in I$,
              so ist $\displaystyle \bigcup_{i \in I} U_i \in \fT$
    \end{enumerate}
    Die Elemente von $\fT$ heißen \textbf{offene Teilmengen} von $X$. 

    $A \setminus X$ heißt \textbf{abgeschlossen}, wenn $X \setminus A$ offen ist.

\end{definition}

Es gibt auch Mengen, die weder abgeschlossen, noch offen sind.

\begin{beispieleX}
    \begin{enumerate}[1)]
        \item $X = \mdr^n$ mit der euklidischen Metrik.\\
              $U \subseteq \mdr^n$ offen $\gdw$ für jedes $x \in U$ 
              gibt es $r > 0$, sodass $B_r(x) = \Set{y \in \mdr^n | d(x,y) < r} \subseteq U$
        \item Allgemeiner: $(X, d)$ metrischer Raum
        \item $X$ Menge, $\fT = \Set{\emptyset, X}$ heißt \enquote{triviale Menge} \index{Menge!triviale}
        \item $X$ Menge, $\fT = \powerset{X}$ heißt \enquote{diskrete Topologie} \index{Topologie!diskrete}
        \item $X :=\mdr, \fT_Z := \Set{U \subseteq \mdr | \mdr \setminus U \text{ endlich}} \cup \Set{\emptyset}$ heißt \enquote{Zariski-Topologie} \index{Topologie!Zariski}\\
              Beobachtung: $U \in \fT_Z \gdw \exists f \in \mdr[X]$, sodass $\mdr \setminus U = V(f) = \Set{x \in \mdr | f(x) = 0}$
        \item $X := \mdr^n, \fT_Z = \{U \subseteq \mdr^n | \text{Es gibt Polynome } f_1, \dots, f_r \in \mdr[X_1, \dots, X_n] \text{ sodass }\\\mdr^n \setminus U = V(f_1, \dots, f_r)\}$
        \item $X = \Set{0,1}, \fT = \Set{\emptyset, \Set{0,1}, \Set{0}}$\\
              abgeschlossene Mengen: $\emptyset, \Set{0,1}, \Set{1}$
    \end{enumerate}
\end{beispieleX}

\begin{definition} \index{Umgebung}
    Sei $(X, \fT)$ ein topologischer Raum, $x \in X$.

    Eine Teilmenge $U \subseteq X$ heißt \textbf{Umgebung} von $x$,
    wenn es ein $U_0 \in \fT$ gibt mit $x \in U_0$ und $U_0 \subseteq U$.
\end{definition}

\begin{definition}
    Sei $(X, \fT)$ ein topologischer Raum, $M \subseteq X$ eine Teilmenge.
    \begin{enumerate}[a)]
        \item $M^\circ := \Set{x \in M | M \text{ ist Umgebung von } x}$ heißt \textbf{Inneres} oder \textbf{ offener Kern} von $M$. \index{Inneres} \index{Kern!offener}
        \item $\displaystyle \overline{M} := \bigcup_{\stackrel{M \subseteq A}{A \text{ abgeschlossen}}} A$ heißt \textbf{abgeschlossene Hülle} oder \textbf{Abschluss} von $M$. \index{Abschluss}
        \item $\partial M := \overline{M} \setminus M^\circ$ heißt \textbf{Rand} von $M$. \index{Rand}
        \item $M$ heißt \textbf{dicht} in $X$, wenn $\overline{M} = X$ ist. \index{dicht}
    \end{enumerate}
\end{definition}
