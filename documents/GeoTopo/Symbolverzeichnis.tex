%!TEX root = GeoTopo.tex
\markboth{Symbolverzeichnis}{Symbolverzeichnis}
\twocolumn
\chapter*{Symbolverzeichnis}
\addcontentsline{toc}{chapter}{Symbolverzeichnis}
%%%%%%%%%%%%%%%%%%%%%%%%%%%%%%%%%%%%%%%%%%%%%%%%%%%%%%%%%%%%%%%%%%%%%
% Mengenoperationen                                                 %
%%%%%%%%%%%%%%%%%%%%%%%%%%%%%%%%%%%%%%%%%%%%%%%%%%%%%%%%%%%%%%%%%%%%%
\section*{Mengenoperationen}

Seien $A, B$ und $M$ Mengen.

% Set \mylengtha to widest element in first column; adjust
% \mylengthb so that the width of the table is \columnwidth
\settowidth\mylengtha{$A \subsetneq B$}
\setlength\mylengthb{\dimexpr\columnwidth-\mylengtha-2\tabcolsep\relax}

\begin{xtabular}{@{} p{\mylengtha} P{\mylengthb} @{}}
$A^C $           & Komplement von $A$\\
$\mathcal{P}(M)$ & Potenzmenge von $M$\\
$\overline{M}$   & Abschluss von $M$\\
$\partial M$     & Rand der Menge $M$\\
$M^\circ$        & Inneres der Menge $M$\\
$A \times B$     & Kreuzprodukt\\
$A \subseteq B$  & Teilmengenbeziehung\\
$A \subsetneq B$ & echte Teilmengenbeziehung\\
$A \setminus B$  & Differenzmenge\\
$A \cup B$       & Vereinigung\\
$A \dcup B$      & Disjunkte Vereinigung\\
$A \cap B$       & Schnitt\\
\end{xtabular}
%%%%%%%%%%%%%%%%%%%%%%%%%%%%%%%%%%%%%%%%%%%%%%%%%%%%%%%%%%%%%%%%%%%%%
% Geometrie                                                 %
%%%%%%%%%%%%%%%%%%%%%%%%%%%%%%%%%%%%%%%%%%%%%%%%%%%%%%%%%%%%%%%%%%%%%
\section*{Geometrie}

\settowidth\mylengtha{$\overline{AB} \cong \overline{CD}$}
\setlength\mylengthb{\dimexpr\columnwidth-\mylengtha-2\tabcolsep\relax}

\begin{xtabular}{@{} p{\mylengtha} P{\mylengthb} @{}}
$AB$                               & Gerade durch die Punkte $A$ und $B$\\
$\overline{AB}$                    & Strecke mit Endpunkten $A$ und $B$\\
$\triangle ABC$                    & Dreieck mit Eckpunkten $A, B, C$\\
$\overline{AB} \cong \overline{CD}$& Die Strecken $\overline{AB}$ und $\overline{CD}$ sind isometrisch\\
$|K|$                              & Geometrische Realisierung des Simplizialkomplexes~$K$\\
\end{xtabular}
%%%%%%%%%%%%%%%%%%%%%%%%%%%%%%%%%%%%%%%%%%%%%%%%%%%%%%%%%%%%%%%%%%%%%
% Gruppen                                                           %
%%%%%%%%%%%%%%%%%%%%%%%%%%%%%%%%%%%%%%%%%%%%%%%%%%%%%%%%%%%%%%%%%%%%%
\section*{Gruppen}

Sei $X$ ein topologischer Raum und $K$ ein Körper.

\settowidth\mylengtha{$\Homoo(X)$}
\setlength\mylengthb{\dimexpr\columnwidth-\mylengtha-2\tabcolsep\relax}

\begin{xtabular}{@{} p{\mylengtha} P{\mylengthb} @{}}
$\Homoo(X)$ & Homöomorphis\-men\-gruppe\\
$\Iso(X)$   & Isometrien\-gruppe\\
$\GL_n(K)$  & Allgemeine lineare Gruppe (von \textit{\textbf{G}eneral \textbf{L}inear Group})\\
$\SL_n(K)$  & Spezielle lineare Gruppe\\
$\PSL_n(K)$ & Projektive lineare Gruppe\\
$\Perm(X)$  & Permutations\-gruppe\\
$\Sym(X)$   & Symmetrische Gruppe\\
\end{xtabular}
%%%%%%%%%%%%%%%%%%%%%%%%%%%%%%%%%%%%%%%%%%%%%%%%%%%%%%%%%%%%%%%%%%%%%
% Wege                                                              %
%%%%%%%%%%%%%%%%%%%%%%%%%%%%%%%%%%%%%%%%%%%%%%%%%%%%%%%%%%%%%%%%%%%%%
\section*{Wege}

Sei $\gamma: I \rightarrow X$ ein Weg.

\settowidth\mylengtha{$\gamma_1 \sim \gamma_2$}
\setlength\mylengthb{\dimexpr\columnwidth-\mylengtha-2\tabcolsep\relax}

\begin{xtabular}{@{} p{\mylengtha} P{\mylengthb} @{}}
$[\gamma]$               & Homotopieklasse von $\gamma$\\
$\gamma_1 * \gamma_2$    & Zusammenhängen von Wegen\\
$\gamma_1 \sim \gamma_2$ & Homotopie von Wegen\\
$\overline{\gamma}(x)$   & Inverser Weg, also $\overline{\gamma}(x) := \gamma(1-x)$\\
$C$                      & Bild eines Weges $\gamma$, also $C := \gamma([0,1])$
\end{xtabular}
%%%%%%%%%%%%%%%%%%%%%%%%%%%%%%%%%%%%%%%%%%%%%%%%%%%%%%%%%%%%%%%%%%%%%
% Weiteres                                                          %
%%%%%%%%%%%%%%%%%%%%%%%%%%%%%%%%%%%%%%%%%%%%%%%%%%%%%%%%%%%%%%%%%%%%%
\section*{Weiteres}

\settowidth\mylengtha{$\fB_\delta(x)$}
\setlength\mylengthb{\dimexpr\columnwidth-\mylengtha-2\tabcolsep\relax}

\begin{xtabular}{@{} p{\mylengtha} P{\mylengthb} @{}}
$\fB$          & Basis einer Topologie\\
$\fB_\delta(x)$& $\delta$-Kugel um $x$\\
$\calS$        & Subbasis einer Topologie\\
$\fT$          & Topologie\\
\end{xtabular}

\settowidth\mylengtha{$X /_\sim$}
\setlength\mylengthb{\dimexpr\columnwidth-\mylengtha-2\tabcolsep\relax}

\begin{xtabular}{@{} p{\mylengtha} P{\mylengthb} @{}}
$\atlas$                        & Atlas\\
$\praum$                        & Projektiver Raum\\
$\langle \cdot , \cdot \rangle$ & Skalarprodukt\\
$X /_\sim$                      & $X$ modulo $\sim$\\
$[x]_\sim$                      & Äquivalenzklassen von $x$ bzgl. $\sim$\\
$\| x \|$                       & Norm von $x$\\
$| x |$                         & Betrag von $x$\\
$\langle a \rangle$             & Erzeugnis von $a$\\
\end{xtabular}

$S^n\;\;\;$ Sphäre\\
$T^n\;\;\;$ Torus\\

\settowidth\mylengtha{$f^{-1}(M)$}
\setlength\mylengthb{\dimexpr\columnwidth-\mylengtha-2\tabcolsep\relax}

\begin{xtabular}{@{} p{\mylengtha} P{\mylengthb} @{}}
$f \circ g$&Verkettung von $f$ und $g$\\
$\pi_X$    &Projektion auf $X$\\
$f|_U$ $f$ &eingeschränkt auf $U$\\
$f^{-1}(M)$&Urbild von $M$\\
$\rang(M)$ & Rang von $M$\\
$\chi(K)$  & Euler-Charakteristik von $K$\\
$\Delta^k$ & Standard-Simplex\\
$X \# Y$   & Verklebung von $X$ und $Y$\\
$d_n$      & Lineare Abbildung aus \cref{kor:9.11}\\
$A \cong B$& $A$ ist isometrisch zu $B$\\
$f_*$      & Abbildung zwischen Fundamentalgruppen (vgl. \cpageref{korr:11.5})
\end{xtabular}

\onecolumn

%%%%%%%%%%%%%%%%%%%%%%%%%%%%%%%%%%%%%%%%%%%%%%%%%%%%%%%%%%%%%%%%%%%%%
% Zahlenmengen                                                      %
%%%%%%%%%%%%%%%%%%%%%%%%%%%%%%%%%%%%%%%%%%%%%%%%%%%%%%%%%%%%%%%%%%%%%
\section*{Zahlenmengen}
$\mdn = \Set{1, 2, 3, \dots} \;\;\;$ Natürliche Zahlen\\
$\mdz = \mdn \cup \Set{0, -1, -2, \dots} \;\;\;$ Ganze Zahlen\\
$\mdq = \mdz \cup \Set{\frac{1}{2}, \frac{1}{3}, \frac{2}{3}} = \Set{\frac{z}{n} \text{ mit } z \in \mdz \text{ und } n \in \mdz \setminus \Set{0}} \;\;\;$ Rationale Zahlen\\
$\mdr = \mdq \cup \Set{\sqrt{2}, -\sqrt[3]{3}, \dots}\;\;\;$ Reele Zahlen\\
$\mdr_+\;$ Echt positive reele Zahlen\\
$\mdr_{+,0}^n := \Set{(x_1, \dots, x_n) \in \mdr^n | x_n \geq 0}\;\;\;$ Halbraum\\
$\mdr^\times = \mdr \setminus \Set{0} \;$ Einheitengruppe von $\mdr$\\
$\mdc = \Set{a+ib|a,b \in \mdr}\;\;\;$ Komplexe Zahlen\\
$\mdp = \Set{2, 3, 5, 7, \dots}\;\;\;$ Primzahlen\\
$\mdh = \Set{z \in \mdc | \Im{z} > 0}\;\;\;$ obere Halbebene\\
$I = [0,1] \subsetneq \mdr\;\;\;$ Einheitsintervall\\

\settowidth\mylengtha{$f:S^1 \hookrightarrow \mdr^2$}
\setlength\mylengthb{\dimexpr\columnwidth-\mylengtha-2\tabcolsep\relax}

\begin{xtabular}{@{} p{\mylengtha} P{\mylengthb} @{}}
$f:S^1 \hookrightarrow \mdr^2$& Einbettung der Kreislinie in die Ebene\\
$\pi_1(X,x)$                  & Fundamentalgruppe im topologischen Raum $X$ um $x \in X$\\
$\Fix(f)$                     & Menge der Fixpunkte der Abbildung $f$\\
$\|\cdot\|_2$                 & 2-Norm; Euklidische Norm\\
$\kappa$                      & Krümmung\\
$\kappa_{\ts{Nor}}$           & Normalenkrümmung\\
$V(f)$                        & Nullstellenmenge von $f$\footnotemark
\end{xtabular}
\footnotetext{von \textit{\textbf{V}anishing Set}}
%%%%%%%%%%%%%%%%%%%%%%%%%%%%%%%%%%%%%%%%%%%%%%%%%%%%%%%%%%%%%%%%%%%%%
% Krümmung                                                          %
%%%%%%%%%%%%%%%%%%%%%%%%%%%%%%%%%%%%%%%%%%%%%%%%%%%%%%%%%%%%%%%%%%%%%
\section*{Krümmung}

\settowidth\mylengtha{$D_p F: \mdr^2 \rightarrow \mdr^3$}
\setlength\mylengthb{\dimexpr\columnwidth-\mylengtha-2\tabcolsep\relax}

\begin{xtabular}{@{} p{\mylengtha} P{\mylengthb} @{}}
$D_p F: \mdr^2 \rightarrow \mdr^3$& Lineare Abbildung mit Jacobi-Matrix in $p$ (siehe \cpageref{def:Tangentialebene})\\
$T_s S$                           & Tangentialebene an $S \subseteq \mdr^3$ durch $s \in S$\\
$d_s n(x)$                        & Weingarten-Abbildung\\
\end{xtabular}

\index{Faser|see{Urbild}}
\index{kongruent|see{isometrisch}}
\index{Kongruenz|see{Isometrie}}