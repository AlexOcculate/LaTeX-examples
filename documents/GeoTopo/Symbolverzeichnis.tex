\twocolumn
\chapter*{Symbolverzeichnis\markboth{Symbolverzeichnis}{Symbolverzeichnis}}
\addcontentsline{toc}{chapter}{Symbolverzeichnis}
%%%%%%%%%%%%%%%%%%%%%%%%%%%%%%%%%%%%%%%%%%%%%%%%%%%%%%%%%%%%%%%%%%%%%
% Mengenoperationen                                                 %
%%%%%%%%%%%%%%%%%%%%%%%%%%%%%%%%%%%%%%%%%%%%%%%%%%%%%%%%%%%%%%%%%%%%%
\section*{Mengenoperationen}
$A^C\;\;\;$ Komplement der Menge $A$\\
$\mathcal{P}(M)\;\;\;$ Potenzmenge von $M$\\
$\overline{M}\;\;\;$ Abschluss der Menge $M$\\
$\partial M\;\;\;$ Rand der Menge $M$\\
$M^\circ\;\;\;$ Inneres der Menge $M$\\
$A \times B\;\;\;$ Kreuzprodukt zweier Mengen\\
$A \subseteq B\;\;\;$ Teilmengenbeziehung\\
$A \subsetneq B\;\;\;$ echte Teilmengenbeziehung\\
$A \setminus B\;\;\;$ $A$ ohne $B$\\
$A \cup B\;\;\;$ Vereinigung\\
$A \dcup B\;\;\;$ Disjunkte Vereinigung\\
$A \cap B\;\;\;$ Schnitt\\

\section*{Geometrie}
$AB\;\;\;$ Gerade durch die Punkte $A$ und $B$\\
$\overline{AB}\;\;\;$ Strecke mit Endpunkten $A$ und $B$\\
$\triangle ABC\;\;\;$ Dreieck mit Eckpunkten $A, B, C$\\
%%%%%%%%%%%%%%%%%%%%%%%%%%%%%%%%%%%%%%%%%%%%%%%%%%%%%%%%%%%%%%%%%%%%%
% Gruppen                                                           %
%%%%%%%%%%%%%%%%%%%%%%%%%%%%%%%%%%%%%%%%%%%%%%%%%%%%%%%%%%%%%%%%%%%%%
\section*{Gruppen}
$\Homoo(X)\;\;\;$ Homöomorphismengruppe\\
$\Iso(X)\;\;\;$ Isometriengruppe\\
$\GL_n(K)\;\;\;$ Allgemeine lineare Gruppe\footnote{von \textit{\textbf{G}eneral \textbf{L}inear Group}}\\
$\SL_n(K)\;\;\;$ Spezielle lineare Gruppe\\
$\PSL_n(K)\;\;\;$ Projektive lineare Gruppe\\
$\Perm(X)\;\;\;$ Permutationsgruppe\\
$\Sym(X)\;\;\;$ Symmetrische Gruppe
%%%%%%%%%%%%%%%%%%%%%%%%%%%%%%%%%%%%%%%%%%%%%%%%%%%%%%%%%%%%%%%%%%%%%
% Weiteres                                                          %
%%%%%%%%%%%%%%%%%%%%%%%%%%%%%%%%%%%%%%%%%%%%%%%%%%%%%%%%%%%%%%%%%%%%%
\section*{Weiteres}
$\fB\;\;\;$ Basis einer Topologie\\
$\calS\;\;\;$ Subbasis einer Topologie\\
$\fB_\delta(x)\;\;\;$ $\delta$-Kugel um $x$\\
$\fT\;\;\;$ Topologie\\

$\praum\;\;\;$ Projektiver Raum\\
$\langle \cdot , \cdot \rangle\;\;\;$ Skalarprodukt\\
$X /_\sim\;\;\;$ $X$ modulo $\sim$\\
$[x]_\sim\;\;\;$ Äquivalenzklassen von $x$ bzgl. $\sim$\\
$\| x \|\;\;\;$ Norm von $x$\\
$| x |\;\;\;$ Betrag von $x$\\

$S^n\;\;\;$ Sphäre\\
$T^n\;\;\;$ Torus\\

$f \circ g\;\;\;$ Verkettung von $f$ und $g$\\
$[\gamma]\;\;\;$ Homotopieklasse eines Weges $\gamma$\\
$\pi_X\;\;\;$ Projektion auf $X$\\
$f|_U\;\;\;$ $f$ eingeschränkt auf $U$\\
$f^{-1}(M)\;\;\;$ Urbild von $M$\\
$\Rg(M)\;\;\;$ Rang von $M$\\
$\chi(K)\;\;\;$ Euler-Charakteristik von $K$\\
$\Delta^k\;\;\;$ Standard-Simplex\\
$X \# Y\;\;\;$ Verklebung von $X$ und $Y$\\
$\gamma_1 * \gamma_2\;\;\;$ Zusammenhängen von Wegen\\
$d_n\;\;\;$ Lineare Abbildung aus \cref{kor:9.11}
\onecolumn

%%%%%%%%%%%%%%%%%%%%%%%%%%%%%%%%%%%%%%%%%%%%%%%%%%%%%%%%%%%%%%%%%%%%%
% Zahlenmengen                                                      %
%%%%%%%%%%%%%%%%%%%%%%%%%%%%%%%%%%%%%%%%%%%%%%%%%%%%%%%%%%%%%%%%%%%%%
\section*{Zahlenmengen}
$\mdn = \Set{1, 2, 3, \dots} \;\;\;$ Natürliche Zahlen\\
$\mdz = \mdn \cup \Set{0, -1, -2, \dots} \;\;\;$ Ganze Zahlen\\
$\mdq = \mdz \cup \Set{\frac{1}{2}, \frac{1}{3}, \frac{2}{3}} = \Set{\frac{z}{n} \text{ mit } z \in \mdz \text{ und } n \in \mdz \setminus \Set{0}} \;\;\;$ Rationale Zahlen\\
$\mdr = \mdq \cup \Set{\sqrt{2}, -\sqrt[3]{3}, \dots}\;\;\;$ Reele Zahlen\\
$\mdr^+\;$ Echt positive reele Zahlen\\
$\mdr^\times = \mdr \setminus \Set{0} \;$ Einheitengruppe von $\mdr$\\
$\mdc = \Set{a+ib|a,b \in \mdr}\;\;\;$ Komplexe Zahlen\\
$\mdp = \Set{2, 3, 5, 7, \dots}\;\;\;$ Primzahlen\\
$\mdh = \Set{z \in \mdc | \Im{z} > 0}\;\;\;$ obere Halbebene\\

$f:S^1 \hookrightarrow \mdr^2\;\;\;$ Einbettung der Kreislinie in die Ebene\\
$\pi_1(X,x)\;\;\;$ Fundamentalgruppe im topologischen Raum $X$ um $x \in X$\\
$\Fix(f)\;\;\;$ Menge der Fixpunkte der Abbildung $f$\\
$\|\cdot\|_2\;\;\;$ 2-Norm; Euklidische Norm\\
$\kappa\;\;\;$ Krümmung\\
$\kappa_{\ts{Nor}}$
$V(f)\;\;\;$ Nullstellenmenge von $f$\footnote{von \textit{\textbf{V}anishing Set}}

\index{Faser|see{Urbild}}
\index{kongruent|see{isometrisch}}
\index{Kongruenz|see{Isometrie}}
