%%%%%%%%%%%%%%%%%%%%%%%%%%%%%%%%%%%%%%%%%%%%%%%%%%%%%%%%%%%%%%%%%%%%%%
% Begriffslexikon zur Beschreibung des Produkts						 %
%%%%%%%%%%%%%%%%%%%%%%%%%%%%%%%%%%%%%%%%%%%%%%%%%%%%%%%%%%%%%%%%%%%%%%
%\newglossaryentry{sortierschluessel}
%{
%  name=Sortierschlüssel,
%  description={ein Schlüssel, anhand dessen diese Einträge sortiert werden}
%}
%\newacronym{abc}{Blub}{Bananarama}

\newglossaryentry{Abschluss}
{
  name={\ensuremath{\overline{M}}},
  description={Abschluss der Menge $M$},
  sort=Abschluss
}

\newglossaryentry{Rand}
{
  name={\ensuremath{\partial M}},
  description={Rand der Menge $M$},
  sort=Rand
}

\newglossaryentry{Inneres}
{
  name={\ensuremath{M^\circ}},
  description={Inneres der Menge $M$},
  sort=Inneres
}

\newglossaryentry{Kreuzprodukt}
{
  name={\ensuremath{A \times B}},
  description={Kreuzprodukt zweier Mengen},
  sort=Kreuzprodukt
}
\newglossaryentry{subseteq}
{
  name={\ensuremath{A \subseteq B}},
  description={Teilmengenbeziehung},
  sort=subseteq
}
\newglossaryentry{subsetneq}
{
  name={\ensuremath{A \subsetneq B}},
  description={echte Teilmengenbeziehung},
  sort=subsetneq
}

\newglossaryentry{R}
{
  name={\ensuremath{\mdr}},
  description={Reele Zahlen},
  sort=KoerperR
}

\newglossaryentry{Q}
{
  name={\ensuremath{\mdq}},
  description={Rationale Zahlen},
  sort=KoerperQ
}

\newglossaryentry{Z}
{
  name={\ensuremath{\mdz}},
  description={Ganze Zahlen},
  sort=KoerperZ
}

\newglossaryentry{Einheitengruppe}
{
  name={\ensuremath{\mdr^\times}},
  description={Multiplikative Einheitengruppe von $\mdr$},
  sort=GruppeEinheiten
}

% Setze den richtigen Namen für das Glossar
\renewcommand*{\glossaryname}{\glossarName}
\deftranslation{Glossary}{\glossarName}

% Drucke das gesamte Glossar
\glsaddall
\printglossaries

% Trage das Glossar in das Inhaltsverzeichnis ein
%\stepcounter{section}
%\addcontentsline{toc}{section}{\numberline {\thesection} \glossarName}
