%%%%%%%%%%%%%%%%%%%%%%%%%%%%%%%%%%%%%%%%%%%%%%%%%%%%%%%%%%%%%%%%%%%%%%
% Begriffslexikon zur Beschreibung des Produkts						 %
%%%%%%%%%%%%%%%%%%%%%%%%%%%%%%%%%%%%%%%%%%%%%%%%%%%%%%%%%%%%%%%%%%%%%%
%\newglossaryentry{sortierschluessel}
%{
%  name=Sortierschlüssel,
%  description={ein Schlüssel, anhand dessen diese Einträge sortiert werden}
%}
%\newacronym{abc}{Blub}{Bananarama}

%%%%%%%%%%%%%%%%%%%%%%%%%%%%%%%%%%%%%%%%%%%%%%%%%%%%%%%%%%%%%%%%%%%%%
% Mengenoperationen                                                 %
%%%%%%%%%%%%%%%%%%%%%%%%%%%%%%%%%%%%%%%%%%%%%%%%%%%%%%%%%%%%%%%%%%%%%
\newglossaryentry{Potenzmenge}
{
  name={\ensuremath{\mathcal{P}(M)}},
  description={Potenzmenge von $M$},
  sort=MengenoperationNPotenzmenge
}

\newglossaryentry{Abschluss}
{
  name={\ensuremath{\overline{M}}},
  description={Abschluss der Menge $M$},
  sort=MengenoperationFAbschluss
}

\newglossaryentry{Rand}
{
  name={\ensuremath{\partial M}},
  description={Rand der Menge $M$},
  sort=MengenoperationFRand
}

\newglossaryentry{Inneres}
{
  name={\ensuremath{M^\circ}},
  description={Inneres der Menge $M$},
  sort=MengenoperationFInneres
}

\newglossaryentry{Kreuzprodukt}
{
  name={\ensuremath{A \times B}},
  description={Kreuzprodukt zweier Mengen},
  sort=MengenoperationNKreuzprodukt
}
\newglossaryentry{subseteq}
{
  name={\ensuremath{A \subseteq B}},
  description={Teilmengenbeziehung},
  sort=MengenoperationNSubseteq
}
\newglossaryentry{subsetneq}
{
  name={\ensuremath{A \subsetneq B}},
  description={echte Teilmengenbeziehung},
  sort=MengenoperationNSubsetneq
}

\newglossaryentry{setminus}
{
  name={\ensuremath{A \setminus B}},
  description={$A$ ohne $B$},
  sort=MengenoperationNSetminus
}

%%%%%%%%%%%%%%%%%%%%%%%%%%%%%%%%%%%%%%%%%%%%%%%%%%%%%%%%%%%%%%%%%%%%%
% Zahlenmengen                                                      %
%%%%%%%%%%%%%%%%%%%%%%%%%%%%%%%%%%%%%%%%%%%%%%%%%%%%%%%%%%%%%%%%%%%%%
\newglossaryentry{N}
{
  name={\ensuremath{\mdn}},
  description={Natürliche Zahlen},
  sort=KoerperAN
}

\newglossaryentry{Z}
{
  name={\ensuremath{\mdz}},
  description={Ganze Zahlen},
  sort=KoerperAZ
}

\newglossaryentry{Q}
{
  name={\ensuremath{\mdq}},
  description={Rationale Zahlen},
  sort=KoerperBQ
}

\newglossaryentry{R}
{
  name={\ensuremath{\mdr}},
  description={Reele Zahlen},
  sort=KoerperR
}

\newglossaryentry{Rplus}
{
  name={\ensuremath{\mdr^+}},
  description={Echt positive reele Zahlen},
  sort=KoerperRplus
}

\newglossaryentry{Einheitengruppe}
{
  name={\ensuremath{\mdr^\times}},
  description={Multiplikative Einheitengruppe von $\mdr$},
  sort=KoerperREinheiten
}

\newglossaryentry{Komplexe Zahlen}
{
  name={\ensuremath{\mdc}},
  description={Komplexe Zahlen},
  sort=KoerperSComplexeZahlen
}

\newglossaryentry{Projektiver Raum}
{
  name={\ensuremath{\mdp}},
  description={Projektiver Raum},
  sort=KoerperXProjektion
}

%%%%%%%%%%%%%%%%%%%%%%%%%%%%%%%%%%%%%%%%%%%%%%%%%%%%%%%%%%%%%%%%%%%%%
% Fraktale Symbole                                                  %
%%%%%%%%%%%%%%%%%%%%%%%%%%%%%%%%%%%%%%%%%%%%%%%%%%%%%%%%%%%%%%%%%%%%%
\newglossaryentry{fB}
{
  name={\ensuremath{\fB}},
  description={Basis einer Topologie},
  sort=fB
}

\newglossaryentry{Epsilonumgebung}
{
  name={\ensuremath{\fB_\delta(x)}},
  description={$\delta$-Kugel um $x$},
  sort=fBr
}

\newglossaryentry{fT}
{
  name={\ensuremath{\fT}},
  description={Topologie},
  sort=fT
}

%%%%%%%%%%%%%%%%%%%%%%%%%%%%%%%%%%%%%%%%%%%%%%%%%%%%%%%%%%%%%%%%%%%%%
% Sonstiges                                                         %
%%%%%%%%%%%%%%%%%%%%%%%%%%%%%%%%%%%%%%%%%%%%%%%%%%%%%%%%%%%%%%%%%%%%%
\newglossaryentry{Skalarprodukt}
{
  name={\ensuremath{\langle \cdot , \cdot \rangle}},
  description={Skalarprodukt},
  sort=ZZZSkalarprodukt
}

\newglossaryentry{Modulo}
{
  name={\ensuremath{X /_\sim}},
  description={$X$ modulo $\sim$},
  sort=ZZZAuequivalenzModulo
}

\newglossaryentry{Modulo-Aequivalenzklasse}
{
  name={\ensuremath{[x]_\sim}},
  description={Äquivalenzklassen von $x$ bzgl. $\sim$},
  sort=ZZZAuequivalenzKlassen
}

\newglossaryentry{Norm}
{
  name={\ensuremath{\| x \|}},
  description={Norm von $x$},
  sort=ZZZNorm
}

\newglossaryentry{Betrag}
{
  name={\ensuremath{| x |}},
  description={Betrag von $x$},
  sort=ZZZNormBetrag
}

\newglossaryentry{Sphaere}
{
  name={\ensuremath{S^n}},
  description={Sphäre},
  sort=ZZZSphaere
}

\newglossaryentry{Torus}
{
  name={\ensuremath{T^n}},
  description={Torus},
  sort=ZZZSphaereTorus
}

\newglossaryentry{Projektion}
{
  name={\ensuremath{\pi_X}},
  description={Projektion auf X},
  sort=ZZZProjektion
}

\newglossaryentry{Urbild}
{
  name={\ensuremath{f^{-1}(M)}},
  description={Urbild von $M$},
  sort=ZZZUrbild
}

\newglossaryentry{Ohne Einschraekung}
{
  name={$\text{\OE}$},
  description={Ohne Einschränkung},
  sort=ZZZOE
}

\newglossaryentry{Allgemeine lineare Gruppe}
{
  name={$\GL_n(K)$},
  description={Allgemeine lineare Gruppe (general linear group)},
  sort=ZZZGL
}

% Setze den richtigen Namen für das Glossar
\renewcommand*{\glossaryname}{\glossarName}
\deftranslation{Glossary}{\glossarName}

% Drucke das gesamte Glossar
\glsaddall
\printglossaries

% Trage das Glossar in das Inhaltsverzeichnis ein
%\stepcounter{section}
%\addcontentsline{toc}{section}{\numberline {\thesection} \glossarName}
