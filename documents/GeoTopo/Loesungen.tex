\chapter*{Lösungen der Übungsaufgaben}
\addcontentsline{toc}{chapter}{Lösungen der Übungsaufgaben}
\begin{solution}[\ref{ub:aufg1}]
    \textbf{Teilaufgabe a)} Es gilt:
    \begin{enumerate}[label=(\roman*)]
        \item $\emptyset, X \in \fT_X$.
        \item $\fT_X$ ist offensichtlich unter Durchschnitten abgeschlossen,
              d.~h. es gilt für alle $U_1, U_2 \in \fT_X: U_1 \cap U_2 \in \fT_X$.
        \item Auch unter beliebigen Vereinigungen ist $\fT_X$ abgeschlossen,
              d.~h. es gilt für eine beliebige Indexmenge $I$ und alle
              $U_i \in \fT_X$ für alle $i \in I: \bigcup_{i \in I} U_i \in \fT_X$
    \end{enumerate}

    Also ist $(X, \fT_X)$ ein topologischer Raum.

    \textbf{Teilaufgabe b)} Wähle $x=1, y=0$. Dann gilt $x \neq y$
    und die einzige Umgebung von $x$ ist $X$. Da $y=0 \in X$ können
    also $x$ und $y$ nicht durch offene Mengen getrennt werden.
    $(X, \fT_X)$ ist also nicht hausdorffsch.

    \textbf{Teilaufgabe c)} Nach Bemerkung \ref{Trennungseigenschaft}
    sind metrische Räume hausdorffsch. Da $(X, \fT_X)$ nach (b) nicht 
    hausdorffsch ist, liefert die Kontraposition der Trennungseigenschaft,
    dass $(X, \fT_X)$ kein metrischer Raum sein kann.
\end{solution}

\begin{solution}[\ref{ub:aufg4}]
    \todo[inline]{Lösung schreiben}
\end{solution}
