\chapter*{Lösungen der Übungsaufgaben}
\addcontentsline{toc}{chapter}{Lösungen der Übungsaufgaben}
\begin{solution}[\ref{ub1:aufg1}]
    \textbf{Teilaufgabe a)} Es gilt:
    \begin{enumerate}[label=(\roman*)]
        \item $\emptyset, X \in \fT_X$.
        \item $\fT_X$ ist offensichtlich unter Durchschnitten abgeschlossen,
              d.~h. es gilt für alle $U_1, U_2 \in \fT_X: U_1 \cap U_2 \in \fT_X$.
        \item Auch unter beliebigen Vereinigungen ist $\fT_X$ abgeschlossen,
              d.~h. es gilt für eine beliebige Indexmenge $I$ und alle
              $U_i \in \fT_X$ für alle $i \in I: \bigcup_{i \in I} U_i \in \fT_X$
    \end{enumerate}

    Also ist $(X, \fT_X)$ ein topologischer Raum.

    \textbf{Teilaufgabe b)} Wähle $x=1, y=0$. Dann gilt $x \neq y$
    und die einzige Umgebung von $x$ ist $X$. Da $y=0 \in X$ können
    also $x$ und $y$ nicht durch offene Mengen getrennt werden.
    $(X, \fT_X)$ ist also nicht hausdorffsch.

    \textbf{Teilaufgabe c)} Nach Bemerkung \ref{Trennungseigenschaft}
    sind metrische Räume hausdorffsch. Da $(X, \fT_X)$ nach (b) nicht 
    hausdorffsch ist, liefert die Kontraposition der Trennungseigenschaft,
    dass $(X, \fT_X)$ kein metrischer Raum sein kann.
\end{solution}

\begin{solution}[\ref{ub1:aufg4}]
    \textbf{Teilaufgabe a)} 

    \textbf{Beh.:}  $\forall a \in \mdz: \Set{a}$ ist abgeschlossen.

    Sei $a \in \mdz$ beliebig. Dann gilt:
    \todo[inline]{Hat jemand diesen Beweis?}

    \textbf{Teilaufgabe b)} 

    \textbf{Beh.:} $\Set{-1, 1}$ ist nicht offen

    \textbf{Bew.:} durch Widerspruch

    Annahme: $\Set{-1, 1}$ ist offen.

    Dann gibt es $T \subseteq \fB$, sodass $\bigcup_{M \in T} M = \Set{-1, 1}$.
    Aber alle $U \in \fB$ haben unendlich viele Elemente. Auch endlich
    viele Schnitte von Elementen in $\fB$ haben unendlich viele
    Elemente $\Rightarrow$ keine endliche nicht-leere Menge kann
    in dieser Topologie offen sein $\Rightarrow \Set{-1,1}$ ist
    nicht offen. $\qed$

    \textbf{Teilaufgabe c)} 

    \textbf{Beh.:} Es gibt unendlich viele Primzahlen.

    \textbf{Bew.:} durch Widerspruch

    Annahme:  Es gibt nur endlich viele Primzahlen $p \in \mdp$

    Dann ist 
    \[\mdz \setminus \Set{-1, +1} \overset{\text{FS d. Arithmetik}}= \bigcup_{p \in \mdp} U_{0,p}\]
    endlich. Das ist ein Widerspruch zu $|\mdz|$ ist unendlich und
    $|\Set{-1,1}|$ ist endlich. $\qed$
\end{solution}

\begin{solution}[\ref{ub2:aufg4}]
    \begin{enumerate}[label=(\alph*)]
        \item \textbf{Beh.:} Die offenen Mengen von $P$ sind
              Vereinigungen von Mengen der Form 
              \[\prod_{j \in J} U_j \times \prod_{i \in \mdn, i \neq j} P_i\]
              wobei $J \subseteq \mdn$ endlich und $U_j \subseteq P_j$
              offen ist.
              \begin{beweis}
                Nach Definition der Produkttopologie bilden Mengen
                der Form
                \[\prod_{i \in J} U_j \times \prod_{\stackrel{i \in \mdn}{i \notin J}} P_i, \text{ wobei } J \subseteq \mdn \text{ endlich und } U_j \subseteq P_j \text{offen } \forall{j \in J}\]
                eine Basis der Topologie. Damit sind die offenen 
                Mengen von $P$ Vereinigungen von Mengen der obigen
                Form. $\qed$
              \end{beweis}
        \item \textbf{Beh.:} Die Zusammenhangskomponenten von $P$
              sind alle einpunktig.\xindex{Total Unzusammenhängend}
              \begin{beweis}
                Es seinen $x,y \in P$ und $x$ sowie $y$ liegen in der
                gleichen Zusammenhangskomponente $Z \subseteq P$.
                Da $Z$ zusammenhängend ist und $\forall{i \in I}: p_i : P \rightarrow P_i$
                ist stetig, ist $p_i(Z) \subseteq P_i$ zusammenhängend
                für alle $i \in \mdn$. Die zusammenhängenden Mengen
                von $P_i$ sind genau $\Set{0}$ und $\Set{1}$, d.~h.
                für alle $i \in \mdn$ gilt entweder $p_i(Z) \subseteq \Set{0}$
                oder $p_i(Z) \subseteq \Set{1}$. Es sei $z_i \in \Set{0,1}$
                so, dass $p_i(Z) \subseteq \Set{z_i}$ für alle $i \in \mdn$.
                Dann gilt also: 
                \[\underbrace{p_i(x)}_{= x_i} = z_i = \underbrace{p_i(y)}_{= y_i} \forall i \in \mdn\]
                Somit folgt: $x = y \qed$
                
              \end{beweis}
    \end{enumerate}
\end{solution}

\begin{solution}[\ref{ub4:aufg1}]
    \begin{enumerate}[label=(\alph*)]
        \item \textbf{Vor.:} Sei $M$ eine topologische Mannigfaltigkeit.\\
              \textbf{Beh.:} $M$ ist wegzusammehängend $\gdw M$ ist zusammenhängend
              \begin{beweis}
                \enquote{$\Rightarrow$}: Da $M$ insbesondere ein
                topologischer Raum ist folgt diese Richtung direkt 
                aus Korollar~\ref{kor:wegzusammehang-impliziert-zusammenhang}.

                \enquote{$\Leftarrow$}: Seien $x,y \in M$ und
                \[Z := \Set{z \in M | \exists \text{Weg von } x \text{ nach } z}\]
                Es gilt:
                \begin{enumerate}[label=(\roman*)]
                    \item $Z \neq \emptyset$, da $M$ lokal wegzusammenhängend ist
                    \item $Z$ ist offen, da $M$ lokal wegzusammenhängend ist
                    \item $Z^C := \Set{\tilde{z} \in M | \nexists \text{Weg von } x \text{ nach } \tilde{z}}$ ist offen

                    Da $M$ eine Mannigfaltigkeit ist, existiert zu jedem
                    $\tilde{z} \in Z^C$ eine offene und wegzusammenhängende Umgebung 
                    $U_{\tilde{z}} \subseteq M$.

                    Es gilt sogar $U_{\tilde{z}} \subseteq Z^C$, denn
                    gäbe es ein $U_{\tilde{z}} \ni \overline{z} \in Z$,
                    so gäbe es Wege $\gamma_2:[0,1] \rightarrow M, \gamma_2(0) = \overline{z}, \gamma_2(1) = x$
                    und $\gamma_1:[0,1] \rightarrow M, \gamma_1(0) = \tilde{z}, \gamma_1(1) = \overline{z}$.
                    Dann wäre aber
                    \[\gamma:[0,1] \rightarrow M,\;\;\; \gamma(x) = \begin{cases}
                        \gamma_1(2x)   &\text{falls } 0 \leq x \leq \frac{1}{2}\\
                        \gamma_2(2x-1) &\text{falls } \frac{1}{2} < x \leq 1
                        \end{cases}\]
                    ein stetiger Weg von $\tilde{z}$ nach $x$
                    $\Rightarrow$ Widerspruch.

                    Da $M$ zusammenhängend ist und $M = \underbrace{Z}_{\mathclap{\text{offen}}} \cup \underbrace{Z^C}_{\mathclap{\text{offen}}}$,
                    sowie $Z \neq \emptyset$ folgt $Z^C = \emptyset$.
                    Also ist $M=Z$ wegzusammenhängend.$\qed$
                \end{enumerate}
              \end{beweis}
        \item \textbf{Beh.:} $X$ ist wegzusammenhängend.\\
            \begin{beweis}
                $X:= (\mdr \setminus \Set{0}) \cup \Set{0_1, 0_2}$
                und $(\mdr \setminus \Set{0}) \cup \Set{0_2}$ sind
                homöomorph zu $\mdr$. Also sind die einzigen kritischen
                Punkte, die man nicht verbinden können könnte
                $0_1$ und $0_2$.

                Da $(\mdr \setminus \Set{0}) \cup \Set{0_1}$ homöomorph
                zu $\mdr$ ist, exisitert ein Weg $\gamma_1$ von $0_1$
                zu einem beliebigen Punkt $a \in \mdr \setminus \Set{0}$.
                
                Da $(\mdr \setminus \Set{0}) \cup \Set{0_2}$ ebenfalls
                homöomorph zu $\mdr$ ist, existiert außerdem ein Weg
                $\gamma_2$ von $a$ nach $0_2$. Damit existiert ein
                (nicht einfacher)
                Weg $\gamma$ von $0_1$ nach $0_2$. $\qed$
            \end{beweis}
    \end{enumerate}
\end{solution}
