%%%%%%%%%%%%%%%%%%%%%%%%%%%%%%%%%%%%%%%%%%%%%%%%%%%%%%%%%%%%%%%%%%%%%
% Henriekes Mitschrieb vom 07.11.2013                               %
%%%%%%%%%%%%%%%%%%%%%%%%%%%%%%%%%%%%%%%%%%%%%%%%%%%%%%%%%%%%%%%%%%%%%
\chapter{Mannigfaltigkeiten und Simpizidkomplexe}
\section{Topologische Mannigfaltigkeiten}
\begin{definition}
    Sei $X$ ein topologischer Raum und $n \in \mdn$.
    \begin{enumerate}[label=(\alph*)]
        \item Eine $n$-dimensionale \textbf{Karte}\xindex{Karte} auf
              $X$ ist ein Paar $(U, \varphi)$, wobei $U \subseteq X$
              offen und $\varphi: U \rightarrow V$ Homöomorphismus
              von $U$ auf eine offene Teilmenge $V \subseteq \mdr^n$.
        \item Ein $n$-dimensionaler \textbf{Atlas}\xindex{Atlas} auf $X$ ist eine
              Familie $(U_i, \varphi_i)_{i \in I}$ von Karten auf $X$,
              sodass $\bigcup_{i \in I} U_i = X$.
        \item $X$ heißt (topologische) $n$-dimensionale \textbf{Mannigfaltigkeit}\xindex{Mannigfaltigkeit},
              wenn $X$ hausdorffsch ist, eine abzählbare Basis der 
              Topologie hat und ein $n$-dimensionalen Atlas besitzt.
    \end{enumerate}
\end{definition}

\begin{bemerkung}
    \begin{enumerate}[label=(\alph*)]
        \item Es gibt surjektive, stetige Abbildungen $[0,1] \rightarrow [0,1] \times [0,1]$
        \item Für $n \neq m$ sind $\mdr^n$ und $\mdr^m$ nicht homöomorph.
              Zum Beweis benutzt man den \enquote{Satz von der Gebietstreue} (Brouwer):

              Ist $U \subseteq \mdr^n$ offen und $f: U \rightarrow \mdr^n$
              stetig und injektiv, so ist $f(U)$ offen.

              Ist $n < m$ und $\mdr^m$ homöomorph zu $\mdr^n$, so wäre
              \[f:\mdr^n \rightarrow \mdr^m \rightarrow \mdr^n, \;\;\; (x_1, \dots, x_n) \mapsto (x_1, x_2, \dots, x_n, 0, \dots, 0)\]
              eine stetige injektive Abbildung. Also müsste $f(\mdr^n)$
              offen sein $\Rightarrow$ Widerspruch
    \end{enumerate}
\end{bemerkung}

\begin{beispiel}
    \begin{enumerate}[label=\arabic*)]
        \item Jede offene Teilmenge $U \subseteq \mdr^n$ ist eine 
              $n$-dimensionale Mannigfaltigkeit mit einem Atlas aus 
              einer Karte.
        \item $\mdc^n$ ist eine $2n$-dimensionale Mannigfaltigkeit
              mit einem Atlas aus einer Karte:
              \[(z_1, \dots, z_n) \mapsto (\operatorname{Re} z_1, \operatorname{Im}z_1, \dots, \operatorname{Re}z_n, \operatorname{Im}z_n)\]
        \item $\mdp^n(\mdr) = (\mdr^{n+1} \setminus \Set{0})/_\sim = S^n /_\sim$ und $\mdp^n(\mdc)$ sind Mannigfaltigkeiten 
              der Dimension $n$ bzw. $2n$.

              $\mdp^n(\mdr) = \bigcup_{i=0}^n U_i,$
              \begin{align*}
U_i = \Set{(x_0: \dots : x_n) \in \mdp^n(\mdr) | x_i \neq 0} &\rightarrow \mdr^n\\
                (x_0 : \dots : x_n) &\mapsto \left (\frac{x_0}{x_i}, \dots, \frac{x_i}{x_i}, \dots, \frac{x_n}{x_i} \right )\\
                (y_1 : \dots : y_{i-1} : 1 : y_i : \dots : y_n) &\mapsfrom (y_1, \dots, y_n)
              \end{align*}
              ist bijektiv.

              Die $U_i,\; i = 0, \dots, n$ bilden einen $n$-dimensionalen Atals.
              \begin{align*}
                      x &= (1:0:0)            &y &= (0:1:1) \in U_2 \rightarrow \mdr^2\\
                \in U_0 &\rightarrow \mdr^2   &y &\mapsto (0,1)\\
                      x &\mapsto (0,0)        &&\text{Umgebung: } \fB_1 (0,1) \rightarrow \Set{(w:z:1) | w^2 + z^2 < 1} = V_2
              \end{align*}
              Umgebung $\fB_1(0,1) \rightarrow \Set{(1:u:v) | \|(u,v)\| < 1} = v_1$

              $V_1 \cap V_2 = \emptyset$?

              $(a:b:c) \in V_1 \cap V_2$\\
              $\Rightarrow a \neq 0$ und $(\frac{b}{a})^2 + (\frac{c}{a})^2 < 1 \Rightarrow \frac{c}{a} < 1$\\
              $\Rightarrow c \neq 0$ und $(\frac{a}{c})^2 + (\frac{b}{c})^2 < 1 \Rightarrow \frac{a}{c} < 1$\\
              $\Rightarrow$ Widerspruch
        \item $S^n = \Set{x \in \mdr^{n+1} | \|x\| = 1}$ ist $n$-dimensionale
              Mannigfaltigkeit.

              Karten: $O_i := \Set{(x_1, \dots, x_{n+1}) \in S^n | x_i > 0} \rightarrow \fB_1 (\underbrace{0, \dots, 0}_{\in \mdr^n})$\\
              $(x_1, \dots, x_{n+1}) \mapsto (x_1, \dots, x_i, \dots, x_{n+1})$\\
              $(x_1, \dots, x_{i-1}, \sqrt{1 - \sum x^2}) \mapsfrom (x_1, \dots, x_n)$\todo{was genau steht hier?}\\
              $S^n = \bigcup_{i=1}^{n+1} (c_i \cup D_i)$
        \item $[0,1]$ ist keine Mannigfaltigkeit, denn:\\
              Es gibt keine Umgebung von $0$ in $[0,1]$, die homöomorph
              zu einem offenem Intervall ist.
        \item $V_1 = \Set{(x,y) \in \mdr^2 | x \cdot y = 0}$ ist
              keine Mannigfaltigkeit.
        \item $V_2 = \Set{(x,y) \in \mdr^2 | x^3 = y^2}$ ist eine
              Mannigfaltigkeit.
        \item $X = (\mdr \setminus \Set{0}) \cup (O_1, O_2)$

              \[U \subseteq X \text{ offen } \gdw 
                \begin{cases}
                    U \text{ offen in } \mdr \setminus \Set{0}, &\text{falls } O_1 \notin U, O_2 \in U\\
                    \exists \varepsilon > 0 \text{ mit } (-\varepsilon, \varepsilon) \subseteq U &\text{falls } O_1 \in U, O_2 \in U
                \end{cases}\]
              Insbesondere sind $(\mdr \setminus \Set{0}) \cup \Set{O_1}$
              und $(\mdr \setminus \Set{0}) \cup \Set{O_2}$ offen und
              homöomorph zu $\mdr$.

              \underline{Aber:} $X$ ist nicht hausdorffsch!
              Denn es gibt keine disjunkten Umgebungen von $O_1$ und
              $O_2$.
        \item $\GL_n(\mdr)$ ist eine Mannigfaltigkeit der Dimension 
              $n^2$, weil offene Teilmengen von $\mdr^{n^2}$ eine
              Mannigfaltigkeit bilden.
    \end{enumerate}
\end{beispiel}

% Die Übungsaufgaben sollen ganz am Ende des Kapitels sein.
\clearpage
\section*{Übungsaufgaben}
\addcontentsline{toc}{section}{Übungsaufgaben}

\begin{aufgabe}[Zusammenhang]\label{ub4:aufg1}
    \begin{enumerate}[label=(\alph*)]
        \item Beweisen Sie, dass eine topologische Mannigfaltigkeit
              genau dann wegzusammenhängend ist, wenn sie zusammenhängend
              ist
        \item Betrachten Sie nun wie in \cref{bsp:mannigfaltigkeit8}
              den Raum $X:= (\mdr \setminus \Set{0}) \cup \Set{0_1, 0_2}$
              versehen mit der dort definierten Topologie. Ist $X$
              wegzusammenhängend?
    \end{enumerate}
\end{aufgabe}

