\clearpage
\section*{Übungsaufgaben}
\addcontentsline{toc}{section}{Übungsaufgaben}

\begin{aufgabe}\label{ub11:aufg1}
    Seien $(X, d)$ eine absolute Ebene und $P, Q, R \in X$ Punkte.
    Der \textit{Scheitelwinkel}\xindex{Scheitelwinkel} des Winkels $\angle PQR$ ist
    der Winkel, der aus den Halbgeraden $QP^-$ und $QR^-$ gebildet
    wird. Die \textit{Nebenwinkel}\xindex{Nebenwinkel} von $\angle PQR$
    sind die von $QP^+$ und $QR^-$ bzw. $QP^-$ und $QR^+$ gebildeten
    Winkel.

    Zeigen Sie:
    \begin{aufgabeenum}
        \item Die beiden Nebenwinkel von $\angle PQR$ sind gleich.
        \item Der Winkel $\angle PQR$ ist gleich seinem Scheitelwinkel.
    \end{aufgabeenum}
\end{aufgabe}

\begin{aufgabe}\label{ub11:aufg3}
    Sei $(X, d)$ eine absolute Ebene. Der \textit{Abstand}\xindex{Abstand} eines 
    Punktes $P$ zu einer Menge $Y \subseteq X$ von Punkten ist
    definiert durch $d(P, Y) := \inf{d(P, y) | y \in Y}$.

    Zeigen Sie:
    \begin{aufgabeenum}
        \item \label{ub11:aufg3.a} Ist $\triangle ABC$ ein Dreieck, in dem die Seiten
              $\overline{AB}$ und $\overline{AC}$ kongruent sind, so
              sind die Winkel $\angle ABC$ und $\angle BCA$ gleich.
        \item \label{ub11:aufg3.b} Ist $\triangle ABC$ ein beliebiges Dreieck, so liegt 
              der längeren Seite der größere Winkel gegenüber und
              umgekehrt.
        \item \label{ub11:aufg3.c} Sind $g$ eine Gerade und $P \notin g$ ein Punkt, so gibt
              es eine eindeutige Gerade $h$ mit $P \in h$ und die
              $g$ im rechten Winkel schneidet. Diese Grade heißt 
              \textit{Lot}\xindex{Lot} von $P$ auf $g$ und der 
              Schnittpunkt des Lots mit $g$ heißt \textit{Lotfußpunkt}\xindex{Lotfußpunkt}.
    \end{aufgabeenum}
\end{aufgabe}

\begin{aufgabe}\label{ub-tut-24:a1}
    Seien $f, g, h \in G$ und paarweise verschieden.

    Zeigen Sie: $f \parallel g \land g \parallel h \Rightarrow f \parallel h$
\end{aufgabe}

\begin{aufgabe}\label{ub-tut-24:a3}%
    Beweise den Kongruenzsatz $SSS$.
\end{aufgabe}
