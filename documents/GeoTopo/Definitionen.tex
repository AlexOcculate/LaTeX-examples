%!TEX root = GeoTopo.tex
\markboth{Anhang: Definitionen und Sätze}{Anhang: Definitionen und Sätze}
\chapter*{Anhang: Definitionen und Sätze}
\addcontentsline{toc}{chapter}{Anhang: Definitionen und Sätze}

Da dieses Skript in die Geometrie und Topologie einführen soll, sollten soweit
wie möglich alle benötigten Begriffe definiert und erklärt werden. Die folgenden
Begriffe wurden zwar verwendet, aber nicht erklärt, da sie Bestandteil der
Vorlesungen \enquote{Analysis I und II} sowie \enquote{Lineare Algebra und analytische Geometrie I und II}
sind. Jedoch will ich zumindest die Definitionen bereitstellen.

\begin{definition}\xindex{Häufungspunkt}%
	Sei $D \subseteq \mdr$ und $x_0 \in \mdr$. $x_0$ heißt ein \textbf{Häufungspunkt}
	von $D :\gdw \exists$ Folge $x_n$ in $D \setminus \Set{x_0}$ mit $x_n \rightarrow x_0$.
\end{definition}

Folgende Definition wurde dem Skript von Herrn Prof.~Dr.~Leuzinger für
Lineare Algebra entnommen:

\begin{definition}\xindex{Abbildung!affine}%
	Es seien $V$ und $W$ $\mdk$-Vektorräume und $\mda(V)$ und $\mda(W)$ die 
	zugehörigen affinen Räume. Eine Abbildung $f:V \rightarrow W$ heißt \textbf{affin},
	falls für alle $a, b \in V$ und alle $\lambda, \mu \in \mdk$ mit $\lambda + \mu = 1$ gilt:
	\[f(\lambda a + \mu b) = \lambda f(a) + \mu f(b)\]
\end{definition}

\begin{definition}\xindex{Orthonormalbasis}%
	Sei $V$ ein Vektorraum und $S \subseteq V$ eine Teilmenge.

	$S$ heißt eine \textbf{Orthonormalbasis} von $V$, wenn gilt:
	\begin{defenumprops}
		\item $S$ ist eine Basis von $V$
		\item $\forall v \in S: \|v\| = 1$
		\item $\forall v_1, v_2 \in S: v_1 \neq v_2 \Rightarrow \langle v_1, v_2 \rangle = 0$
	\end{defenumprops}
\end{definition}

\begin{satz*}[Zwischenwertsatz]\xindex{Zwischenwertsatz}%
	Sei $a<b$ und $f \in\ C[a, b]:=C([a, b])$, weiter sei $y_0 \in \mdr$ und 
	$f(a) < y_0 < f(b)$ oder $f(b) < y_0 < f(a)$. Dann existiert ein 
	$x_0 \in [a, b]$ mit $f(x_0) = y_0$.
\end{satz*}