\documentclass[a5paper,oneside]{scrbook}
\usepackage{etoolbox}
\usepackage{amsmath,amssymb}% math symbols / fonts
\usepackage{mathtools}      % \xRightarrow
\usepackage{nicefrac}       % \nicefrac
\usepackage[utf8]{inputenc} % this is needed for umlauts
\usepackage[ngerman]{babel} % this is needed for umlauts
\usepackage[T1]{fontenc}    % this is needed for correct output of umlauts in pdf
\usepackage[framed,amsmath,thmmarks,hyperref]{ntheorem}
\usepackage{framed}
\usepackage{marvosym}
\usepackage{makeidx}        % for automatically generation of an index
\usepackage{xcolor}
\usepackage[bookmarks,bookmarksnumbered,hypertexnames=false,pdfpagelayout=OneColumn,colorlinks,hyperindex=false]{hyperref} % has to be after makeidx
\usepackage{enumitem}       % Better than \usepackage{enumerate}, because it allows to set references
\usepackage{tabto}
\usepackage{braket}         % needed for \Set
\usepackage{csquotes}       % \enquote{}
\usepackage{subfig}         % multiple figures in one
\usepackage{parskip}        % nicer paragraphs
\usepackage{xifthen}        % \isempty
\usepackage{changepage}     % for the adjustwidth environment
\usepackage{pst-solides3d}
\usepackage[colorinlistoftodos]{todonotes}
\usepackage{pgfplots}
\pgfplotsset{compat=1.7}
\usepackage[arrow, matrix, curve]{xy}
\usepackage{caption}        % get newlines within captions
\usepackage{tikz}           % draw
\usepackage{tikz-3dplot}    % draw
\usepackage{tkz-fct}        % draw
\usepackage{tkz-euclide}    % draw
\usetkzobj{all}             % tkz-euclide
\usetikzlibrary{3d,calc,intersections,er,arrows,positioning,shapes.misc,patterns,fadings,decorations.pathreplacing}
\usepackage{tqft}
\usepackage{xspace}   % for new commands; decides weather I want to insert a space after the command
\usepackage[german,nameinlink]{cleveref} % has to be after hyperref, ntheorem, amsthm
\usepackage[left=10mm,right=10mm, top=2mm, bottom=10mm]{geometry}
\usepackage{../shortcuts}

\hypersetup{ 
  pdfauthor   = {Martin Thoma}, 
  pdfkeywords = {Geometrie und Topologie}, 
  pdftitle    = {Fragen zu Definitionen} 
}
\allowdisplaybreaks

%%%%%%%%%%%%%%%%%%%%%%%%%%%%%%%%%%%%%%%%%%%%%%%%%%%%%%%%%%%%%%%%%%%%%
% Begin document                                                    %
%%%%%%%%%%%%%%%%%%%%%%%%%%%%%%%%%%%%%%%%%%%%%%%%%%%%%%%%%%%%%%%%%%%%%
\begin{document}
\chapter{Fragen zu Definitionen}
\section*{6.) Basisbeispiele}
\todo[inline]{Kennst du ein Beispiel für eine Subbasis in einem Topologischen Raum,
die keine Basis ist?}

Wie ist es mit folgendem?

Sei $(X, \fT)$ ein topologischer Raum mit 
  $X = \Set{0,1,2}$ und $\fT = \Set{\emptyset, \Set{0}, \Set{0,1}, X}$.\\
  Dann ist $\calS = \Set{\emptyset, \Set{0,1}, \Set{0,2}}$ eine Subbasis von 
  $\fT$, da gilt:
  \begin{itemize}
    \item $\emptyset \in \calS$
    \item $\Set{0} = \Set{0, 1} \cap \Set{0,2}$
    \item $\Set{0,1} \in \calS$
    \item $X = \Set{0,1} \cup \Set{0,2}$
  \end{itemize}
  Allerings ist $\calS$ keine Basis von $(X, \fT)$, da
  $\Set{0}$ nicht als Vereinigung von Elementen aus $\calS$
  erzeugt werden kann.


\section*{9.) Mannigfaltigkeit mit Rand}
\begin{definition}%
    Sei $X$ ein topologischer Raum und $n \in \mdn$.
    \begin{defenum}
        \item Eine $n$-dimensionale \textbf{Karte}\xindex{Karte} auf
              $X$ ist ein Paar $(U, \varphi)$, wobei $U \subseteq X$
              offen und $\varphi: U \rightarrow V$ Homöomorphismus
              von $U$ auf eine offene Teilmenge $V \subseteq \mdr^n$.
        \item Ein $n$-dimensionaler \textbf{Atlas}\xindex{Atlas} $\atlas$ auf $X$ ist eine
              Familie $(U_i, \varphi_i)_{i \in I}$ von Karten auf $X$,
              sodass $\bigcup_{i \in I} U_i = X$.
        \item $X$ heißt (topologische) $n$-dimensionale \textbf{Mannigfaltigkeit}\xindex{Mannigfaltigkeit},
              wenn $X$ hausdorffsch ist, eine abzählbare Basis der 
              Topologie hat und ein $n$-dimensionalen Atlas besitzt.
    \end{defenum}
\end{definition}
\begin{definition}\xindex{Mannigfaltigkeit!mit Rand}%
    Sei $X$ ein Hausdorffraum mit abzählbarer Basis der Topologie.
    $X$ heißt $n$-dimensionale \textbf{Mannigfaltigkeit mit Rand},
    wenn es einen Atlas $(U_i, \varphi_i)$ gibt, wobei $U_i \subseteq X_i$
    offen und $\varphi_i$ ein Homöomorphismus auf eine offene 
    Teilmenge von 
    \[R_{+,0}^n := \Set{(x_1, \dots, x_n) \in \mdr^n | x_{\color{red}m} \geq 0}\]
    ist.
\end{definition}

\todo[inline]{Sind Mannigfaltigkeiten mit Rand auch Mannigfaltigkeiten? Sollte das rote $m$ eventuell $n$ sein? Oder sollte es ein $i$ sein, mit $i = 1..n$?}

Laut \url{https://de.wikipedia.org/wiki/Mannigfaltigkeit_mit_Rand}:

\enquote{Eine Mannigfaltigkeit mit Rand ist mathematisches Objekt aus der Differentialgeometrie. Es handelt sich hierbei nicht um einen Spezialfall einer Mannigfaltigkeit, sondern ganz im Gegenteil um eine Verallgemeinerung.}

\todo[inline]{Ist die Aussage auf Wikipedia korrekt? Für mich sieht das so aus, also ob folgende Definition auch richtig wäre:}

\begin{definition}\xindex{Mannigfaltigkeit!mit Rand}%
    Sei $X$ eine Mannigfaltigkeit mit Atlas $\atlas$ und
    \[\mdr_{+,0}^n := \Set{(x_1, \dots, x_n) \in \mdr^n | x_m \geq 0}\]

    $X$ heißt \textbf{Mannigfaltigkeit mit Rand}, wenn gilt:
    \[\forall (U, \varphi) \in \atlas: \varphi(U) \subseteq \mdr_{+,0}^n\]
\end{definition}

\section*{11.) Produkttopologie}
\begin{definition}\xindex{Produkttopologie}%
    Seien $X_1, X_2$ topologische Räume.\\
    $U \subseteq X_1 \times X_2$ sei offen, wenn es zu jedem $x = (x_1, x_2) \in U$
    Umgebungen $U_i$ um $x_i$  mit $i=1,2$ gibt, sodass $U_1 \times U_2 \subseteq U$
    gilt.

    $\fT = \Set{U \subseteq X_1 \times X_2 | U \text{ offen}}$
    ist eine Topologie auf $X_1 \times X_2$. Sie heißt \textbf{Produkttopologie}.
    $\fB = \Set{U_1 \times U_2 | U_i \text{ offen in } X_i, i=1,2}$
    ist eine Basis von $\fT$.
\end{definition}

\todo[inline]{Gibt es ein Beispiel, das zegit, dass nicht $\fB = \fT$ gilt?}

\section*{15.) Existenz der Parallelen}
\begin{definition}%
    \begin{enumerate}[label=§\arabic*),ref=§\arabic*,start=5]
        \item \label{axiom:5}\textbf{Parallelenaxiom}\xindex{Parallele}:
            Für jedes $g \in G$ und jedes
            $P \in X \setminus g$ gibt es höchstens ein $h \in G$ mit
            $h \cap g = \emptyset$. $h$ heißt \textbf{Parallele zu $g$ durch $P$}.
    \end{enumerate}
\end{definition}

\todo[inline]{Wie beweist man, dass es genau eine gibt? (Verschiebung der Geraden in den entsprechenden Punkt mit der Isometrie, die die Halbebenen gleich lässt)}

\section*{17.) Simpliziale Abbildungen}
Wenn man Simpliziale Abbildungen wie folgt definiert

\begin{definition}\xindex{Abbildung!simpliziale}%
    Seien $K, L$ Simplizialkomplexe. Eine stetige Abbildung
    \[f:|K| \rightarrow |L|\]
    heißt \textbf{simplizial}, wenn für
    jedes $\Delta \in K$ gilt:
    \begin{defenum}
        \item $f(\Delta) \in L$
        \item $f|_{\Delta} : \Delta \rightarrow f(\Delta)$ ist eine
              affine Abbildung.
    \end{defenum}
\end{definition}

\todo[inline]{Dann ist die Forderung \enquote{$f(\Delta) \in L$} doch immer erfüllt, oder?
Gibt es eine Abbildung
$f:|K| \rightarrow |L|$
mit $f(\Delta) \notin L$?}

\section*{18.) ÜB 1, Aufgabe 2}
\underline{Vor.:} Es sei $(X, d)$ ein metrischer Raum, $A \subseteq X$. 
Weiter bezeichne $\fT$ die von $d$ auf $X$ erzeugte Topologie $\fT'$, die von
der auf $A \times A$ eingeschränkten Metrik $d|_{A \times A}$ erzeugte Topologie.

\underline{Beh.:} Die Topologie $\fT'$ und $\fT|_A$ (Spurtopologie) stimmen überein.

\underline{Bew.:}

\enquote{$\fT|_A \subseteq \fT'$}:

Sei $U \in \fT|_A = \Set{V \cap A | V \in \fT}$.\\
Dann ex. also $V \in \fT$ mit
$U = V \cap A$.\\
Sei $x \in U$.\\
Da $V \in \fT$, ex. nach Bemerkung~3 ein $r > 0$ mit 

\begin{align*}
    \fB_r(x) := \Set{y \in X | d(x,y) < r} &\subseteq V\\
                \Set{y \in A | d(x,y) < r} &\subseteq V \cap A = U
\end{align*}
also ist $U$ offen bzgl. $d|_{A \times A}$.
\todo[inline]{Wieso ist $U$ offen bzgl. $d|_{A \times A}$?}
Da $x \in U$ beliebig gewählt war gilt: $\fT|_A \subseteq \fT'$

\section*{19.) Topologische Gruppe und stetige Gruppenoperation}
\begin{definition}%
    Sei $G$ eine Mannigfaltigkeit und $(G, \circ)$ eine Gruppe.

    \begin{defenum}
        \item $G$ heißt \textbf{topologische Gruppe}\xindex{Gruppe!topologische},
              wenn die Abbildungen $\circ: G \times G \rightarrow G$
              und $\iota: G \rightarrow G$ definiert durch
              \[g \circ h := g \cdot h \text{ und } \iota(g) := g^{-1}\]
              stetig sind.
        \item Ist $G$ eine differenzierbare Mannigfaltigkeit, so heißt
              $G$ \textbf{Lie-Gruppe}\xindex{Lie-Gruppe}, wenn
              $(G, \circ)$ und $(G, \iota)$ differenzierbar sind.
    \end{defenum}
\end{definition}

\begin{definition}
    Sei $G$ eine Gruppe, $X$ ein topologischer Raum und
    $\circ: G \times X \rightarrow X$ eine Gruppenoperation.

    \begin{defenum}
        \item \xindex{Gruppe operiert durch Homöomorphismen}\textbf{$G$ operiert durch Homöomorphismen}, wenn für jedes $g \in G$
              die Abbildung
              \[m_g: X \rightarrow X, x \mapsto g \circ x\]
              ein Homöomorphismus ist.
        \item Ist $G$ eine topologische Gruppe, so heißt die Gruppenoperation $\circ$
              \textbf{stetig}\xindex{Gruppenoperation!stetige}, wenn 
              $\circ: G \times X \rightarrow X$ stetig ist.
    \end{defenum}
\end{definition}

\todo[inline]{Wenn $G$ eine topologische Gruppe ist, dann ist $\circ$ doch auf jeden Fall stetig! Was soll die Definition? Des Weiteren verstehe ich $g \circ h := g \cdot h$ nicht. Was ist $\cdot$?}

\section*{Hyperbolische Metrik und Geraden}
\begin{definition}\xindex{Gerade!hyperbolische}%
    Sei
        \[\mdh:= \Set{z \in \mdc | \Im(z) > 0} = \Set{(x,y) \in \mdr^2 | y > 0}\]
    die obere Halbebene bzw. Poincaré-Halbebene und $G = G_1 \cup G_2$
    mit
        \begin{align*}
            G_1 &= \Set{g_1 \subseteq \mdh | \exists m \in \mdr, r \in \mdr_{>0}: g_1 = \Set{z \in \mdh : |z-m|=r}}\\
            G_2 &= \Set{g_2 \subseteq \mdh | \exists x \in \mdr: g_2 = \Set{z \in \mdh: \Re(z) = x}}
        \end{align*}

    Die Elemente von $\mdh$ heißen \textbf{hyperbolische Geraden}.
\end{definition}
\begin{definition}\xindex{Metrik!hyperbolische}%
    Für $z_1, z_2 \in \mdh$ sei $g_{z_1, z_2}$ die eindeutige hyperbolische
    Gerade durch $z_1$ und $z_2$ und $a_1, a_2$ die
    \enquote{Schnittpunkte} von $g_{z_1, z_2}$ mit $\mdr \cup \Set{\infty}$.

    Dann sei $d(z_1, z_2) := \frac{1}{2} \ln |\DV(a_1, z_4, a_2, z_2) |$
    und heiße \textbf{hyperbolische Metrik}.
\end{definition}

\todo[inline]{Wir haben hyperbolische Geraden mit der euklidischen Metrik beschrieben. Kann man hyperbolische Geraden auch mit der hyperbolischen Metrik beschreiben? Wie?}
vgl. Beweis von Bemerkung 68 b)
\end{document}