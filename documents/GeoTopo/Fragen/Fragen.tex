\documentclass[a5paper,oneside]{scrbook}
\usepackage{etoolbox}
\usepackage{amsmath,amssymb}% math symbols / fonts
\usepackage{mathtools}      % \xRightarrow
\usepackage{nicefrac}       % \nicefrac
\usepackage[utf8]{inputenc} % this is needed for umlauts
\usepackage[ngerman]{babel} % this is needed for umlauts
\usepackage[T1]{fontenc}    % this is needed for correct output of umlauts in pdf
\usepackage[framed,amsmath,thmmarks,hyperref]{ntheorem}
\usepackage{framed}
\usepackage{marvosym}
\usepackage{makeidx}        % for automatically generation of an index
\usepackage{xcolor}
\usepackage[bookmarks,bookmarksnumbered,hypertexnames=false,pdfpagelayout=OneColumn,colorlinks,hyperindex=false]{hyperref} % has to be after makeidx
\usepackage{enumitem}       % Better than \usepackage{enumerate}, because it allows to set references
\usepackage{tabto}
\usepackage{braket}         % needed for \Set
\usepackage{csquotes}       % \enquote{}
\usepackage{subfig}         % multiple figures in one
\usepackage{parskip}        % nicer paragraphs
\usepackage{xifthen}        % \isempty
\usepackage{changepage}     % for the adjustwidth environment
\usepackage{pst-solides3d}
\usepackage[colorinlistoftodos]{todonotes}
\usepackage{pgfplots}
\pgfplotsset{compat=1.7}
\usepackage[arrow, matrix, curve]{xy}
\usepackage{caption}        % get newlines within captions
\usepackage{cancel}
\usepackage{tikz}           % draw
\usepackage{tikz-3dplot}    % draw
\usepackage{tkz-fct}        % draw
\usepackage{tkz-euclide}    % draw
\usetkzobj{all}             % tkz-euclide
\usetikzlibrary{3d,calc,intersections,er,arrows,positioning,shapes.misc,patterns,fadings,decorations.pathreplacing}
\usepackage{tqft}
\usepackage{xspace}   % for new commands; decides weather I want to insert a space after the command
\usepackage[german,nameinlink]{cleveref} % has to be after hyperref, ntheorem, amsthm
\usepackage[left=10mm,right=10mm, top=2mm, bottom=10mm]{geometry}
\usepackage{../shortcuts}

\hypersetup{ 
  pdfauthor   = {Martin Thoma}, 
  pdfkeywords = {Geometrie und Topologie}, 
  pdftitle    = {Fragen zu Definitionen} 
}
\allowdisplaybreaks

%%%%%%%%%%%%%%%%%%%%%%%%%%%%%%%%%%%%%%%%%%%%%%%%%%%%%%%%%%%%%%%%%%%%%
% Begin document                                                    %
%%%%%%%%%%%%%%%%%%%%%%%%%%%%%%%%%%%%%%%%%%%%%%%%%%%%%%%%%%%%%%%%%%%%%
\begin{document}
\chapter{Fragen zu Definitionen}

\section*{17.) Simpliziale Abbildungen}
Wenn man Simpliziale Abbildungen wie folgt definiert

\begin{definition}\xindex{Abbildung!simpliziale}%
    Seien $K, L$ Simplizialkomplexe. Eine stetige Abbildung
    \[f:|K| \rightarrow |L|\]
    heißt \textbf{simplizial}, wenn für
    jedes $\Delta \in K$ gilt:
    \begin{defenum}
        \item $f(\Delta) \in L$
        \item $f|_{\Delta} : \Delta \rightarrow f(\Delta)$ ist eine
              affine Abbildung.
    \end{defenum}
\end{definition}

\todo[inline]{Dann ist die Forderung \enquote{$f(\Delta) \in L$} doch immer erfüllt, oder?
Gibt es eine Abbildung
$f:|K| \rightarrow |L|$
mit $f(\Delta) \notin L$?}

\section*{18.) ÜB 1, Aufgabe 2}
\underline{Vor.:} Es sei $(X, d)$ ein metrischer Raum, $A \subseteq X$. 
Weiter bezeichne $\fT$ die von $d$ auf $X$ erzeugte Topologie $\fT'$, die von
der auf $A \times A$ eingeschränkten Metrik $d|_{A \times A}$ erzeugte Topologie.

\underline{Beh.:} Die Topologie $\fT'$ und $\fT|_A$ (Spurtopologie) stimmen überein.

\underline{Bew.:}

\enquote{$\fT|_A \subseteq \fT'$}:

Sei $U \in \fT|_A = \Set{V \cap A | V \in \fT}$.\\
Dann ex. also $V \in \fT$ mit
$U = V \cap A$.\\
Sei $x \in U$.\\
Da $V \in \fT$, ex. nach Bemerkung~3 ein $r > 0$ mit 

\begin{align*}
    \fB_r(x) := \Set{y \in X | d(x,y) < r} &\subseteq V\\
                \Set{y \in A | d(x,y) < r} &\subseteq V \cap A = U
\end{align*}
also ist $U$ offen bzgl. $d|_{A \times A}$.
\todo[inline]{Wieso ist $U$ offen bzgl. $d|_{A \times A}$?}
Da $x \in U$ beliebig gewählt war gilt: $\fT|_A \subseteq \fT'$

\section*{19.) Topologische Gruppe und stetige Gruppenoperation}
\begin{definition}%
    Sei $G$ eine Mannigfaltigkeit und $(G, \circ)$ eine Gruppe.

    \begin{defenum}
        \item $G$ heißt \textbf{topologische Gruppe}\xindex{Gruppe!topologische},
              wenn die Abbildungen $\circ: G \times G \rightarrow G$
              und $\iota: G \rightarrow G$ definiert durch
              \[g \circ h := g \cdot h \text{ und } \iota(g) := g^{-1}\]
              stetig sind.
        \item Ist $G$ eine differenzierbare Mannigfaltigkeit, so heißt
              $G$ \textbf{Lie-Gruppe}\xindex{Lie-Gruppe}, wenn
              $(G, \circ)$ und $(G, \iota)$ differenzierbar sind.
    \end{defenum}
\end{definition}

\begin{definition}
    Sei $G$ eine Gruppe, $X$ ein topologischer Raum und
    $\circ: G \times X \rightarrow X$ eine Gruppenoperation.

    \begin{defenum}
        \item \xindex{Gruppe operiert durch Homöomorphismen}\textbf{$G$ operiert durch Homöomorphismen}, wenn für jedes $g \in G$
              die Abbildung
              \[m_g: X \rightarrow X, x \mapsto g \circ x\]
              ein Homöomorphismus ist.
        \item Ist $G$ eine topologische Gruppe, so heißt die Gruppenoperation $\circ$
              \textbf{stetig}\xindex{Gruppenoperation!stetige}, wenn 
              $\circ: G \times X \rightarrow X$ stetig ist.
    \end{defenum}
\end{definition}

\todo[inline]{Wenn $G$ eine topologische Gruppe ist, dann ist $\circ$ doch auf jeden Fall stetig! Was soll die Definition? Des Weiteren verstehe ich $g \circ h := g \cdot h$ nicht. Was ist $\cdot$?}

\section*{22.) MF-Beispiel}
$\praum^n(\mdr) = (\mdr^{n+1} \setminus \Set{0})/_\sim = S^n /_\sim$ und $\praum^n(\mdc)$ sind Mannigfaltigkeiten
der Dimension $n$ bzw. $2n$, da gilt:

Sei $U_i := \Set{(x_0: \dots : x_n) \in \praum^n(\mdr) | x_i \neq 0}\;\forall i \in 0, \dots, n$.
Dann ist $\praum^n(\mdr) = \bigcup_{i=0}^n U_i$ und die Abbildung
\begin{align*}
  U_i &\rightarrow \mdr^n\\
  (x_0 : \dots : x_n) &\mapsto \left (\frac{x_0}{x_i}, \dots, \cancel{\frac{x_i}{x_i}}, \dots, \frac{x_n}{x_i} \right )\\
  (y_1 : \dots : y_{i-1} : 1 : y_i : \dots : y_n) &\mapsfrom (y_1, \dots, y_n)
\end{align*}
ist bijektiv.
\todo[inline]{Was wird im Folgenden gemacht?}
Die $U_i$ mit $i = 0, \dots, n$ bilden einen $n$-dimensionalen Atlas:
\begin{align*}
        x &= (1:0:0) \in U_0 \rightarrow \mdr^2 & x &\mapsto (0,0)\\
        y &= (0:1:1) \in U_2 \rightarrow \mdr^2 & y &\mapsto (0,1)
\end{align*}
$\text{Umgebung: } \fB_1 (0,1) \rightarrow \Set{(1:u:v) | \|(u,v)\| < 1} = V_1$\\
$\text{Umgebung: } \fB_1 (0,1) \rightarrow \Set{(w:z:1) | w^2 + z^2 < 1} = V_2$\\

$V_1 \cap V_2 = \emptyset$?

$(a:b:c) \in V_1 \cap V_2$\\
$\Rightarrow a \neq 0$ und $(\frac{b}{a})^2 + (\frac{c}{a})^2 < 1 \Rightarrow \frac{c}{a} < 1$\\
$\Rightarrow c \neq 0$ und $(\frac{a}{c})^2 + (\frac{b}{c})^2 < 1 \Rightarrow \frac{a}{c} < 1$\\
$\Rightarrow$ Widerspruch


\section*{23) Hyperbolische Geraden erfüllen 3.ii}
\begin{bemerkung}[Eigenschaften der hyperbolischen Geraden]
    Die hyperbolischen Geraden erfüllen das Anordnungsaxiom 3 ii
\end{bemerkung}

\begin{beweis}\leavevmode
 Sei $g \in G_1 \dcup G_2$ eine hyperbolische Gerade.\\
              \underline{Fall 1:} $g = \Set{z \in \mdh | |z-m| = r} \in G_1$\\
              Dann gilt:
              \[\mdh = \underbrace{\Set{z \in \mdh | |z-m| < r}}_{=:H_1 \text{ (Kreisinneres)}} \dcup \underbrace{\Set{z \in \mdh | |z-m| < r}}_{=:H_2 \text{ (Kreisäußeres)}}\]
              Da $r > 0$ ist $H_1$ nicht leer, da $r \in \mdr$ ist $H_2$ nicht leer.

              \underline{Zu zeigen:} $\forall A \in H_i$, $B \in H_j$ mit
                      $i,j \in \Set{1,2}$ gilt: 
                      $\overline{AB} \cap g \neq \emptyset \Leftrightarrow i \neq j$\\
              \enquote{$\Leftarrow$}: Da $d_\mdh$ stetig ist, folgt diese Richtung
              direkt. Alle Punkte in $H_1$ haben einen Abstand von $m$ der kleiner
              ist als $r$ und alle Punkte in $H_2$ haben einen Abstand von $m$ der
              größer ist als $r$. Da man jede Strecke von $A$ nach $B$ insbesondere
              auch als stetige Abbildung $f: \mdr \rightarrow \mdr_{>0}$ auffassen
              kann, greift der Zwischenwertsatz $\Rightarrow$ $\overline{AB} \cap g \neq \emptyset$

              \enquote{$\Rightarrow$}:
              \todo[inline]{TODO}

              \underline{Fall 2:} $g = \Set{z \in \mdh | \Re{z} = x} \in G_2$\\
              Die disjunkte Zerlegung ist:
              \[\mdh = \underbrace{\Set{z \in \mdh | \Re(z) < x}}_{=: H_1 \text{ (Links)}} \dcup \underbrace{\Set{z \in \mdh | \Re(z) > x}}_{=: H_2 \text{ (Rechts)}}\]

              \underline{Zu zeigen:} $\forall A \in H_i$, $B \in H_j$ mit
                      $i,j \in \Set{1,2}$ gilt: 
                      $\overline{AB} \cap g \neq \emptyset \Leftrightarrow i \neq j$\\
              \enquote{$\Leftarrow$}: Wie zuvor mit dem Zwischenwertsatz.

              \enquote{$\Rightarrow$}:
              \todo[inline]{TODO}
\end{beweis}

\section*{25.) Fragen}
\begin{enumerate}
  \item Kapitel II:
  \begin{enumerate}
    \item Frage 7: Anschaulich ist mir klar, warum durch Verkleben gegenüberliegernder Seiten ein Torus entsteht. Was wird hier erwartet?
  \end{enumerate}
  \item Kapitel III
  \begin{enumerate}
    \item Deformationsretrakt: Das hatten wir nicht in der Vorlesung, oder? Ich meine mich zwar an das Wort zu erinnern (aus einem Übungsblatt? Einem Tutorium?) Könntest du bitte nochmals erklären was das ist?
Das ist zwar auf Blatt 7 und 8 vorgekommen, aber sonst nie.
    \item Damit verbunden: Was genau ist eine "Einbettung"? 
    \item Was bedeutet der Pfeil: $f:S^1 \hookrightarrow \mdr^2\;\;\;$ Einbettung der Kreislinie in die Ebene
    \item Was ist eine Inklusionsabbildung?
    \item Was ist ein Homotopietyp? (Ist das eventuell die Anzahl der Homotopieklassen?)
    \item Frage 4: Was ist eine Rose?
    \item Frage 5: Wieso ist $\GL(n, \mdr)$ eine Lie-Gruppe?
  \end{enumerate}
\end{enumerate}
\end{document}