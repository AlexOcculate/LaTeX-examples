\documentclass[a5paper,oneside]{scrbook}
\usepackage{etoolbox}
\usepackage{amsmath,amssymb}% math symbols / fonts
\usepackage{mathtools}      % \xRightarrow
\usepackage{nicefrac}       % \nicefrac
\usepackage[utf8]{inputenc} % this is needed for umlauts
\usepackage[ngerman]{babel} % this is needed for umlauts
\usepackage[T1]{fontenc}    % this is needed for correct output of umlauts in pdf
\usepackage[framed,amsmath,thmmarks,hyperref]{ntheorem}
\usepackage{framed}
\usepackage{marvosym}
\usepackage{makeidx}        % for automatically generation of an index
\usepackage{xcolor}
\usepackage[bookmarks,bookmarksnumbered,hypertexnames=false,pdfpagelayout=OneColumn,colorlinks,hyperindex=false]{hyperref} % has to be after makeidx
\usepackage{enumitem}       % Better than \usepackage{enumerate}, because it allows to set references
\usepackage{tabto}
\usepackage{braket}         % needed for \Set
\usepackage{csquotes}       % \enquote{}
\usepackage{subfig}         % multiple figures in one
\usepackage{parskip}        % nicer paragraphs
\usepackage{xifthen}        % \isempty
\usepackage{changepage}     % for the adjustwidth environment
\usepackage{pst-solides3d}
\usepackage[colorinlistoftodos]{todonotes}
\usepackage{pgfplots}
\pgfplotsset{compat=1.7}
\usepackage[arrow, matrix, curve]{xy}
\usepackage{caption}        % get newlines within captions
\usepackage{tikz}           % draw
\usepackage{tikz-3dplot}    % draw
\usepackage{tkz-fct}        % draw
\usepackage{tkz-euclide}    % draw
\usetkzobj{all}             % tkz-euclide
\usetikzlibrary{3d,calc,intersections,er,arrows,positioning,shapes.misc,patterns,fadings,decorations.pathreplacing}
\usepackage{tqft}
\usepackage{xspace}   % for new commands; decides weather I want to insert a space after the command
\usepackage[german,nameinlink]{cleveref} % has to be after hyperref, ntheorem, amsthm
\usepackage[left=10mm,right=10mm, top=2mm, bottom=10mm]{geometry}
\usepackage{../shortcuts}

\hypersetup{ 
  pdfauthor   = {Martin Thoma}, 
  pdfkeywords = {Geometrie und Topologie}, 
  pdftitle    = {Fragen zu Definitionen} 
}
\allowdisplaybreaks

%%%%%%%%%%%%%%%%%%%%%%%%%%%%%%%%%%%%%%%%%%%%%%%%%%%%%%%%%%%%%%%%%%%%%
% Begin document                                                    %
%%%%%%%%%%%%%%%%%%%%%%%%%%%%%%%%%%%%%%%%%%%%%%%%%%%%%%%%%%%%%%%%%%%%%
\begin{document}
\chapter{Fragen zu Definitionen}
\section{Topologischer Raum}
\begin{definition}\xindex{Raum!topologischer}\xindex{Menge!offene}\xindex{Menge!abgeschlossene}%
    Ein \textbf{topologischer Raum} ist ein Paar $(X, \fT)$ bestehend
    aus einer Menge $X$ und $\fT \subseteq \powerset{X}$ mit
    folgenden Eigenschaften
    \begin{defenumprops}
        \item $\emptyset, X \in \fT$
        \item \label{def:topologie.ii} Sind $U_1, U_2 \in \fT$, so ist $U_1 \cap U_2 \in \fT$
        \item Ist $I$ eine Menge und $U_i \in \fT$ für jedes $i \in I$,
              so ist $\displaystyle \bigcup_{i \in I} U_i \in \fT$
    \end{defenumprops}
    Die Elemente von $\fT$ heißen \textbf{offene Teilmengen} von $X$. 

    $A \subseteq X$ heißt \textbf{abgeschlossen}, wenn $X \setminus A$ offen ist.
\end{definition}

Ich glaube es ist unnötig in (i) zu fordern, dass $\emptyset in \fT$ gilt,
da man das mit (iii) bereits abdeckt:

Sei in (iii) die Indexmenge $I = \emptyset$. Dann muss gelten:

$\displaystyle \bigcup_{i \in \emptyset} U_i = \emptyset \in \fT$

\section{Diskret}
\begin{definition}
    Sei $X$ ein topologischer Raum und $M \subseteq X$.

    $M$ heißt \textbf{diskret} in $X$, wenn $M$ in $X$ keinen 
    Häufungspunkt hat.
\end{definition}

Laut \url{http://www.uni-protokolle.de/Lexikon/Diskreter_Raum.html#Diskrete_Teilmenge_eines_topologischen_Raums}
könnte man \textbf{diskret} wie folgt definieren:

\begin{definition}
    Sei $X$ ein topologischer Raum.
    \begin{defenum}
        \item Ein Punkt $x \in X$ heißt \textbf{isolierter Punkt}, wenn $\Set{ x }$ offen ist.
        \item Ein topologischer Raum heißt \textbf{diskreter topologischer}, Raum wenn jeder seiner Punkte isoliert ist.
    \end{defenum}
\end{definition}

Sind diese beiden Definitionen äquivalent? Falls ja, finde ich die 
zweite besser. Da benötigt man den Begriff \enquote{Häufungspunkt}
nicht, den wir nicht definiert hatten.

\section{Simpliziale Abbildung}
\begin{definition}
    Seien $K, L$ Simplizialkomplexe. Eine stetige Abbildung
    \[f:|K| \rightarrow |L|\]
    heißt \textbf{simplizial}, wenn für
    jedes $\Delta \in K$ gilt:
    \begin{defenum}
        \item $f(\Delta) \in L$
        \item $f|_{\Delta} : \Delta \rightarrow f(\Delta)$ ist eine
              affine Abbildung.
    \end{defenum}
\end{definition}

Ist die Definition so richtig? Was bedeutet $|K|$ und $|L|$ in
    \[f:|K| \rightarrow |L|\]

\section{Knotendiagramm}
\begin{definition}\xindex{Knotendiagramm}%
    Ein \textbf{Knotendiagramm} eines Knotens $\gamma$ ist eine 
    Projektion $\pi: \mdr^3 \rightarrow E$ auf eine Ebene $E$, sodass
    $|(\pi|C)^{-1}(x)| \leq 2$ für jedes $x \in D$.

    Ist $(\pi|C)^{-1}(x) = \Set{y_1, y_2}$, so \textbf{liegt $y_1$ über $y_2$},
    wenn $(y_1-x) = \lambda (y_2 - x)$ für ein $\lambda > 1$ ist.
\end{definition}

Sollte das jeweils $\pi|_C$ (sprich: \enquote{$\pi$ eingeschränkt auf $C$})
sein? Was ist $C$?

\section{Homotope Abbildungen und äquivalente Knoten}
\begin{definition}
    Zwei Knoten $\gamma_1, \gamma_2: S^1 \rightarrow \mdr^3$ heißen
    \textbf{äquivalent}, wenn es eine stetige Abbildung
    \[H: S^1 \times [0,1] \rightarrow \mdr^3\]
    gibt mit 
    \begin{align*}
        H(z,0) &= \gamma_1(z)\\
        H(z,1) &= \gamma_2(z)
    \end{align*}
    und für jedes
    feste $t \in [0,1]$ ist 
    \[H_z: S^1 \rightarrow \mdr^2, z \mapsto H(z,t)\]
    ein Knoten. Die Abbildung $H$ heißt \textbf{Isotopie} zwischen
    $\gamma_1$ und $\gamma_2$.
\end{definition}

Fehlt hier nicht etwas wie \enquote{$\forall z \in S^1$}?

\begin{definition}\xindex{Abbildung!homotope}%
    Seien $X, Y$ topologische Räume, $x_0 \in X, y_0 \in Y, f, g: X \rightarrow Y$
    stetig mit $f(x_0) = y_0 = g(x_0)$.

    $f$ und $g$ heißen \textbf{homotop} ($f \sim g$), wenn es eine stetige
    Abbildung $H: X \times I \rightarrow Y$ mit 
    \begin{align*}
        H(x,0)    &= f(x) \; \forall x \in X\\
        H(x,1)    &= g(x) \; \forall x \in X\\
        H(x_0, s) &= y_0  \; \forall s \in I
    \end{align*}
    gibt.
\end{definition}

Mir scheint der Begriff \enquote{homotope Abbildung} bis auf die
Eigenschaft \enquote{$H(x_0, s) = y_0  \; \forall s \in I$} mit 
dem Begriff \enquote{äquivalente Knoten} übereinzustimmen.
Der Knoten-Begriff ist dafür etwas spezieller nur auf Knoten bezogen.
Stimmt das?

\section{Basis und Subbasis}
\begin{itemize}
    \item Kennst du ein Beispiel für eine Subbasis in einem Topologischen Raum,
die zugleich eine Basis ist?
    \item Kennst du ein Beispiel für eine Subbasis in einem Topologischen Raum,
die keine Basis ist?
    \item Kennst du ein Beispiel für eine Basis in einem Topologischen Raum,
die keine Subbasis ist?
\end{itemize}

\section{Homotopie}
\begin{definition}%
    Sei $X$ ein topologischer Raum, $a, b \in X$, 
    $\gamma_1, \gamma_2: [0,1] \rightarrow X$ Wege von $a$ nach $b$,
    d.~h. $\gamma_1(0) = \gamma_2(0) = a$, $\gamma_1(1) = \gamma_2(1) = b$

    \begin{defenum}
        \item $\gamma_1$ und $\gamma_2$ heißen \textbf{homotop}\xindex{Weg!homotope},
              wenn es eine stetige Abbildung $H : I \times I \rightarrow X$ mit
              \begin{align*}
                H(t,0) &= \gamma_1(t)\;\forall t \in [0,1] =: I\\
                H(t,1) &= \gamma_2(t)\;\forall t \in [0,1] =: I
              \end{align*}
              und $H(0,s) = a$ und $H(1,s) = b$ für alle $s \in I$ gibt.
              Dann schreibt man: $\gamma_1 \sim \gamma_2$

              $H$ heißt \textbf{Homotopie}\xindex{Homotopie} zwischen
              $\gamma_1$ und $\gamma_2$.
        \item $\gamma_s: I \rightarrow X, \gamma_s(t) = H(t,s)$ ist
              Weg in $X$ von $a$ nach $b$ für jedes $s \in I$.
    \end{defenum}
\end{definition}

Diese Definition finde ich seltsam. Sollte b) nicht eine Bedingung für \enquote{Homotopie}
sein? Falls nicht: Was wird in b) definiert?

\section{Mannigfaltigkeit und MF mit Rand}
\begin{definition}%
    Sei $X$ ein topologischer Raum und $n \in \mdn$.
    \begin{defenum}
        \item Eine $n$-dimensionale \textbf{Karte}\xindex{Karte} auf
              $X$ ist ein Paar $(U, \varphi)$, wobei $U \subseteq X$
              offen und $\varphi: U \rightarrow V$ Homöomorphismus
              von $U$ auf eine offene Teilmenge $V \subseteq \mdr^n$.
        \item Ein $n$-dimensionaler \textbf{Atlas}\xindex{Atlas} $\atlas$ auf $X$ ist eine
              Familie $(U_i, \varphi_i)_{i \in I}$ von Karten auf $X$,
              sodass $\bigcup_{i \in I} U_i = X$.
        \item $X$ heißt (topologische) $n$-dimensionale \textbf{Mannigfaltigkeit}\xindex{Mannigfaltigkeit},
              wenn $X$ hausdorffsch ist, eine abzählbare Basis der 
              Topologie hat und ein $n$-dimensionalen Atlas besitzt.
    \end{defenum}
\end{definition}
\begin{definition}\xindex{Mannigfaltigkeit!mit Rand}%
    Sei $X$ ein Hausdorffraum mit abzählbarer Basis der Topologie.
    $X$ heißt $n$-dimensionale \textbf{Mannigfaltigkeit mit Rand},
    wenn es einen Atlas $(U_i, \varphi_i)$ gibt, wobei $U_i \subseteq X_i$
    offen und $\varphi_i$ ein Homöomorphismus auf eine offene 
    Teilmenge von 
    \[R_{+,0}^n := \Set{(x_1, \dots, x_n) \in \mdr^n | x_m \geq 0}\]
    ist.
\end{definition}

Wieso wird bei der Mannigfaltigkeit mit Rand nicht gefordert, dass
sie eine abzählbare Basis haben soll? Sollte man nicht vielleicht
hinzufügen, dass der Atlas $n$-dimensional sein soll?

\section{Standard-Simplex}
\begin{definition}
    \begin{defenum}
        \item Sei $\Delta^k = \conv(e_0, \dots, e_k) \subseteq \mdr^{n+1}$
              die konvexe Hülle der Standard-Basisvektoren $e_0, \dots, e_k$.

              Dann heißt $\Delta^k$ \textbf{Standard-Simplex}\xindex{Standard-Simplex}
              und $k$ die Dimension des Simplex.
        \item Für Punkte $v_0, \dots, v_k$ im $\mdr^n$ in allgemeiner
              Lage heißt $\Delta (v_0, \dots, v_k) = \conv(v_0, \dots, v_k)$
              ein \textbf{$k$-Simplex}\xindex{Simplex} in $\mdr^n$.
        \item Ist $\Delta (v_0, \dots, v_k)$ ein $k$-Simplex und
              $I = \Set{i_0, \dots, i_r} \subseteq \Set{0, \dots, k}$,
              so heißt $s_{i_0, \dots, i_r} := \conv(v_{i_0}, \dots, v_{i_r})$
              \textbf{Teilsimplex}\xindex{Teilsimplex} oder \textbf{Seite}\xindex{Seite}
              von $\Delta$. 

              $s_{i_0, \dots, i_r}$ ist $r$-Simplex.
    \end{defenum}
\end{definition}
Kann man bei der Definition des Standard-Simplex $k$ durch $n$ ersetzen?
Es gilt doch auf jeden Fall $0 \geq k \geq n$, oder? (Also auch für die anderen Definitionen).

\section{Produkttopologie}
\begin{definition}\xindex{Produkttopologie}%
    Seien $X_1, X_2$ topologische Räume.\\
    $U \subseteq X_1 \times X_2$ sei offen, wenn es zu jedem $x = (x_1, x_2) \in U$
    Umgebungen $U_i$ um $x_i$  mit $i=1,2$ gibt, sodass $U_1 \times U_2 \subseteq U$
    gilt.

    $\fT = \Set{U \subseteq X_1 \times X_2 | U \text{ offen}}$
    ist eine Topologie auf $X_1 \times X_2$. Sie heißt \textbf{Produkttopologie}.
    $\fB = \Set{U_1 \times U_2 | U_i \text{ offen in } X_i, i=1,2}$
    ist eine Basis von $\fT$.
\end{definition}
Gibt es ein Beispiel, das zegit, dass nicht $\fB = \fT$ gilt?

\end{document}
