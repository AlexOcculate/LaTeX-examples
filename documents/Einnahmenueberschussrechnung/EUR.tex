\documentclass[a4paper, 12pt]{article}
\usepackage[utf8]{inputenc} % this is needed for umlauts
\usepackage[ngerman]{babel} % this is needed for umlauts
\usepackage[T1]{fontenc}    % needed for right umlaut output in pdf
\usepackage[ngerman, num]{isodate} % get DD.MM.YYYY dates
\usepackage[locale=DE]{siunitx}
\usepackage{parskip}
\usepackage{fancyhdr}
\usepackage[rmargin=2.4cm]{geometry}
\usepackage{csvsimple}
\usepackage{pgfplots}

% Anpassen %%%%%%%%%%%%%%%%%%%%%%%%%%%%%%%%%%%%%%%%%%%%%%%%%%%%%%%%%%
\newcommand{\Year}{2013}                                            %
\newcommand{\Idnr}{72 DDD DDD DDD} % Deine Steuernummer (IdNr.)     %
\newcommand{\Vorname}{Martin}     % Vorname                         %
\newcommand{\Nachname}{Thoma}     % Nachname                        %
\newcommand{\Strasse}{Parkstraße} % Deine Straße                    %
\newcommand{\Hausnummer}{17}      % Deine Hausnummer                %
\newcommand{\PLZ}{76131}          % Deine PLZ                       %
\newcommand{\Ort}{Karlsruhe}      % Dein Ort                        %
                                                                    %
\newcommand{\Empfaenger}{Finanzamt Karlsruhe-Stadt} % Der Empfänger %
\newcommand{\EStrasse}{Schlossplatz 14}     % Straße des Empfängers %
\newcommand{\EPLZ}{76131}                   % PLZ des Empfängers    %
\newcommand{\EOrt}{Karlsruhe}               % Ort des Empfängers    %
                                                                    %
\newcommand{\DocTitle}{Einnahmenüberschussrechnung} %Titel des Dokuments%
%%%%%%%%%%%%%%%%%%%%%%%%%%%%%%%%%%%%%%%%%%%%%%%%%%%%%%%%%%%%%%%%%%%%%

% pdfinfo
\pdfinfo{
   /Author (\Nachname, \Vorname)
   /Title  (\DocTitle)
   /Subject (\DocTitle)
   /Keywords (Einnahmenüberschussrechnung, EÜR, Steuern, Finanzen, Steuererklärung)
}

\fancypagestyle{appendix}{%
    \fancyhead{}
    \renewcommand{\headrulewidth}{0pt}
}
 
\setlength{\headheight}{15pt} % fixes \headheight warning
\lhead{\Vorname{}~\Nachname{}, \Strasse{}, \PLZ{}~\Ort}
\rhead{Id-Nr. \Idnr}

% Begin document %%%%%%%%%%%%%%%%%%%%%%%%%%%%%%%%%%%%%%%%%%%%%%%%%%%%
\begin{document}
\pagestyle{fancy}
\section*{Überschussrechnung \Year}
\subsection*{Einnahmen}
\csvreader[before reading=\def\einnahmen{0},separator=semicolon]{2013-einnahmen-selbststaendig.csv}{Betrag=\Betrag}{%
\pgfmathsetmacro{\einnahmen}{\einnahmen+\Betrag}%
}

\begin{tabular}{|p{2.5cm}|p{2.5cm}|p{7cm}|p{2cm}|}\hline%
    \textbf{Rechnungs-\newline{}nummer} & \textbf{Rechnungs-\newline{datum}} & \textbf{Text} & \textbf{Betrag\newline{}in EUR}
\csvreader[head to column names,separator=semicolon]{2013-einnahmen-selbststaendig.csv}{}%
{\\\hline\Rechnungsnummer & \Rechnungsdatum & \Text &\hfill\num{\Betrag}}%
\\\hline\hline
\multicolumn{3}{|l|}{Summe} & \hfill\num[round-mode=places,round-precision=2]{\einnahmen}
\\\hline
\end{tabular}

\subsection*{Ausgaben}
\csvreader[before reading=\def\ausgaben{0},separator=semicolon]{2013-ausgaben.csv}{Betrag=\Betrag}{%
\pgfmathsetmacro{\ausgaben}{\ausgaben+\Betrag}%
}

\begin{tabular}{|p{2cm}|p{10.4cm}|p{2cm}|}\hline%
\textbf{Datum} & \textbf{Text} & \textbf{Betrag}
\csvreader[head to column names,separator=semicolon]{2013-ausgaben.csv}{}%
{\\\hline\Datum & \Text &\hfill\num{\Betrag}}%
\\\hline\hline
\multicolumn{2}{|l|}{Summe} & \hfill\num[round-mode=places,round-precision=2]{\ausgaben}
\\\hline
\end{tabular}


\vfill\pgfmathsetmacro{\gewinn}{\einnahmen-\ausgaben}
\textbf{Gewinn / Verlust:}\hfill\num[round-mode=places,round-precision=2]{\gewinn}
\clearpage

\subsection*{Sonstige Einnahmen aus selbstständiger Tätigkeit}
\csvreader[before reading=\def\selbststaendig{0},separator=semicolon]{2013-einnahmen-sonstiges.csv}{Betrag=\Betrag}{%
\pgfmathsetmacro{\selbststaendig}{\selbststaendig+\Betrag}%
}

\begin{tabular}{|p{2cm}|p{10.4cm}|p{2cm}|}\hline%
\textbf{Datum} & \textbf{Text} & \textbf{Betrag}
\csvreader[head to column names,separator=semicolon]{2013-einnahmen-sonstiges.csv}{}%
{\\\hline\Datum & \Text &\hfill\num{\Betrag}}%
\\\hline\hline
\multicolumn{2}{|l|}{Summe} & \hfill\num[round-mode=places,round-precision=2]{\selbststaendig}
\\\hline
\end{tabular}

\subsection*{Einnahmen aus Kapitalerträgen}
\csvreader[before reading=\def\kapitalsum{0},separator=semicolon]{2013-kapitalertraege.csv}{Betrag=\Betrag}{%
\pgfmathsetmacro{\kapitalsum}{\kapitalsum+\Betrag}%
}

\begin{tabular}{|p{2cm}|p{10.4cm}|p{2cm}|}\hline%
\textbf{Datum} & \textbf{Text} & \textbf{Betrag}
\csvreader[head to column names,separator=semicolon]{2013-kapitalertraege.csv}{}%
{\\\hline\Datum & \Text &\hfill\num{\Betrag}}%
\\\hline\hline
\multicolumn{2}{|l|}{Summe} & \hfill\num[round-mode=places,round-precision=2]{\kapitalsum}
\\\hline
\end{tabular}

\end{document}
