\documentclass[a4paper,12pt]{scrartcl}
\usepackage{amssymb, amsmath} % needed for math
\usepackage[utf8]{inputenc} % this is needed for umlauts
\usepackage[ngerman]{babel} % this is needed for umlauts
\usepackage[T1]{fontenc}    % this is needed for correct output of umlauts in pdf
\usepackage[top=3cm, bottom=3cm, left=4cm, right=2cm]{geometry} %layout
\usepackage{hyperref}   % links im text
\usepackage{color}
\usepackage{framed}
\usepackage{enumerate}  % for advanced numbering of lists
\usepackage{pdfpages}  % Signatureinbingung und includepdf
\usepackage{parskip}   % spaces instead of intendation between paragraphs
\usepackage{cite}
\usepackage[scaled=.90]{helvet}% Helvetica, served as a model for arial
\linespread{1.45}     % 1,45-Facher Zeilenabstand

\usepackage{titlesec}
%\titlespacing{command}{left spacing}{before spacing}{after spacing}[right]
\titlespacing\section{0pt}{12pt plus 3pt minus 2pt}{0pt plus 2pt minus 1pt}
\usepackage[framemethod=tikz,xcolor=true]{mdframed}

\usepackage{enumitem}
\usepackage{braket} % needed for nice printing of sets

\usepackage{fancyhdr}  % needed for the footer
\usepackage{lastpage}  % needed for the footer

\usepackage{tikz} % needed for arrow in document
\usetikzlibrary{shapes.arrows} % needed for arrow in document

\clubpenalty  = 10000   % Schusterjungen verhindern
\widowpenalty = 10000   % Hurenkinder verhindern

\hypersetup{ 
  pdfauthor   = {Martin Thoma}, 
  pdfkeywords = {Asymmetrische Verschlüsselungsverfahren; RSA-Kryptosystems}, 
  pdftitle    = {Asymmetrische Verschlüsselungsverfahren am Beispiel des RSA-Kryptosystems},
  pdfborder = {0 0 0} % remove red box around hyperlinks
} 
\usepackage[german,nameinlink]{cleveref}
\crefname{section}{Kapitel}{Kapitel}
\pagestyle{fancy}
\fancyhf{}
\renewcommand{\headrulewidth}{0pt}
\renewcommand{\footrulewidth}{0pt}
\fancyfoot[R]{Seite~\thepage~von \pageref{LastPage}}

% From http://www.matthewflickinger.com/blog/archives/2005/02/20/latex_mod_spacing.asp
% Thanks!
\makeatletter
\def\imod#1{\allowbreak\mkern10mu({\operator@font mod}\,\,#1)}
\makeatother

\usepackage{minted} % needed for the inclusion of source code

\begin{document}

\def\myDownArrow{\smash{\begin{tikzpicture}[baseline=6mm]\useasboundingbox (-2,0);\node[single arrow,draw=black,fill=black!10,minimum height=4cm,minimum width=1.2cm,shape border rotate=270] at (0,-1) {};\end{tikzpicture}}}
\def\myUpArrow{\smash{\begin{tikzpicture}[baseline=6mm]\useasboundingbox (-2,0);\node[single arrow,draw=black,fill=black!10,minimum height=4cm,minimum width=1.2cm,shape border rotate=90] at (0,-1) {};\end{tikzpicture}}}

\def\myDownArrowB{\smash{\begin{tikzpicture}[baseline=6mm]\useasboundingbox (-1,0);\node[single arrow,draw=black,fill=black!10,minimum height=4cm,minimum width=1.2cm,shape border rotate=270] at (0,-1) {};\end{tikzpicture}}}
\def\myUpArrowB{\smash{\begin{tikzpicture}[baseline=6mm]\useasboundingbox (-1,0);\node[single arrow,draw=black,fill=black!10,minimum height=4cm,minimum width=1.2cm,shape border rotate=90] at (0,-1) {};\end{tikzpicture}}}

\setcounter{page}{0}
\pagenumbering{roman} 

%\thispagestyle{empty}

\begin{center}
{\Huge Paul-Klee-Gymnasium}


Facharbeit aus der Mathematik



Thema:

Asymmetrische Verschlüsselungsverfahren
am Beispiel des RSA-Kryptosystems



\begin{tabular}{lll}
Verfasser       &:& Martin Andreas Thoma\\
Kursleiter      &:& Claudia Wenninger\\
Abgegeben am    &:& 20.01.2010 (verändert am 06.04.2010)\\
\\
\\
Erzielte Note   &:& \line(1,0){120}\\
Erzielte Punkteanzahl   &:& \line(1,0){120}\\
\end{tabular} 
\end{center}


\includepdf[pages=1]{Titelseite.pdf}
\clearpage

\thispagestyle{empty}
\tableofcontents 
\clearpage


\pagenumbering{arabic} 
\setcounter{page}{2}

% Start der eigentlichen Arbeit
In diesem Kapitel sei $X \neq \emptyset$ eine Menge.

\begin{definition}
    \index{$\sigma$-!Algebra}
    Sei $\fa\subseteq\mathcal{P}(X)$, $\fa$ heißt eine 
    \textbf{$\sigma$-Algebra} auf $X$, wenn gilt:
    \begin{enumerate}
        \item[($\sigma_1$)] $X\in\fa$
        \item[($\sigma_2$)] $A\in\fa \implies A^c\in\fa$
        \item[($\sigma_3$)] $(A_j)$ ist eine Folge in $\fa \implies$
                            $\bigcup A_j\in\fa$.
    \end{enumerate}
\end{definition}

\begin{beispieleX}
    \begin{enumerate}
        \item $\Set{X,\emptyset}$ und $\mathcal{P}(X)$ sind 
              $\sigma$-Algebren auf $X$.
        \item Sei $A\subseteq X$, dann ist $\Set{X,\emptyset, A, A^c}$ 
              eine $\sigma$-Algebra auf $X$.
        \item $\fa:=\Set{A\subseteq X | A \text{ abzählbar oder } A^c \text{ abzählbar}}$ 
              ist eine $\sigma$-Algebra auf $X$.
    \end{enumerate}
\end{beispieleX}

\begin{lemma}
\label{Lemma 1.1}
Sei $\fa$ eine $\sigma$-Algebra auf $X$, dann:
\begin{enumerate}
\item $\emptyset\in\fa$
\item Ist $(A_j)$ eine Folge in $\fa$, so ist $\bigcap A_j\in\fa$.
\item Sind $A_1,\dots,A_n\in\fa$, so gilt:
\begin{enumerate}
\item $A_1\cup\dots\cup A_n\in\fa$
\item $A_1\cap\dots\cap A_n\in\fa$
\item $A_1\setminus A_2\in\fa$
\end{enumerate}
\end{enumerate}
\end{lemma}

\begin{beweis}
    \begin{enumerate}
    \item \folgtnach{$\sigma_2$} $\emptyset=X^c\in\fa$.
    \item $D:=\bigcap A_j$. $D^c=\bigcup A_j^c\in\fa$ (nach 
          ($\sigma_2$) und ($\sigma_3$)), also gilt auch 
          $D=(D^c)^c\in\fa$.
    \item \begin{enumerate}
            \item \folgtnach{($\sigma_3$) mit $A_{n+j}:=\emptyset$ ($j\ge 1$)} 
                  $A_1\cup\dots\cup A_n\in\fa$.
            \item \folgtnach{(2) mit $A_{n+j}:=X$ ($j\ge 1)$} 
                  $A_1\cap\dots\cap A_n\in\fa$.
            \item $A_1\setminus A_2=A_1\cap A_2^c\in\fa$
          \end{enumerate}
    \end{enumerate}
\end{beweis}

\begin{lemma}
    \label{Lemma 1.2}
    Sei $\cf \neq \emptyset$ eine Menge von $\sigma$-Algebren auf $X$. 
    Dann ist 
    \[\fa_0:=\bigcap_{\fa\in\cf}\fa\]
    eine $\sigma$-Algebra auf $X$.
\end{lemma}

\begin{beweis}
    \begin{enumerate}
        \item[($\sigma_1$)] $\forall\fa\in\cf:X\in\fa\implies X\in\fa_0$.
        \item[($\sigma_2$)] Sei $A\in\fa_0$, dann gilt:
          \begin{align*}
            \forall\fa\in\cf:A\in\fa &\implies \forall\fa\in\cf:A^c\in\fa\\
                                     &\implies A^c\in\fa_0
          \end{align*}
        \item[($\sigma_3$)] Sei $(A_j)$ eine Folge in $\fa_0$, dann 
            ist $(A_j)$ Folge in $\fa$ für alle $\fa\in\cf$, dann gilt:
          \begin{align*}
            \forall\fa\in\cf:\bigcap A_j\in\fa \implies \bigcap A_j\in\fa_0
          \end{align*}
    \end{enumerate}
\end{beweis}

\begin{definition}
    \index{Erzeuger}
    Sei $\emptyset \neq \mathcal{E} \subseteq \mathcal{P}(X)$ und 
    $\cf:=\{\fa:\fa$ ist $\sigma$-Algebra auf $X$ mit 
    $\mathcal{E}\subseteq\fa\}$. Definiere
    \[\sigma(\mathcal{E}):=\bigcap_{\fa\in\cf}\fa\]
    \folgtnach{1.2} $\sigma(\mathcal{E})$ ist eine $\sigma$-Algebra 
    auf $X$. $\sigma(\mathcal{E})$ heißt die 
    \textbf{von $\mathcal{E}$ erzeugte $\sigma$-Algebra}. 
    $\mathcal{E}$ heißt ein \textbf{Erzeuger} von 
    $\sigma(\mathcal{E})$.
\end{definition}

\begin{lemma}
    \label{Lemma 1.3}
    Sei $\emptyset\ne\mathcal{E}\subseteq\mathcal{P}(X)$.
    \begin{enumerate}
        \item $\mathcal{E}\subseteq\sigma(\mathcal{E})$. 
              $\sigma(\mathcal{E})$ ist die "`kleinste"' 
              $\sigma$-Algebra auf $X$, die $\mathcal{E}$ enthält.
        \item Ist $\mathcal{E}$ eine $\sigma$-Algebra, so ist 
              $\sigma(\mathcal{E})=\mathcal{E}$.
        \item Ist $\mathcal{E}\subseteq\mathcal{E}'$, so ist 
              $\sigma(\mathcal{E})\subseteq\sigma(\mathcal{E}')$.
    \end{enumerate}
\end{lemma}

\begin{beweis}
    \begin{enumerate}
        \item Klar nach Definition.
        \item $\fa:=\mathcal{E}$, dann gilt 
              $\fa\subseteq\sigma(\mathcal{E})\subseteq\fa$.
        \item $\mathcal{E}\subseteq\mathcal{E}'\subseteq\sigma(\mathcal{E}')$, 
              also folgt nach Definition 
              $\sigma(\mathcal{E})\subseteq\sigma(\mathcal{E}')$.
    \end{enumerate}
\end{beweis}

\begin{beispiel}
    \begin{enumerate}
        \item Sei $A\subseteq X$ und $\mathcal{E}:=\{A\}$. Dann ist 
              $\sigma(\mathcal{E})=\{X,\emptyset,A,A^c\}$.
        \item $X:=\{1,2,3,4,5\}, \mathcal{E}:=\{\{1\},\{1,2\}\}$. 
              Dann gilt:
              \[\sigma(\mathcal{E}):=\{X,\emptyset, \{1\},\{2\},\{1,2\},\{3,4,5\},\{1,3,4,5\},\{2,3,4,5\}\}\]
    \end{enumerate}
\end{beispiel}

\begin{erinnerung}
    \index{Offenheit}\index{Abgeschlossenheit}
    Sei $d\in\mdn, X\subseteq\mdr^d$. $A\subseteq X$ heißt 
    \textbf{offen} (\textbf{abgeschlossen}) in $X$, genau dann wenn 
    ein offenes (abgeschlossenes) $G\subseteq\mdr^d$ existiert mit 
    $A=X\cap G$.\\
    Beachte: $A$ abgeschlossen in $X$ $\iff$ $X\setminus A$ offen in 
    $X$.
\end{erinnerung}

\begin{definition}
    \index{Borel!$\sigma$-Algebra}\index{$\sigma$-!Algebra, Borelsche}
    \index{Borel!Mengen}
    Sei $X\subseteq\mdr^d$.
    \begin{enumerate}
        \item $\mathcal{O}(X):=\Set{A\subseteq X | A \text{ ist offen in } X}$
        \item $\fb(X):=\sigma(\mathcal{O}(X))$ heißt 
              \textbf{Borelsche $\sigma$-Algebra} auf $X$.
        \item $\fb_d:=\fb(\mdr^d)$. Die Elemente von $\fb_d$ heißen 
              \textbf{Borelsche Mengen} oder \textbf{Borel-Mengen}.
    \end{enumerate}
\end{definition}

\begin{beispiel}
    \begin{enumerate}
        \item Sei $\emptyset \neq X\subseteq\mdr^d$. Ist $A\subseteq$ 
              $\stackrel{\hbox{offen}}{\hbox{abgeschlossen}}$
              in $X$, so ist $A\in\fb(X)$.
        \item Ist $A\subseteq\mdr^d$ 
              $\stackrel{\hbox{offen}}{\hbox{abgeschlossen}}$,
              so ist $A\in\fb_d$.
        \item Sei $d=1, A=\mdq$. $\mdq$ ist abzählbar, also 
              $\mdq=\{r_1,r_2,\dots\}$ (mit $r_i\ne r_j$ für $i\ne j$). 
              Also ist $\mdq=\bigcup \{r_j\}$. Sei nun $r\in\mdq$, 
              dann ist $B:=(-\infty,r)\cup(r,\infty)\in\fb_1$. Daraus 
              folgt $\{r_j\}\in\fb_1$, also auch $\mdq\in\fb_1$.\\
              Allgemeiner lässt sich zeigen: 
              $\mdq^d:=\{(x_1,\dots,x_n):x_j\in\mdq (j=1,\dots,n)\}\in\fb_d$.
        \item Sei $x_0 \in \mdr^d, \Set{x_0}$ ist abgeschlossen
              $\Rightarrow \Set{x_0} \in \fb$
    \end{enumerate}
\end{beispiel}

\begin{definition}
    \index{Intervall}
    \index{Halbraum}
    \begin{enumerate}
    \item Seien $I_1,\dots,I_d$ Intervalle in $\mdr$. 
          Dann heißt $I_1\times\dots\times I_d$ ein \textbf{Intervall} 
          in $\mdr^d$.
    \item Seien $a=(a_1,\dots,a_d), b=(b_1,\dots,b_d)\in\mdr^d$.
          \[a\le b:\iff a_j\le b_j \quad \forall j \in \Set{1, \dots, d}\]
    \item Seien $a,b\in\mdr^d$ und $a\le b$.
          \begin{align*}
            (a,b) &:= (a_1,b_1)\times(a_2,b_2)\times\dots\times(a_d,b_d)\\
            (a,b] &:= (a_1,b_1]\times(a_2,b_2]\times\dots\times(a_d,b_d]\\
            [a,b) &:= [a_1,b_1)\times[a_2,b_2)\times\dots\times[a_d,b_d)\\
            [a,b] &:= [a_1,b_1]\times[a_2,b_2]\times\dots\times[a_d,b_d]
        \end{align*}
        mit der Festlegung $(a,b):=(a,b]:=[a,b):=\emptyset$, falls 
        $a_j=b_j$ für ein $j\in\{1,\dots,d\}$.
    \item Für $k\in\{1,\dots,d\}$ und $\alpha\in\mdr$ definiere die 
          folgenden \textbf{Halbräume}:
          \begin{align*}
            H_k^-(\alpha) &:= \Set{(x_1,\dots,x_d)\in\mdr^d:x_k\le\alpha}\\
            H_k^+(\alpha) &:= \Set{(x_1,\dots,x_d)\in\mdr^d:x_k\ge\alpha}
          \end{align*}
    \end{enumerate}
\end{definition}

Beispiel für ein Intervall $(a_1, b_1) \times [a_2, b_2]$ und
die beiden Halbräume:\\
\begin{tikzpicture}
    % Draw axes
    \draw [<->,thick] (0,2.5) node (yaxis) [above] {$x_2$}
        |- (2.5,0) node (xaxis) [right] {$x_1$};

    % Draw two intersecting lines
    \draw[thick, dashed] (1,1) coordinate (a) -- (2,1) coordinate (b);
    \draw[thick, dashed] (a) -- (1,2) coordinate (d);
    \draw[thick]         (d) -- (2,2) coordinate (c);
    \draw[thick]         (b) -- (2,2);

    \fill[green!15] (a) -- (b) -- (c) -- (d) -- (a);

    % Draw lines indicating intersection with y and x axis. Here we 
    % use the perpendicular coordinate system
    \draw[dotted] (yaxis |- a) node[left] {$a_2$}
        -| (xaxis -| a) node[below] {$a_1$};

    \draw[dotted] (yaxis |- c) node[left] {$b_2$}
        -| (xaxis -| c) node[below] {$b_1$};
\end{tikzpicture}
\begin{tikzpicture}
    \pgfdeclarepatternformonly{north east lines wide}%
    {\pgfqpoint{-1pt}{-1pt}}%
    {\pgfqpoint{10pt}{10pt}}%
    {\pgfqpoint{9pt}{9pt}}%
    {
    \pgfsetlinewidth{0.7pt}
    \pgfpathmoveto{\pgfqpoint{0pt}{0pt}}
    \pgfpathlineto{\pgfqpoint{9.1pt}{9.1pt}}
    \pgfusepath{stroke}
    }

    \pgfdeclarepatternformonly{north west lines wide}
    {\pgfqpoint{-1pt}{-1pt}}%
    {\pgfqpoint{7pt}{7pt}}%
    {\pgfqpoint{6pt}{6pt}}%
    {
    \pgfsetlinewidth{0.7pt}
    \pgfpathmoveto{\pgfqpoint{0pt}{6pt}}
    \pgfpathlineto{\pgfqpoint{6.1pt}{-0.1pt}}
    \pgfusepath{stroke}
    }

    % Draw two intersecting lines
    \draw[thick, red]   (-1,-1) coordinate (a) -- (2,-1) coordinate (b);
    \draw[thick, green] ( 1,-1) coordinate (c) -- (1, 2) coordinate (d);

    \fill[pattern=north east lines wide, pattern color=red!50]   (a) -- (b) -- (2,2) -- (-1,2) -- (a);
    \fill[pattern=north west lines wide, pattern color=green!50] (a) -- (1,-1) -- (1,2) -- (-1,2) -- (a);

    \draw[thick, green] (c) -- (d);
    \draw[thick, red]   (a) -- (b);


    % Draw axes
    \draw [<->,thick] (0,2.5) node (yaxis) [above] {$x_2$}
        |- (2.5,0) node (xaxis) [right] {$x_1$};
    \node[red]   at (1.5,2.8) {$H_2^+(-1)$};
    \node[green] at (1.5,2.3) {$H_1^-(1)$};
\end{tikzpicture}

\begin{satz}[Erzeuger der Borelschen $\sigma$-Algebra auf $\mdr^d$]
\label{Satz 1.4}
Es seien $\ce_1,\ce_2,\ce_3$ wie folgt definiert:
\begin{align*}
\ce_1&:=\Set{(a,b) | a,b\in\mdq^d,a\le b}\\
\ce_2&:=\Set{(a,b] | a,b\in\mdq^d, a\le b}\\
\ce_3&:=\Set{H^-_k(\alpha) | \alpha\in\mdq, k \in \Set{1,\dots,d}}
\end{align*}
Dann gilt:
\[\fb_d=\sigma(\ce_1)=\sigma(\ce_2)=\sigma(\ce_3)\]
Entsprechendes gilt für die anderen Typen von Intervallen und Halbräumen.
\end{satz}

\begin{beweis}
\[\fb_d 
    \stackrel{(1)}{\subseteq} \sigma(\ce_1) 
    \stackrel{(2)}{\subseteq} \sigma(\ce_2) 
    \stackrel{(3)}{\subseteq} \sigma(\ce_3) 
    \stackrel{(4)}{\subseteq} \fb_d
\]
\begin{enumerate}
    \item Sei $G\in\co(\mdr^d), \fm:=\Set{(a,b) | a,b \in \mdq^d, \; a\le b, \; (a,b)\subseteq G}$.\\
          Dann ist $\fm$ abzählbar und $G=\bigcup_{I\in\fm}I$.\\
          Also gilt:
          \[\co(\mdr^d) \subseteq \sigma(\ce_1)\]
          \[G\in\sigma(\ce_1)\implies \fb_d=\sigma(\co(\mdr^d))\stackrel{1.3}{\subseteq}\sigma(\ce_1)\]
    \item Sei $a=(a_1, \dots,a_d), b=(b_1,\dots,b_d) \in \mdq^d$ und $a \leq b$ sowie $(a, b)\in\ce_1$.\\
          \textbf{Fall 1:} $(a,b)=\emptyset\in\ce_2\subseteq\sigma(\ce_2)$\\
          \textbf{Fall 2:} $(a,b)\ne\emptyset$.\\
          Dann gilt für alle $j\in\{1,\dots,d\}:a_j<b_j$. Also gilt auch:
          \[\exists N\in\mdn:\forall n\ge N: \forall j\in\{1,\dots,d\}:a_j<b_j-\frac1n\]
          Definiere $c_n:=(\frac1n,\dots,\frac1n)\in\mdq^d$. Dann gilt:
          \[(a,b)=\bigcup_{n\ge N}(a,b-c_n]\in\sigma(\ce_2)\]
          Also auch $\ce_1\subseteq\sigma(\ce_2)$ und damit 
          $\sigma(\ce_1)\subseteq\sigma(\ce_2)$.
    \item Seien $a = (a_1,\dots,a_d), b=(b_1,\dots,b_d) \in \mdq^d$ 
          mit $a \leq b$. 
          Nachrechnen:
          \[(a,b] = \bigcap_{k=1}^d (H^-_k(b_k) \cap H^-_k(a_k)^c) \in \sigma(\ce_3). \]
          Das heißt $\ce_2 \subseteq \sigma(\ce_3)$ und damit auch 
          $\sigma(\ce_2) \subseteq \sigma(\ce_3)$. 
    \item $H^-_k(\alpha)$ ist abgeschlossen, somit ist 
          $H^-_k(\alpha)^c$ offen und damit $H^-_k(\alpha)^c \in \fb_d$, 
          also auch $H^-_k(\alpha) \in \fb_d$. Damit ist 
          $\ce_3 \subseteq \fb_d \implies \sigma(\ce_3) \subseteq \fb_d$. 
\end{enumerate}
\end{beweis}

\begin{definition}
    \index{Spur}
    Sei $\emptyset \neq \fm \subseteq \mathcal{P}(X)$ und 
    $\emptyset \neq Y \subseteq X$. 
    \[\fm_Y := \{A \cap Y : A \in \fm\}\] 
    heißt die \textbf{Spur von $\fm$ in $Y$}.
\end{definition}

\begin{beispiel}
    $X = \mdr^d, \fm \subseteq \sigma(\mdr^d), \; Y \subseteq X$. 
    Dann: $(\co(\mdr^d))_Y = \sigma(Y)$
\end{beispiel}

\begin{satz}[Spuren und $\sigma$-Algebren]
    \label{Satz 1.5}
    Sei $\emptyset \neq Y \subseteq X$ und $\fa$ eine 
    $\sigma$-Algebra auf $X$.
    \begin{enumerate}
        \item $\fa_Y$ ist eine $\sigma$-Algebra auf $Y$.
        \item $\fa_Y \subseteq \fa \iff Y \in \fa$
        \item Ist $\emptyset \neq \ce \subseteq \mathcal{P}(X)$, so 
              ist $\sigma(\ce_Y) = \sigma(\ce)_Y$.
    \end{enumerate}
\end{satz}

\begin{beweis}
    \begin{enumerate}
        \item 
          \begin{enumerate}
            \item[($\sigma_1$)] Es ist $Y=Y\cap X\in\fa_Y$, da $X\in\fa$.
            \item[($\sigma_2$)] Sei $B\in\fa_Y$, dann existiert ein 
                            $A\in\fa$ mit $B=A\cap Y$.\\
                            Also ist 
                            $Y\setminus B=\overbrace{(X\setminus A)}^{\in\fa} \cap Y\in\fa_Y$.
            \item[($\sigma_3$)] Sei $(B_j)$ eine Folge in $\fa_Y$, dann 
                            existiert eine Folge $(A_j)\in\fa^\mdn$ 
                            mit $B_j=A_j\cap Y$. Es gilt:
                            \[\bigcup B_j=\bigcup(A_j\cap Y)=(\bigcup A_j)\cap Y\in\fa_Y\]
            \end{enumerate}
        \item Der Beweis erfolgt durch Implikation in beiden Richtungen:
              \begin{enumerate}
                \item["`$\implies$"'] Es gilt $Y\in\fa_Y\subseteq\fa$.
                \item["`$\impliedby$"'] Sei $B\in\fa_Y$, dann existiert ein $A\in\fa$ mit $B=A\cap Y\in\fa$.
              \end{enumerate}
        \item Es gilt:
        \begin{align*}
        \ce\subseteq\sigma(\ce)&\implies\ce_Y\subseteq\sigma(\ce)_Y\\
        &\implies\sigma(\ce_Y)\subseteq\sigma(\ce)_Y
        \end{align*}
        Sei nun:
        \[\cd:=\{A\subseteq X:A\cap Y\in\sigma(\ce_Y)\}\]
        Übung: $\cd$ ist eine $\sigma$-Algebra auf $X$.\\
        Sei $E\in\ce$ dann ist $E\cap Y\in\ce_Y\subseteq\sigma(\ce_Y)$ also $E\in\cd$ und damit $\ce\subseteq\cd$. Daraus folgt:
        \begin{align*}
        \sigma(\ce)_Y&\subseteq\sigma(\cd)_Y=\cd_Y=\{A\cap Y:A\in\cd\}\\
        &\subseteq\sigma(\ce_Y)
        \end{align*}
    \end{enumerate}
\end{beweis}

\begin{folgerungen}
    Sei $X\subseteq\mdr^d$. Dann gilt:
    \begin{enumerate}
        \item $\fb(X)=(\fb_d)_X$
        \item \importantbox{\text{Ist } X\in\fb_d \text{, so ist } \fb(X)=\Set{A\in\fb_d:A\subseteq X}\subseteq\fb_d}
    \end{enumerate}
\end{folgerungen}

\begin{definition}
Wir fügen $\mdr$ ein zusätzliches Symbol $+\infty$ hinzu. Es soll gelten:
\begin{enumerate}
    \item $(+\infty)+(+\infty):=+\infty$
    \item $\forall a\in\mdr:a<+\infty$
    \item $\pm a+(+\infty):=+\infty=:(+\infty)\pm a$
\end{enumerate}
Außerdem sei $[0,+\infty]:=[0,\infty)\cup\{+\infty\}$.
\begin{enumerate}
    \item Sei $(x_n)$ eine Folge in $[0,+\infty]$. Es gilt:
          \[x_n\stackrel{n\to\infty}{\to}\infty:\iff \forall c>0\;\exists n_c\in\mdn:\forall n\ge n_c: x_n> c\]
    \item Sei $(a_n)$ eine Folge in $[0,+\infty]$. Es gilt
          \[\sum_{n=1}^\infty a_n=\sum a_n = +\infty :\Leftrightarrow
            \begin{cases}
                \exists n \in \mdn \text{ mit } a_n = +\infty \text{ oder }\\
                \sum a_n \text{ divergiert}
            \end{cases} 
          \]
\end{enumerate} 
Wegen Ana I, 13.1 können Reihen der obigen Form beliebig umgeordnet 
werden, ohne dass sich ihr Wert verändert.
\end{definition}

\begin{definition}
\index{Maß}
\index{$\sigma$-!Additivität}
\index{Maßraum}
\index{Maß!endliches}
\index{Wahrscheinlichkeitsmaß}\index{Maß!Wahrscheinlichkeits-}
Sei $\fa$ eine $\sigma$-Algebra auf $X$ und $\mu:\fa\to[0,+\infty]$ 
eine Abbildung. $\mu$ heißt ein \textbf{Maß} auf $\fa$, genau dann 
wenn gilt:
\begin{enumerate}
\item[$(M_1)$] $\mu(\emptyset)=0$
\item[$(M_2)$] Ist $(A_j)$ eine disjunkte Folge in $\fa$, so ist 
$\mu(\bigcup A_j)=\sum\mu(A_j)$. Diese Eigenschaft heißt 
\textbf{$\sigma$-Additivität}.
\end{enumerate}
In diesem Fall heißt $(X,\fa,\mu)$ ein \textbf{Maßraum}.\\
Ein Maß $\mu$ heißt \textbf{endlich} $:\Leftrightarrow \mu(X)<\infty$.\\
Ein Maß $\mu$ heißt ein \textbf{Wahrscheinlichkeitsmaß} $:\Leftrightarrow\mu(X)=1$ ist.
\end{definition}

\begin{beispiel}
\index{Punktmaß}\index{Maß!Punkt-}
\index{Dirac-Maß}\index{Maß!Dirac-}
\index{Zählmaß}\index{Maß!Zähl-}
\begin{enumerate}
    \item Sei $\fa:=\cp(X)$ und $x_0\in X$. 
          $\delta_{x_0}:\fa\to[0,+\infty]$ sei definiert durch:
          \[\delta_{x_0}(A):=
          \begin{cases}
          1,\ x_0\in A\\
          0,\ x_0\not\in A
          \end{cases}\]
          Klar ist, dass $\delta_{x_0}(\emptyset)=0$ ist.\\
          Sei $(A_j)$ eine disjunkte Folge in $\fa$.
          \[\delta_{x_0}(\bigcup A_j)=
          \left.\begin{cases}
          1,\ x_0\in\bigcup A_j\\
          0,\ x_0\not\in\bigcup A_j
          \end{cases}\right\}=\sum\delta_{x_0}(A_j)\]
          $\delta_{x_0}$ ist ein Maß auf $\fa$ und heißt 
          \textbf{Punktmaß} oder \textbf{Dirac-Maß}.
    \item Sei $X:=\mdn$, $\fa:=\cp(X)$ und $(p_j)$ eine Folge in 
          $[0,+\infty]$. Definiere $\mu:\fa\to[0,+\infty]$ durch:
          \begin{align*}
          \text{Für } A \in \fa: \quad 
          \mu(A):=
          \begin{cases}
              0                &\text{, falls } A=\emptyset\\
              \sum_{j\in A}p_j &\text{, falls } A\ne\emptyset
          \end{cases}
          \end{align*}
          Übung: $\mu$ ist ein Maß auf $\fa=\cp(\mdn)$ und heißt ein \textbf{Zählmaß}. 
          Sind alle $p_j=1$, so ist $\mu(A)$ gerade die Anzahl der 
          Elemente von $A$.
    \item Sei $(X,\fa,\mu)$ ein Maßraum, $\emptyset\ne Y\subseteq X$ 
          und $\fa_0\subseteq\fa$ eine $\sigma$-Algebra auf $Y$. 
          Definiere $\mu_0:\fa_0\to[0,+\infty]$ durch 
          $\mu_0(A):=\mu(A)$ ($A\in\fa_0$).\\
          Dann ist 
          $(Y,\fa_0,\mu_0)$ ein Maßraum.\\
          Ist spezieller $Y\in\fa$, so ist $\fa_0:=\fa_Y\subseteq\fa$ 
          und man definiert $\mu_{|Y}:\fa_Y\to[0,+\infty]$ durch 
          $\mu_{|Y}(A):=\mu(A)$ ist ein Maß auf $\fa_Y$.
\end{enumerate}
\end{beispiel}

\begin{satz}
\label{Satz 1.7}
\((X,\fa,\mu)\) sei ein Maßraum, es seien \(A,B\in\fa\) und 
\((A_{j})\) sei eine Folge in \(\fa\). Dann:
\begin{enumerate}
\item \(A\subseteq B\,\implies\,\mu(A)\leq\mu(B)\)
\item Ist \(\mu(A)<\infty\) und \(A\subseteq B,\implies\,\mu(B\setminus A)=\mu(B)-\mu(A)\)
\item Ist \(\mu\) endlich, dann ist \(\mu(A)<\infty\) und \(\mu(A^{c})=\mu(X)-\mu(A)\)
\item \(\mu\left(\bigcup A_{j}\right)\leq\sum{\mu(A_{j})}\) (\(\sigma\)-Subadditivität)
\item Ist \(A_{1}\subseteq A_{2}\subseteq A_{3}\subseteq\dots\), so ist \(\mu(\bigcup A_{j})=\lim_{n\to\infty}{\mu(A_{n})}\)
\item Ist \(A_{1}\supseteq A_{2}\supseteq A_{3}\supseteq\dots\) und \(\mu(A)<\infty\), so ist
	\(\mu(\bigcap A_{j})=\lim_{n\to\infty}{\mu(A_{n})}\)
\end{enumerate}
\end{satz}
\begin{beweis}
\begin{enumerate}
% Eigentlich muesste es in folgender Zeile statt B=(B\setminus A)\cup A korrekt 
% heissen: B=(B\setminus A)\cupdot A -- Spaeter
\item[(1)-(3)] \(B=(B\setminus A)\cup A\). Dann: \(\mu(B)=\underbrace{\mu(B\setminus A)}_{\geq0}+\mu(A)\geq\mu(A)\)
\item[(4)] % Das muesste jetzt eigentlich Punkt 4 sein
\(B_{1}=A_{1},\,B_{k}:=A_{k}\setminus\bigcup_{j=1}^{k-1}{A_{j}}\quad(k\geq 2)\)

Dann: \(B_{j}\in\fa,\,B_{j}\subseteq A_{j}\,(j\in\MdN);\,(B_{j})\) disjunkt und \(\bigcup A_{j}=\bigcup B_{j}\). Dann:
\[
\mu\left(\bigcup A_{j}\right)=\mu\left(\bigcup B_{j}\right)=\sum{\underbrace{\mu(B_{j})}_{\leq\mu(A_{j})}}\leq\sum{\mu(A_{j})}
\]
\item[(5)] % Das muesste jetzt eigentlich Punkt 5 sein
\(B_{1}=A_{1},\,B_{k}=A_{k}\setminus A_{k-1}\,(k\geq 2)\)

Dann: \(B_{j}\subseteq\fa;\,B_{j}\subseteq A_{j}\,(j\in\MdN);\,\bigcup A_{j}=\bigcup B_{j}\) und \(A_{n}=\bigcup_{j=1}^{n}{B_{j}}\)%\bigcupdot_{j=1}^{n}{B_{j}}\)

Dann: \(\mu(\bigcup A_{j})=\mu(\bigcup B_{j})=\sum{\mu(B_{j})}=\lim_{n\to\infty}{\underbrace{\sum_{j=1}^{n}{\mu(B_{j})}}_{=\mu\left(\bigcup_{j=1}^{n}{B_{j}}\right)=\mu(A_{n})}}\)
\item[(6)] Übung
\end{enumerate}
\end{beweis}

\clearpage
\section{Einwegfunktionen}
Eine Einwegfunktion ist in der Mathematik eine Beziehung zwischen 
zwei Mengen, die "`komplexitätstheoretisch "`schwer"' umzukehren ist"'\footnote{[Beutelspacher], S. 114}. 
Ein Beispiel für eine Einwegfunktion ist die Multiplikation zweier 
Zahlen. Die Laufzeit des Schönhage-Strassen-Algorithmus zur 
Multiplikation zweier $n$-stelliger ganzer Zahlen ist mit
$\mathcal{O}(n \cdot \log(n) \cdot \log(log(n)))$\footnote{[Pethö], S. 25}
deutlich kleiner als die Laufzeit von  des Zahlkörpersiebs 
$\mathcal{O}(e^{(1,92+o(1)) \sqrt[3]{\ln n} \sqrt[3]{(\ln \ln n)^2}})$\footnote{[Rothe], S. 384}, 
das der Faktorisierung dient.

Die Sicherheit des RSA-Verfahrens zur asymmetrischen 
Verschlüsselung basiert auf der Annahme, dass die Faktorisierung 
einer großen Zahl deutlich länger dauert als das Multiplizieren der 
Primfaktoren. Falls es keinen besseren Algorithmus zur Faktorisierung 
als zur Multiplikation gibt, ist diese Annahme korrekt. Nach dem 
Stand von 2009 ist dies der Fall.

Weitere Hinweise zur Sicherheit des RSA-Kryptosystems sind in \cref{sec:Security} zu finden.

\section{Restklassen}
Teilt man eine ganze Zahl $a$ durch eine ganze Zahl $m \neq 0$, so 
bleibt ein Rest $r \in \mathbb{N}_0$. Anhand aller möglichen Reste
$0 \leq r < m$ teilt man nun alle Zahlen in $|m|$ Teilmengen ein. 
Diese Teilmengen nennt man Restklassen. Man sagt, alle Zahlen, die 
den selben Rest $r$ beim Teilen durch $m$ lassen, gehören der selben 
Restklasse modulo $m$ an\footnote{[Forster], S. 45}. 
Ein Beispiel aus dem Alltag sind Zeitangaben. Man schreibt nicht 348 
Minuten, sondern 5 Stunden und 48 Minuten. Es wird also modulo 60 
gerechnet. Auch in der Grund-schule rechnet man mit Restklassen 
modulo 10, wenn man ganze Zahlen in Einer, Zehner und Hunderter 
unterteilt.\\
Ein weiteres Beispiel ist die Einteilung in  gerade und ungerade 
Zahlen. Bleibt bei einer Zahl kein Rest beim Teilen durch zwei, so 
wird sie als "`gerade"' bezeichnet und ist in einer Restklasse 
modulo 2 mit allen anderen geraden Zahlen.

\clearpage
In diesem Kapitel sei $\emptyset\ne X\in\fb_d$. Wir schreiben außerdem $\lambda$ statt $\lambda_d$.

\begin{definition}
\index{Lebesgueintegral}
Sei $f:X\to [0,\infty)$ eine einfache Funktion mit der Normalform $f=\sum_{j=1}^m y_j\mathds{1}_{A_j}$.\\
Das \textbf{Lebesgueintegral} von $f$ ist definiert durch:
\[\int_X f(x)\text{ d}x:=\sum_{j=1}^m y_j\lambda(A_j)\]
\end{definition}

\begin{satz}
\label{Satz 4.1}
Sei $f:X\to[0,\infty)$ einfach, $z_1,\dots,z_k\in[0,\infty)$ und $B_1,\dots,B_k\in\fb(X)$ mit $\bigcup B_j=X$ und $f=\sum_{j=1}^k z_j\mathds{1}_{B_j}$. Dann gilt:
\[\int_X f(x)\text{ d}x=\sum_{j=1}^k z_j\lambda(B_j)\]
\end{satz}

\begin{beweis}
In der großen Übung.
\end{beweis}

\begin{satz}
\label{Satz 4.2}
Seien $f,g:X\to[0,\infty)$ einfach, $\alpha, \beta\in[0,\infty)$ und $A\in\fb(X)$.
\begin{enumerate}
\item $\int_X \mathds{1}_A(x)\text{ d}x=\lambda(A)$
\item $\int_X (\alpha f+\beta g)(x)\text{ d}x = \alpha\int_X f(x)\text{ d}x + \beta\int_X g(x)\text{ d}x$
\item Ist $f\le g$ auf $X$, so ist $\int_X f(x)\text{ d}x\le \int_X g(x)\text{ d}x$.
\end{enumerate}
\end{satz}

\begin{beweis}
\begin{enumerate}
\item Folgt aus der Definition und \ref{Satz 4.1}.
\item Es seien $f=\sum_{j=1}^m y_j \mathds{1}_{A_j}$ und $g=\sum_{j=1}^k z_j \mathds{1}_{B_j}$ die Normalformen von $f$ und $g$. Dann gilt:
\[\alpha f+ \beta g=\sum_{j=1}^m \alpha y_j\mathds{1}_{A_j}+\sum_{j=1}^k \beta z_j\mathds{1}_{B_j}\]
Dann gilt:
\begin{align*}
\int_X (\alpha f+\beta g) &\stackrel{\ref{Satz 4.1}}= \sum_{j=1}^m \alpha y_j \lambda(A_j) + \sum_{j=1}^k \beta z_j \lambda(B_j)\\
&= \alpha \sum_{j=1}^m y_j \lambda(A_j) + \beta \sum_{j=1}^k z_j \lambda(B_j)\\
&= \alpha \int_X f(x)\text{ d}x + \beta \int_X g(x)\text{ d}x
\end{align*}
\item Definiere $h:=g-f$. Dann ist $h\ge 0$ und einfach. Sei $h=\sum_{j=1}^m x_j\mathds{1}_{C_j}$ die Normalform von $h$, d.h. $x_1,\dots,x_m\ge 0$. Dann gilt:
\[\int_X h(x)\text{ d}x = \sum_{j=1}^m x_j\lambda(C_j)\ge 0\]
Also folgt aus $g=f+h$ und (2):
\[\int_X g(x)\text{ d}x=\int_X f(x)\text{ d}x +\int_X h(x)\text{ d}x\ge \int_X f(x)\text{ d}x\]
\end{enumerate}
\end{beweis}

\begin{definition}
\index{Lebesgueintegral}
Sei $f:X\to[0,\infty]$ messbar. $(f_n)$ sei eine für $f$ zulässige Folge. Das \textbf{Lebesgueintegral} von $f$ ist definiert als:
\begin{align*}
\tag{$*$}\int_X f(x)\text{ d}x:=\lim_{n\to\infty}\int_X f_n(x)\text{ d}x
\end{align*}
\end{definition}

\begin{bemerkung}\ 
\begin{enumerate}
\item In \ref{Satz 4.3} werden wir sehen, dass $(*)$ unabhängig ist von der Wahl der für $f$ zulässigen Folge $(f_n)$.
\item $(f_n(x))$ ist wachsend für alle $x\in X$, d.h.:
\[f(x)=\lim_{n\to\infty} f_n(x)=(\sup_{n\in\mdn} f_n)(x)\]
\item Aus \ref{Satz 4.2}(3) folgt dass $(\int_X f_n(x)\text{ d}x)$ wachsend ist, d.h.:
\[\lim_{n\to\infty} \int_X f_n(x)\text{ d}x = \sup\Set{\int_X f_n(x)\text{ d}x | n\in\mdn}=\int_X f_(x)\text{ d}x\]
\end{enumerate}
\end{bemerkung}

\textbf{Bezeichnung:}\\
Für messbare Funktionen $f:X\to[0,\infty]$ definiere
\[M(f):=\Set{\int_X g\text{ d}x\mid g:X\to[0,\infty) \text{ einfach und }g\le f\text{ auf }X}\]

\begin{satz}
\label{Satz 4.3}
Ist $f:X\to[0,\infty]$ messbar und $(f_n)$ zulässig für $f$, so gilt:
\[L:=\lim_{n\to\infty}\int_X f_n\text{ d}x=\sup M(f)\]
Insbesondere ist $\int_X f(x) \text{ d}x$ wohldefiniert.
\end{satz}

\begin{folgerungen}
\label{Folgerung 4.4}
Ist $f:X\to[0,\infty]$ messbar, so ist $\int_X f(x) \text{ d}x=\sup M(f)$.
\end{folgerungen}

\begin{beweis}
Sei \(\int_Xf_n\,dx\in M(f) \,\forall\natn \). Dann ist \[L = \sup\left\{\int_Xf_n\,dx\mid\natn\right\} \leq \sup M(f)\]\\
Sei nun $g$ einfach und \(0\leq g\leq f\). Sei weiter \[g=\sum^m_{j=1}y_j\mathds{1}_{A_j}\] die Normalform von $g$.\\
Sei \(\alpha>1\) und \(B_n:=\{\alpha f_n\geq g\}\). Dann ist \[B_n\in\fb(X) \text{ und }(B_n\subseteq B_{n+1}\text{, sowie } \mathds{1}_{B_n}g\leq\alpha f_n.\]
Sei \(x\in X\).\\
\textbf{Fall 1:} Ist \(f(x)=0\), so ist wegen \(0\leq g\leq f\) auch \(g(x)=0\). Somit ist \(x\in B_n\) für jedes \(\natn\).\\
\textbf{Fall 2:} Ist  \(f(x)>0\), so ist \[\frac{1}{\alpha}g(x)<f(x)\] (Dies ist klar für \(g(x)=0\) und falls gilt: \(g(x)>0\), so ist \(\frac{1}{\alpha}g(x)<g(x)\leq f(x) \) )\\
Da $f_n$ zulässig für $f$ ist, gilt: \(f_n(x)\to f(x)\  (n\to\infty)\), weshalb ein \(n(x)\in\mdn\) existiert mit:
\[\frac{1}{\alpha}g(x)<f(x)\text{für jedes } n\geq n(x)\]
Es folgt \(x\in B_n\) für jedes \(n\geq n(x)\).\\
\textbf{Fazit:} \(X=\bigcup B_n\). \[A_j=A_j\cap X=A_j\cap\left(\bigcup B_n\right) = \bigcup(A_j\cap B_n) \text{ und } A_j\cap B_n\subseteq A_j\cap B_{n+1} \]
Aus \ref{Satz 1.7} folgt \(\lambda(A_j)=\lim\limits_{n\to\infty}\lambda(A_j\cap B_n)\). Das liefert:
\begin{align*}
   \int\limits_Xg\,dx &= \sum\limits_{j=1}^m y_j\lambda(A_j) 
   = \sum\limits_{j=1}^m y_j\lim\limits_{n\to\infty}\lambda(A_j\cap B_n)\\ 
   &=\lim\limits_{n\to\infty}\sum\limits_{j=1}^m y_j\lambda(A_j\cap B_n)
   \overset{\ref{Satz 4.1}}= \lim\limits_{n\to\infty} \int\limits_X \mathds{1}_{B_n}g\,dx\\
   &\leq  \lim\limits_{n\to\infty} \int\limits_X \alpha f_n\,dx
   =\alpha L
\end{align*}
g war einfach und \(0\leq g\leq f\) beliebig, sodass \[\sup M(f)\leq\alpha L \overset{\alpha\to 1}\implies \sup M(f)\leq L \]
\end{beweis}

\begin{satz}
\label{Satz 4.5}
Seien $f,g:X\to[0,\infty]$ messbar und $\alpha,\beta\ge0$.
\begin{enumerate}
\item $\int_X (\alpha f+\beta g)(x) \text{ d}x=\alpha\int_X f(x) \text{ d}x+\beta\int_X g(x) \text{ d}x$
\item Ist $f\le g$ auf $X$, so gilt $\int_X f(x) \text{ d}x\le \int_X g(x) \text{ d}x$
\item $\int_X f(x) \text{ d}x=0 \iff \lambda(\{f>0\})=0$
\end{enumerate}
\end{satz}

\begin{beweis}
\begin{enumerate}
\item \((f_n)\) und \((g_n)\) seien zulässig für $f$ bzw. $g$. Weiter sei \((h_n):=\alpha (f_n)+\beta (g_n) \).
Dann ist wegen \ref{Satz 3.7} und \(\alpha , \beta \geq 0\), dass \((h_n)\) zulässig für \(\alpha f+\beta g\) ist. Dann:
\begin{align*}
\int_X(\alpha f + \beta g)\,dx
&= \lim\limits_{n\to\infty}\int_X \left( \alpha (f_n)+\beta (g_n) \right)\,dx\\
&\overset{\ref{Satz 4.2}}= \alpha\lim\limits_{n\to\infty}\int_X(f_n)\,dx + \beta\lim\limits_{n\to\infty}\int_X(g_n)\,dx\\
&=\alpha\int_Xf\,dx + \beta\int_Xg\,dx
\end{align*}
\item Wegen \(f\leq g\) auf $X$ ist \(M(f)\subseteq M(g)\) und somit auch \(\sup M(f)\leq\sup M(g)\). Aus \ref{Folgerung 4.4} folgt nun die Behauptung.
\item Setze \(A:=\{f>0\}=\{x\in X:f(x)>0\}\).
\begin{enumerate}
\item["'$\implies$"'] Sei \(\int_Xf\,dx=0\) und \(A_n:=\{f>\frac{1}{n}\}\). Dann ist \(A=\bigcup A_n\) und \(f\geq\frac{1}{n}\mathds{1}_{A_n}\). Damit folgt:
\begin{align*}
0 = \int_Xf\,dx 
\overset{\text{(2)}}\geq \int_X\frac1{n}\mathds{1}_{A_n}\,dx
=\frac1{n}\lambda(A_n)
\intertext{Es ist also \(\lambda(A_n)=0\) und damit gilt weiter}
\lambda(A)=\lambda(\bigcup A_n) \overset{\ref{Satz 1.7}}\leq \sum\lambda(A_n)=0
\end{align*}
Also ist auch \(\lambda(A)=0\).
\item["'$\impliedby$"'] Sei \(\lambda(A)=0\), \((f_n)\) zulässig für $f$ und \(c_n:=\max\{f_n(x):x\in X\}\). Dann ist \(f_n\leq c_n\mathds{1}_A\) und es gilt:
\[0 \leq \int_Xf_n\,dx\overset{\text{(2)}} \leq \int_Xc_n\mathds{1}_A\,dx = c_n\lambda(A) \overset{\text{Vor.}} = 0 \]
Es ist also  \(\int_Xf_n\,dx=0\) für jedes $\natn$ und somit auch \(\int_Xf\,dx=0\)
\end{enumerate}
\end{enumerate}
\end{beweis}

\begin{satz}[Satz von Beppo Levi (Version I)]
\label{Satz 4.6}
Sei $(f_n)$ eine Folge messbarer Funktionen $f_n:X\to[0,\infty]$ und es gelte $f_n\le f_{n+1}$ auf $X$ für jedes $n\in\mdn$.
\begin{enumerate}
\item Für alle $x\in X$ existiert $\lim_{n\to\infty} f_n(x)$.
\item Die Funktion $f:X\to[0,\infty]$ definiert durch:
\[f(x):=\lim_{n\to\infty} f_n(x)\]
ist messbar.
\item $\int_X \lim\limits_{n\to\infty}f_n(x) \text{ d}x=\int_X f(x) \text{ d}x=\lim\limits_{n\to\infty}\int_X f_n(x) \text{ d}x$
\end{enumerate}
\end{satz}

\begin{beweis}
\begin{enumerate}
\item Für alle $x\in X$ ist \(\left(f_n(x)\right)\) wachsend, also konvergent in \([0,+\infty]\).
\item folgt aus \ref{Satz 3.5}.
\item Sei \( \left(u_j^{(n)}\right)_{j\in\mdn} \) zulässig für $f_n$ und \(v_j:=\max\left\{u_j^{(1)}, u_j^{(2)}, \dots , u_j^{(j)} \right\} \).
Aus \ref{Satz 3.7} folgt, dass $v_j$ einfach ist und aus der Konstruktion lässt sich nachrechnen, dass gilt:
 \[0\leq v_j\leq v_{j+1} \text{ und } v_j\leq f_n\leq f \text{ und } f_n=\sup\limits_{j\in\mdn}u_j^{(n)} \leq \sup\limits_{j\in\mdn}v_j \text{ (auf $X$)}\]
Damit ist $(v_j)$ zulässig für $f$ und es gilt:
\[ \int_Xf\,dx=\lim\limits_{j\to\infty}\int_Xv_j\,dx\leq\lim\limits_{j\to\infty}\int_Xf_j\,dx\leq\int_Xf\,dx \]
\end{enumerate}
\end{beweis}

\begin{satz}[Satz von Beppo Levi (Version II)]
\label{Satz 4.7}
Sei $(f_n)$ eine Folge messbarer Funktionen $f_n:X\to[0,\infty]$.
\begin{enumerate}
\item Für alle $x\in X$ existiert $s(x):=\sum_{j=1}^\infty f_j(x)$.
\item $s:X\to[0,\infty]$ ist messbar.
\item $\int_X \sum_{j=1}^\infty f_j(x) \text{ d}x= \sum_{j=1}^\infty \int_X f_j(x) \text{ d}x$
\end{enumerate}
\end{satz}

\begin{beweis}
Setze \[s_n:=\sum\limits_{j=1}^nf_j\]
Dann erfüllt \((s_n)\) die Voraussetzungen von \ref{Satz 4.6}. Aus 4.6 und \ref{Satz 4.5}(1) folgt die Behauptung.
\end{beweis}

\begin{satz}
\label{Satz 4.8}
Sei $f:X\to[0,\infty]$ messbar und es sei $\emptyset\ne Y\in\fb(X)$ (also $Y\subseteq X$ und $Y\in\fb_d$). Dann sind die Funktionen $f_{|Y}:Y\to[0,\infty]$ und $\mathds{1}_Y\cdot f:X\to[0,\infty]$ messbar und es gilt:
\[\int_Y f(x) \text{ d}x:=\int_Y f_{|Y}(x) \text{ d}x=\int_X (\mathds{1}_Y\cdot f)(x) \text{ d}x\]
\end{satz}

\begin{beweis}
\textbf{Fall 1:} Die Behauptung ist klar, falls $f$ einfach ist. (Übung!)\\
\textbf{Fall 2:} Sei \((f_n)\) zulässig für $f$ und \(g_n:=f_{n|Y} , h_n:=\mathds{1}_Y f_n\)
Dann ist \((g_n)\) zulässig für \(f_{|Y}\) und \((h_n)\) ist zulässig für \(\mathds{1}_Y f_n\).
Insbesondere sind  \(f_{n|Y}\) und \(\mathds{1}_Y f_n\) nach \ref{Satz 3.5} messbar.
Weiter gilt:
\[ \int_Y f_{|Y}\,dx \overset{n\to\infty}\longleftarrow \int_Yg_n\,dx \overset{Fall 1}=\int_Xh_n\,dx\overset{n\to\infty}\longrightarrow \int_X\mathds{1}_Yf\,dx   \]
\end{beweis}

\begin{definition}
\index{integrierbar}\index{Integral}\index{Lebesgueintegral}
Sei $f:X\to\imdr$ messbar. $f$ heißt (Lebesgue-)\textbf{integrierbar} (über $X$), genau dann wenn $\int_X f_+(x) \text{ d}x<\infty$ \textbf{und} $\int_X f_-(x) \text{ d}x<\infty$.\\
In diesem Fall heißt:
\[\int_X f(x) \text{ d}x:=\int_X f_+(x) \text{ d}x-\int_X f_-(x) \text{ d}x\]
das (Lebesgue-)\textbf{Integral} von $f$ (über $X$).
\end{definition}

\textbf{Beachte:}\\
Ist $f:X\to[0,\infty]$ messbar, so ist $f$ genau dann integrierbar, wenn gilt:
\[\int_X f(x) \text{ d}x<\infty\]

\begin{beispiel}
Sei $X \in \fb_1$, $f(x) := \begin{cases} 1&,x\in X\cap\MdQ\\ 0&,x\in X\setminus\MdQ\end{cases} = \mathds{1}_{X\cap\MdQ}$.
$X, \MdQ \in \fb_1 \implies X \cap \MdQ \in \fb_1 \implies f$ ist messbar.
\[0 \leq \int_X f(x) \text{ d}x = \int_X \mathds{1}_{X\cap\MdQ} \text{ d}x = \lambda(X\cap\MdQ) \leq \lambda(\MdQ) = 0\]
\textbf{Das heißt:} $f \in \fl^1(X)$, $\int_X f \text{ d}x = 0$.
Ist speziell $X = [a,b]\quad (a<b)$, so gilt: $f \in \fl^1([a,b])$, aber $f \not\in R([a,b])$. 
\end{beispiel}

\begin{satz}[Charakterisierung der Integrierbarkeit]
\label{Satz 4.9}
Sei $f: X \to \imdr$ messbar. Die folgenden Aussagen sind äquivalent:
\begin{enumerate}
 \item $f$ ist integrierbar.
 \item Es existieren integrierbare Funktionen $u, v: X \to [0,+\infty]$ mit $u(x)=v(x)=\infty$ für \textbf{kein} $x \in X$ und $f=u-v$ auf $X$.
 \item Es existiert eine integrierbare Funktion $g: X \to [0,+\infty]$ mit $\lvert f \rvert \leq g$ auf $X$.
 \item $\lvert f \rvert$ ist integrierbar.
\end{enumerate}
\end{satz}

\textbf{Zusatz:}
\begin{enumerate}
 \item $\fl^1(X) = \{f: X \to \mdr \mid f$ ist messbar und $\int_X \lvert f \rvert \text{ d}x < \infty\}$ (folgt aus (1)-(4)).
 \item Sind $u,v$ wie in (2), so gilt: $ \int_X f \text{ d}x = \int_X u \text{ d}x - \int_X v \text{ d}x$.
\end{enumerate}


\begin{beweis}[des Satzes]
\begin{enumerate}
 \item[(1) $\Rightarrow$ (2)] $u:= f_+$, $v := f_-$.
 \item[(2) $\Rightarrow$ (3)] $g := u+v$, dann ist $u,v \geq 0$, $g \geq 0$, $\int_X g \text{ d}x \stackrel{4.5}{=} \int_X u \text{ d}x + \int_X v \text{ d}x < \infty$. $\implies g$ ist integrierbar und: $|f| = |u-v| \leq |u| + |v| = u+v = g$ auf $X$.
 \item[(3) $\Rightarrow$ (4)] \ref{Satz 4.5} $\implies \int_X |f| \text{ d}x \leq \int_X g \text{ d}x < \infty \implies f$ ist integrierbar.
 \item[(4) $\Rightarrow$ (1)] $f_+, f_- \leq |f|$ auf $X$. $\implies 0 \leq \int_X f_\pm \text{ d}x \leq \int_X |f| \text{ d}x < \infty \stackrel{Def.}{\implies} f$ ist integrierbar.
\end{enumerate}
\end{beweis}

\begin{beweis}[des Zusatzes]
\begin{enumerate}
 \item \checkmark
 \item Es ist $f = u-v = f_+ - f_- \implies u+f_- = f_+ + v$.
\[\implies \int_X u \text{ d}x + \int_X f_- \text{ d}x \stackrel{4.5}{=} \int_X (u+ f_-) \text{ d}x = \int_X (f_+ + v) \text{ d}x \stackrel{4.5}{=} \int_X f_+ \text{ d}x + \int_X v \text{ d}x\]
\[\implies \int_X u \text{ d}x - \int_X v \text{ d}x = \int_X f_+ \text{ d}x - \int_X f_- \text{ d}x \stackrel{Def.}{=} \int_X f \text{ d}x. \]
\end{enumerate}
\end{beweis}

\begin{folgerungen}
\label{Folgerung 4.10}
\label{Satz 4.10}
Sei $f:X\to\imdr$ integrierbar und $N := \{\lvert f \rvert = +\infty\} = \{x\in X : \lvert f(x) \rvert = + \infty\}$. Dann ist $N\in \fb(X)$ und $\lambda(N) = 0$.
\end{folgerungen}

\begin{beweis}
 $\ref{Satz 3.4} \implies N \in \fb(X).$ $n\mathds{1}_N \leq \lvert f \rvert$ für alle $n\in \MdN$. Dann: 
\[n \cdot \lambda(N) = \int_X n\mathds{1}_N \text{ d}x \stackrel{4.5}{\leq} \int_X \lvert f \rvert \text{ d}x \stackrel{4.9}{<} \infty \text{  für alle } n \in \mdn\]
Also: $0 \leq n\lambda(N) \leq \int_X \lvert f \rvert \text{ d}x \quad \forall n \in \mdn \implies \lambda(N) = 0$ 
\end{beweis}

\begin{satz}
\label{Satz 4.11}
$f, g: X \to \imdr$ seien integrierbar und es sei $\alpha \in \mdr$.
\begin{enumerate}
 \item $\alpha f$ ist integrierbar und $\int_X (\alpha f) \text{ d}x = \alpha \int_X f \text{ d}x$.
 \item Ist $f+g:X\to\imdr$ auf $X$ definiert, so ist $f+g$ integrierbar und es gilt:
 \[\int_X (f+g)\text{ d}x = \int_X f \text{ d}x + \int_X g \text{ d}x\]
(Für $f=+\infty$ und $g=-\infty$ ist $f+g$ beispielsweise nicht definiert.)
 \item $\fl^1(X)$ ist ein reeller Vektorraum und die Abbildung $f \mapsto \int_X f \text{ d}x$ ist linear auf $\fl^1(X)$.
 \item $\max\{f,g\}$ und $\min\{f,g\}$ sind integrierbar.
 \item Ist $f\leq g$ auf $X$, so ist $\int_X f \text{ d}x \leq \int_X g \text{ d}x$.
 \item $\lvert \int_X f \text{ d}x \rvert \leq \int_X \lvert f \rvert \text{ d}x$. (Dreiecksungleichung für Integrale)
 \item Sei $\emptyset\ne Y \in \fb(X)$. Dann sind die Funktionen $f_{|Y}: Y \to \imdr$ und $\mathds{1}_Y\cdot f: X \to \imdr$ integrierbar und
\[\int_Y f(x) \text{ d}x := \int_Y f_{|Y} (x) \text{ d}x = \int_X(\mathds{1}_Y \cdot f)(x) \text{ d}x\]
 \item Sei $\lambda(X) < \infty$ und $h: X \to \mdr$ sei messbar und beschränkt. Dann: $h \in \fl^1(X)$ und $\lvert \int_X h \text{ d}x\rvert \leq \|h\|_\infty \lambda(X) \quad$ (mit $\|h\|_\infty := \sup\{|h(x)| : x\in X\}$) 
\end{enumerate}
\end{satz}

\begin{beweis}
\begin{enumerate} 
\item folgt aus \(\alpha f)_{\pm}=\alpha f_{\pm}\), falls \(\alpha\geq0\) und \(\alpha f)_{\pm}=-\alpha f_{\mp}\), falls 
    \(\alpha<0\).
\item Es gilt \(f+g=\underbrace{f_{+}+g_{+}}_{=:u}-\underbrace{(f_{-}+g_{-})}_{=:v}=u-v\). Dann:
\[
\int_{X}{u\mathrm{d}x}=\int_{X}{f_{+}+g_{+}\mathrm{d}x}\overset{\ref{Satz 4.5}}{=}\int_{X}{f_{+}\mathrm{d}x}+\int_{X}{g_{+}\mathrm{d}x}<\infty
\]
Genauso: \(\int_{X}{v\mathrm{d}x}<\infty\)\\
Mit Satz \ref{Satz 4.9} folgt: \(f+g\) ist integrierbar. Weiter:
\begin{align*}
\int_{X}{(f+g)\mathrm{d}x}&\overset{\ref{Satz 4.9}}{=}\int_{X}{u\mathrm{d}x}-\int_{X}{v\mathrm{d}x}\\
    &=\int_{X}{f_{+}\mathrm{d}x}+\int_{X}{g_{+}\mathrm{d}x}-\left(\int_{X}{f_{-}\mathrm{d}x}+\int_{X}{g_{-}\mathrm{d}x}\right)\\
    &=\int_{X}{f\mathrm{d}x}+\int_{X}{g\mathrm{d}x}
\end{align*}
\item folgt aus (1) und (2).
\item Mit Satz \ref{Satz 3.5} folgt: \(\max\{f,g\}\) ist messbar. Es gilt:
\[
0\leq\lvert\max\{f,g\}\rvert\leq\lvert f\rvert+\lvert g\rvert
\]
Mit \ref{Satz 4.9} und Aussage (2) folgt \(\lvert f\rvert+\lvert g\rvert\) ist integrierbar. Dann folgt mit Satz \ref{Satz 4.9}:
\(\max\{f,g\}\) ist integrierbar.\\
Analog zeigt man: \(\min\{f,g\}\) ist integrierbar.
\item Nach Voraussetzung ist \(f\leq g\) auf \(X\). Dann gilt: \(f_{+}\leq g_{+}\) auf \(X\) und \(f_{-}\geq g_{-}\) auf \(X\).
Es folgt:
\[
\int_{X}{f\mathrm{d}x}=\int_{X}{f_{+}\mathrm{d}x}-\int_{X}{f_{-}\mathrm{d}x}\overset{\ref{Satz 4.5}}{\leq}\int_{X}{g_{+}\mathrm{d}x}-\int_{X}{g_{-}\mathrm{d}x}=\int_{X}{g\mathrm{d}x}
\]
\item Es ist \(\pm f\leq\lvert f\rvert\). Mit Aussage (1) und (5) folgt: 
    \(\pm\int_{X}{f\mathrm{d}x}=\int_{X}{(\pm f)\mathrm{d}x}\leq\int_{X}{\lvert f\rvert\mathrm{d}x}\).\\
Es ist \(\int_{X}{f\mathrm{d}x}=\lvert\int_{X}{f\mathrm{d}x}\rvert\) oder \(-\int_{X}{f\mathrm{d}x}=\lvert\int_{X}{f\mathrm{d}x}\rvert\)
\item Mit Bemerkung (2) vor \ref{Satz 3.1} und Satz \ref{Satz 3.6}.(2) folgt: \(f_{|Y}\) und \(\mathds{1}_{Y}\cdot f\) sind
messbar. Es gilt: \((f_{|Y})_{\pm}=(f_{\pm})_{|Y}\) und \((\mathds{1}_{Y}\cdot f)_{\pm}=\mathds{1}\cdot f_{\pm}\). Weiterhin 
gilt \(0\leq\mathds{1}_{Y}f_{\pm}\leq f_{\pm}\). Mit \ref{Satz 4.9} folgt dann, daß\ \(\mathds{1}_{Y}f_{\pm}\) integrierbar
ist. Dann:
\begin{align*}
\int_{X}{(\mathds{1}_{Y}f)\mathrm{d}x}&=\int_{X}{\mathds{1}f_{+}\mathrm{d}x}-\int_{X}{\mathds{1}_{Y}f\mathrm{d}x}\\
    &=\underbrace{\int_{Y}{(f_{+})_{|Y}\mathrm{d}x}}_{<\infty}-\underbrace{\int_{Y}{(f_{-})_{|Y}\mathrm{d}x}}_{<\infty}
\end{align*}
Es folgt: \(f_{|Y}\) ist integrierbar und \(\int_{Y}{f_{|Y}\mathrm{d}x}=\int_{Y}{(f_{+})_{|Y}\mathrm{d}x}-\int_{Y}{(f_{-})_{|Y}\mathrm{d}x}=\int_{X}{(\mathds{1}_{Y}f)\mathrm{d}x}\).
\item Es ist \(\lvert h\rvert\leq\lVert h\rVert_{\infty}\cdot\mathds{1}_{X}\). Dann folgt:
\[
\int_{X}{\lvert h\rvert\mathrm{d}x}\leq\int_{X}{\lVert h\rVert_{\infty}\mathds{1}_{X}\mathrm{d}x}=\lVert h\rVert_{\infty}\lambda(X)<\infty
\]
Damit: \(\lvert h\rvert\) ist integrierbar und mit \ref{Satz 4.9} auch \(h\). Da \(h\) beschränkt ist, folgt: 
\(h\in\fl^{1}(X)\). Schließlich:
\[
\left\lvert\int_{X}{h\mathrm{d}x}\right\rvert\leq\int_{X}{\lvert h\rvert\mathrm{d}x}\leq\lVert h\lVert_{\infty}\lambda(X)
\]
\end{enumerate}
\end{beweis}

\begin{satz}
\label{Satz 4.12}
\begin{enumerate}
 \item Sind $\emptyset\ne A,B \in \fb(X)$ disjunkt, $X = A \cup B$ und ist $f: X \to \imdr$ integrierbar (über $X$), so ist $f$ integrierbar über $A$ und integrierbar über $B$ und es gilt:
 \[\int_X f \text{ d}x = \int_A f \text{ d}x + \int_B f \text{ d}x\]
 \item Ist $\emptyset \neq K \subseteq \mdr^d $ kompakt und $f:K\to\mdr$ stetig, so ist $f \in \fl^1(K)$.
\end{enumerate}

\end{satz}

\begin{beweis}
\begin{enumerate}
 \item Aus \ref{Satz 4.11}(7) folgt: $f$ ist integrierbar über $A$ und integrierbar über $B$. Es ist 
\[ \int_X f(x) \text{ d}x = \int_X \left( \mathds{1}_{A\cup B} \cdot f \right)(x) \text{ d}x = \int_X \left( \left( \mathds{1}_A + \mathds{1}_B \right) f\right)(x) \text{ d}x \]
\[= \int_X \left(\mathds{1}_A f + \mathds{1}_B f \right)(x) \text{ d}x \stackrel{4.11(2)}{=} \int_X \mathds{1}_A f \text{ d}x + \int_X \mathds{1}_B f \text{ d}x \stackrel{4.11(7)}{=} \int_A f \text{ d}x + \int_B f \text{ d}x.\]

 \item $K$ ist kompakt, also gilt: $\lambda(K) < \infty$. Aus \ref{Satz 3.2}(1) folgt, dass $f$ messbar ist. Analysis II (\glqq stetige Funktionen auf kompakten Mengen nehmen Minimum und Maximum an\grqq ) liefert: $f$ ist beschränkt. Insgesamt folgt mit \ref{Satz 4.11}(8) schließlich: $f \in \fl^1(K)$.
\end{enumerate}
\end{beweis}

\begin{satz}
\label{Satz 4.13}
Seien $a,b\in\mdr$, $a<b$, $X:=[a,b]$ und $f\in C(X)$. Dann ist $f\in\fl^1(X)$ und es gilt:
\[L-\int_X f(x) \text{ d}x=R-\int_a^b f(x) \text{ d}x\]
\end{satz}

\begin{beweis}
Sei $\natn$, $t_j^{(n)}:=a+j\frac{b-a}{n}$ ($j=0,\dots,n$) und $I_j^{(n)}:=\left[t_{j-1}^{(n)},t_j^{(n)}\right]$ ($j=1,\dots,n$).
\begin{align*}
S_n:=\sum^n_{j=1} f \left(t_j^{(n)}\right) \underbrace{ \frac{b-a}{n}}_{= \lambda_1 \left(I_j^{(n)}\right)} \text{ ist Riemannsche Zwischensumme für R-} \int_a^bf(x)\,dx.
\end{align*}
Aus Analysis I folgt $S_n\to\text{R-}\int_a^bf(x)\,dx$ ($n\to\infty$). 
Definiere $f_n:=\sum^n_{j=1}f \left(t_j^{(n)} \right) \mathds{1}_{I_j^{(n)}} $. Dann ist $f_n$ einfach und 
\[\int_X f_n(x)\,dx=\sum_{j=1}^n f \left(t_j^{(n)} \right) \lambda_1 \left(I_j^{(n)}\right)=S_n\]
$f$ ist auf $X$ gleichmäßig stetig also konvergiert $f_n$ auf $X$ gleichmäßig gegen $f$ (Übung!), also gilt:
\[\lVert f_n-f \rVert_{\infty}=\text{sup} \left \{ \lvert f_n(x)-f(x) \rvert : x\in X \right\} \to 0 \  (n\to \infty)\]
Aus \ref{Satz 4.12}(2) folgt $f\in \mathfrak{L}^1(X)$
\begin{align*}
\left\lvert \text{L-} \int \limits_X f(x)\,dx -S_n \right\rvert = \left\lvert \text{L-} \int \limits_X (f-f_n)\,dx \right\rvert \stackrel{\text{4.11}}\leq \int \limits_X(f-f_n)\,dx \stackrel{\text{4.11}}\leq \lVert f-f_n \rVert_{\infty} \underbrace{\lambda(X)}_{=b-a} \to 0
\end{align*}
Daraus folgt $S_n \to$ L- $\int_X f\,dx$
\end{beweis}

\begin{satz}
\label{Satz 4.14}
Sei $a\in\mdr, X:=[a,\infty)$ und $f\in C(X)$. Dann gilt:
\begin{enumerate}
\item $f$ ist messbar.
\item $f\in\fl^1(X)$ genau dann wenn das uneigentliche Riemann-Integral $\int_a^\infty f(x) \text{ d}x$ \textbf{absolut} konvergent ist. In diesem Fall gilt:
\[L-\int_X f(x) \text{ d}x=R-\int_a^\infty f(x) \text{ d}x\]
Entsprechendes gilt für die anderen Typen uneigentlicher Riemann-Integrale.
\end{enumerate}
\end{satz}

\begin{beweis}
Eine Hälfte des Beweises folgt in Kapitel \ref{Kapitel 6}.
\end{beweis}

\begin{beispiel}
\begin{enumerate}
\item Sei $X=(0,1]$, $f(x)=\frac{1}{\sqrt{x}}$. Aus Analysis I wissen wir, dass R-$\int^1_0\frac{1}{\sqrt{x}}\,dx$ (absolut) konvergent ist. Also ist $f\in\mathfrak{L}^1(X)$.\\
Außerdem wissen wir aus Analysis I, dass R-$\int_0^1\frac{1}{x}$ divergent ist. Also ist $f^2\notin\mathfrak{L}^1(X)$.
\item Sei $X=[0,\infty)$, $f(x)=\frac{\sin(x)}{x}$. Aus Analysis I wissen wir, dass R-$\int^{\infty}_1f(x)\,dx$ konvergent, aber nicht absolut konvergent ist. Also ist $f\notin\mathfrak{L}^1(X)$.
\end{enumerate}
\end{beispiel}

\section{Lineare Kongruenzen}
\subsection{Allgemeine Informationen}
Zwei Zahlen $a, b \in \mathbb{Z}$ heißen kongruent modulo $m \in \mathbb{N}$, 
falls $a$ und $b$ bei der Division durch $m$ den den gleichen Rest lassen. 
Man schreibt $a \equiv b \imod{m}$\footnote{[Reiss], S. 179f}.

Gilt $ax \equiv b \imod{m}$, für $a, b, x \in \mathbb{Z}$ und $m \in \mathbb{N}$,
dann bedeutet das, dass $m | (ax - b)$ für ein passendes $x$. 
Man nennt $ax \equiv b \imod{m}$ ein lineares Kongruenzsystem. 
\clearpage 

\subsection{Chinesischer Restsatz}
Der Chinesische Restsatz sagt, ob lineare Kongruenzsysteme lösbar 
sind und wie diese Lösungen aussehen:

\begin{mdframed}[tikzsetting={draw=red,ultra thick}, innertopmargin=0.6cm]
Seien $m_1, m_2, ..., m_n$ paarweise teilerfremde natürliche Zahlen und
$a_1, a_2, \dots, a_n$ ganze Zahlen.

Dann ist das System linearer Kongruenzen
\vspace{-0.4cm}
\[x \equiv a_1 \imod{m_1},\;\;\; x \equiv a_2 \imod{m_2},\;\;\;\dots,\;\;\; x \equiv a_n \imod{m_n}\]
lösbar. Alle Lösungen des Systems liegen in einer gemeinsamen
 Restklasse modulo $M=\prod_{i = 1}^n m_i$
\end{mdframed}

\textbf{Beweis nach [Reiss], S. 221f:}
\begin{enumerate}[label=(\Roman{*}),labelsep=0.5em,noitemsep]
    \item $M_j = \frac{M}{m_j}$ für $j = 1, \dots, n$
    \item $y_j \cdot M_j \equiv 1 \imod{m_j}$, $y_j$ mit dem erweitertem Euklidischem Algorithmus bestimmen
    \item $a_j \cdot y_j \cdot M_j \equiv a_j \imod{m_j}$ für $j = 1, \dots, n$\\
Weil $m_j$ für $i \neq j$ ein Teiler von  $M_i$ ist, gilt auch:
    \item $a_i \cdot y_i \cdot M_i \equiv 0 \imod{m_j}$ für alle $i, j = 1, \dots, n$ mit $i \neq j$
\end{enumerate}

Da alle Summanden bis auf Einen ($j = i$) gleich Null sind, stimmt dieser Ausdruck:
\begin{align*}
a_i \cdot y_i \cdot M_i &\equiv \sum_{j=1}^n {a_j \cdot y_j \cdot M_j} \imod{m_i}\\
a_i &\equiv \sum_{j=1}^n {a_j \cdot y_j \cdot M_j} \imod{m_i}\text{, da }y_i \cdot M_i \equiv 1 \imod{m_i}
\end{align*}

$a_i$ ist die Lösung des Kongruenzsystems. Alle Lösungen liegen in dieser Restklasse.


\subsubsection*{Beispielaufgabe}
Folgende Aufgabe wurde [Berendt] entnommen:

\hangindent2em
\hangafter=0
17 chinesische Piraten erbeuten eine Truhe mit Goldstücken. Beim Versuch, diese gleichmäßig zu verteilen, bleiben 7 Goldstücke übrig. Um diese entbrennt ein heftiger Streit, bei dem einer der Piraten das Leben lässt. Die verbleibenden 16 versuchen erneut, die Goldstücke gerecht zu verteilen, behalten jedoch elf Stücke übrig. Bei der folgenden Auseinandersetzung geht wieder einer der Streitenden über Bord. Den 15 Überlebenden gelingt dann die Teilung. Wie viele Goldstücke müssen es mindestens gewesen sein?

\subsubsection*{Restklassensystem} % This should semantically rather be subsubsubsection
\begin{align*}
x &:= \text{Anzahl der Goldstücke}\\
x &\equiv 7 \imod{17}\\
x &\equiv 11 \imod{16}\\
x &\equiv 0 \imod{15}
\end{align*}

\subsubsection*{Lösung}
I Produkte
\begin{align*}
M   &= 17 \cdot 16 \cdot 15 = 4080\\
M_1 &= \frac{4080}{17} = 240\\
M_2 &= \frac{4080}{16} = 255\\
M_3 &= \frac{4080}{15} = 272
\end{align*}

II Multiplikativ Inverses der Restklassensysteme
\begin{align*}
 9 \cdot 240 &\equiv 1 \imod{17}\\
15 \cdot 255 &\equiv 1 \imod{16}\\
8 \cdot 272 &\equiv 1 \imod{15}
\end{align*}

III  Multiplikation der Restklassensysteme mit $a_j$
\begin{align*}
7 \cdot 9 \cdot 240     &\equiv 7   \imod{17}\\
11 \cdot 15 \cdot 255   &\equiv 11  \imod{16}\\
8 \cdot 272             &\equiv 0   \imod{15}
\end{align*}

IV Berechnung der Lösung des Restklassensystem
\begin{align*}
x = \sum_{j = 1}^3 a_j \cdot y_j \cdot M_j \imod{15 \cdot 16 \cdot 17} = 7 \cdot 240 \cdot 9 + 11 \cdot 255 \cdot 15 = 57195\\
57195 \equiv 75 \imod{4080}\\
75 \text{ ist die kleinste positive Lösung des Kongruenzsystems.}
\end{align*}

\subsubsection*{Antwort:}
Die Anzahl der von den Piraten erbeuteten Goldstücken muss mindestens $75$ betragen, kann aber auch $75 + 1 \cdot 4080$, $75 + 2 \cdot 4080$  oder ein beliebiger anderer positiver Vertreter dieser Restklasse$\imod{4080}$ sein.

\clearpage
Stets in diesem Kapitel: \(\emptyset\neq X\in\fb_{d}\)

\begin{lemma}[Lemma von Fatou]
\label{Lemma 6.1}
Sei \((f_{n})\) eine Folge messbarer Funktionen \(f_{n}:\,X\to[0,+\infty]\).
\begin{enumerate}
\item Es gilt:
\[\int_{X}{\left (\liminf_{n\to\infty}f_{n} \right)(x)\mathrm{d}x}\leq\liminf_{n\to\infty}{\int_{X}{f_{n}(x)\mathrm{d}x}}\]
\item Ist \(f: X\to[0,+\infty]\) messbar und gilt \(f_{n}\to f\) fast überall,
so ist
\[
\int_{X}{f\mathrm{d}x}\leq\liminf_{n\to\infty}{\int_{X}{f_{n}\mathrm{d}x}}
\]
\item Ist \(f\) wie in (2) und ist \(\left(\int_{X}{f_{n}\mathrm{d}x}\right)\)
beschränkt, so ist \(f\) integrierbar.
\end{enumerate}
\end{lemma}

\begin{beweis}
\begin{enumerate}
\item \(g_{j}:=\inf_{n\geq j}{f_{n}}\) \folgtnach{\ref{Satz 3.5}} \(g_{j}\) ist messbar. Klar: \(g_{j}\leq g_{j+1}\) auf
\(X\); \(\displaystyle \sup_{j\in\mdn}{g_{j}}=\liminf_{n\to\infty}{f_{n}}\)

Weiter: \(g_{j}\leq f_{n}\,(n\geq j)\)

Dann:
\begin{align*}
    \int_{X}{\liminf_{n\to\infty}f_{n}\mathrm{d}x}&=\int_{X}{\sup_{j\in\mdn}g_{j}\mathrm{d}x}\\
	&=\int_{X}{\lim_{j\to\infty}g_{j}(x)\mathrm{d}x}\\
	&\overset{\ref{Satz 4.6}}{=}\lim_{j\to\infty}\int_{X}{g_{j}\mathrm{d}x}\\
	&=\sup_{j\in\mdn}\underbrace{\int_{X}{g_{j}\mathrm{d}x}}_{\leq\inf_{n\geq j}\int_{X}{f_{n}\mathrm{d}x}}\\
	&\leq\sup_{j\in\mdn}\left\{\inf_{n\geq j}\int_{X}{f_{n}\mathrm{d}x}\right\}\\
	&=\liminf_{n\to\infty}\int_{X}{f_{n}\mathrm{d}x}
\end{align*}
\item Es existiert eine Nullmenge \(N\subseteq X\): \(f_{n}(x)\to f(x)\;\forall x\in X\setminus N\).\\
Dann gilt \(f=\mathds{1}_{X\setminus N}\cdot f\) fast überall.
\begin{align*}
    \int_{X}{f\mathrm{d}x}
    &\overset{\text{\ref{Satz 5.3}.(3)}}{=}\int_{X}{\mathds{1}_{X\setminus N}\cdot f\mathrm{d}x}\\
    &=\int_{X}{\left (\lim_{n\to\infty}\mathds{1}_{X\setminus N}f_{n} \right )\mathrm{d}x}\\
    &\overset{(1)}{\leq}\liminf_{n\to\infty}\int_{X}{\mathds{1}_{X\setminus N}f_{n}\mathrm{d}x}\\
    &\overset{\text{\ref{Satz 5.3}.(3)}}{=}\liminf_{n\to\infty}\int_{X}{f_{n}\mathrm{d}x}
\end{align*}
\item folgt aus (2). Nach Voraussetzung gilt 
\[
0\leq\int_{X}{f\mathrm{d}x}\overset{\text{(2)}}{\leq}\liminf_{n\to\infty}\int_{X}{f_{n}\mathrm{d}x}<\infty
\]
\end{enumerate}
\end{beweis}

\begin{satz}[Konvergenzsatz von Lebesgue (Majorisierte Konvergenz)]
\label{Satz 6.2}
\((f_{n})\) sei eine Folge messbarer Funktionen \(f_{n}:X\to\imdr\), \((f_{n})\)
konvergiere fast überall und es sei \(g:X\to[0,+\infty]\) integrierbar. Für
jedes \(n\in\mdn\) gelte \(\lvert f_{n}\rvert\leq g\) fast überall.\\
Dann sind alle \(f_{n}\) integrierbar und es existiert ein \(f\in\fl^{1}(X)\) mit:
\begin{enumerate}
\item \(f_{n}\to f\) fast überall
\item \(\int_{X}{f_{n}\mathrm{d}x}\to\int_{X}{f\mathrm{d}x}\)
\item \(\int_{X}{\lvert f_{n}-f\rvert\mathrm{d}x}\to 0\)
\end{enumerate}
\end{satz}

\begin{beispiele}
% Hier fehlt eventuell eine Grafik
\item Sei \(X=\mdr,\,f_{n}:=n\mathds{1}_{(0,\frac{1}{n})}\). Dann:
\[
\int_{X}{f_{n}\mathrm{d}x}=n\cdot\lambda_{1}\left(\left(0,\frac{1}{n}\right)\right)=n\cdot\frac{1}{n}=1\quad\forall n\in\mdn
\]
Es gilt \(f_{n}\to f:=0\) punktweise und \(\int_{X}{f\mathrm{d}x}=0 \neq 1 = \int_{X}{f_{n}\mathrm{d}x}\). 
$\Rightarrow$ \ref{Satz 6.2} ist ohne die integrierbare Majorante 
$g$ im allgemeinen falsch.
\item Sei $X = [1, \infty), \alpha > 1, f_n(x) := \frac{1}{x^\alpha} \sin{\frac{x}{n}} (x \in X, n \in \mathbb{N})$.\\
Berechne $\lim_{n \rightarrow \infty} \int_X f_n(x) \mathrm{d}x$\\
$f_n(x) \rightarrow 0 =: f(x) \quad \forall x \in X$. $|f_n(x)| = \frac{1}{x^\alpha} |\sin{\frac{x}{n}}| \leq \frac{1}{x^\alpha} =: g(x)$
\folgtnach{AI} $R-\int_1^\infty g(x) \mathrm{d}x$ konvergiert absolut \folgtnach{4.14} $g \in lebeq^1(X)$
\folgtnach{6.2} $\int_X f_n \mathrm{d}x \rightarrow \int_X f \mathrm{d}x = 0, \int_X |f_n| \mathrm{d}x \rightarrow 0$
\end{beispiele}

\begin{beweis}
% Nummerierung vernuenftig zurechtbasteln
\begin{enumerate}
\item Aus \ref{Satz 5.4} folgt: Es existiert \(\hat{f}:X\to\imdr\) messbar mit \(f_{n}\to\hat{f}\) fast überall.
Es existiert eine Nullmenge \(N_{0}\subseteq X:\,f_{n}(x)\to\hat{f}(x)\,\forall x\in X\setminus N_{0}\)
\item Für alle \(n\in\mdn\) existiert eine Nullmenge \(N_{n}\subseteq X:\,\lvert f_{n}(x)\rvert\leq g(x)\,\forall x\in X\setminus N_{n}\).

Setze \(N:=\bigcup_{n=0}^{\infty}{N_{n}}\). Mit \ref{Lemma 5.1} folgt: \(N\) ist eine Nullmenge.

Wir haben: \(\lvert f_{n}(x)\rvert\leq g(x)\,\forall x\in X\setminus N\forall n\in\mdn\) und
\(\lvert\hat{f}(x)\rvert\leq g(x)\,\forall x\in X\setminus N\).
\item \(f_{n}=\mathds{1}_{X\setminus N}f_{n}\) fast überall und \(\hat{f}=\mathds{1}_{X\setminus N}\hat{f}\)
fast überall.

Es gilt \(\lvert\mathds{1}_{X\setminus N}f_{n}\rvert\leq g\) und \(\lvert\mathds{1}_{X\setminus N}\hat{f}\rvert\leq g\). Mit
\ref{Satz 4.9} folgt: \(\mathds{1}_{X\setminus N}f_{n}\) und \(\mathds{1}_{X\setminus N}\hat{f}\) sind integrierbar.

Mit \ref{Satz 5.3}.(1) folgt: \(f_{n}\) und \(\hat{f}\) sind integrierbar.
\item \(\tilde{N}:=N\cup\{\lvert\hat{f}\rvert=\infty\}\cup\{g=\infty\}\). Mit \ref{Folgerung 4.10} und \ref{Lemma 5.1} folgt:
\(\tilde{N}\) ist eine Nullmenge.

Setze \(f:=\mathds{1}_{X\setminus N}\hat{f}\). Dann: \(f\) ist messbar; es ist \(\lvert f\rvert\leq\lvert\hat{f}\rvert\).
Mit \ref{Satz 4.9} folgt: \(f\) ist integrierbar.

Es ist \(f(X)\subseteq\mdr\). Also: \(f\in\fl^{1}(X)\).

Sei \(x\in X\setminus\tilde{N}:\,f(x)=\tilde{f}(x)=\lim_{n\to\infty}f_{n}(x)\).
D.h. \(f_{n}\to f\) fast überall.
\item Definiere $g_n:=|f|+\mathds{1}_{X\setminus \tilde N}g-\mathds{1}_{X\setminus \tilde N}|f_n-f|$. Es ist fast überall
\begin{align*}
\mathds{1}_{X\setminus \tilde N}g=g&&\mathds{1}_{X\setminus \tilde N}|f_n-f|=|f_n-f|
\end{align*}
Nach \ref{Satz 5.3}(1) ist $g$ integrierbar und $g_n\to |f|+g$ fast überall. Es gilt:
\begin{align*}
|f_n-f|\le|f_n|+|f|\le g+|f| \text{ auf} X\setminus\tilde N
\end{align*}
D.h. es ist $g\ge0$ auf X.
\item Es gilt:
\begin{align*}
\int_X(|f|+g)\text{ d}x&\stackrel{\ref{Lemma 6.1}(2)}\le \liminf_{n\to\infty} \int_X g_n \text{ d}x\\
&=\liminf \left(\int_{\tilde N} g_n\text{ d}x+\int_{X\setminus\tilde N}g_n\text{ d}x\right)\\
&=\liminf \int_{X\setminus\tilde N}g_n\text{ d}x\\
&=\liminf \int_{X\setminus\tilde N}(|f|+g-|f_n-f|)\text{ d}x\\
&=\int_{X\setminus\tilde N} (|f|+g)\text{ d}x-\limsup \int_{X\setminus\tilde N}|f_n-f|\text{ d}x\\
&\stackrel{\ref{Satz 5.2}(3)}= \int_X |f|+g\text{ d}x-\limsup\int_X |f_n-f|\text{ d}x
\end{align*}
Daraus folgt:
\[\limsup\int_x|f_n-f|\text{ d}x\le 0\]
Also gilt auch:
\[|\int_Xf_n\text{ d}x-\int_Xf\text{ d}x|=|\int_X(f_n-f)\text{ d}x\le \int_X|f_n-f|\text{ d}x\to 0\]
\end{enumerate}
\end{beweis}

\begin{beispiel}
Sei \(X:=[1,\infty)\) und \(f_n(x):=\frac1{x^\frac32}\sin\left(\frac xn \right) \) für alle \(x\in X, n\in\mdn\) mit \(f_n(x)\to f(x)\equiv 0\) für jedes \(x\in X\).
Dann ist \(\lvert f_n(x) \rvert\leq \frac1{x^\frac32}\) für jedes \(x\in X\) und $\natn$. 
Definiere nun \[g(x):=\frac1{x^\frac32}\]
Aus Analysis I ist bekannt, dass \(\int^\infty_1 g(x)\,dx\) (absolut) konvergent ist 
und aus \ref{Satz 4.14} folgt \[g\in\mathfrak{L}^1(X) \text{ sowie } \int_X g(x)\,dx = \text{R-}\int^\infty_1 g(x)\,dx\]
Weiter folgen aus \ref{Satz 6.2}:
\[\int_X f_n\,dx\to 0 \text{ und } \int_X\lvert f_n\rvert\,dx\to 0 \ (n\to\infty) \]
\end{beispiel}

\begin{folgerung}[aus \ref{Satz 6.2}]
\label{Folgerung 6.3}
\begin{enumerate}
	\item 	Sei \(f:X\to\imdr\) messbar und \((A_n)\) sei eine Folge in \(\fb(X)\) mit \(A_n\subseteq A_{n+1}\) für jedes $\natn$ und \(X=\bigcup A_n\). Weiter sei
		\begin{align*}
		f_n:=\mathds{1}_{A_n}\cdot f \text{ integrierbar für alle } \natn \intertext{und} \left(\int_{A_n}\lvert f\rvert\,dx\right) \text{ sei beschränkt. }
		\end{align*}
		Dann ist $f$ integrierbar und es gilt: \[\int_{A_n}f\,dx \to \int_Xf\,dx \quad \text{für } n \to \infty\]
	\item 	Sei \(a\in\mdr\), \(X:=[a,\infty]\) und \(f:X\to\mdr\) sei stetig. Weiter sei R-\(\int_a^\infty f\,dx\) \textbf{absolut} konvergent. Dann ist \(f\in\mathfrak{L}^1(X)\) und wie in 					\ref{Satz 4.14}:
		\[\text{L-}\int_Xf\,dx=\text{R-}\int^\infty_a f\,dx \]
\end{enumerate}
\end{folgerung}

\begin{beweis}
\begin{enumerate}
	\item 	Sei \(x\in X\). Es exisitert ein $m\in\mdn$, für das \(x\in A_m\) ist und somit auch \(x\in A_n \) für jedes \(n\geq m\). Nach der Definition von $f_n$ gilt dann \(f_n(x)=f(x)\) für jedes 				\(n\geq m\) und somit \(f_n\to f\) auf $X$. Damit gilt auch \[\lvert f_n\rvert\to\lvert f\rvert \text{ auf } X\] Durch die Konstruktion der $f_n$ ergibt sich: 
		\[ \lvert f_n\rvert=\lvert \mathds{1}_{A_n}f\rvert=\mathds{1}_{A_n}\lvert f\rvert \leq \mathds{1}_{A_{n+1}}\lvert f\rvert=\lvert f_{n+1}\rvert \]
		Dann gilt:
		\[ \int_X \lvert f\rvert\,dx \overset{\ref{Satz 4.6}}=\lim\int_X \lvert f_n\rvert\,dx = \lim\int_{A_n} \lvert f\rvert\,dx \overset{Vor.}<\infty \]
		Es folgt, dass \(\lvert f\rvert\) integrierbar ist und somit ist nach \ref{Satz 4.9} auch $f$ integrierbar. Da \(\lvert f_n\rvert \leq \lvert f\rvert\) auf $X$ für jedes \(\natn\) gilt, ist $f$ eine 		
		integrierbare Majorante und es folgt mit \ref{Satz 6.2}:
		\[ \int_Xf\,dx = \lim\int_Xf_n\,dx = \lim\int_{A_n}f\,dx \]
	\item 	Setze \(A_n:=[a,n]\ (\natn)\) und es gelte o.B.d.A.: \(a\leq 1\). Dann gilt:
		\[ \int_{A_n}\lvert f\rvert\,dx \overset{\ref{Satz 4.13}}= \text{R-}\int^n_a \lvert f\rvert\,dx \overset{Vor.}\longrightarrow \text{R-}\int^\infty_a \lvert f\rvert\,dx \]
		D.h.\(\left(\int_{A_n}\lvert f\rvert\,dx\right)\) ist beschränkt. Definiere \(f_n:=\mathds{1}_{A_n}f\) mit \ref{Satz 4.13} folgt daraus, dass $f_n$ integrierbar ist. Weiter folgt
		aus (1) \(f\in\mathfrak{L}^1(X)\) (denn es ist \(f(X)\subseteq\mdr\)) und 
		\[ \text{L-}\int_Xf\,dx = \lim\int_{A_n}f\,dx \overset{\ref{Satz 4.13}}= \lim\left(\text{R-}\int^n_a f\,dx \right) = \text{R-}\int^\infty_a f\,dx. \]
\end{enumerate}
\end{beweis}

\begin{bemerkung}
\ref{Folgerung 6.3}(2) gilt entsprechend für die anderen Typen uneigentlicher Riemann-Integrale.
\end{bemerkung}

\begin{folgerung}
\label{Folgerung 6.4}
\begin{enumerate}
	\item 	\((f_n)\) sei eine Folge integrierbarer Funktionen \(f_n\colon X\to\imdr\), \(g\colon X\to[0,+\infty]\) sei ebenfalls integrierbar und 
		\[g_n:=f_1+f_2+\dots+f_n \ (\natn)\]
		Weiter sei $N$ eine Nullmenge in $X$ so, dass \((g_n(x))\) für jedes \(x\in X\setminus N\) in $\imdr$  konvergiert und 
		\[\lvert g_n(x)\rvert \leq g(x) \text{ für jedes } \natn \text{ und } x\in X\setminus N\]
		Setzt man
		\[f(x):=\sum^\infty_{j=1}f_j(x):=	
		\begin{cases}
			0, 				& \text{falls } x\in N 			\\
			\lim\limits_{n\to\infty}g_n(x), & \text{falls } x\in X\setminus N
		\end{cases}\quad,\]
		so gilt, dass $f$ integrierbar ist und
		\[\int_X \left( \sum^\infty_{j=1}f_j(x) \right)\,dx = \sum^\infty_{j=1}\left( \int_Xf_j(x)\,dx \right) \]
	\item 	Sei \(f\in\mathfrak{L}^1(X)\) und \((A_n)\) eine \textbf{disjunkte} Folge in \(\fb(X)\) mit \(X=\dot\bigcup A_n\). Dann gilt
		\[\int_Xf\,dx = \sum^\infty_{j=1}\int_{A_j}f\,dx \]
\end{enumerate}
\end{folgerung}

\begin{beweis}
\begin{enumerate}
	\item 	Fast überall gelten \(g_n\to f\) und für jedes \(\natn\) auch \(\lvert g_n\rvert \leq g\). Aus \ref{Satz 6.2} folgt
		\begin{align*}
			\int_X \left(\sum^\infty_{j=1}f_j(x)\right) \,dx 
			&= \int_Xf\,dx  					\\
			&\overset{\ref{Satz 6.2}}= \lim\int_Xg_n\,dx 	\\
			&= \lim\int_X\left(\sum^n_{j=1}f_j\right)\,dx 	\\
			&=\lim\sum^n_{j=1}\int_Xf_j(x)\,dx 			\\
			&=\sum^\infty_{j=1}\int_Xf_j\,dx 			\\
		\end{align*}
	\item 	Setze \(f_j:=\mathds{1}_{A_j}f\), \(g:=\lvert f\rvert\), \(g_n:=f_1+\dots+f_n\). Dann ist
		\[\lvert g_n\rvert = \lvert \mathds{1}_{A_1\cup\dots\cup A_n}\cdot f\rvert \leq \lvert f\rvert =g \]
		Es gilt: \(g_n\to f\) auf $X$. Aus (1) folgt
		\[ \int_Xf\,dx = \sum^\infty_{j=1}\int_Xf_j\,dx = \sum^\infty_{j=1}\int_{A_j}f\,dx \]
\end{enumerate}
\end{beweis}

\clearpage
In diesem Kapitel sei stets \(\emptyset\neq X\in \fb_d\).

\begin{satz}
\label{Satz 7.1}
Sei \(U\in\fb_k, t_0\in U\) und es sei \(f\colon U\times X\to \mdr\) eine Funktion mit:
\begin{enumerate}
	\item 	Für jedes \(t\in U\) ist \(x\mapsto f(t,x)\) messbar.
	\item 	Es existiert eine Nullmenge \(N\subseteq X\) so, dass \(t\mapsto f(t,x)\) für jedes \(x\in X\setminus N\) stetig in $t_0$ ist.
	\item 	Es existiert eine integrierbare Funktion \(g\colon X\to [0,\infty]\) und zu jedem \(t\in U\) existiert eine Nullmenge \(N_t\subseteq X\) so, dass für 
		jedes \(t\in U\) und jedes \(x\in X\setminus N_t\) gilt: \[ \lvert f(t,x)\rvert \leq g(x) \]
\end{enumerate}
Dann ist \(x\mapsto f(t,x)\) für jedes \(t\in U\) integrierbar. Ist \(F\colon U\to\mdr\) definiert durch
\[ F(t):=\int_Xf(t,x)\,dx,\]
so ist $F$ stetig in $t_0$.
\end{satz}

Also: \[ \lim\limits_{t\to t_0}\int_X f(t,x)\,dx = \lim\limits_{t\to t_0}F(t)=F(t_0) = \int_X f(t_0,x)\,dx =\int_X\lim\limits_{t\to t_0} f(t,x)\,dx \]

\begin{beweis}
Aus (1) und (3) folgt, dass \(x\mapsto f(t,x)\) für jedes \(t\in U\) integrierbar ist (zur Übung). Sei \((t_n)\) eine Folge in $U$ mit \(t_n\to t_0\) und 
\[g_n(x):=f(t_n,x) \ (\natn, x\in X) \]
Setze \[ \tilde N := N\cup \left(\bigcup^\infty_{n=1}N_{t_n} \right) \]
Aus \ref{Lemma 5.1} folgt, dass \(\tilde N\) eine Nullmenge ist. Voraussetzung (2) liefert \(g_n(x)\to f(t_0,x)\) für jedes \(x\in X\setminus\tilde N\), also gilt 
\[g_n(x)\to f(t_0,x) \text{ fast überall auf } X\]
Voraussetzung (3) liefert \(\lvert g_n(x)\rvert = \lvert f(t_n,x)\rvert \leq g(x) \) für jedes \(\natn\) und \(x\in X\setminus\tilde N\). Aus \ref{Satz 6.2} folgt
\[ F(t_n) = \int_X f(t_n,x)\,dx = \int_Xg_n\,dx \longrightarrow \int_X f(t_0,x)\,dx = F(t_0) \]
\end{beweis}

\textbf{Bezeichnung}\\
Sei \(I\subseteq\mdr\) ein Intervall, \(a:=\inf I\) und \(b:=\sup I\), wobei \(a=-\infty\) oder \(b=+\infty\) zugelassen sind. Weiter sei \(f\colon I\to\imdr\) integrierbar 
(oder $f$ ist messbar und \(\geq 0\)) und 
\[\int\limits^b_af(x)\,dx:=\int\limits_{(a,b)}f_{|(a,b)}(x)\,dx \]
Dann ist 
\[ \int_I f(x) dx = \int_{(a,b)} f(x) dx\]
Ist z.B. \(I=[a,b)\), dann gilt, da \(\{a\}\) eine Nullmenge ist: \[\int_If\,dx=\int_{\{a\}}f\,dx + \int_{(a,b)}f\,dx= \int_{(a,b)}f\,dx \] 

\begin{folgerung}
\label{Folgerung 7.2}
Sei \(I\subseteq\mdr\) ein Intervall, \(a=\inf I\) und \(f\colon I\to\mdr\) sei integrierbar. Definiert man \(F\colon I\to\mdr\) durch 
\[F(t):=\int^t_a f(x)\,dx,\] so ist \(F\in C(I)\).
\end{folgerung}

\begin{beweis}
Für \(x,t\in I\) definiere \(h(t,x):=\mathds{1}_{(a,t)}f(x)\). Dann ist \(F(t)=\int_I h(t,x)\,dx\) und 
\[\lvert h(t,x)\rvert = \mathds{1}_{(a,t)}\cdot \lvert f(x)\rvert \leq \lvert f(x)\rvert \text{ für alle } t,x\in I\]
Aus \ref{Satz 4.9} folgt, dass \(\lvert f\rvert\) integrierbar ist. Sei \(t_0\in I\) und \(N:=\{t_0\}\), also eine Nullmenge.
Dann ist \(t\mapsto h(t,x)\) für jedes \(x\in I\setminus N\) stetig in \(t_0\) (zur Übung). Die Behauptung folgt aus \ref{Satz 7.1}.
\end{beweis}

\begin{satz}
\label{Satz 7.3}
Sei \(U\subseteq \mdr^k\) offen, \(f\colon U\times X\to\mdr\) eine Funktion. Es sei \(g\colon X\to [0,+\infty]\) integrierbar und \(N\subseteq X\) sei eine Nullmenge.
Weiter gelte:
\begin{enumerate}
	\item 	für jedes \(t\in U\) sei \(x\mapsto f(t,x)\) integrierbar.
	\item 	für jedes \(x\in X\setminus N\) sei \(t\mapsto f(t,x)\) partiell differenzierbar auf $U$.
	\item 	\(\left\lvert \frac{ \partial f}{\partial t_j} \right\rvert \leq g(x) \) für jedes \(x\in X\setminus N\), jedes \(t\in U\) und jedes \(j\in\{1,\dots,k\}\)
\end{enumerate}
Ist dann \(F\colon U\to\mdr\) definiert durch \[F(t):=\int_X f(t,x)dx\] so ist $F$ auf $U$ partiell differenzierbar und für jedes \( t\in U\) sowie jedes \( j\in\{1,\dots,k\}\) gilt:
\[ \frac{\partial F}{\partial t_j}(t) = \int_X\frac{\partial f}{\partial t_j}(t,x)\,dx \]
\end{satz}
\textbf{Also: } \( \frac{\partial}{\partial t_j}\int_X f(t,x)\,dx = \int_X \frac{\partial f}{\partial t_j}(t,x)\,dx \).

\begin{beweis}
Sei o.B.d.A. \(k=1\), also \(U\subseteq\mdr\). Sei \(t_0\in U\) und \((h_n)\) eine Folge mit \(h_n\to 0\) und \(h_n\neq 0\) für alle \(\natn\).
Setze \[g_n(x):=\frac{f(t_0+h_n,x)-f(t_0,x)}{h_n} \ \ (x\in X, \natn) \]
Aus Voraussetzung (2) folgt für jedes \(x\in X\setminus N\): \[ g_n(x)\to \frac{\partial f}{\partial t}(t_0,x) \ \ (n\to\infty) \]
Nach dem Mittelwertsatz aus Analysis 1 existiert für jedes \(x\in X\setminus N\) und jedes \(\natn\) ein \(s_n=s_n(x)\) zwischen \(t_0\) und \(t_0+h_n\) mit:
\[ \left\lvert g_n(x) \right\rvert = \left\lvert \frac{\partial f}{\partial t}(s_n,x)\right\rvert \overset{(3)}\leq g(x) \]
Aus \ref{Satz 6.2} folgt \[ \int_X g_n\,dx \longrightarrow \int_X\frac{\partial f}{\partial t}(t_0,x)\,dx \]
Es ist nach Konstruktion  gerade \(\int_X g_n\,dx =\frac{F(t_0+h_n)-F(t_0)}{h_n} \).
\end{beweis}

\clearpage

\section{Literaturverzeichnis}

\textbf{Beutelspacher, A., Neumann, H. B. und Schwarzpaul, T.:}\\
Kryptografie in Theorie und Praxis.\\
Wiesbaden, Vieweg+Teubner Verlag, 2005.

\textbf{Brill, M.:}\\
Mathematik für Informatiker.\\
Wien, Hanser Verlag, 2004.

\textbf{Forster, O.:}\\
Algorithmische Zahlentheorie.\\
Wiesbaden, Vieweg Braunschweig/Wiesbaden, 1996.

\textbf{Pethö, A. und Pohst, M.:}\\
Algebraische Algorithmen.\\
Wiesbaden, Vieweg+Teubner Verlag, 1999.

\textbf{Reiss, K. und Schmieder, G.:}\\
Basiswissen Zahlentheorie.\\
Berlin Heidelberg, Springer-Verlag, 2007.

\textbf{Rothe, J.:}\\
Komplexitätstheorie und Kryptologie.\\
Berlin, Springer, 2008.

\textbf{Wrixon, F. B.:}\\
Geheimsprachen.\\
Königswinter, Tandem Verlag GmbH, 2006.

\newpage
\subsection*{Internetadressen}

TODO

\clearpage

\section{Anhang}

\inputminted[linenos, numbersep=5pt, tabsize=4]{pascal}{Factoring.ari}

\clearpage

\newpage
\vspace{10cm}

Ich erkläre, dass ich die Facharbeit ohne fremde Hilfe angefertigt 
und nur die im Literaturverzeichnis angeführten Quellen und 
Hilfsmittel benützt habe.

\vspace{10cm}

\titledate{Ort, Datum}


\end{document}
