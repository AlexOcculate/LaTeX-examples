\section{Eulersche $\varphi$-Funktion}
Die Eulersche $\varphi$-Funktion gibt für jede natürliche Zahl $n$ an, 
wie viele positive ganze Zahlen $a \leq n $ zu ihr relativ prim sind\footnote{[Brill], S. 148}.
$a$ ist zu $n$ relativ prim, wenn $ggT(a,n) = 1$ gilt, also wenn $a$ 
und $n$ keinen größeren gemeinsamen Teiler als $1$ haben. Man sagt 
auch "`a und b sind teilerfremd"'.

$\varphi(n)$ ist zugleich die Ordnung der multiplikativen Gruppe $(\mathbb{Z}/n \mathbb{Z})^*$. 
$\varphi(n)$ gibt also an, wie viele Zahlen im Restklassenring modulo $n$ ein multiplikativ Inverses haben. Mehr dazu in Kapitel 6.	% TODO

Für Primzahlen gilt $\varphi(p) = p - 1$ , da eine Primzahl nur 
durch sich und eins teilbar ist. Sei $A$ die multiplikative Gruppe 
einer Primzahl $p$, $B$ die multiplikative Gruppe einer Primzahl $q$ 
und $C$ die multiplikative Gruppe von $p \cdot q$. Dann ist $|C| = |A| \cdot |B|$ und 
$\varphi(p \cdot q) = |C|$, $\varphi(p) = |A|$ sowie $\varphi(q) = |B|$.

Daraus folgt, dass $\varphi(pq) = \varphi(p) \cdot \varphi(q) = (p-1) \cdot (q - 1)$ für zwei Primzahlen $p \neq q$ gilt.
