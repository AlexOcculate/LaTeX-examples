Für den DYCOS-Algorithmus wurde in \cite{aggarwal2011} bewiesen,
dass sich nach Ausführung von DYCOS für einen unbeschrifteten
Knoten mit einer Wahrscheinlichkeit von höchstens
$(1-k)\cdot e^{-l \cdot b^2 / 2}$ eine Knotenbeschriftung ergibt, deren
relative Häufigkeit weniger als $b$ der häufigsten Beschriftung ist.
Dabei ist $k$ die Anzahl der Klassen und $l$ die Länge der 
Random-Walks.

Außerdem wurde experimentell anhand des DBLP-Datensatzes\footnote{http://dblp.uni-trier.de/}
und des CORA-Datensatzes\footnote{http://people.cs.umass.edu/~mccallum/data/cora-classify.tar.gz}
gezeigt, dass die Klassifikationsgüte nicht wesentlich von der
maximalen Listenlänge $a$ und der Anzahl der Wörter mit
höchstem Gini-Koeffizient $m$ abhängt. Obwohl es sich nicht sagen lässt,
wie genau die Ergebnisse aus \cite{aggarwal2011} zustande gekommen sind,
eignet sich das Kreuzvalidierungsverfahren zur Bestimmung der Klassifikationsgüte
wie es in \cite{Lavesson,Stone1974} vorgestellt wird:
\begin{enumerate}
    \item Betrachte nur $V_{L,T}$.
    \item Unterteile $V_{L,T}$ zufällig in $k$ disjunkte Mengen $M_1, \dots, M_k$.
    \item \label{schritt3} Teste die Klassifikationsgüte, wenn die Knotenbeschriftungen
          aller Knoten in $M_i$ für DYCOS verborgen werden für $i=1,\dots, k$.
    \item Bilde den Durchschnitt der Klassifikationsgüten aus \cref{schritt3}.
\end{enumerate}

Es wird $k=10$ vorgeschlagen.


