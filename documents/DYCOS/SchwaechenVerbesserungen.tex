Der in \cite{aggarwal2011} vorgestellte Algorithmus hat einige Probleme,
die im Folgenden erläutert werden. Außerdem werden Verbesserungen
vorgeschlagen, die es allerdings noch zu untersuchen gilt.

\subsection{Schwächen von DYCOS}
\subsubsection{Anzahl der Labels}
So, wie er vorgestellt wurde, können nur Graphen bearbeitet werden, 
deren Knoten höchstens ein Label haben. In vielen Fällen, wie z.~B. 
Wikipedia mit Kategorien als Labels haben Knoten jedoch viele Labels.

Auf einen ersten Blick ist diese Schwäche einfach zu beheben, indem 
man beim zählen der Labels für jeden Knoten jedes Label zählt. Dann
wäre noch die Frage zu klären, mit wie vielen Labels der betrachtete
Knoten gelabelt werden soll.

Jedoch ist z.~B. bei Wikipedia-Artikeln auf den Knoten eine 
Hirarchie definiert. So ist die Kategorie \enquote{Klassifikationsverfahren}
eine Unterkategorie von \enquote{Klassifikation}. Bei dem Kategorisieren
von Artikeln sind möglichst spezifische Kategorien vorzuziehen, also
kann man nicht einfach bei dem Auftreten der Kategorie \enquote{Klassifikationsverfahren}
sowohl für diese Kategorie als auch für die Kategorie \enquote{Klassifikation}
zählen.


\subsubsection{Überanpassung und Reklassifizierung}
Aggarwal und Li beschreiben in \cite{aggarwal2011} nicht, auf welche
Knoten der Klassifizierungsalgorithmus angewand werden soll. Jedoch
ist die Reihenfolge der Klassifizierung relevant. Dazu folgendes 
minimale Beispiel:

Gegeben sei ein dynamischer Graph ohne textuelle Inhalte. Zum Zeitpunkt
$t=1$ habe dieser Graph genau einen Knoten $v_1$ und $v_1$  sei
mit dem Label $A$ beschriftet. Zum Zeitpunkt $t=2$ komme ein nicht-gelabelter
Knoten $v_2$ sowie die Kante $(v_1, v_2)$ hinzu.\\
Nun wird der DYCOS-Algorithmus auf diesen Knoten angewendet und
$v_2$ mit $A$ gelabelt.\\
Zum Zeitpunkt $t=3$ komme ein Knoten $v_3$, der mit $B$ gelabelt ist,
und die Kante $(v_3, v_2)$ hinzu.

\begin{figure}[ht]
    \centering
    \subfloat[$t=1$]{
        \tikzstyle{vertex}=[draw,black,circle,minimum size=10pt,inner sep=0pt]
\tikzstyle{edge}=[very thick]
\begin{tikzpicture}[scale=1,framed]
    \node (a)[vertex,label=$A$] at (0,0) {$v_1$};
    \node (b)[vertex, white] at (1,0) {$v_2$};
    \node (struktur)[label={[label distance=-0.2cm]0:$t=1$}] at (-2,1) {};
\end{tikzpicture}

        \label{fig:graph-t1}
    }%
    \subfloat[$t=2$]{
        \tikzstyle{vertex}=[draw,black,circle,minimum size=10pt,inner sep=0pt]
\tikzstyle{edge}=[very thick]
\begin{tikzpicture}[scale=1,framed]
    \node (a)[vertex,label=$A$] at (0,0) {$v_1$};
    \node (b)[vertex,label={\color{blue}$A$}] at (1,0) {$v_2$};
    \draw[->] (b) -- (a);
    \node (struktur)[label={[label distance=-0.2cm]0:$t=2$}] at (-2,1) {};
\end{tikzpicture}

        \label{fig:graph-t2}
    }

    \subfloat[$t=3$]{
        \tikzstyle{vertex}=[draw,black,circle,minimum size=10pt,inner sep=0pt]
\tikzstyle{edge}=[very thick]
\begin{tikzpicture}[scale=1,framed]
    \node (a)[vertex,label=$A$] at (0,0) {$v_1$};
    \node (b)[vertex,label={\color{blue}$A$}] at (1,0) {$v_2$};
    \node (c)[vertex,label=$B$] at (2,0) {$v_3$};
    \draw[->] (b) -- (a);
    \draw[->] (b) -- (c);
    \node (struktur)[label={[label distance=-0.2cm]0:$t=3$}] at (-1,1) {};
\end{tikzpicture}

        \label{fig:graph-t3}
    }%
    \label{Formen}
    \caption{Minimalbeispiel für den Einfluss früherer DYCOS-Anwendungen}
\end{figure}

Würde man nun den DYCOS-Algorithmus erst jetzt, also anstelle von
Zeitpunkt $t=2$ zum Zeitpunkt $t=3$ auf den Knoten $v_2$ anwenden, so
würde eine $50\%$-Wahrscheinlichkeit bestehen, dass dieser mit $B$ 
gelabelt wird. Aber in diesem Beispiel wurde der Knoten schon
zum Zeitpunkt $t=2$ gelabelt. Obwohl es in diesem kleinem Beispiel
noch keine Rolle spielt, wird das Problem klar, wennn man weitere
Knoten einfügt:

Wird zum Zeitpunkt $t=4$ ein ungelabelter Knoten $v_4$ und die Kanten
$(v_1, v_4)$, $(v_2, v_4)$, $(v_3, v_4)$ hinzugefügt, so ist die 
Wahrscheinlichkeit, dass $v_4$ mit $A$ gelabelt wird bei $75\%$.
Werden die als ungelabelten Knoten jedoch erst jetzt und alle gemeinsam
gelabelt, so ist die Wahrscheinlichkeit für $A$ als Label bei nur $50\%$.
Bei dem DYCOS-Algorithmus findet also eine Überanpassung and vergangene
Labels statt.

Das Reklassifizieren von Knoten könnte eine mögliche Lösung für dieses
Problem sein. Knoten, die durch den DYCOS-Algorithmus gelabelt wurden
könnten eine Lebenszeit bekommen (TTL, Time to Live). Ist diese 
abgelaufen, wird der DYCOS-Algorithmus erneut auf den Knoten angewendet.

\subsection{Schwächen des Papers}
Die Ergebnise der experimentelle Analyse können aus folgenden Gründen
nicht überprüft werden:
\begin{itemize}
    \item DYCOS verwendet als Vokabular die Top-$m$-Wörter mit dem
          höchsten Gini-Index aus einer Sample-Menge von Texten, die
          wie in \cite{Vitter} beschrieben
          erzeugt wird. Allerdings wird niemals erklärt, wie $m \in \mathbb{N}$
          bestimmt wird. Es ist nicht einmal klar, ob $m$ für den
          Algorithmus als konstant anzusehen ist oder ob $m$ sich
          bei der Vokabularbestimmung ändern könnte.
    \item DYCOS beschränkt sich bei inhaltlichen Mehrfachsprüngen
          auf die Top-$q$-Wortknoten, also die $q$ ähnlichsten Knoten
          gemessen mit der Aggregatanalyse. Auch hier wird nicht erklärt wie
          $q \in \mathbb{N}$ bestimmt oder nach welchen Überlegungen $q$ gesetzt 
          wurde. Allerings ist hier wenigstens klar, dass $q$ für
          den DYCOS-Algorithmus konstant ist. Für die Experimentelle
          Analyse wurde zwar erwähnt, dass $q$ ein Parameter des
          Algorithmus ist\cite[S. 362]{aggarwal2011}, aber nicht welcher
          Wert in der Analyse des DBLP-Datensatzes genutzt wurde.
          Für den CORA-Datensatz wurde $q=10$ gewählt\cite[S. 364]{aggarwal2011}.
    \item Für die Analyse der CORA-Datensatzes\footnote{inzwischen unter \href{http://people.cs.umass.edu/~mccallum/data/cora-classify.tar.gz}{http://people.cs.umass.edu/~mccallum/data/cora-classify.tar.gz}} analysiert.
          Dieser beinhaltet Forschungsarbeiten, wobei die 
          Forschungsgebiete die in einen Baum mit 73 Blättern 
          eingeordnet wurden. Aus diesen 73 Blättern wurden 5 Klassen
          extrahiert und der Graph, der keine Zeitpunkte beinhaltet,
          künstlich in 10 Graphen mit Zeitpunkten unterteilt. Wie
          jedoch die TODO
\end{itemize}
