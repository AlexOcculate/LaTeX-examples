\subsection{Schwächen des Papers}
In \cite{aggarwal2011} wurde eine experimentelle Analyse mithilfe 
des DBLP-Datensatzes\footnote{http://dblp.uni-trier.de/} und des
CORA-Datensatzes\footnote{\href{http://people.cs.umass.edu/~mccallum/data/cora-classify.tar.gz}{http://people.cs.umass.edu/~mccallum/data/cora-classify.tar.gz}} durchgeführt.
Die Ergebnisse dieser Analyse können aus folgenden Gründen
nicht überprüft werden:
\begin{itemize}
    \item Der Parameter $a \in \mathbb{N}$, der die Anzahl der ausgehenden Kanten
          aller Wortknoten beschränkt, wird erst mit der Experimentellen
          Analyse auf S.~362 eingeführt.
          Es ist nicht klar, wie entschieden wird welche Kanten
          gespeichert werden und welche nicht.
    \item Für die Analyse der CORA-Datensatzes analysiert.
          Dieser beinhaltet Forschungsarbeiten, wobei die 
          Forschungsgebiete die in einen Baum mit 73 Blättern 
          eingeordnet wurden. Aus diesen 73 Blättern wurden 5 Klassen
          extrahiert und der Graph, der keine Zeitpunkte beinhaltet,
          künstlich in 10 Graphen mit Zeitpunkten unterteilt. Wie
          jedoch diese Unterteilung genau durchgeführt wurde kann nicht
          nachvollzogen werden.
    \item Der auf S. 360 in \enquote{Algorithm 1} vorgestellte
          Pseudocode soll den DYCOS-Algorithmus darstellen. Allerdings
          werden die bereits klassifizierten Knoten $\T_t$ neu klassifiziert
          und mit $\theta$ die Klassifikationsgüte gemessen.
\end{itemize}
