\documentclass[11pt,a4paper,oneside]{scrartcl}
\usepackage{amssymb, amsmath} % needed for math
\usepackage[utf8]{inputenc} % this is needed for umlauts
\usepackage[ngerman]{babel} % this is needed for umlauts
\usepackage[T1]{fontenc}    % this is needed for correct output of umlauts in pdf
\usepackage[margin=2.5cm]{geometry} %layout
\usepackage{hyperref}       % links im text
\usepackage{enumerate}      % for advanced numbering of lists
\usepackage{fancyheadings}  % Kopfzeile
\usepackage{array}          % needed for m{1cm} in tabular
\usepackage{enumitem}
\usepackage{wasysym}
\usepackage{framed}

%%%%%%%%%%%%%%%%%%%%%%%%%%%%%%%%%%%%%%%%%%%%%%%%%%%%%%%%%%%%%%%%%%%%%
% Change the following lines:                                       %
%%%%%%%%%%%%%%%%%%%%%%%%%%%%%%%%%%%%%%%%%%%%%%%%%%%%%%%%%%%%%%%%%%%%%
\newcommand\yourName{Martin Thoma}
\newcommand\yourTutorial{10}
%%%%%%%%%%%%%%%%%%%%%%%%%%%%%%%%%%%%%%%%%%%%%%%%%%%%%%%%%%%%%%%%%%%%%

\hypersetup { 
  pdfauthor   = {Martin Thoma}, 
  pdfkeywords = {Feedback}, 
  pdftitle    = {Feedback} 
}

\pagestyle{fancy}% eigenen Seitestil aktivieren}
\lhead{Tutor: \yourName}
\rhead{Tutorium Nr. \yourTutorial}

\begin{document}
\section*{Über mich}
Ich bin \dots\\
\begin{tabular}{lllll}
{\huge \Square} Info      & {\huge \Square} InWi & {\huge \Square} Schüler        & {\huge \Square} Sonstiges: \_\_\_\_\_\_\_\_\_\_\_\_ \\
\end{tabular}

\section*{Über das Tutorium}
Das \textbf{Quiz} ist \dots\\
\begin{tabular}{lllll}
{\huge \Square} lehrreich      & {\huge \Square} informativ & {\huge \Square} nervig        & {\huge \Square} lustig    & \_\_\_\_\_\_\_\_\_\_\_\_\\
{\huge \Square} viel zu selten & {\huge \Square} zu selten  & {\huge \Square} genau richtig & {\huge \Square} zu häufig & {\huge \Square} viel zu häufig\\
{\huge \Square} viel zu leicht & {\huge \Square} zu leicht  & {\huge \Square} genau richtig & {\huge \Square} zu schwer & {\huge \Square} viel zu schwer\\
\end{tabular}\\

\noindent Die \textbf{Praxisaufgaben} sind \dots\\
\begin{tabular}{lllll}
{\huge \Square} lehrreich      & {\huge \Square} informativ & {\huge \Square} nervig        & {\huge \Square} lustig    & \_\_\_\_\_\_\_\_\_\_\_\_\\
{\huge \Square} viel zu selten & {\huge \Square} zu selten  & {\huge \Square} genau richtig & {\huge \Square} zu häufig & {\huge \Square} viel zu häufig\\
{\huge \Square} viel zu leicht & {\huge \Square} zu leicht  & {\huge \Square} genau richtig & {\huge \Square} zu schwer & {\huge \Square} viel zu schwer\\
\end{tabular}\\

\noindent Die \textbf{Folien} sind \dots\\
\begin{tabular}{lllll}
{\huge \Square} übersichtlich   & {\huge \Square} verwirrend & {\huge \Square} informativ   & {\huge \Square} gut    & \_\_\_\_\_\_\_\_\_\_\_\_\\
\end{tabular}\\

\noindent Die \textbf{Wiederholung des Vorlesungsstoffs} war\dots\\
\begin{tabular}{lllll}
{\huge \Square} viel zu selten & {\huge \Square} zu selten  & {\huge \Square} genau richtig & {\huge \Square} zu häufig & {\huge \Square} viel zu häufig\\
\end{tabular}

\section*{Über den Tutor}
Der Tutor \textbf{spricht} \dots\\
\begin{tabular}{lllll}
{\huge \Square} viel zu leise & {\huge \Square} zu leise  & {\huge \Square} genau richtig & {\huge \Square} zu laut & {\huge \Square} viel zu laut\\
{\huge \Square} viel zu langsam & {\huge \Square} zu langsam  & {\huge \Square} genau richtig & {\huge \Square} zu schnell & {\huge \Square} viel zu schnell\\
\end{tabular}

\section*{Freitext}
Diese Themen sollten wiederholt werden:
\begin{framed}
  \hfill\vspace{3cm}
\end{framed}

\noindent \textbf{Andere Kommentare}: z.B. Was ist unklar? 
Was war heute gut / schlecht?
\begin{framed}
  \hfill\vspace{3cm}
\end{framed}
\end{document}
