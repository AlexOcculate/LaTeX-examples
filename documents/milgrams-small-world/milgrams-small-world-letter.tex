\documentclass[a4paper, 12pt, KOMAold, sections]{scrlttr2}
\usepackage[utf8]{inputenc} % this is needed for umlauts
\usepackage[ngerman]{babel} % this is needed for umlauts
\usepackage[T1]{fontenc}    % needed for right umlaut output in pdf
\usepackage[ngerman, num]{isodate} % get DD.MM.YYYY dates

\usepackage{hyperref}
 
% Anpassen %%%%%%%%%%%%%%%%%%%%%%%%%%%%%%%%%%%%%%%%%%%%%%%%%%%%%%%%%%
\newcommand{\Vorname}{Martin}     % Vorname des Senders             %
\newcommand{\Nachname}{Thoma}     % Nachname des Senders            %
\newcommand{\Strasse}{Parkstraße} % Deine Straße                    %
\newcommand{\Hausnummer}{17}      % Deine Hausnummer                %
\newcommand{\PLZ}{76131}          % Deine PLZ                       %
\newcommand{\Ort}{Karlsruhe}      % Dein Ort                        %
\newcommand{\Kundennr}{123456}    % Deine Kundennummer              %
                                                                    %
\newcommand{\Empfaenger}{Lisa Müller} % Der Empfänger               %
\newcommand{\EStrasse}{Poststr. 17}   % Straße des Empfängers       %
\newcommand{\EPLZ}{12345}             % PLZ des Empfängers          %
\newcommand{\EOrt}{Berlin}            % Ort des Empfängers          %
                                                                    %
\newcommand{\DocTitle}{Milgrams Small World Experiment} %Titel des Dokuments%
%%%%%%%%%%%%%%%%%%%%%%%%%%%%%%%%%%%%%%%%%%%%%%%%%%%%%%%%%%%%%%%%%%%%%
 
 
% pdfinfo
\hypersetup{ 
  pdfauthor   = {\Nachname, \Vorname}, 
  pdfkeywords = {Experiment, Fun, Web}, 
  pdftitle    = {\DocTitle} 
}
 
% set letter variables
\signature{\Vorname~\Nachname}
\backaddress{\Vorname~\Nachname, \Strasse~\Hausnummer, \PLZ~\Ort}
\newcommand{\section}[1]{\noindent\textbf{#1}\newline}
 
% Begin document %%%%%%%%%%%%%%%%%%%%%%%%%%%%%%%%%%%%%%%%%%%%%%%%%%%%
\begin{document}
    \begin{letter}{\Empfaenger \\ \EStrasse \\ \EPLZ~\EOrt}
    \date{\today}%Change this if you want a different date than today
    \subject{\DocTitle}
    \opening{Hallo \Empfaenger,}
    ich schreibe dir diesen Brief, weil ich Milgrams Experiment wiederhole und
    dabei auf deine Hilfe angewiesen bin.\\

    \section{Was ist Milgrams Experiment?}
    Ist es dir auch schon einmal passiert, dass du überraschend mit einem Freund 
    oder eine Freundin einen gemeinsamen Bekannten hattest? Kennst du
    das "`Freundesfreunde"'-System aus sozialen Netzwerken?

    Dann weißt du im Prinzip schon worum es hier geht. Die Welt ist heutzutage
    durch viele Beziehungen zu anderen Menschen, die teilweise sogar in anderen
    Weltteilen sind, klein geworden. Das hat schon Stanley Milgram 1967 vermutet.
    Er hat seine Vermutung dadurch bestätigt, dass er 60 Personen zufällig
    gewählt hat, die einen Brief an eine Zielperson in Bosten senden sollte.
    Allerdings sollten sie den Brief nur an Personen schicken, die sie kennen
    und die die Zielperson eventuell kennen könnten.\\

    \section{Wie kannst du helfen?}
    Im Anhang ist ein Blatt auf dem die Zielperson steht, an die du diesen Brief
    und alle Blätter im Anhang schicken sollst. 

    Da ich nicht will, dass du einfach die Adresse im Internet
    suchst, gebe ich dir nur den Ort, den Beruf und einen Vornamen. Nun musst 
    du den Brief an einen Bekannten schicken, der diese Person kennen könnte.
    Wenn deine Zielperson also in den USA lebt, könntest du den Brief zuerst
    an einen Englisch-sprachigen Bekannten schicken. Oder du schickst ihn an
    eine Person, von der du weißt dass sie sehr viele weitere Personen kennt.
    Wenn du keine Ahnung hast wer die Zielperson kennen könnte, kannst du den
    Brief auch an einen zufälligen Bekannten schicken. Egal wie, bitte schicke
    den Brief bald weiter! Es wäre schade, wenn er bei dir liegen bleibt.\\

    \section{Ich bin die Zielperson, was nun?}
    Bitte schicke den Brief zurück an:\\

    \noindent Martin Thoma\\
    Parkstraße 17\\
    76131 Karlsruhe\\

    \section{Wo werden die Ergebnisse veröffentlicht?}
    Die Ergebnisse wirst du auf meinem Blog unter \href{http://martin-thoma.com/milgram}{martin-thoma.com/milgram}
    finden.
 
    \closing{Mit freundlichen Grüßen,}
    \end{letter}
\end{document}
