\documentclass[a4paper,10pt]{article}
\usepackage{amssymb, amsmath}
\DeclareMathOperator{\arcsinh}{arcsinh}
\DeclareMathOperator{\arccosh}{arccosh}
\DeclareMathOperator{\arctanh}{arctanh}
\usepackage[utf8]{inputenc} % this is needed for umlauts
\usepackage[ngerman]{babel} % this is needed for umlauts
\usepackage[T1]{fontenc}    % this is needed for correct output of umlauts in pdf
%layout
\usepackage[margin=2.5cm]{geometry}
\usepackage{parskip}

\pdfinfo{
   /Author (Peter Merkert, Martin Thoma)
   /Title  (Wichtige Formeln der Analysis I)
   /CreationDate (D:20120221095400)
   /Subject (Analysis I)
   /Keywords (Analysis I; Formeln)
}

%\everymath={\displaystyle}

\begin{document}

\title{Analysis Formelsammlung}
\author{Peter Merkert, Martin Thoma}
\date{21. Februar 2012}

\section{Grenzwerte}
\begin{table}[ht]
\begin{minipage}[b]{0.5\linewidth}\centering

\begin{align*}
    \lim_{x \to 0} \frac {\sin x}{x}  &= 1 \\
    \lim_{x \to 0} \frac {e^x - 1}{x} &= 1 \\
    \lim_{h \to 0} \frac {e^{{x_0} + h} - e^{x_0}}{h} &= e^{x_0} \\
    \sum_{n = 0}^{\infty} (-1)^n \frac {(-1)^{n + 1}}{n} &= \log 2 \\
    \cos x    &= \sum_{n = 0}^{\infty} (-1)^n \frac {x^{2n}}{(2n)!}  \\
    \sin x    &= \sum_{n = 0}^{\infty} (-1)^n \frac {x^{2n + 1}}{(2n + 1)!}
\end{align*}

\end{minipage}
\hspace{0.5cm}
\begin{minipage}[b]{0.5\linewidth}
\centering

\begin{align*}
\cosh x = \frac {1}{2} (e^x + e^{-x}) &= \scriptstyle \sum_{n = 0}^{\infty} \frac {x^{2n}}{(2n)!} \\
\sinh x = \frac {1}{2} (e^x - e^{-x}) &= \sum_{n = 0}^{\infty} \frac {x^{2n + 1}}{(2n + 1)!} \\
e^x &= \sum_{n = 0}^{\infty} \frac {x^n}{n!} \\
\sum_{n = 0}^{\infty} (-1)^n \frac {x^{n + 1}}{n + 1} &= \log (1+x)   (x \in (-1,1)) \\
\sum_{n = 0}^{\infty} x^n &= \frac {1}{1 - x}    (x \in (-1,1)) \\
0,\bar{3} &= \sum_{n = 1}^{\infty} \frac {3}{(10)^n}
\end{align*}

\end{minipage}
\end{table}

\section{Zusammenhänge}
\begin{align*}
    (\cos x)^2 + (\sin x)^2 &= 1 \\
    (\cosh x)^2 - (\sinh x)^2 &= 1 \\
    \tan x  &= \frac {\sin x}{\cos x} \\
    \tanh x &= \frac {\sinh x}{\cosh x} \\
  (x + y)^n &= \sum_{k=0}^{n} \binom{n}{k} x^{n-k} y^k
\end{align*}

%%%%%%%%%%%%%%%%%%%%%%%%%%%%%%%%%%%%%%%%%%%%%%%%%%%%%%%%%%%%%%%%%%%%%
\section{Ableitungen}
%%%%%%%%%%%%%%%%%%%%%%%%%%%%%%%%%%%%%%%%%%%%%%%%%%%%%%%%%%%%%%%%%%%%%
\begin{table}[ht]
\begin{minipage}[b]{0.3\linewidth}\centering
\begin{align*}
    (\sin x)'    &= \cos x \\
    (\cos x)'    &= -\sin x \\
    (\tan x)'    &= \frac{1}{\cos^2 x} \\
    (\sinh x)'   &= \cosh x \\
    (\cosh x)'   &= \sinh x \\
\end{align*}

\end{minipage}
\hspace{0.1cm}
\begin{minipage}[b]{0.3\linewidth}
\centering

\begin{align*}
    (\arcsin x)'  &=   \frac {1}{\sqrt{1-x^2}} \\
    (\arccos x)'  &= - \frac {1}{\sqrt{1-x^2}} \\
    (\arctan x)'  &=   \frac {1}{1 + x^2} \\
    % (\arcsinh x)' &=   \frac {1}{\sqrt{1+x^2}} \\
    % (\arccosh x)' &=   \frac {1}{\sqrt{(1-x^2) \cdot (1+x^2)}} \\
    % (\arctanh x)' &=   \frac {1}{1 - x^2}
\end{align*}
\end{minipage}
\hspace{0.1cm}
\begin{minipage}[b]{0.3\linewidth}
\centering
\begin{align*}
    (\log x)' &= \frac{1}{x} \\
\end{align*}
\end{minipage}
\end{table}

\section{Werte}
\begin{table}[h]
    \centering
    \begin{tabular}{llll}
        \(\arctan(0) = 0\)             & \(\sin(0) = 0\)                & \(\cos(0) = 1\) \\
        \(\arctan(1) = \frac{\pi}{4}\) & \(\sin(\frac{\pi}{2}) = 1\)    & \(\cos(\frac{\pi}{2}) = 0\)\\
    \end{tabular}
\end{table}

\end{document}
