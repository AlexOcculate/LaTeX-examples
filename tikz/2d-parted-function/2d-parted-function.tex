\documentclass[varwidth=true, border=2pt]{standalone}

\usepackage{pgfplots}
\usepackage{tikz}

\begin{document}
\begin{tikzpicture}
    \begin{axis}[
    legend pos=south west,
        axis x line=middle,
        axis y line=middle,
        grid = major,
        %width=9cm,
        %height=4.5cm,
        grid style={dashed, gray!30},
        xmin=-1,     % start the diagram at this x-coordinate
        xmax= 6,    % end   the diagram at this x-coordinate
        ymin=-0.25,     % start the diagram at this y-coordinate
        ymax= 2.25,   % end   the diagram at this y-coordinate
        axis background/.style={fill=white},
        xlabel=x,
        ylabel=y,
        %xticklabels={-2,-1.6,...,7},
        %yticklabels={-8,-7,...,8},
        tick align=outside,
        minor tick num=-3,
        enlargelimits=true,
        tension=0.08]
      % plot the stirling-formulae
      \addplot[domain=0:1, red, thick,samples=20] {0.5*x*x}; 
      \addplot[domain=1:2, green, thick,samples=20] {x-0.5}; 
      \addplot[domain=2:3, blue, thick,samples=500] {-0.5*(x-2)*(x-2)+x-0.5}; 
      \addplot[domain=3:5, purple, thick,samples=20] {5-x}; 
      \addplot[domain=5:7, orange, thick,samples=3] {0}; 
      \addplot[domain=-3:0, orange, thick,samples=3] {0}; 
      %\addlegendentry{$f_1(x)=\frac{1}{2}x^2$}
      %\addlegendentry{$f_2(x)=x-\frac{1}{2}$}
      %\addlegendentry{$f_2(x)=-\frac{1}{2} (x-2)^2+x-\frac{1}{2}$}
    \end{axis} 
\end{tikzpicture}
\end{document}
