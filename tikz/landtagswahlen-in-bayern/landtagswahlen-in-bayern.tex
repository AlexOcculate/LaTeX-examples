\documentclass[varwidth=false, border=2pt]{standalone}
\usepackage[utf8]{inputenc} % this is needed for umlauts
\usepackage[ngerman]{babel} % this is needed for umlauts
\usepackage[T1]{fontenc}    % this is needed for correct output of umlauts in pdf
\usepackage[margin=2.5cm]{geometry} %layout

\usepackage{pgfplots}
\usepgfplotslibrary{dateplot}
\usetikzlibrary{pgfplots.dateplot}

\begin{document}
\begin{tikzpicture}
    \begin{axis}[
            date coordinates in=x,
            xticklabel={\year},
            x tick label style={align=center, rotate=45},
            yticklabel=$\pgfmathprintnumber{\tick}$\,\%,
            date ZERO=1946-06-30,
            xmin={1946-01-01},
            xmax={2010-01-01},
            extra y ticks={5},
            /pgfplots/ytick={0,10,...,100},
            ymin=0, ymax=100,
            width=15cm, height=8cm,     % size of the image
            grid = major,
            grid style={dashed, gray!30},
            legend style={at={(1.15,1)}, anchor=north}
         ]
          \addplot[blue, dashed, mark=triangle*] table [x=Wahltag, y=Wahlbeteiligung, col sep=comma] {landtagswahlen-in-bayern.csv};
          \addplot[black,mark=square*] table [x=Wahltag, y=CSU, col sep=comma] {landtagswahlen-in-bayern.csv};
          \addplot[red,mark=square*] table [x=Wahltag, y=SPD, col sep=comma] {landtagswahlen-in-bayern.csv};
          \addplot[green,mark=square*] table [x=Wahltag, y=GRÜNE, col sep=comma] {landtagswahlen-in-bayern.csv};
          \addplot[yellow,mark=square*] table [x=Wahltag, y=FDP, col sep=comma] {landtagswahlen-in-bayern.csv};
          \legend{Wahlbeteiligung,CSU,SPD,Grüne,FDP}
    \end{axis}
\end{tikzpicture}
\end{document}
