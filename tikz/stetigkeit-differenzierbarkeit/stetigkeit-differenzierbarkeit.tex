\documentclass{article}

\usepackage[utf8]{inputenc} % this is needed for umlauts
\usepackage[ngerman]{babel} % this is needed for umlauts
\usepackage[T1]{fontenc}    % this is needed for correct output of umlauts in pdf

\usepackage[pdftex,active,tightpage]{preview}
\setlength\PreviewBorder{2mm}
\usepackage{tikz}
\usetikzlibrary{shapes,snakes,calc,patterns} 
\usepackage{amsmath,amssymb}
\begin{document}
\begin{preview}

%\begin{align*}
%    f: \mathbb{R} \rightarrow \mathbb{R}\\
%    g: \mathbb{R} \rightarrow \mathbb{R}\\
%\end{align*}

\begin{tikzpicture}[%
    auto,
    example/.style={
      rectangle,
      draw=blue,
      thick,
      fill=blue!20,
      text width=4.5em,
      align=center,
      rounded corners,
      minimum height=2em
    },
    algebraicName/.style={
      text width=7em,
      align=center,
      minimum height=2em
    },
    explanation/.style={
      text width=10em,
      align=left,
      minimum height=3em
    }
  ]
\pgfdeclarepatternformonly{north east lines wide}%
   {\pgfqpoint{-1pt}{-1pt}}%
   {\pgfqpoint{10pt}{10pt}}%
   {\pgfqpoint{9pt}{9pt}}%
   {
     \pgfsetlinewidth{3pt}
     \pgfpathmoveto{\pgfqpoint{0pt}{0pt}}
     \pgfpathlineto{\pgfqpoint{9.1pt}{9.1pt}}
     \pgfusepath{stroke}
    }


    \draw[fill=yellow!20,yellow!20, rounded corners] (-1.85, 0.70) rectangle (13.4,-6.85);
    \draw[fill=lime!20,lime!20, rounded corners]     (-1.75, 0.60) rectangle (7.3,-6.75);
    \draw[fill=purple!20,purple!20, rounded corners] (-1.65,-1.55) rectangle (7.2,-6.65);
    \draw[fill=blue!20,blue!20, rounded corners]     ( 4.55,-3.45) rectangle (13.1,-6.55);
    \draw (0, 0) node[algebraicName] (A) {gleichmäßig stetig}
          (3, 0) node[explanation]   (B) {
            \begin{minipage}{0.90\textwidth}
                \tiny 
                $\forall \varepsilon >0 \ \exists \delta=\delta(\varepsilon)>0\colon\\ |f(x)-f(z)| < \varepsilon\\ \forall x,z \in D \text{ mit } |x-z|<\delta$
            \end{minipage}
          }
          (6, 0) node[example, draw=lime, fill=lime!15] (X) {\tiny$f_5(x)=\sin(x)$}
          (6,-1) node[example, draw=lime, fill=lime!15] (X) {\tiny$f_6(x)=\cos(x)$}
          (4,-1) node[example, draw=lime, fill=lime!15] (X) {\tiny$f_9(x)=\sqrt x$}
          (0,-2) node[algebraicName, purple] (C) {Lipschitz-stetig}
          (3.5,-2) node[explanation]   (X) {
            \begin{minipage}{90\textwidth}
                \tiny 
$f$ heißt auf $D$ \textbf{Lipschitz-stetig}\\
$:\Leftrightarrow \exists L\ge 0: |f(x)-f(z)|\le L|x-z|\ \forall x,z \in D$
            \end{minipage}
          }
          (12,-6) node[example, draw=blue, fill=black!15] (G) {\tiny$f_2(x) = e^x$}

          (0,-6) node[example, draw=purple, fill=red!15] (K) {\tiny$f_4(x) = |x|$}
          (6,-6) node[example, draw=purple, fill=red!15, pattern=north east lines wide, pattern color=black!25] (N) {\tiny$f_7(x) = 42$}
          (6,-4) node[example, draw=purple, fill=red!15, pattern=north east lines wide, pattern color=black!25] (ANCHORD) {\tiny$f_3(x) = 42$}

          (12,-2) node[example, draw=yellow, fill=yellow!15] (Q) {\tiny$f_1(x) = |x|$}


          (9, 0) node[algebraicName] (O) {stetig}
          (12,0) node[explanation]   (X) {
            \begin{minipage}{0.9\textwidth}
                \tiny 
                $f$ heißt stetig in $x_0 :\Leftrightarrow$\\
                $\forall \varepsilon > 0\  \exists \delta = \delta(\varepsilon)\colon$\\
                $|f(x)-f(x_0)|<\varepsilon$ \\
                $\forall x\in D_\delta(x_0)$
            \end{minipage}
          }

          (12,-4) node[example, draw=blue, fill=black!15] (P) {\tiny$g_1(x) = \frac{1}{x}$}
          (12,-5) node[example, draw=blue, fill=black!15] (P) {\tiny$f_8(x) = x^2$}


          (9, -4) node[algebraicName] (random1) {differenzierbar}
          (9.8, -4.7) node[explanation]   (X) {
            \begin{minipage}{0.9\textwidth}
                \tiny 
                $f$ heißt differenzierbar in $x_0 :\Leftrightarrow$\\
                $\lim_{h \rightarrow 0} \frac{f(x_0+h) - f(x_0)}{h}$
                existiert
            \end{minipage}
          };


    % LP-Stetig
    \draw[purple, thick, rounded corners] ($(C.north west)+(-0.3,0.1)$) rectangle ($(N.south east)+(0.3,-0.3)$);
    % gleichmäßig stetig
    \draw[lime, thick, rounded corners]   ($(A.north west)+(-0.4,0.1)$) rectangle ($(N.south east)+(0.4,-0.4)$);
    % stetige funktionen
    \draw[yellow, thick, rounded corners] ($(A.north west)+(-0.5,0.2)$) rectangle ($(G.south east)+(0.5,-0.5)$);
    % differenzierbar
    \draw[blue, thick, rounded corners] ($(ANCHORD.north west)+(-0.5,0.2)$) rectangle ($(G.south east)+(0.2,-0.2)$);
\end{tikzpicture}
\end{preview}
\end{document}
