\documentclass[varwidth=true, border=2pt]{standalone}

\usepackage[utf8]{inputenc}
\usepackage[TS1,T1]{fontenc}
\usepackage{fourier}
\usepackage{array, booktabs}
\usepackage{graphicx}
\usepackage[x11names]{xcolor}
\usepackage{colortbl}
\usepackage{caption}
\DeclareCaptionFont{blue}{\color{LightSteelBlue3}}

\newcommand{\foo}{\color{LightSteelBlue3}\makebox[0pt]{\textbullet}\hskip-0.5pt\vrule width 1pt\hspace{\labelsep}}

\begin{document}
\begin{table}
\renewcommand\arraystretch{1.4}\arrayrulecolor{LightSteelBlue3}
\captionsetup{singlelinecheck=false, font=blue, labelfont=sc, labelsep=quad}
\caption{Zeitverlauf}\vskip -1.5ex
\begin{tabular}{@{\,}r <{\hskip 2pt} !{\foo} >{\raggedright\arraybackslash}p{5cm}}
\toprule
\addlinespace[1.5ex]
2000 & Planungsarbeiten beginnen bei der Dornier GmbH (EADS, Friedrichshafen)\\
2002 & Flugversuche auf Edwards Air Force Base (Kalifornien) mit RQ-4A Global Hawk + EADS Sensoren\\
21. Oktober 2003 &  6 Testflüge mit dem Prototyp 01 der RQ-4A + EADS-Sensor\\
2007 & BWB, EuroHawk GmbH, Northrop Grumman, EADS schließen Vertrag für die Entwicklung, Erprobung und Unterstützung des UAS-Demonstrators bis 2010 erteilt - Auftragsvolumen: 450 Mio. Euro -- 1300 Mio. Euro\\
8. Oktober 2009 & Rollout\\
29. Juni 2010   & Erstflug von Palmdale zur Edwards Air Force Base\\
21. Juli 2011   & Erste Maschine trifft in Manching zur Einrüstung der Aufklärungselektronik ein. Es gab Schwierigkeiten bei der Überführung, weil US-Behörden Überflugrechte über die USA verweigerten.\\
11. Januar 2013 & Erster Testflug über Deutschland ohne Schwierigkeiten\\
Mai 2013 & De Maizière beendet projekt, weil kein Antikollisionssystem vorhanden ist, {obwohl ein Antikollisionssystem laut Hersteller vorhanden ist / sein sollte}
\end{tabular}
\end{table}
\end{document}
